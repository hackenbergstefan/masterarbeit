\chapter{Tabellen}

Im Folgenden stellen wir die mit Hilfe der vorgestellten Algorithmen
berechneten Werte vor. Dabei ist folgende Legende zu beachten:
\begin{description}
  \item[$q, p,r$] sind die Daten des betrachteten Grundkörpers $\F_q$, wobei
    $q = p^r$ gilt.
  \item[$\CN(q,n)$] gibt die Anzahl der vollständig normalen Elemente
    der Erweiterung von Grad $n$ über $\F_q$ an.
  \item[$\PCN(q,n)$] gibt die Anzahl der primitiv vollständig normalen Elemente 
    der Erweiterung von Grad $n$ über $\F_q$ an.
  \item[\normalfont Erzeuger.] Hier ist die Anzahl der vollständigen Erzeuger
    der Zerlegung nach \thref{satz:zerlegungssatz} gegeben, wobei ein Datum
    $(k,t,\pi):\, N$ bedeutet, dass für den Kreisteilungsmodul 
    $\C_{k,t\pi}$ gerade $N$ vollständige Erzeuger in $\F_q$ existieren.
  \item[$(\speq.)^\ast$] gibt bei Vorhandensein in der Spalte $\CN(q,n)$ an, 
    ob die aktuelle Körpererweiterung einfach (\thref{def:einfach}) ist.
    Falls ja, so gilt per definitionem 
    $\CN(q,n) = \cal N(q,n)$ und $\PCN(q,n) = \PN(q,n)$.
  \item[$(\speq.)^\dagger$] gibt bei Vorhandensein hinter einem Erzeuger-Datum
    an, ob dieser regulär ist (\thref{def:regulaer}).
\end{description}

\begin{longtable}{llllllp{7cm}}
  \caption{Enumerationen $p=2$}\\
  $q$ & $p$ & $r$ & $n$ & $\CN(q,n)$ & $\PCN(q,n)$ & Erzeuger \\\hline
  \endhead
  2 & 2 & 1 & 2 & 2$^\ast$ & 2 & $(1,1,2)^\dagger$: 2\\
2 & 2 & 1 & 3 & 3$^\ast$ & 3 & $(1,1,1)^\dagger$: 1,\ $(3,1,1)^\dagger$: 3\\
2 & 2 & 1 & 4 & 8$^\ast$ & 4 & $(1,1,4)^\dagger$: 8\\
2 & 2 & 1 & 5 & 15$^\ast$ & 15 & $(1,1,1)^\dagger$: 1,\ $(5,1,1)^\dagger$: 15\\
2 & 2 & 1 & 6 & 12 & 6 & $(1,1,2)^\dagger$: 2,\ $(3,1,2)$: 6\\
2 & 2 & 1 & 7 & 49$^\ast$ & 49 & $(1,1,1)^\dagger$: 1,\ $(7,1,1)^\dagger$: 49\\
2 & 2 & 1 & 8 & 128$^\ast$ & 56 & $(1,1,8)^\dagger$: 128\\
2 & 2 & 1 & 9 & 189$^\ast$ & 171 & $(1,1,1)^\dagger$: 1,\ $(3,1,1)^\dagger$: 3,\ $(9,1,1)^\dagger$: 63\\
2 & 2 & 1 & 10 & 420 & 250 & $(1,1,2)^\dagger$: 2,\ $(5,1,2)$: 210\\
2 & 2 & 1 & 11 & 1023$^\ast$ & 957 & $(1,1,1)^\dagger$: 1,\ $(11,1,1)^\dagger$: 1023\\
2 & 2 & 1 & 12 & 768 & 360 & $(1,1,4)^\dagger$: 8,\ $(3,1,4)$: 96\\
2 & 2 & 1 & 13 & 4095$^\ast$ & 4095 & $(1,1,1)^\dagger$: 1,\ $(13,1,1)^\dagger$: 4095\\
2 & 2 & 1 & 14 & 6272$^\ast$ & 4074 & $(1,1,2)^\dagger$: 2,\ $(7,1,2)^\dagger$: 3136\\
2 & 2 & 1 & 15 & 10125$^\ast$ & 8430 & $(1,1,1)^\dagger$: 1,\ $(3,1,1)^\dagger$: 3,\ $(5,1,1)^\dagger$: 15,\ $(15,1,1)^\dagger$: 225\\
2 & 2 & 1 & 16 & 32768$^\ast$ & 16272 & $(1,1,16)^\dagger$: 32768\\
2 & 2 & 1 & 17 & 65025$^\ast$ & 65025 & $(1,1,1)^\dagger$: 1,\ $(17,1,1)^\dagger$: 65025\\
2 & 2 & 1 & 18 & 46872 & 24948 & $(1,1,2)^\dagger$: 2,\ $(3,1,2)$: 6,\ $(9,1,2)$: 3906\\
2 & 2 & 1 & 19 & 262143$^\ast$ & 262143 & $(1,1,1)^\dagger$: 1,\ $(19,1,1)^\dagger$: 262143\\
2 & 2 & 1 & 20 & 329280 & 150320 & $(1,1,4)^\dagger$: 8,\ $(5,1,4)$: 61440\\
2 & 2 & 1 & 21 & 259308 & 220374 & $(1,1,1)^\dagger$: 1,\ $(3,1,1)^\dagger$: 3,\ $(7,3,1)$: 194481\\
2 & 2 & 1 & 22 & 2091012 & 1317250 & $(1,1,2)^\dagger$: 2,\ $(11,1,2)$: 1047552\\
2 & 2 & 1 & 23 & 4190209$^\ast$ & 4099957 & $(1,1,1)^\dagger$: 1,\ $(23,1,1)^\dagger$: 4190209\\
2 & 2 & 1 & 24 & 3145728 & 1246752 & $(1,1,8)^\dagger$: 128,\ $(3,1,8)$: 49152\\
2 & 2 & 1 & 25 & 15728625$^\ast$ & 15188050 & $(1,1,1)^\dagger$: 1,\ $(5,1,1)^\dagger$: 15,\ $(25,1,1)^\dagger$: 1048575\\
2 & 2 & 1 & 26 & 33529860 & 22345232 & $(1,1,2)^\dagger$: 2,\ $(13,1,2)$: 16764930\\
2 & 2 & 1 & 27 & 47258883 & 39950874 & $(1,1,1)^\dagger$: 1,\ $(3,1,1)^\dagger$: 3,\ $(9,1,1)^\dagger$: 63,\ $(27,1,1)^\dagger$: 250047\\
2 & 2 & 1 & 28 & 102760448$^\ast$ & 50821260 & $(1,1,4)^\dagger$: 8,\ $(7,1,4)^\dagger$: 12845056\\
2 & 2 & 1 & 29 & 268435455$^\ast$ & 266908663 & $(1,1,1)^\dagger$: 1,\ $(29,1,1)^\dagger$: 268435455\\
2 & 2 & 1 & 30 & 111132000 & 55308540 & $(1,1,2)^\dagger$: 2,\ $(3,1,2)$: 6,\ $(5,1,2)$: 210,\ $(15,1,2)$: 44100\\
2 & 2 & 1 & 31 & 887503681$^\ast$ & 887503681 & $(1,1,1)^\dagger$: 1,\ $(31,1,1)^\dagger$: 887503681\\
4 & 2 & 2 & 2 & 12$^\ast$ & 8 & $(1,1,2)^\dagger$: 12\\
4 & 2 & 2 & 3 & 27$^\ast$ & 18 & $(1,1,1)^\dagger$: 3,\ $(3,1,1)^\dagger$: 9\\
4 & 2 & 2 & 4 & 192$^\ast$ & 96 & $(1,1,4)^\dagger$: 192\\
4 & 2 & 2 & 5 & 675$^\ast$ & 400 & $(1,1,1)^\dagger$: 3,\ $(5,1,1)^\dagger$: 225\\
4 & 2 & 2 & 6 & 1728$^\ast$ & 792 & $(1,1,2)^\dagger$: 12,\ $(3,1,2)^\dagger$: 144\\
4 & 2 & 2 & 7 & 11907$^\ast$ & 7784 & $(1,1,1)^\dagger$: 3,\ $(7,1,1)^\dagger$: 3969\\
4 & 2 & 2 & 8 & 49152$^\ast$ & 24448 & $(1,1,8)^\dagger$: 49152\\
4 & 2 & 2 & 9 & 107163$^\ast$ & 57186 & $(1,1,1)^\dagger$: 3,\ $(3,1,1)^\dagger$: 9,\ $(9,1,1)^\dagger$: 3969\\
4 & 2 & 2 & 10 & 529200 & 241400 & $(1,1,2)^\dagger$: 12,\ $(5,1,2)$: 44100\\
4 & 2 & 2 & 11 & 3139587$^\ast$ & 1978020 & $(1,1,1)^\dagger$: 3,\ $(11,1,1)^\dagger$: 1046529\\
4 & 2 & 2 & 12 & 7077888$^\ast$ & 2803392 & $(1,1,4)^\dagger$: 192,\ $(3,1,4)^\dagger$: 36864\\
4 & 2 & 2 & 13 & 50307075$^\ast$ & 33525908 & $(1,1,1)^\dagger$: 3,\ $(13,1,1)^\dagger$: 16769025\\
4 & 2 & 2 & 14 & 195084288$^\ast$ & 96481224 & $(1,1,2)^\dagger$: 12,\ $(7,1,2)^\dagger$: 16257024\\
8 & 2 & 3 & 2 & 56$^\ast$ & 36 & $(1,1,2)^\dagger$: 56\\
8 & 2 & 3 & 3 & 441$^\ast$ & 378 & $(1,1,1)^\dagger$: 7,\ $(3,1,1)^\dagger$: 63\\
8 & 2 & 3 & 4 & 3584$^\ast$ & 1512 & $(1,1,4)^\dagger$: 3584\\
8 & 2 & 3 & 5 & 28665$^\ast$ & 23760 & $(1,1,1)^\dagger$: 7,\ $(5,1,1)^\dagger$: 4095\\
8 & 2 & 3 & 6 & 218736 & 117288 & $(1,1,2)^\dagger$: 56,\ $(3,1,2)$: 3906\\
8 & 2 & 3 & 7 & 823543$^\ast$ & 698544 & $(1,1,1)^\dagger$: 7,\ $(7,1,1)^\dagger$: 117649\\
8 & 2 & 3 & 8 & 14680064$^\ast$ & 5804640 & $(1,1,8)^\dagger$: 14680064\\
8 & 2 & 3 & 9 & 110270727$^\ast$ & 93223872 & $(1,1,1)^\dagger$: 7,\ $(3,1,1)^\dagger$: 63,\ $(9,1,1)^\dagger$: 250047\\
16 & 2 & 4 & 3 & 3375$^\ast$ & 1440 & $(1,1,1)^\dagger$: 15,\ $(3,1,1)^\dagger$: 225\\
16 & 2 & 4 & 4 & 61440$^\ast$ & 30720 & $(1,1,4)^\dagger$: 61440\\
16 & 2 & 4 & 6 & 13824000$^\ast$ & 5469696 & $(1,1,2)^\dagger$: 240,\ $(3,1,2)^\dagger$: 57600\\
32 & 2 & 5 & 3 & 31713$^\ast$ & 26100 & $(1,1,1)^\dagger$: 31,\ $(3,1,1)^\dagger$: 1023\\
32 & 2 & 5 & 4 & 1015808$^\ast$ & 465000 & $(1,1,4)^\dagger$: 1015808\\
32 & 2 & 5 & 6 & 1037141952 & 516358800 & $(1,1,2)^\dagger$: 992,\ $(3,1,2)$: 1045506\\
64 & 2 & 6 & 3 & 250047$^\ast$ & 134136 & $(1,1,1)^\dagger$: 63,\ $(3,1,1)^\dagger$: 3969\\
64 & 2 & 6 & 4 & 16515072$^\ast$ & 6531840 & $(1,1,4)^\dagger$: 16515072\\
128 & 2 & 7 & 3 & 2080641$^\ast$ & 1764882 & $(1,1,1)^\dagger$: 127,\ $(3,1,1)^\dagger$: 16383\\
128 & 2 & 7 & 4 & 266338304$^\ast$ & 131721408 & $(1,1,4)^\dagger$: 266338304\\
256 & 2 & 8 & 3 & 16581375$^\ast$ & 6561792 & $(1,1,1)^\dagger$: 255,\ $(3,1,1)^\dagger$: 65025\\
512 & 2 & 9 & 3 & 133955073$^\ast$ & 113245776 & $(1,1,1)^\dagger$: 511,\ $(3,1,1)^\dagger$: 262143\\

\end{longtable}

\begin{longtable}{llllllp{7cm}}
  \caption{Enumerationen $p=3$}\\
  $q$ & $p$ & $r$ & $n$ & $\CN(q,n)$ & $\PCN(q,n)$ & Erzeuger \\\hline
  3 & 3 & 1 & 2 & 4$^\ast$ & 4 & $(1,1,1)^\dagger$: 2,\ $(2,1,1)^\dagger$: 2\\
3 & 3 & 1 & 3 & 18$^\ast$ & 9 & $(1,1,3)^\dagger$: 18\\
3 & 3 & 1 & 4 & 32$^\ast$ & 16 & $(1,1,1)^\dagger$: 2,\ $(2,1,1)^\dagger$: 2,\ $(4,1,1)^\dagger$: 8\\
3 & 3 & 1 & 5 & 160$^\ast$ & 75 & $(1,1,1)^\dagger$: 2,\ $(5,1,1)^\dagger$: 80\\
3 & 3 & 1 & 6 & 324$^\ast$ & 144 & $(1,1,3)^\dagger$: 18,\ $(2,1,3)^\dagger$: 18\\
3 & 3 & 1 & 7 & 1456$^\ast$ & 728 & $(1,1,1)^\dagger$: 2,\ $(7,1,1)^\dagger$: 728\\
3 & 3 & 1 & 8 & 1536 & 576 & $(1,1,1)^\dagger$: 2,\ $(2,1,1)^\dagger$: 2,\ $(4,1,1)^\dagger$: 8,\ $(8,1,1)^\dagger$: 48\\
3 & 3 & 1 & 9 & 13122$^\ast$ & 6075 & $(1,1,9)^\dagger$: 13122\\
3 & 3 & 1 & 10 & 24960 & 11160 & $(1,1,1)^\dagger$: 2,\ $(2,1,1)^\dagger$: 2,\ $(5,2,1)$: 6240\\
3 & 3 & 1 & 11 & 117128$^\ast$ & 55979 & $(1,1,1)^\dagger$: 2,\ $(11,1,1)^\dagger$: 58564\\
3 & 3 & 1 & 12 & 209952$^\ast$ & 65424 & $(1,1,3)^\dagger$: 18,\ $(2,1,3)^\dagger$: 18,\ $(4,1,3)^\dagger$: 648\\
3 & 3 & 1 & 13 & 913952$^\ast$ & 456976 & $(1,1,1)^\dagger$: 2,\ $(13,1,1)^\dagger$: 456976\\
3 & 3 & 1 & 14 & 2114112 & 1054368 & $(1,1,1)^\dagger$: 2,\ $(2,1,1)^\dagger$: 2,\ $(7,2,1)$: 529984\\
3 & 3 & 1 & 15 & 9447840$^\ast$ & 3962700 & $(1,1,3)^\dagger$: 18,\ $(5,1,3)^\dagger$: 524880\\
3 & 3 & 1 & 16 & 6291456 & 2289984 & $(1,1,1)^\dagger$: 2,\ $(2,1,1)^\dagger$: 2,\ $(4,1,1)^\dagger$: 8,\ $(8,1,1)^\dagger$: 64,\ $(16,1,1)^\dagger$: 6400\\
3 & 3 & 1 & 17 & 86093440$^\ast$ & 43022053 & $(1,1,1)^\dagger$: 2,\ $(17,1,1)^\dagger$: 43046720\\
3 & 3 & 1 & 18 & 172186884$^\ast$ & 62696736 & $(1,1,9)^\dagger$: 13122,\ $(2,1,9)^\dagger$: 13122\\
3 & 3 & 1 & 19 & 774840976$^\ast$ & 387177364 & $(1,1,1)^\dagger$: 2,\ $(19,1,1)^\dagger$: 387420488\\
3 & 3 & 1 & 20 & 1184481280 & 423266160 & $(1,1,1)^\dagger$: 2,\ $(2,1,1)^\dagger$: 2,\ $(4,1,1)^\dagger$: 8,\ $(5,4,1)$: 37015040\\
3 & 3 & 1 & 21 & 6935383728 & -- & $(1,1,3)^\dagger$: 18,\ $(7,1,3)$: 385299096\\
3 & 3 & 1 & 22 & 13718968384$^\ast$ & -- & $(1,1,1)^\dagger$: 2,\ $(2,1,1)^\dagger$: 2,\ $(11,1,1)^\dagger$: 58564,\ $(22,1,1)^\dagger$: 58564\\
9 & 3 & 2 & 2 & 64$^\ast$ & 32 & $(1,1,1)^\dagger$: 8,\ $(2,1,1)^\dagger$: 8\\
9 & 3 & 2 & 3 & 648$^\ast$ & 264 & $(1,1,3)^\dagger$: 648\\
9 & 3 & 2 & 4 & 4096$^\ast$ & 1536 & $(1,1,1)^\dagger$: 8,\ $(2,1,1)^\dagger$: 8,\ $(4,1,1)^\dagger$: 64\\
9 & 3 & 2 & 5 & 51200$^\ast$ & 23000 & $(1,1,1)^\dagger$: 8,\ $(5,1,1)^\dagger$: 6400\\
9 & 3 & 2 & 6 & 419904$^\ast$ & 130848 & $(1,1,3)^\dagger$: 648,\ $(2,1,3)^\dagger$: 648\\
9 & 3 & 2 & 7 & 4239872$^\ast$ & 2115008 & $(1,1,1)^\dagger$: 8,\ $(7,1,1)^\dagger$: 529984\\
9 & 3 & 2 & 8 & 16777216$^\ast$ & 6117376 & $(1,1,1)^\dagger$: 8,\ $(2,1,1)^\dagger$: 8,\ $(4,1,1)^\dagger$: 64,\ $(8,1,1)^\dagger$: 4096\\
9 & 3 & 2 & 9 & 344373768$^\ast$ & 125421768 & $(1,1,9)^\dagger$: 344373768\\
27 & 3 & 3 & 3 & 18954$^\ast$ & 8748 & $(1,1,3)^\dagger$: 18954\\
27 & 3 & 3 & 4 & 492128$^\ast$ & 154368 & $(1,1,1)^\dagger$: 26,\ $(2,1,1)^\dagger$: 26,\ $(4,1,1)^\dagger$: 728\\
27 & 3 & 3 & 6 & 359254116$^\ast$ & 130838112 & $(1,1,3)^\dagger$: 18954,\ $(2,1,3)^\dagger$: 18954\\
81 & 3 & 4 & 3 & 524880$^\ast$ & 163584 & $(1,1,3)^\dagger$: 524880\\
81 & 3 & 4 & 4 & 40960000$^\ast$ & 14962688 & $(1,1,1)^\dagger$: 80,\ $(2,1,1)^\dagger$: 80,\ $(4,1,1)^\dagger$: 6400\\
243 & 3 & 5 & 3 & 14289858$^\ast$ & 5994450 & $(1,1,3)^\dagger$: 14289858\\
243 & 3 & 5 & 4 & 3458087072$^\ast$ & 1235872000 & $(1,1,1)^\dagger$: 242,\ $(2,1,1)^\dagger$: 242,\ $(4,1,1)^\dagger$: 59048\\
729 & 3 & 6 & 3 & 386889048$^\ast$ & 140901120 & $(1,1,3)^\dagger$: 386889048\\

\end{longtable}

\begin{longtable}{llllllp{7cm}}
  \caption{Enumerationen $p=5$}\\
  $q$ & $p$ & $r$ & $n$ & $\CN(q,n)$ & $\PCN(q,n)$ & Erzeuger \\\hline
  \endhead
  5 & 5 & 1 & 2 & 16$^\ast$ & 8 & $(1,1,1)^\dagger$: 4,\ $(2,1,1)^\dagger$: 4\\
5 & 5 & 1 & 3 & 96$^\ast$ & 48 & $(1,1,1)^\dagger$: 4,\ $(3,1,1)^\dagger$: 24\\
5 & 5 & 1 & 4 & 256$^\ast$ & 64 & $(1,1,1)^\dagger$: 4,\ $(2,1,1)^\dagger$: 4,\ $(4,1,1)^\dagger$: 16\\
5 & 5 & 1 & 5 & 2500$^\ast$ & 1130 & $(1,1,5)^\dagger$: 2500\\
5 & 5 & 1 & 6 & 8448 & 2376 & $(1,1,1)^\dagger$: 4,\ $(2,1,1)^\dagger$: 4,\ $(3,2,1)$: 528\\
5 & 5 & 1 & 7 & 62496$^\ast$ & 31248 & $(1,1,1)^\dagger$: 4,\ $(7,1,1)^\dagger$: 15624\\
5 & 5 & 1 & 8 & 147456$^\ast$ & 44928 & $(1,1,1)^\dagger$: 4,\ $(2,1,1)^\dagger$: 4,\ $(4,1,1)^\dagger$: 16,\ $(8,1,1)^\dagger$: 576\\
5 & 5 & 1 & 9 & 1499904$^\ast$ & 687132 & $(1,1,1)^\dagger$: 4,\ $(3,1,1)^\dagger$: 24,\ $(9,1,1)^\dagger$: 15624\\
5 & 5 & 1 & 10 & 6250000$^\ast$ & 1862760 & $(1,1,5)^\dagger$: 2500,\ $(2,1,5)^\dagger$: 2500\\
5 & 5 & 1 & 11 & 39037504$^\ast$ & 19518752 & $(1,1,1)^\dagger$: 4,\ $(11,1,1)^\dagger$: 9759376\\
5 & 5 & 1 & 12 & 71368704 & 18178944 & $(1,1,1)^\dagger$: 4,\ $(2,1,1)^\dagger$: 4,\ $(3,2,1)$: 528,\ $(4,1,1)^\dagger$: 16,\ $(12,1,1)$: 528\\
25 & 5 & 2 & 3 & 13824$^\ast$ & 3888 & $(1,1,1)^\dagger$: 24,\ $(3,1,1)^\dagger$: 576\\
25 & 5 & 2 & 4 & 331776$^\ast$ & 101376 & $(1,1,1)^\dagger$: 24,\ $(2,1,1)^\dagger$: 24,\ $(4,1,1)^\dagger$: 576\\
25 & 5 & 2 & 6 & 191102976$^\ast$ & 48691008 & $(1,1,1)^\dagger$: 24,\ $(2,1,1)^\dagger$: 24,\ $(3,1,1)^\dagger$: 576,\ $(6,1,1)^\dagger$: 576\\
125 & 5 & 3 & 3 & 1937376$^\ast$ & 887220 & $(1,1,1)^\dagger$: 124,\ $(3,1,1)^\dagger$: 15624\\
125 & 5 & 3 & 4 & 236421376$^\ast$ & 60235200 & $(1,1,1)^\dagger$: 124,\ $(2,1,1)^\dagger$: 124,\ $(4,1,1)^\dagger$: 15376\\
625 & 5 & 4 & 3 & 242970624$^\ast$ & 61910784 & $(1,1,1)^\dagger$: 624,\ $(3,1,1)^\dagger$: 389376\\

\end{longtable}

\begin{longtable}{llllllp{7cm}}
  \caption{Enumerationen $p=7$}\\
  $q$ & $p$ & $r$ & $n$ & $\CN(q,n)$ & $\PCN(q,n)$ & Erzeuger \\\hline
  \endhead
  7 & 7 & 1 & 2 & 36$^\ast$ & 16 & $(1,1,1)^\dagger$: 6,\ $(2,1,1)^\dagger$: 6\\7 & 7 & 1 & 3 & 216$^\ast$ & 72 & $(1,1,1)^\dagger$: 6,\ $(3,1,1)^\dagger$: 36\\7 & 7 & 1 & 4 & 1728$^\ast$ & 480 & $(1,1,1)^\dagger$: 6,\ $(2,1,1)^\dagger$: 6,\ $(4,1,1)^\dagger$: 48\\7 & 7 & 1 & 5 & 14400$^\ast$ & 4800 & $(1,1,1)^\dagger$: 6,\ $(5,1,1)^\dagger$: 2400\\7 & 7 & 1 & 6 & 46656$^\ast$ & 14832 & $(1,1,1)^\dagger$: 6,\ $(2,1,1)^\dagger$: 6,\ $(3,1,1)^\dagger$: 36,\ $(6,1,1)^\dagger$: 36\\7 & 7 & 1 & 7 & 705894$^\ast$ & 227010 & $(1,1,7)^\dagger$: 705894\\7 & 7 & 1 & 8 & 3815424 & 1016320 & $(1,1,1)^\dagger$: 6,\ $(2,1,1)^\dagger$: 6,\ $(4,1,1)^\dagger$: 48,\ $(8,1,1)^\dagger$: 2208\\7 & 7 & 1 & 9 & 25264224$^\ast$ & 7753806 & $(1,1,1)^\dagger$: 6,\ $(3,1,1)^\dagger$: 36,\ $(9,1,1)^\dagger$: 116964\\7 & 7 & 1 & 10 & 207187200 & 62435920 & $(1,1,1)^\dagger$: 6,\ $(2,1,1)^\dagger$: 6,\ $(5,2,1)$: 5755200\\7 & 7 & 1 & 11 & 1694851488$^\ast$ & 564443264 & $(1,1,1)^\dagger$: 6,\ $(11,1,1)^\dagger$: 282475248\\
\end{longtable}

\begin{longtable}{llllllp{7cm}}
  \caption{Enumerationen $n=3$}\\
  $q$ & $p$ & $r$ & $n$ & $\CN(q,n)$ & $\PCN(q,n)$ & Erzeuger \\\hline
  \endhead
  2 & 2 & 1 & 3 & 3$^\ast$ & 3 & $(1,1,1)^\dagger$: 1,\ $(3,1,1)^\dagger$: 3\\3 & 3 & 1 & 3 & 18$^\ast$ & 9 & $(1,1,3)^\dagger$: 18\\4 & 2 & 2 & 3 & 27$^\ast$ & 18 & $(1,1,1)^\dagger$: 3,\ $(3,1,1)^\dagger$: 9\\5 & 5 & 1 & 3 & 96$^\ast$ & 48 & $(1,1,1)^\dagger$: 4,\ $(3,1,1)^\dagger$: 24\\7 & 7 & 1 & 3 & 216$^\ast$ & 72 & $(1,1,1)^\dagger$: 6,\ $(3,1,1)^\dagger$: 36\\8 & 2 & 3 & 3 & 441$^\ast$ & 378 & $(1,1,1)^\dagger$: 7,\ $(3,1,1)^\dagger$: 63\\9 & 3 & 2 & 3 & 648$^\ast$ & 264 & $(1,1,3)^\dagger$: 648\\11 & 11 & 1 & 3 & 1200$^\ast$ & 384 & $(1,1,1)^\dagger$: 10,\ $(3,1,1)^\dagger$: 120\\13 & 13 & 1 & 3 & 1728$^\ast$ & 576 & $(1,1,1)^\dagger$: 12,\ $(3,1,1)^\dagger$: 144\\16 & 2 & 4 & 3 & 3375$^\ast$ & 1440 & $(1,1,1)^\dagger$: 15,\ $(3,1,1)^\dagger$: 225\\17 & 17 & 1 & 3 & 4608$^\ast$ & 2304 & $(1,1,1)^\dagger$: 16,\ $(3,1,1)^\dagger$: 288\\19 & 19 & 1 & 3 & 5832$^\ast$ & 1944 & $(1,1,1)^\dagger$: 18,\ $(3,1,1)^\dagger$: 324\\23 & 23 & 1 & 3 & 11616$^\ast$ & 4440 & $(1,1,1)^\dagger$: 22,\ $(3,1,1)^\dagger$: 528\\25 & 5 & 2 & 3 & 13824$^\ast$ & 3888 & $(1,1,1)^\dagger$: 24,\ $(3,1,1)^\dagger$: 576\\27 & 3 & 3 & 3 & 18954$^\ast$ & 8748 & $(1,1,3)^\dagger$: 18954\\29 & 29 & 1 & 3 & 23520$^\ast$ & 9180 & $(1,1,1)^\dagger$: 28,\ $(3,1,1)^\dagger$: 840\\31 & 31 & 1 & 3 & 27000$^\ast$ & 7200 & $(1,1,1)^\dagger$: 30,\ $(3,1,1)^\dagger$: 900\\32 & 2 & 5 & 3 & 31713$^\ast$ & 26100 & $(1,1,1)^\dagger$: 31,\ $(3,1,1)^\dagger$: 1023\\37 & 37 & 1 & 3 & 46656$^\ast$ & 13176 & $(1,1,1)^\dagger$: 36,\ $(3,1,1)^\dagger$: 1296\\41 & 41 & 1 & 3 & 67200$^\ast$ & 26880 & $(1,1,1)^\dagger$: 40,\ $(3,1,1)^\dagger$: 1680\\43 & 43 & 1 & 3 & 74088$^\ast$ & 21168 & $(1,1,1)^\dagger$: 42,\ $(3,1,1)^\dagger$: 1764\\47 & 47 & 1 & 3 & 101568$^\ast$ & 46596 & $(1,1,1)^\dagger$: 46,\ $(3,1,1)^\dagger$: 2208\\49 & 7 & 2 & 3 & 110592$^\ast$ & 34272 & $(1,1,1)^\dagger$: 48,\ $(3,1,1)^\dagger$: 2304\\53 & 53 & 1 & 3 & 146016$^\ast$ & 57744 & $(1,1,1)^\dagger$: 52,\ $(3,1,1)^\dagger$: 2808\\59 & 59 & 1 & 3 & 201840$^\ast$ & 97440 & $(1,1,1)^\dagger$: 58,\ $(3,1,1)^\dagger$: 3480\\61 & 61 & 1 & 3 & 216000$^\ast$ & 52848 & $(1,1,1)^\dagger$: 60,\ $(3,1,1)^\dagger$: 3600\\64 & 2 & 6 & 3 & 250047$^\ast$ & 134136 & $(1,1,1)^\dagger$: 63,\ $(3,1,1)^\dagger$: 3969\\67 & 67 & 1 & 3 & 287496$^\ast$ & 72000 & $(1,1,1)^\dagger$: 66,\ $(3,1,1)^\dagger$: 4356\\71 & 71 & 1 & 3 & 352800$^\ast$ & 120960 & $(1,1,1)^\dagger$: 70,\ $(3,1,1)^\dagger$: 5040\\73 & 73 & 1 & 3 & 373248$^\ast$ & 124416 & $(1,1,1)^\dagger$: 72,\ $(3,1,1)^\dagger$: 5184\\79 & 79 & 1 & 3 & 474552$^\ast$ & 122040 & $(1,1,1)^\dagger$: 78,\ $(3,1,1)^\dagger$: 6084\\81 & 3 & 4 & 3 & 524880$^\ast$ & 163584 & $(1,1,3)^\dagger$: 524880\\83 & 83 & 1 & 3 & 564816$^\ast$ & 260280 & $(1,1,1)^\dagger$: 82,\ $(3,1,1)^\dagger$: 6888\\89 & 89 & 1 & 3 & 696960$^\ast$ & 316800 & $(1,1,1)^\dagger$: 88,\ $(3,1,1)^\dagger$: 7920\\97 & 97 & 1 & 3 & 884736$^\ast$ & 294912 & $(1,1,1)^\dagger$: 96,\ $(3,1,1)^\dagger$: 9216\\121 & 11 & 2 & 3 & 1728000$^\ast$ & 364608 & $(1,1,1)^\dagger$: 120,\ $(3,1,1)^\dagger$: 14400\\125 & 5 & 3 & 3 & 1937376$^\ast$ & 887220 & $(1,1,1)^\dagger$: 124,\ $(3,1,1)^\dagger$: 15624\\128 & 2 & 7 & 3 & 2080641$^\ast$ & 1764882 & $(1,1,1)^\dagger$: 127,\ $(3,1,1)^\dagger$: 16383\\169 & 13 & 2 & 3 & 4741632$^\ast$ & 1325376 & $(1,1,1)^\dagger$: 168,\ $(3,1,1)^\dagger$: 28224\\243 & 3 & 5 & 3 & 14289858$^\ast$ & 5994450 & $(1,1,3)^\dagger$: 14289858\\256 & 2 & 8 & 3 & 16581375$^\ast$ & 6561792 & $(1,1,1)^\dagger$: 255,\ $(3,1,1)^\dagger$: 65025\\289 & 17 & 2 & 3 & 23887872$^\ast$ & 6283008 & $(1,1,1)^\dagger$: 288,\ $(3,1,1)^\dagger$: 82944\\343 & 7 & 3 & 3 & 40001688$^\ast$ & 12279276 & $(1,1,1)^\dagger$: 342,\ $(3,1,1)^\dagger$: 116964\\361 & 19 & 2 & 3 & 46656000$^\ast$ & 10584000 & $(1,1,1)^\dagger$: 360,\ $(3,1,1)^\dagger$: 129600\\512 & 2 & 9 & 3 & 133955073$^\ast$ & 113245776 & $(1,1,1)^\dagger$: 511,\ $(3,1,1)^\dagger$: 262143\\529 & 23 & 2 & 3 & 147197952$^\ast$ & 34848000 & $(1,1,1)^\dagger$: 528,\ $(3,1,1)^\dagger$: 278784\\625 & 5 & 4 & 3 & 242970624$^\ast$ & 61910784 & $(1,1,1)^\dagger$: 624,\ $(3,1,1)^\dagger$: 389376\\729 & 3 & 6 & 3 & 386889048$^\ast$ & 140901120 & $(1,1,3)^\dagger$: 386889048\\841 & 29 & 2 & 3 & 592704000$^\ast$ & 122760576 & $(1,1,1)^\dagger$: 840,\ $(3,1,1)^\dagger$: 705600\\961 & 31 & 2 & 3 & 884736000$^\ast$ & 191020032 & $(1,1,1)^\dagger$: 960,\ $(3,1,1)^\dagger$: 921600\\
\end{longtable}

\begin{longtable}{llllllp{7cm}}
  \caption{Enumerationen $n=4$}\\
  $q$ & $p$ & $r$ & $n$ & $\CN(q,n)$ & $\PCN(q,n)$ & Erzeuger \\\hline
  \endhead
  2 & 2 & 1 & 4 & 8$^\ast$ & 4 & $(1,1,4)^\dagger$: 8\\
3 & 3 & 1 & 4 & 32$^\ast$ & 16 & $(1,1,1)^\dagger$: 2,\ $(2,1,1)^\dagger$: 2,\ $(4,1,1)^\dagger$: 8\\
4 & 2 & 2 & 4 & 192$^\ast$ & 96 & $(1,1,4)^\dagger$: 192\\
5 & 5 & 1 & 4 & 256$^\ast$ & 64 & $(1,1,1)^\dagger$: 4,\ $(2,1,1)^\dagger$: 4,\ $(4,1,1)^\dagger$: 16\\
7 & 7 & 1 & 4 & 1728$^\ast$ & 480 & $(1,1,1)^\dagger$: 6,\ $(2,1,1)^\dagger$: 6,\ $(4,1,1)^\dagger$: 48\\
8 & 2 & 3 & 4 & 3584$^\ast$ & 1512 & $(1,1,4)^\dagger$: 3584\\
9 & 3 & 2 & 4 & 4096$^\ast$ & 1536 & $(1,1,1)^\dagger$: 8,\ $(2,1,1)^\dagger$: 8,\ $(4,1,1)^\dagger$: 64\\
11 & 11 & 1 & 4 & 12000$^\ast$ & 3200 & $(1,1,1)^\dagger$: 10,\ $(2,1,1)^\dagger$: 10,\ $(4,1,1)^\dagger$: 120\\
13 & 13 & 1 & 4 & 20736$^\ast$ & 4352 & $(1,1,1)^\dagger$: 12,\ $(2,1,1)^\dagger$: 12,\ $(4,1,1)^\dagger$: 144\\
16 & 2 & 4 & 4 & 61440$^\ast$ & 30720 & $(1,1,4)^\dagger$: 61440\\
17 & 17 & 1 & 4 & 65536$^\ast$ & 16896 & $(1,1,1)^\dagger$: 16,\ $(2,1,1)^\dagger$: 16,\ $(4,1,1)^\dagger$: 256\\
19 & 19 & 1 & 4 & 116640$^\ast$ & 31104 & $(1,1,1)^\dagger$: 18,\ $(2,1,1)^\dagger$: 18,\ $(4,1,1)^\dagger$: 360\\
23 & 23 & 1 & 4 & 255552$^\ast$ & 60640 & $(1,1,1)^\dagger$: 22,\ $(2,1,1)^\dagger$: 22,\ $(4,1,1)^\dagger$: 528\\
25 & 5 & 2 & 4 & 331776$^\ast$ & 101376 & $(1,1,1)^\dagger$: 24,\ $(2,1,1)^\dagger$: 24,\ $(4,1,1)^\dagger$: 576\\
27 & 3 & 3 & 4 & 492128$^\ast$ & 154368 & $(1,1,1)^\dagger$: 26,\ $(2,1,1)^\dagger$: 26,\ $(4,1,1)^\dagger$: 728\\
29 & 29 & 1 & 4 & 614656$^\ast$ & 139776 & $(1,1,1)^\dagger$: 28,\ $(2,1,1)^\dagger$: 28,\ $(4,1,1)^\dagger$: 784\\
31 & 31 & 1 & 4 & 864000$^\ast$ & 207360 & $(1,1,1)^\dagger$: 30,\ $(2,1,1)^\dagger$: 30,\ $(4,1,1)^\dagger$: 960\\
32 & 2 & 5 & 4 & 1015808$^\ast$ & 465000 & $(1,1,4)^\dagger$: 1015808\\
37 & 37 & 1 & 4 & 1679616$^\ast$ & 420864 & $(1,1,1)^\dagger$: 36,\ $(2,1,1)^\dagger$: 36,\ $(4,1,1)^\dagger$: 1296\\
41 & 41 & 1 & 4 & 2560000$^\ast$ & 564224 & $(1,1,1)^\dagger$: 40,\ $(2,1,1)^\dagger$: 40,\ $(4,1,1)^\dagger$: 1600\\
43 & 43 & 1 & 4 & 3259872$^\ast$ & 659712 & $(1,1,1)^\dagger$: 42,\ $(2,1,1)^\dagger$: 42,\ $(4,1,1)^\dagger$: 1848\\
47 & 47 & 1 & 4 & 4672128$^\ast$ & 1036288 & $(1,1,1)^\dagger$: 46,\ $(2,1,1)^\dagger$: 46,\ $(4,1,1)^\dagger$: 2208\\
49 & 7 & 2 & 4 & 5308416$^\ast$ & 1413120 & $(1,1,1)^\dagger$: 48,\ $(2,1,1)^\dagger$: 48,\ $(4,1,1)^\dagger$: 2304\\
53 & 53 & 1 & 4 & 7311616$^\ast$ & 1794816 & $(1,1,1)^\dagger$: 52,\ $(2,1,1)^\dagger$: 52,\ $(4,1,1)^\dagger$: 2704\\
59 & 59 & 1 & 4 & 11706720$^\ast$ & 3014144 & $(1,1,1)^\dagger$: 58,\ $(2,1,1)^\dagger$: 58,\ $(4,1,1)^\dagger$: 3480\\
61 & 61 & 1 & 4 & 12960000$^\ast$ & 3340800 & $(1,1,1)^\dagger$: 60,\ $(2,1,1)^\dagger$: 60,\ $(4,1,1)^\dagger$: 3600\\
64 & 2 & 6 & 4 & 16515072$^\ast$ & 6531840 & $(1,1,4)^\dagger$: 16515072\\
67 & 67 & 1 & 4 & 19549728$^\ast$ & 4453760 & $(1,1,1)^\dagger$: 66,\ $(2,1,1)^\dagger$: 66,\ $(4,1,1)^\dagger$: 4488\\
71 & 71 & 1 & 4 & 24696000$^\ast$ & 5644800 & $(1,1,1)^\dagger$: 70,\ $(2,1,1)^\dagger$: 70,\ $(4,1,1)^\dagger$: 5040\\
73 & 73 & 1 & 4 & 26873856$^\ast$ & 6279168 & $(1,1,1)^\dagger$: 72,\ $(2,1,1)^\dagger$: 72,\ $(4,1,1)^\dagger$: 5184\\
79 & 79 & 1 & 4 & 37964160$^\ast$ & 9345024 & $(1,1,1)^\dagger$: 78,\ $(2,1,1)^\dagger$: 78,\ $(4,1,1)^\dagger$: 6240\\
81 & 3 & 4 & 4 & 40960000$^\ast$ & 14962688 & $(1,1,1)^\dagger$: 80,\ $(2,1,1)^\dagger$: 80,\ $(4,1,1)^\dagger$: 6400\\
83 & 83 & 1 & 4 & 46314912$^\ast$ & 9351040 & $(1,1,1)^\dagger$: 82,\ $(2,1,1)^\dagger$: 82,\ $(4,1,1)^\dagger$: 6888\\
89 & 89 & 1 & 4 & 59969536$^\ast$ & 13620480 & $(1,1,1)^\dagger$: 88,\ $(2,1,1)^\dagger$: 88,\ $(4,1,1)^\dagger$: 7744\\
97 & 97 & 1 & 4 & 84934656$^\ast$ & 19390976 & $(1,1,1)^\dagger$: 96,\ $(2,1,1)^\dagger$: 96,\ $(4,1,1)^\dagger$: 9216\\
121 & 11 & 2 & 4 & 207360000$^\ast$ & 54374400 & $(1,1,1)^\dagger$: 120,\ $(2,1,1)^\dagger$: 120,\ $(4,1,1)^\dagger$: 14400\\
125 & 5 & 3 & 4 & 236421376$^\ast$ & 60235200 & $(1,1,1)^\dagger$: 124,\ $(2,1,1)^\dagger$: 124,\ $(4,1,1)^\dagger$: 15376\\
128 & 2 & 7 & 4 & 266338304$^\ast$ & 131721408 & $(1,1,4)^\dagger$: 266338304\\
169 & 13 & 2 & 4 & 796594176$^\ast$ & 171343872 & $(1,1,1)^\dagger$: 168,\ $(2,1,1)^\dagger$: 168,\ $(4,1,1)^\dagger$: 28224\\
243 & 3 & 5 & 4 & 3458087072$^\ast$ & 1235872000 & $(1,1,1)^\dagger$: 242,\ $(2,1,1)^\dagger$: 242,\ $(4,1,1)^\dagger$: 59048\\

\end{longtable}

\begin{longtable}{llllllp{7cm}}
  \caption{Enumerationen $n=6$}\\
  $q$ & $p$ & $r$ & $n$ & $\CN(q,n)$ & $\PCN(q,n)$ & Erzeuger \\\hline
  \endhead
  2 & 2 & 1 & 6 & 12 & 6 & $(1,1,2)^\dagger$: 2,\ $(3,1,2)$: 6\\
3 & 3 & 1 & 6 & 324$^\ast$ & 144 & $(1,1,3)^\dagger$: 18,\ $(2,1,3)^\dagger$: 18\\
4 & 2 & 2 & 6 & 1728$^\ast$ & 792 & $(1,1,2)^\dagger$: 12,\ $(3,1,2)^\dagger$: 144\\
5 & 5 & 1 & 6 & 8448 & 2376 & $(1,1,1)^\dagger$: 4,\ $(2,1,1)^\dagger$: 4,\ $(3,2,1)$: 528\\
7 & 7 & 1 & 6 & 46656$^\ast$ & 14832 & $(1,1,1)^\dagger$: 6,\ $(2,1,1)^\dagger$: 6,\ $(3,1,1)^\dagger$: 36,\ $(6,1,1)^\dagger$: 36\\
8 & 2 & 3 & 6 & 218736 & 117288 & $(1,1,2)^\dagger$: 56,\ $(3,1,2)$: 3906\\
9 & 3 & 2 & 6 & 419904$^\ast$ & 130848 & $(1,1,3)^\dagger$: 648,\ $(2,1,3)^\dagger$: 648\\
11 & 11 & 1 & 6 & 1416000 & 298848 & $(1,1,1)^\dagger$: 10,\ $(2,1,1)^\dagger$: 10,\ $(3,2,1)$: 14160\\
13 & 13 & 1 & 6 & 2985984$^\ast$ & 834048 & $(1,1,1)^\dagger$: 12,\ $(2,1,1)^\dagger$: 12,\ $(3,1,1)^\dagger$: 144,\ $(6,1,1)^\dagger$: 144\\
16 & 2 & 4 & 6 & 13824000$^\ast$ & 5469696 & $(1,1,2)^\dagger$: 240,\ $(3,1,2)^\dagger$: 57600\\
17 & 17 & 1 & 6 & 21086208 & 5546304 & $(1,1,1)^\dagger$: 16,\ $(2,1,1)^\dagger$: 16,\ $(3,2,1)$: 82368\\
19 & 19 & 1 & 6 & 34012224$^\ast$ & 7711200 & $(1,1,1)^\dagger$: 18,\ $(2,1,1)^\dagger$: 18,\ $(3,1,1)^\dagger$: 324,\ $(6,1,1)^\dagger$: 324\\
23 & 23 & 1 & 6 & 134420352 & 31821840 & $(1,1,1)^\dagger$: 22,\ $(2,1,1)^\dagger$: 22,\ $(3,2,1)$: 277728\\
25 & 5 & 2 & 6 & 191102976$^\ast$ & 48691008 & $(1,1,1)^\dagger$: 24,\ $(2,1,1)^\dagger$: 24,\ $(3,1,1)^\dagger$: 576,\ $(6,1,1)^\dagger$: 576\\
27 & 3 & 3 & 6 & 359254116$^\ast$ & 130838112 & $(1,1,3)^\dagger$: 18954,\ $(2,1,3)^\dagger$: 18954\\
29 & 29 & 1 & 6 & 551873280 & 114307056 & $(1,1,1)^\dagger$: 28,\ $(2,1,1)^\dagger$: 28,\ $(3,2,1)$: 703920\\
31 & 31 & 1 & 6 & 729000000$^\ast$ & 157394880 & $(1,1,1)^\dagger$: 30,\ $(2,1,1)^\dagger$: 30,\ $(3,1,1)^\dagger$: 900,\ $(6,1,1)^\dagger$: 900\\
32 & 2 & 5 & 6 & 1037141952 & 516358800 & $(1,1,2)^\dagger$: 992,\ $(3,1,2)$: 1045506\\
37 & 37 & 1 & 6 & 2176782336$^\ast$ & 548654688 & $(1,1,1)^\dagger$: 36,\ $(2,1,1)^\dagger$: 36,\ $(3,1,1)^\dagger$: 1296,\ $(6,1,1)^\dagger$: 1296\\
41 & 41 & 1 & 6 & 215496704 & 1028522880 & $(1,1,1)^\dagger$: 40,\ $(2,1,1)^\dagger$: 40,\ $(3,2,1)$: 2819040\\
43 & 43 & 1 & 6 & 1194064448$^\ast$ & 1304511264 & $(1,1,1)^\dagger$: 42,\ $(2,1,1)^\dagger$: 42,\ $(3,1,1)^\dagger$: 1764,\ $(6,1,1)^\dagger$: 1764\\

\end{longtable}
