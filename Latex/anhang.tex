\chapter{Tabellen}

\section{Enumerationen}

Im Folgenden stellen wir die mit Hilfe der vorgestellten Algorithmen
berechneten Werte vor. Dabei ist folgende Legende zu beachten:
\begin{description}
  \item[$q, p,r$] sind die Daten des betrachteten Grundkörpers $\F_q$, wobei
    $q = p^r$ gilt.
  \item[$\cal N(q,n)$] gibt die Anzahl der normalen Elemente
    der Erweiterung von Grad $n$ über $\F_q$ an.
  \item[$\CN(q,n)$] gibt die Anzahl der vollständig normalen Elemente
    der Erweiterung von Grad $n$ über $\F_q$ an.
  \item[$\PCN(q,n)$] gibt die Anzahl der primitiv vollständig normalen Elemente 
    der Erweiterung von Grad $n$ über $\F_q$ an. Ist in dieser Spalte ein
    -- vorzufinden, so konzentrierte sich diese Berechnung lediglich auf 
    die Daten für $\CN(q,n)$.
  \item[$\PN(q,n)$] gibt die Anzahl der primitiv normalen Elemente 
    der Erweiterung von Grad $n$ über $\F_q$ an.
  \item[\normalfont Erzeuger.] Hier ist die Anzahl der vollständigen Erzeuger
    der Zerlegung nach \thref{satz:zerlegungssatz} gegeben, wobei ein Datum
    $(k,t,\pi):\, N$ bedeutet, dass für den Kreisteilungsmodul 
    $\C_{k,t\pi}$ gerade $N$ vollständige Erzeuger in $\F_q$ existieren.
  \item[$(\speq.)^\ast$] gibt bei Vorhandensein in der Spalte $\CN(q,n)$ an, 
    ob die aktuelle Körpererweiterung einfach (\thref{def:einfach}) ist.
    Falls ja, so gilt per definitionem 
    $\CN(q,n) = \cal N(q,n)$ und $\PCN(q,n) = \PN(q,n)$.
    Daher sind in den Tabellen mit (primitiv) normalen Elementen lediglich
    diejenigen Erweiterungen gelistet, die nicht einfach sind.
  \item[$(\speq.)^\dagger$] gibt bei Vorhandensein hinter einem Erzeuger-Datum
    an, ob dieser regulär ist (\thref{def:regulaer}).
\end{description}

\newcommand{\createPTables}[2]{%
  \foreach \x in {#1}{%
    \begin{longtable}[h]{llllllp{7cm}}
      \caption{Enumerationen $p=\x$}\\
      $q$ & $p$ & $r$ & $n$ & $\CN(q,n)$ & $\PCN(q,n)$ & Erzeuger \\\hline
      \endhead
      \input{./tables/enumerationsPCN_P_\x.tex}
    \end{longtable}

    \begin{longtable}[h]{llllll}
      \caption{Enumerationen $p=\x$}\\
      $q$ & $p$ & $r$ & $n$ & $\cal N(q,n)$ & $\PN(q,n)$ \\\hline
      \endhead
      \input{./tables/enumerationsPN_P_\x.tex}
    \end{longtable}\par}

  \foreach \x in {#2}{%
    \begin{longtable}[h]{llllllp{7cm}}
      \caption{Enumerationen $p=\x$}\\
      $q$ & $p$ & $r$ & $n$ & $\CN(q,n)$ & $\PCN(q,n)$ & Erzeuger \\\hline
      \input{./tables/enumerationsPCN_P_\x.tex}
    \end{longtable}\goodbreak\par}}

\newcommand{\createNTables}[1]{%
  \foreach \x in {#1}{%
    \begin{longtable}[h]{llllllp{7cm}}
      \caption{Enumerationen $n=\x$}\\
      $q$ & $p$ & $r$ & $n$ & $\CN(q,n)$ & $\PCN(q,n)$ & Erzeuger \\\hline
      \input{./tables/enumerationsPCN_N_\x.tex}
    \end{longtable}\par}}


\createPTables{2,3}{2, 3, 5, 7, 11, 13, 17, 19, 23, 29, 31, 37, 41, 43}

\createNTables{3,4,6}
