\chapter{Tabellen}

\section{Enumerationen}

Im Folgenden stellen wir die mit Hilfe der vorgestellten Algorithmen
berechneten Werte vor. Dabei ist folgende Legende zu beachten:
\begin{description}
  \item[$q, p,r$] sind die Daten des betrachteten Grundkörpers $\F_q$, wobei
    $q = p^r$ gilt.
  \item[$\cal N(q,n)$] gibt die Anzahl der normalen Elemente
    der Erweiterung von Grad $n$ über $\F_q$ an.
  \item[$\CN(q,n)$] gibt die Anzahl der vollständig normalen Elemente
    der Erweiterung von Grad $n$ über $\F_q$ an.
  \item[$\PCN(q,n)$] gibt die Anzahl der primitiv vollständig normalen Elemente 
    der Erweiterung von Grad $n$ über $\F_q$ an.
  \item[$\PN(q,n)$] gibt die Anzahl der primitiv normalen Elemente 
    der Erweiterung von Grad $n$ über $\F_q$ an.
  \item[\normalfont vollst. Erzeuger.] Hier ist die Anzahl der vollständigen Erzeuger
    der Zerlegung nach \thref{satz:zerlegungssatz} gegeben, wobei ein Datum
    $(k,t,\pi):\, N$ bedeutet, dass für den Kreisteilungsmodul 
    $\C_{k,t\pi}$ gerade $N$ vollständige Erzeuger in $\F_q$ existieren.
  \item[$(\speq.)^\ast$] gibt bei Vorhandensein in der Spalte $\CN(q,n)$ an, 
    ob die aktuelle Körpererweiterung einfach (\thref{def:einfach}) ist.
    Falls ja, so gilt per definitionem 
    $\CN(q,n) = \cal N(q,n)$ und $\PCN(q,n) = \PN(q,n)$.
    Daher sind in den Tabellen mit (primitiv) normalen Elementen lediglich
    diejenigen Erweiterungen gelistet, die nicht einfach sind.
  \item[$(\speq.)^\dagger$] gibt bei Vorhandensein hinter einem Erzeuger-Datum
    an, ob dieser regulär ist (\thref{def:regulaer}).
\end{description}

\begin{longtable}[h]{llllllp{7cm}}
  \caption{Enumerationen $p=2$}\\
  $q$ & $p$ & $r$ & $n$ & $\CN(q,n)$ & $\PCN(q,n)$ & Erzeuger \\\hline
  \endhead
  2 & 2 & 1 & 2 & 2$^\ast$ & 2 & $(1,1,2)^\dagger$: 2\\
2 & 2 & 1 & 3 & 3$^\ast$ & 3 & $(1,1,1)^\dagger$: 1,\ $(3,1,1)^\dagger$: 3\\
2 & 2 & 1 & 4 & 8$^\ast$ & 4 & $(1,1,4)^\dagger$: 8\\
2 & 2 & 1 & 5 & 15$^\ast$ & 15 & $(1,1,1)^\dagger$: 1,\ $(5,1,1)^\dagger$: 15\\
2 & 2 & 1 & 6 & 12 & 6 & $(1,1,2)^\dagger$: 2,\ $(3,1,2)$: 6\\
2 & 2 & 1 & 7 & 49$^\ast$ & 49 & $(1,1,1)^\dagger$: 1,\ $(7,1,1)^\dagger$: 49\\
2 & 2 & 1 & 8 & 128$^\ast$ & 56 & $(1,1,8)^\dagger$: 128\\
2 & 2 & 1 & 9 & 189$^\ast$ & 171 & $(1,1,1)^\dagger$: 1,\ $(3,1,1)^\dagger$: 3,\ $(9,1,1)^\dagger$: 63\\
2 & 2 & 1 & 10 & 420 & 250 & $(1,1,2)^\dagger$: 2,\ $(5,1,2)$: 210\\
2 & 2 & 1 & 11 & 1023$^\ast$ & 957 & $(1,1,1)^\dagger$: 1,\ $(11,1,1)^\dagger$: 1023\\
2 & 2 & 1 & 12 & 768 & 360 & $(1,1,4)^\dagger$: 8,\ $(3,1,4)$: 96\\
2 & 2 & 1 & 13 & 4095$^\ast$ & 4095 & $(1,1,1)^\dagger$: 1,\ $(13,1,1)^\dagger$: 4095\\
2 & 2 & 1 & 14 & 6272$^\ast$ & 4074 & $(1,1,2)^\dagger$: 2,\ $(7,1,2)^\dagger$: 3136\\
2 & 2 & 1 & 15 & 10125$^\ast$ & 8430 & $(1,1,1)^\dagger$: 1,\ $(3,1,1)^\dagger$: 3,\ $(5,1,1)^\dagger$: 15,\ $(15,1,1)^\dagger$: 225\\
2 & 2 & 1 & 16 & 32768$^\ast$ & 16272 & $(1,1,16)^\dagger$: 32768\\
2 & 2 & 1 & 17 & 65025$^\ast$ & 65025 & $(1,1,1)^\dagger$: 1,\ $(17,1,1)^\dagger$: 65025\\
2 & 2 & 1 & 18 & 46872 & 24948 & $(1,1,2)^\dagger$: 2,\ $(3,1,2)$: 6,\ $(9,1,2)$: 3906\\
2 & 2 & 1 & 19 & 262143$^\ast$ & 262143 & $(1,1,1)^\dagger$: 1,\ $(19,1,1)^\dagger$: 262143\\
2 & 2 & 1 & 20 & 329280 & 150320 & $(1,1,4)^\dagger$: 8,\ $(5,1,4)$: 61440\\
2 & 2 & 1 & 21 & 259308 & 220374 & $(1,1,1)^\dagger$: 1,\ $(3,1,1)^\dagger$: 3,\ $(7,3,1)$: 194481\\
2 & 2 & 1 & 22 & 2091012 & 1317250 & $(1,1,2)^\dagger$: 2,\ $(11,1,2)$: 1047552\\
2 & 2 & 1 & 23 & 4190209$^\ast$ & 4099957 & $(1,1,1)^\dagger$: 1,\ $(23,1,1)^\dagger$: 4190209\\
2 & 2 & 1 & 24 & 3145728 & 1246752 & $(1,1,8)^\dagger$: 128,\ $(3,1,8)$: 49152\\
2 & 2 & 1 & 25 & 15728625$^\ast$ & 15188050 & $(1,1,1)^\dagger$: 1,\ $(5,1,1)^\dagger$: 15,\ $(25,1,1)^\dagger$: 1048575\\
2 & 2 & 1 & 26 & 33529860 & 22345232 & $(1,1,2)^\dagger$: 2,\ $(13,1,2)$: 16764930\\
2 & 2 & 1 & 27 & 47258883 & 39950874 & $(1,1,1)^\dagger$: 1,\ $(3,1,1)^\dagger$: 3,\ $(9,1,1)^\dagger$: 63,\ $(27,1,1)^\dagger$: 250047\\
2 & 2 & 1 & 28 & 102760448$^\ast$ & 50821260 & $(1,1,4)^\dagger$: 8,\ $(7,1,4)^\dagger$: 12845056\\
2 & 2 & 1 & 29 & 268435455$^\ast$ & 266908663 & $(1,1,1)^\dagger$: 1,\ $(29,1,1)^\dagger$: 268435455\\
2 & 2 & 1 & 30 & 111132000 & 55308540 & $(1,1,2)^\dagger$: 2,\ $(3,1,2)$: 6,\ $(5,1,2)$: 210,\ $(15,1,2)$: 44100\\
2 & 2 & 1 & 31 & 887503681$^\ast$ & 887503681 & $(1,1,1)^\dagger$: 1,\ $(31,1,1)^\dagger$: 887503681\\
4 & 2 & 2 & 2 & 12$^\ast$ & 8 & $(1,1,2)^\dagger$: 12\\
4 & 2 & 2 & 3 & 27$^\ast$ & 18 & $(1,1,1)^\dagger$: 3,\ $(3,1,1)^\dagger$: 9\\
4 & 2 & 2 & 4 & 192$^\ast$ & 96 & $(1,1,4)^\dagger$: 192\\
4 & 2 & 2 & 5 & 675$^\ast$ & 400 & $(1,1,1)^\dagger$: 3,\ $(5,1,1)^\dagger$: 225\\
4 & 2 & 2 & 6 & 1728$^\ast$ & 792 & $(1,1,2)^\dagger$: 12,\ $(3,1,2)^\dagger$: 144\\
4 & 2 & 2 & 7 & 11907$^\ast$ & 7784 & $(1,1,1)^\dagger$: 3,\ $(7,1,1)^\dagger$: 3969\\
4 & 2 & 2 & 8 & 49152$^\ast$ & 24448 & $(1,1,8)^\dagger$: 49152\\
4 & 2 & 2 & 9 & 107163$^\ast$ & 57186 & $(1,1,1)^\dagger$: 3,\ $(3,1,1)^\dagger$: 9,\ $(9,1,1)^\dagger$: 3969\\
4 & 2 & 2 & 10 & 529200 & 241400 & $(1,1,2)^\dagger$: 12,\ $(5,1,2)$: 44100\\
4 & 2 & 2 & 11 & 3139587$^\ast$ & 1978020 & $(1,1,1)^\dagger$: 3,\ $(11,1,1)^\dagger$: 1046529\\
4 & 2 & 2 & 12 & 7077888$^\ast$ & 2803392 & $(1,1,4)^\dagger$: 192,\ $(3,1,4)^\dagger$: 36864\\
4 & 2 & 2 & 13 & 50307075$^\ast$ & 33525908 & $(1,1,1)^\dagger$: 3,\ $(13,1,1)^\dagger$: 16769025\\
4 & 2 & 2 & 14 & 195084288$^\ast$ & 96481224 & $(1,1,2)^\dagger$: 12,\ $(7,1,2)^\dagger$: 16257024\\
8 & 2 & 3 & 2 & 56$^\ast$ & 36 & $(1,1,2)^\dagger$: 56\\
8 & 2 & 3 & 3 & 441$^\ast$ & 378 & $(1,1,1)^\dagger$: 7,\ $(3,1,1)^\dagger$: 63\\
8 & 2 & 3 & 4 & 3584$^\ast$ & 1512 & $(1,1,4)^\dagger$: 3584\\
8 & 2 & 3 & 5 & 28665$^\ast$ & 23760 & $(1,1,1)^\dagger$: 7,\ $(5,1,1)^\dagger$: 4095\\
8 & 2 & 3 & 6 & 218736 & 117288 & $(1,1,2)^\dagger$: 56,\ $(3,1,2)$: 3906\\
8 & 2 & 3 & 7 & 823543$^\ast$ & 698544 & $(1,1,1)^\dagger$: 7,\ $(7,1,1)^\dagger$: 117649\\
8 & 2 & 3 & 8 & 14680064$^\ast$ & 5804640 & $(1,1,8)^\dagger$: 14680064\\
8 & 2 & 3 & 9 & 110270727$^\ast$ & 93223872 & $(1,1,1)^\dagger$: 7,\ $(3,1,1)^\dagger$: 63,\ $(9,1,1)^\dagger$: 250047\\
16 & 2 & 4 & 3 & 3375$^\ast$ & 1440 & $(1,1,1)^\dagger$: 15,\ $(3,1,1)^\dagger$: 225\\
16 & 2 & 4 & 4 & 61440$^\ast$ & 30720 & $(1,1,4)^\dagger$: 61440\\
16 & 2 & 4 & 6 & 13824000$^\ast$ & 5469696 & $(1,1,2)^\dagger$: 240,\ $(3,1,2)^\dagger$: 57600\\
32 & 2 & 5 & 3 & 31713$^\ast$ & 26100 & $(1,1,1)^\dagger$: 31,\ $(3,1,1)^\dagger$: 1023\\
32 & 2 & 5 & 4 & 1015808$^\ast$ & 465000 & $(1,1,4)^\dagger$: 1015808\\
32 & 2 & 5 & 6 & 1037141952 & 516358800 & $(1,1,2)^\dagger$: 992,\ $(3,1,2)$: 1045506\\
64 & 2 & 6 & 3 & 250047$^\ast$ & 134136 & $(1,1,1)^\dagger$: 63,\ $(3,1,1)^\dagger$: 3969\\
64 & 2 & 6 & 4 & 16515072$^\ast$ & 6531840 & $(1,1,4)^\dagger$: 16515072\\
128 & 2 & 7 & 3 & 2080641$^\ast$ & 1764882 & $(1,1,1)^\dagger$: 127,\ $(3,1,1)^\dagger$: 16383\\
128 & 2 & 7 & 4 & 266338304$^\ast$ & 131721408 & $(1,1,4)^\dagger$: 266338304\\
256 & 2 & 8 & 3 & 16581375$^\ast$ & 6561792 & $(1,1,1)^\dagger$: 255,\ $(3,1,1)^\dagger$: 65025\\
512 & 2 & 9 & 3 & 133955073$^\ast$ & 113245776 & $(1,1,1)^\dagger$: 511,\ $(3,1,1)^\dagger$: 262143\\

\end{longtable}

\begin{longtable}[h]{llllllp{7cm}}
  \caption{Enumerationen $p=2$}\\
  $q$ & $p$ & $r$ & $n$ & $\cal N(q,n)$ & $\PN(q,n)$ & Erzeuger \\\hline
  \endhead
  2 & 2 & 1 & 6 & 24 & 18\\
2 & 2 & 1 & 10 & 480 & 290\\
2 & 2 & 1 & 12 & 1536 & 624\\
2 & 2 & 1 & 18 & 96768 & 51660\\
2 & 2 & 1 & 20 & 491520 & 225100\\
2 & 2 & 1 & 21 & 583443 & 495159\\
2 & 2 & 1 & 22 & 2095104 & 1319692\\
2 & 2 & 1 & 24 & 6291456 & 2488320\\
2 & 2 & 1 & 26 & 33546240 & 22356074\\
2 & 2 & 1 & 27 & 49545027 & 41883129\\
4 & 2 & 2 & 10 & 691200 & 316760\\

\end{longtable}

\begin{longtable}[h]{llllllp{7cm}}
  \caption{Enumerationen $p=3$}\\
  $q$ & $p$ & $r$ & $n$ & $\CN(q,n)$ & $\PCN(q,n)$ & Erzeuger \\\hline
  3 & 3 & 1 & 2 & 4$^\ast$ & 4 & $(1,1,1)^\dagger$: 2,\ $(2,1,1)^\dagger$: 2\\
3 & 3 & 1 & 3 & 18$^\ast$ & 9 & $(1,1,3)^\dagger$: 18\\
3 & 3 & 1 & 4 & 32$^\ast$ & 16 & $(1,1,1)^\dagger$: 2,\ $(2,1,1)^\dagger$: 2,\ $(4,1,1)^\dagger$: 8\\
3 & 3 & 1 & 5 & 160$^\ast$ & 75 & $(1,1,1)^\dagger$: 2,\ $(5,1,1)^\dagger$: 80\\
3 & 3 & 1 & 6 & 324$^\ast$ & 144 & $(1,1,3)^\dagger$: 18,\ $(2,1,3)^\dagger$: 18\\
3 & 3 & 1 & 7 & 1456$^\ast$ & 728 & $(1,1,1)^\dagger$: 2,\ $(7,1,1)^\dagger$: 728\\
3 & 3 & 1 & 8 & 1536 & 576 & $(1,1,1)^\dagger$: 2,\ $(2,1,1)^\dagger$: 2,\ $(4,1,1)^\dagger$: 8,\ $(8,1,1)^\dagger$: 48\\
3 & 3 & 1 & 9 & 13122$^\ast$ & 6075 & $(1,1,9)^\dagger$: 13122\\
3 & 3 & 1 & 10 & 24960 & 11160 & $(1,1,1)^\dagger$: 2,\ $(2,1,1)^\dagger$: 2,\ $(5,2,1)$: 6240\\
3 & 3 & 1 & 11 & 117128$^\ast$ & 55979 & $(1,1,1)^\dagger$: 2,\ $(11,1,1)^\dagger$: 58564\\
3 & 3 & 1 & 12 & 209952$^\ast$ & 65424 & $(1,1,3)^\dagger$: 18,\ $(2,1,3)^\dagger$: 18,\ $(4,1,3)^\dagger$: 648\\
3 & 3 & 1 & 13 & 913952$^\ast$ & 456976 & $(1,1,1)^\dagger$: 2,\ $(13,1,1)^\dagger$: 456976\\
3 & 3 & 1 & 14 & 2114112 & 1054368 & $(1,1,1)^\dagger$: 2,\ $(2,1,1)^\dagger$: 2,\ $(7,2,1)$: 529984\\
3 & 3 & 1 & 15 & 9447840$^\ast$ & 3962700 & $(1,1,3)^\dagger$: 18,\ $(5,1,3)^\dagger$: 524880\\
3 & 3 & 1 & 16 & 6291456 & 2289984 & $(1,1,1)^\dagger$: 2,\ $(2,1,1)^\dagger$: 2,\ $(4,1,1)^\dagger$: 8,\ $(8,1,1)^\dagger$: 64,\ $(16,1,1)^\dagger$: 6400\\
3 & 3 & 1 & 17 & 86093440$^\ast$ & 43022053 & $(1,1,1)^\dagger$: 2,\ $(17,1,1)^\dagger$: 43046720\\
3 & 3 & 1 & 18 & 172186884$^\ast$ & 62696736 & $(1,1,9)^\dagger$: 13122,\ $(2,1,9)^\dagger$: 13122\\
3 & 3 & 1 & 19 & 774840976$^\ast$ & 387177364 & $(1,1,1)^\dagger$: 2,\ $(19,1,1)^\dagger$: 387420488\\
3 & 3 & 1 & 20 & 1184481280 & 423266160 & $(1,1,1)^\dagger$: 2,\ $(2,1,1)^\dagger$: 2,\ $(4,1,1)^\dagger$: 8,\ $(5,4,1)$: 37015040\\
3 & 3 & 1 & 21 & 6935383728 & -- & $(1,1,3)^\dagger$: 18,\ $(7,1,3)$: 385299096\\
3 & 3 & 1 & 22 & 13718968384$^\ast$ & -- & $(1,1,1)^\dagger$: 2,\ $(2,1,1)^\dagger$: 2,\ $(11,1,1)^\dagger$: 58564,\ $(22,1,1)^\dagger$: 58564\\
9 & 3 & 2 & 2 & 64$^\ast$ & 32 & $(1,1,1)^\dagger$: 8,\ $(2,1,1)^\dagger$: 8\\
9 & 3 & 2 & 3 & 648$^\ast$ & 264 & $(1,1,3)^\dagger$: 648\\
9 & 3 & 2 & 4 & 4096$^\ast$ & 1536 & $(1,1,1)^\dagger$: 8,\ $(2,1,1)^\dagger$: 8,\ $(4,1,1)^\dagger$: 64\\
9 & 3 & 2 & 5 & 51200$^\ast$ & 23000 & $(1,1,1)^\dagger$: 8,\ $(5,1,1)^\dagger$: 6400\\
9 & 3 & 2 & 6 & 419904$^\ast$ & 130848 & $(1,1,3)^\dagger$: 648,\ $(2,1,3)^\dagger$: 648\\
9 & 3 & 2 & 7 & 4239872$^\ast$ & 2115008 & $(1,1,1)^\dagger$: 8,\ $(7,1,1)^\dagger$: 529984\\
9 & 3 & 2 & 8 & 16777216$^\ast$ & 6117376 & $(1,1,1)^\dagger$: 8,\ $(2,1,1)^\dagger$: 8,\ $(4,1,1)^\dagger$: 64,\ $(8,1,1)^\dagger$: 4096\\
9 & 3 & 2 & 9 & 344373768$^\ast$ & 125421768 & $(1,1,9)^\dagger$: 344373768\\
27 & 3 & 3 & 3 & 18954$^\ast$ & 8748 & $(1,1,3)^\dagger$: 18954\\
27 & 3 & 3 & 4 & 492128$^\ast$ & 154368 & $(1,1,1)^\dagger$: 26,\ $(2,1,1)^\dagger$: 26,\ $(4,1,1)^\dagger$: 728\\
27 & 3 & 3 & 6 & 359254116$^\ast$ & 130838112 & $(1,1,3)^\dagger$: 18954,\ $(2,1,3)^\dagger$: 18954\\
81 & 3 & 4 & 3 & 524880$^\ast$ & 163584 & $(1,1,3)^\dagger$: 524880\\
81 & 3 & 4 & 4 & 40960000$^\ast$ & 14962688 & $(1,1,1)^\dagger$: 80,\ $(2,1,1)^\dagger$: 80,\ $(4,1,1)^\dagger$: 6400\\
243 & 3 & 5 & 3 & 14289858$^\ast$ & 5994450 & $(1,1,3)^\dagger$: 14289858\\
243 & 3 & 5 & 4 & 3458087072$^\ast$ & 1235872000 & $(1,1,1)^\dagger$: 242,\ $(2,1,1)^\dagger$: 242,\ $(4,1,1)^\dagger$: 59048\\
729 & 3 & 6 & 3 & 386889048$^\ast$ & 140901120 & $(1,1,3)^\dagger$: 386889048\\

\end{longtable}

\begin{longtable}[h]{llllllp{7cm}}
  \caption{Enumerationen $p=3$}\\
  $q$ & $p$ & $r$ & $n$ & $\cal N(q,n)$ & $\PN(q,n)$ & Erzeuger \\\hline
  \endhead
  3 & 3 & 1 & 8 & 2048 & 832\\
3 & 3 & 1 & 10 & 25600 & 11520\\
3 & 3 & 1 & 14 & 2119936 & 1057392\\
3 & 3 & 1 & 16 & 13107200 & 4790656\\
3 & 3 & 1 & 20 & 1310720000 & 468392880\\

\end{longtable}

\begin{longtable}[h]{llllllp{7cm}}
  \caption{Enumerationen $p=5$}\\
  $q$ & $p$ & $r$ & $n$ & $\CN(q,n)$ & $\PCN(q,n)$ & Erzeuger \\\hline
  \endhead
  5 & 5 & 1 & 2 & 16$^\ast$ & 8 & $(1,1,1)^\dagger$: 4,\ $(2,1,1)^\dagger$: 4\\
5 & 5 & 1 & 3 & 96$^\ast$ & 48 & $(1,1,1)^\dagger$: 4,\ $(3,1,1)^\dagger$: 24\\
5 & 5 & 1 & 4 & 256$^\ast$ & 64 & $(1,1,1)^\dagger$: 4,\ $(2,1,1)^\dagger$: 4,\ $(4,1,1)^\dagger$: 16\\
5 & 5 & 1 & 5 & 2500$^\ast$ & 1130 & $(1,1,5)^\dagger$: 2500\\
5 & 5 & 1 & 6 & 8448 & 2376 & $(1,1,1)^\dagger$: 4,\ $(2,1,1)^\dagger$: 4,\ $(3,2,1)$: 528\\
5 & 5 & 1 & 7 & 62496$^\ast$ & 31248 & $(1,1,1)^\dagger$: 4,\ $(7,1,1)^\dagger$: 15624\\
5 & 5 & 1 & 8 & 147456$^\ast$ & 44928 & $(1,1,1)^\dagger$: 4,\ $(2,1,1)^\dagger$: 4,\ $(4,1,1)^\dagger$: 16,\ $(8,1,1)^\dagger$: 576\\
5 & 5 & 1 & 9 & 1499904$^\ast$ & 687132 & $(1,1,1)^\dagger$: 4,\ $(3,1,1)^\dagger$: 24,\ $(9,1,1)^\dagger$: 15624\\
5 & 5 & 1 & 10 & 6250000$^\ast$ & 1862760 & $(1,1,5)^\dagger$: 2500,\ $(2,1,5)^\dagger$: 2500\\
5 & 5 & 1 & 11 & 39037504$^\ast$ & 19518752 & $(1,1,1)^\dagger$: 4,\ $(11,1,1)^\dagger$: 9759376\\
5 & 5 & 1 & 12 & 71368704 & 18178944 & $(1,1,1)^\dagger$: 4,\ $(2,1,1)^\dagger$: 4,\ $(3,2,1)$: 528,\ $(4,1,1)^\dagger$: 16,\ $(12,1,1)$: 528\\
25 & 5 & 2 & 3 & 13824$^\ast$ & 3888 & $(1,1,1)^\dagger$: 24,\ $(3,1,1)^\dagger$: 576\\
25 & 5 & 2 & 4 & 331776$^\ast$ & 101376 & $(1,1,1)^\dagger$: 24,\ $(2,1,1)^\dagger$: 24,\ $(4,1,1)^\dagger$: 576\\
25 & 5 & 2 & 6 & 191102976$^\ast$ & 48691008 & $(1,1,1)^\dagger$: 24,\ $(2,1,1)^\dagger$: 24,\ $(3,1,1)^\dagger$: 576,\ $(6,1,1)^\dagger$: 576\\
125 & 5 & 3 & 3 & 1937376$^\ast$ & 887220 & $(1,1,1)^\dagger$: 124,\ $(3,1,1)^\dagger$: 15624\\
125 & 5 & 3 & 4 & 236421376$^\ast$ & 60235200 & $(1,1,1)^\dagger$: 124,\ $(2,1,1)^\dagger$: 124,\ $(4,1,1)^\dagger$: 15376\\
625 & 5 & 4 & 3 & 242970624$^\ast$ & 61910784 & $(1,1,1)^\dagger$: 624,\ $(3,1,1)^\dagger$: 389376\\

\end{longtable}

\begin{longtable}[h]{llllllp{7cm}}
  \caption{Enumerationen $p=7$}\\
  $q$ & $p$ & $r$ & $n$ & $\CN(q,n)$ & $\PCN(q,n)$ & Erzeuger \\\hline
  \endhead
  7 & 7 & 1 & 2 & 36$^\ast$ & 16 & $(1,1,1)^\dagger$: 6,\ $(2,1,1)^\dagger$: 6\\7 & 7 & 1 & 3 & 216$^\ast$ & 72 & $(1,1,1)^\dagger$: 6,\ $(3,1,1)^\dagger$: 36\\7 & 7 & 1 & 4 & 1728$^\ast$ & 480 & $(1,1,1)^\dagger$: 6,\ $(2,1,1)^\dagger$: 6,\ $(4,1,1)^\dagger$: 48\\7 & 7 & 1 & 5 & 14400$^\ast$ & 4800 & $(1,1,1)^\dagger$: 6,\ $(5,1,1)^\dagger$: 2400\\7 & 7 & 1 & 6 & 46656$^\ast$ & 14832 & $(1,1,1)^\dagger$: 6,\ $(2,1,1)^\dagger$: 6,\ $(3,1,1)^\dagger$: 36,\ $(6,1,1)^\dagger$: 36\\7 & 7 & 1 & 7 & 705894$^\ast$ & 227010 & $(1,1,7)^\dagger$: 705894\\7 & 7 & 1 & 8 & 3815424 & 1016320 & $(1,1,1)^\dagger$: 6,\ $(2,1,1)^\dagger$: 6,\ $(4,1,1)^\dagger$: 48,\ $(8,1,1)^\dagger$: 2208\\7 & 7 & 1 & 9 & 25264224$^\ast$ & 7753806 & $(1,1,1)^\dagger$: 6,\ $(3,1,1)^\dagger$: 36,\ $(9,1,1)^\dagger$: 116964\\7 & 7 & 1 & 10 & 207187200 & 62435920 & $(1,1,1)^\dagger$: 6,\ $(2,1,1)^\dagger$: 6,\ $(5,2,1)$: 5755200\\7 & 7 & 1 & 11 & 1694851488$^\ast$ & 564443264 & $(1,1,1)^\dagger$: 6,\ $(11,1,1)^\dagger$: 282475248\\
\end{longtable}

\begin{longtable}[h]{llllllp{7cm}}
  \caption{Enumerationen $n=3$}\\
  $q$ & $p$ & $r$ & $n$ & $\CN(q,n)$ & $\PCN(q,n)$ & Erzeuger \\\hline
  \endhead
  2 & 2 & 1 & 3 & 3$^\ast$ & 3 & $(1,1,1)^\dagger$: 1,\ $(3,1,1)^\dagger$: 3\\3 & 3 & 1 & 3 & 18$^\ast$ & 9 & $(1,1,3)^\dagger$: 18\\4 & 2 & 2 & 3 & 27$^\ast$ & 18 & $(1,1,1)^\dagger$: 3,\ $(3,1,1)^\dagger$: 9\\5 & 5 & 1 & 3 & 96$^\ast$ & 48 & $(1,1,1)^\dagger$: 4,\ $(3,1,1)^\dagger$: 24\\7 & 7 & 1 & 3 & 216$^\ast$ & 72 & $(1,1,1)^\dagger$: 6,\ $(3,1,1)^\dagger$: 36\\8 & 2 & 3 & 3 & 441$^\ast$ & 378 & $(1,1,1)^\dagger$: 7,\ $(3,1,1)^\dagger$: 63\\9 & 3 & 2 & 3 & 648$^\ast$ & 264 & $(1,1,3)^\dagger$: 648\\11 & 11 & 1 & 3 & 1200$^\ast$ & 384 & $(1,1,1)^\dagger$: 10,\ $(3,1,1)^\dagger$: 120\\13 & 13 & 1 & 3 & 1728$^\ast$ & 576 & $(1,1,1)^\dagger$: 12,\ $(3,1,1)^\dagger$: 144\\16 & 2 & 4 & 3 & 3375$^\ast$ & 1440 & $(1,1,1)^\dagger$: 15,\ $(3,1,1)^\dagger$: 225\\17 & 17 & 1 & 3 & 4608$^\ast$ & 2304 & $(1,1,1)^\dagger$: 16,\ $(3,1,1)^\dagger$: 288\\19 & 19 & 1 & 3 & 5832$^\ast$ & 1944 & $(1,1,1)^\dagger$: 18,\ $(3,1,1)^\dagger$: 324\\23 & 23 & 1 & 3 & 11616$^\ast$ & 4440 & $(1,1,1)^\dagger$: 22,\ $(3,1,1)^\dagger$: 528\\25 & 5 & 2 & 3 & 13824$^\ast$ & 3888 & $(1,1,1)^\dagger$: 24,\ $(3,1,1)^\dagger$: 576\\27 & 3 & 3 & 3 & 18954$^\ast$ & 8748 & $(1,1,3)^\dagger$: 18954\\29 & 29 & 1 & 3 & 23520$^\ast$ & 9180 & $(1,1,1)^\dagger$: 28,\ $(3,1,1)^\dagger$: 840\\31 & 31 & 1 & 3 & 27000$^\ast$ & 7200 & $(1,1,1)^\dagger$: 30,\ $(3,1,1)^\dagger$: 900\\32 & 2 & 5 & 3 & 31713$^\ast$ & 26100 & $(1,1,1)^\dagger$: 31,\ $(3,1,1)^\dagger$: 1023\\37 & 37 & 1 & 3 & 46656$^\ast$ & 13176 & $(1,1,1)^\dagger$: 36,\ $(3,1,1)^\dagger$: 1296\\41 & 41 & 1 & 3 & 67200$^\ast$ & 26880 & $(1,1,1)^\dagger$: 40,\ $(3,1,1)^\dagger$: 1680\\43 & 43 & 1 & 3 & 74088$^\ast$ & 21168 & $(1,1,1)^\dagger$: 42,\ $(3,1,1)^\dagger$: 1764\\47 & 47 & 1 & 3 & 101568$^\ast$ & 46596 & $(1,1,1)^\dagger$: 46,\ $(3,1,1)^\dagger$: 2208\\49 & 7 & 2 & 3 & 110592$^\ast$ & 34272 & $(1,1,1)^\dagger$: 48,\ $(3,1,1)^\dagger$: 2304\\53 & 53 & 1 & 3 & 146016$^\ast$ & 57744 & $(1,1,1)^\dagger$: 52,\ $(3,1,1)^\dagger$: 2808\\59 & 59 & 1 & 3 & 201840$^\ast$ & 97440 & $(1,1,1)^\dagger$: 58,\ $(3,1,1)^\dagger$: 3480\\61 & 61 & 1 & 3 & 216000$^\ast$ & 52848 & $(1,1,1)^\dagger$: 60,\ $(3,1,1)^\dagger$: 3600\\64 & 2 & 6 & 3 & 250047$^\ast$ & 134136 & $(1,1,1)^\dagger$: 63,\ $(3,1,1)^\dagger$: 3969\\67 & 67 & 1 & 3 & 287496$^\ast$ & 72000 & $(1,1,1)^\dagger$: 66,\ $(3,1,1)^\dagger$: 4356\\71 & 71 & 1 & 3 & 352800$^\ast$ & 120960 & $(1,1,1)^\dagger$: 70,\ $(3,1,1)^\dagger$: 5040\\73 & 73 & 1 & 3 & 373248$^\ast$ & 124416 & $(1,1,1)^\dagger$: 72,\ $(3,1,1)^\dagger$: 5184\\79 & 79 & 1 & 3 & 474552$^\ast$ & 122040 & $(1,1,1)^\dagger$: 78,\ $(3,1,1)^\dagger$: 6084\\81 & 3 & 4 & 3 & 524880$^\ast$ & 163584 & $(1,1,3)^\dagger$: 524880\\83 & 83 & 1 & 3 & 564816$^\ast$ & 260280 & $(1,1,1)^\dagger$: 82,\ $(3,1,1)^\dagger$: 6888\\89 & 89 & 1 & 3 & 696960$^\ast$ & 316800 & $(1,1,1)^\dagger$: 88,\ $(3,1,1)^\dagger$: 7920\\97 & 97 & 1 & 3 & 884736$^\ast$ & 294912 & $(1,1,1)^\dagger$: 96,\ $(3,1,1)^\dagger$: 9216\\121 & 11 & 2 & 3 & 1728000$^\ast$ & 364608 & $(1,1,1)^\dagger$: 120,\ $(3,1,1)^\dagger$: 14400\\125 & 5 & 3 & 3 & 1937376$^\ast$ & 887220 & $(1,1,1)^\dagger$: 124,\ $(3,1,1)^\dagger$: 15624\\128 & 2 & 7 & 3 & 2080641$^\ast$ & 1764882 & $(1,1,1)^\dagger$: 127,\ $(3,1,1)^\dagger$: 16383\\169 & 13 & 2 & 3 & 4741632$^\ast$ & 1325376 & $(1,1,1)^\dagger$: 168,\ $(3,1,1)^\dagger$: 28224\\243 & 3 & 5 & 3 & 14289858$^\ast$ & 5994450 & $(1,1,3)^\dagger$: 14289858\\256 & 2 & 8 & 3 & 16581375$^\ast$ & 6561792 & $(1,1,1)^\dagger$: 255,\ $(3,1,1)^\dagger$: 65025\\289 & 17 & 2 & 3 & 23887872$^\ast$ & 6283008 & $(1,1,1)^\dagger$: 288,\ $(3,1,1)^\dagger$: 82944\\343 & 7 & 3 & 3 & 40001688$^\ast$ & 12279276 & $(1,1,1)^\dagger$: 342,\ $(3,1,1)^\dagger$: 116964\\361 & 19 & 2 & 3 & 46656000$^\ast$ & 10584000 & $(1,1,1)^\dagger$: 360,\ $(3,1,1)^\dagger$: 129600\\512 & 2 & 9 & 3 & 133955073$^\ast$ & 113245776 & $(1,1,1)^\dagger$: 511,\ $(3,1,1)^\dagger$: 262143\\529 & 23 & 2 & 3 & 147197952$^\ast$ & 34848000 & $(1,1,1)^\dagger$: 528,\ $(3,1,1)^\dagger$: 278784\\625 & 5 & 4 & 3 & 242970624$^\ast$ & 61910784 & $(1,1,1)^\dagger$: 624,\ $(3,1,1)^\dagger$: 389376\\729 & 3 & 6 & 3 & 386889048$^\ast$ & 140901120 & $(1,1,3)^\dagger$: 386889048\\841 & 29 & 2 & 3 & 592704000$^\ast$ & 122760576 & $(1,1,1)^\dagger$: 840,\ $(3,1,1)^\dagger$: 705600\\961 & 31 & 2 & 3 & 884736000$^\ast$ & 191020032 & $(1,1,1)^\dagger$: 960,\ $(3,1,1)^\dagger$: 921600\\
\end{longtable}

\begin{longtable}[h]{llllllp{7cm}}
  \caption{Enumerationen $n=4$}\\
  $q$ & $p$ & $r$ & $n$ & $\CN(q,n)$ & $\PCN(q,n)$ & Erzeuger \\\hline
  \endhead
  2 & 2 & 1 & 4 & 8$^\ast$ & 4 & $(1,1,4)^\dagger$: 8\\
3 & 3 & 1 & 4 & 32$^\ast$ & 16 & $(1,1,1)^\dagger$: 2,\ $(2,1,1)^\dagger$: 2,\ $(4,1,1)^\dagger$: 8\\
4 & 2 & 2 & 4 & 192$^\ast$ & 96 & $(1,1,4)^\dagger$: 192\\
5 & 5 & 1 & 4 & 256$^\ast$ & 64 & $(1,1,1)^\dagger$: 4,\ $(2,1,1)^\dagger$: 4,\ $(4,1,1)^\dagger$: 16\\
7 & 7 & 1 & 4 & 1728$^\ast$ & 480 & $(1,1,1)^\dagger$: 6,\ $(2,1,1)^\dagger$: 6,\ $(4,1,1)^\dagger$: 48\\
8 & 2 & 3 & 4 & 3584$^\ast$ & 1512 & $(1,1,4)^\dagger$: 3584\\
9 & 3 & 2 & 4 & 4096$^\ast$ & 1536 & $(1,1,1)^\dagger$: 8,\ $(2,1,1)^\dagger$: 8,\ $(4,1,1)^\dagger$: 64\\
11 & 11 & 1 & 4 & 12000$^\ast$ & 3200 & $(1,1,1)^\dagger$: 10,\ $(2,1,1)^\dagger$: 10,\ $(4,1,1)^\dagger$: 120\\
13 & 13 & 1 & 4 & 20736$^\ast$ & 4352 & $(1,1,1)^\dagger$: 12,\ $(2,1,1)^\dagger$: 12,\ $(4,1,1)^\dagger$: 144\\
16 & 2 & 4 & 4 & 61440$^\ast$ & 30720 & $(1,1,4)^\dagger$: 61440\\
17 & 17 & 1 & 4 & 65536$^\ast$ & 16896 & $(1,1,1)^\dagger$: 16,\ $(2,1,1)^\dagger$: 16,\ $(4,1,1)^\dagger$: 256\\
19 & 19 & 1 & 4 & 116640$^\ast$ & 31104 & $(1,1,1)^\dagger$: 18,\ $(2,1,1)^\dagger$: 18,\ $(4,1,1)^\dagger$: 360\\
23 & 23 & 1 & 4 & 255552$^\ast$ & 60640 & $(1,1,1)^\dagger$: 22,\ $(2,1,1)^\dagger$: 22,\ $(4,1,1)^\dagger$: 528\\
25 & 5 & 2 & 4 & 331776$^\ast$ & 101376 & $(1,1,1)^\dagger$: 24,\ $(2,1,1)^\dagger$: 24,\ $(4,1,1)^\dagger$: 576\\
27 & 3 & 3 & 4 & 492128$^\ast$ & 154368 & $(1,1,1)^\dagger$: 26,\ $(2,1,1)^\dagger$: 26,\ $(4,1,1)^\dagger$: 728\\
29 & 29 & 1 & 4 & 614656$^\ast$ & 139776 & $(1,1,1)^\dagger$: 28,\ $(2,1,1)^\dagger$: 28,\ $(4,1,1)^\dagger$: 784\\
31 & 31 & 1 & 4 & 864000$^\ast$ & 207360 & $(1,1,1)^\dagger$: 30,\ $(2,1,1)^\dagger$: 30,\ $(4,1,1)^\dagger$: 960\\
32 & 2 & 5 & 4 & 1015808$^\ast$ & 465000 & $(1,1,4)^\dagger$: 1015808\\
37 & 37 & 1 & 4 & 1679616$^\ast$ & 420864 & $(1,1,1)^\dagger$: 36,\ $(2,1,1)^\dagger$: 36,\ $(4,1,1)^\dagger$: 1296\\
41 & 41 & 1 & 4 & 2560000$^\ast$ & 564224 & $(1,1,1)^\dagger$: 40,\ $(2,1,1)^\dagger$: 40,\ $(4,1,1)^\dagger$: 1600\\
43 & 43 & 1 & 4 & 3259872$^\ast$ & 659712 & $(1,1,1)^\dagger$: 42,\ $(2,1,1)^\dagger$: 42,\ $(4,1,1)^\dagger$: 1848\\
47 & 47 & 1 & 4 & 4672128$^\ast$ & 1036288 & $(1,1,1)^\dagger$: 46,\ $(2,1,1)^\dagger$: 46,\ $(4,1,1)^\dagger$: 2208\\
49 & 7 & 2 & 4 & 5308416$^\ast$ & 1413120 & $(1,1,1)^\dagger$: 48,\ $(2,1,1)^\dagger$: 48,\ $(4,1,1)^\dagger$: 2304\\
53 & 53 & 1 & 4 & 7311616$^\ast$ & 1794816 & $(1,1,1)^\dagger$: 52,\ $(2,1,1)^\dagger$: 52,\ $(4,1,1)^\dagger$: 2704\\
59 & 59 & 1 & 4 & 11706720$^\ast$ & 3014144 & $(1,1,1)^\dagger$: 58,\ $(2,1,1)^\dagger$: 58,\ $(4,1,1)^\dagger$: 3480\\
61 & 61 & 1 & 4 & 12960000$^\ast$ & 3340800 & $(1,1,1)^\dagger$: 60,\ $(2,1,1)^\dagger$: 60,\ $(4,1,1)^\dagger$: 3600\\
64 & 2 & 6 & 4 & 16515072$^\ast$ & 6531840 & $(1,1,4)^\dagger$: 16515072\\
67 & 67 & 1 & 4 & 19549728$^\ast$ & 4453760 & $(1,1,1)^\dagger$: 66,\ $(2,1,1)^\dagger$: 66,\ $(4,1,1)^\dagger$: 4488\\
71 & 71 & 1 & 4 & 24696000$^\ast$ & 5644800 & $(1,1,1)^\dagger$: 70,\ $(2,1,1)^\dagger$: 70,\ $(4,1,1)^\dagger$: 5040\\
73 & 73 & 1 & 4 & 26873856$^\ast$ & 6279168 & $(1,1,1)^\dagger$: 72,\ $(2,1,1)^\dagger$: 72,\ $(4,1,1)^\dagger$: 5184\\
79 & 79 & 1 & 4 & 37964160$^\ast$ & 9345024 & $(1,1,1)^\dagger$: 78,\ $(2,1,1)^\dagger$: 78,\ $(4,1,1)^\dagger$: 6240\\
81 & 3 & 4 & 4 & 40960000$^\ast$ & 14962688 & $(1,1,1)^\dagger$: 80,\ $(2,1,1)^\dagger$: 80,\ $(4,1,1)^\dagger$: 6400\\
83 & 83 & 1 & 4 & 46314912$^\ast$ & 9351040 & $(1,1,1)^\dagger$: 82,\ $(2,1,1)^\dagger$: 82,\ $(4,1,1)^\dagger$: 6888\\
89 & 89 & 1 & 4 & 59969536$^\ast$ & 13620480 & $(1,1,1)^\dagger$: 88,\ $(2,1,1)^\dagger$: 88,\ $(4,1,1)^\dagger$: 7744\\
97 & 97 & 1 & 4 & 84934656$^\ast$ & 19390976 & $(1,1,1)^\dagger$: 96,\ $(2,1,1)^\dagger$: 96,\ $(4,1,1)^\dagger$: 9216\\
121 & 11 & 2 & 4 & 207360000$^\ast$ & 54374400 & $(1,1,1)^\dagger$: 120,\ $(2,1,1)^\dagger$: 120,\ $(4,1,1)^\dagger$: 14400\\
125 & 5 & 3 & 4 & 236421376$^\ast$ & 60235200 & $(1,1,1)^\dagger$: 124,\ $(2,1,1)^\dagger$: 124,\ $(4,1,1)^\dagger$: 15376\\
128 & 2 & 7 & 4 & 266338304$^\ast$ & 131721408 & $(1,1,4)^\dagger$: 266338304\\
169 & 13 & 2 & 4 & 796594176$^\ast$ & 171343872 & $(1,1,1)^\dagger$: 168,\ $(2,1,1)^\dagger$: 168,\ $(4,1,1)^\dagger$: 28224\\
243 & 3 & 5 & 4 & 3458087072$^\ast$ & 1235872000 & $(1,1,1)^\dagger$: 242,\ $(2,1,1)^\dagger$: 242,\ $(4,1,1)^\dagger$: 59048\\

\end{longtable}

\begin{longtable}[h]{llllllp{7cm}}
  \caption{Enumerationen $n=6$}\\
  $q$ & $p$ & $r$ & $n$ & $\CN(q,n)$ & $\PCN(q,n)$ & Erzeuger \\\hline
  \endhead
  2 & 2 & 1 & 6 & 12 & 6 & $(1,1,2)^\dagger$: 2,\ $(3,1,2)$: 6\\
3 & 3 & 1 & 6 & 324$^\ast$ & 144 & $(1,1,3)^\dagger$: 18,\ $(2,1,3)^\dagger$: 18\\
4 & 2 & 2 & 6 & 1728$^\ast$ & 792 & $(1,1,2)^\dagger$: 12,\ $(3,1,2)^\dagger$: 144\\
5 & 5 & 1 & 6 & 8448 & 2376 & $(1,1,1)^\dagger$: 4,\ $(2,1,1)^\dagger$: 4,\ $(3,2,1)$: 528\\
7 & 7 & 1 & 6 & 46656$^\ast$ & 14832 & $(1,1,1)^\dagger$: 6,\ $(2,1,1)^\dagger$: 6,\ $(3,1,1)^\dagger$: 36,\ $(6,1,1)^\dagger$: 36\\
8 & 2 & 3 & 6 & 218736 & 117288 & $(1,1,2)^\dagger$: 56,\ $(3,1,2)$: 3906\\
9 & 3 & 2 & 6 & 419904$^\ast$ & 130848 & $(1,1,3)^\dagger$: 648,\ $(2,1,3)^\dagger$: 648\\
11 & 11 & 1 & 6 & 1416000 & 298848 & $(1,1,1)^\dagger$: 10,\ $(2,1,1)^\dagger$: 10,\ $(3,2,1)$: 14160\\
13 & 13 & 1 & 6 & 2985984$^\ast$ & 834048 & $(1,1,1)^\dagger$: 12,\ $(2,1,1)^\dagger$: 12,\ $(3,1,1)^\dagger$: 144,\ $(6,1,1)^\dagger$: 144\\
16 & 2 & 4 & 6 & 13824000$^\ast$ & 5469696 & $(1,1,2)^\dagger$: 240,\ $(3,1,2)^\dagger$: 57600\\
17 & 17 & 1 & 6 & 21086208 & 5546304 & $(1,1,1)^\dagger$: 16,\ $(2,1,1)^\dagger$: 16,\ $(3,2,1)$: 82368\\
19 & 19 & 1 & 6 & 34012224$^\ast$ & 7711200 & $(1,1,1)^\dagger$: 18,\ $(2,1,1)^\dagger$: 18,\ $(3,1,1)^\dagger$: 324,\ $(6,1,1)^\dagger$: 324\\
23 & 23 & 1 & 6 & 134420352 & 31821840 & $(1,1,1)^\dagger$: 22,\ $(2,1,1)^\dagger$: 22,\ $(3,2,1)$: 277728\\
25 & 5 & 2 & 6 & 191102976$^\ast$ & 48691008 & $(1,1,1)^\dagger$: 24,\ $(2,1,1)^\dagger$: 24,\ $(3,1,1)^\dagger$: 576,\ $(6,1,1)^\dagger$: 576\\
27 & 3 & 3 & 6 & 359254116$^\ast$ & 130838112 & $(1,1,3)^\dagger$: 18954,\ $(2,1,3)^\dagger$: 18954\\
29 & 29 & 1 & 6 & 551873280 & 114307056 & $(1,1,1)^\dagger$: 28,\ $(2,1,1)^\dagger$: 28,\ $(3,2,1)$: 703920\\
31 & 31 & 1 & 6 & 729000000$^\ast$ & 157394880 & $(1,1,1)^\dagger$: 30,\ $(2,1,1)^\dagger$: 30,\ $(3,1,1)^\dagger$: 900,\ $(6,1,1)^\dagger$: 900\\
32 & 2 & 5 & 6 & 1037141952 & 516358800 & $(1,1,2)^\dagger$: 992,\ $(3,1,2)$: 1045506\\
37 & 37 & 1 & 6 & 2176782336$^\ast$ & 548654688 & $(1,1,1)^\dagger$: 36,\ $(2,1,1)^\dagger$: 36,\ $(3,1,1)^\dagger$: 1296,\ $(6,1,1)^\dagger$: 1296\\
41 & 41 & 1 & 6 & 215496704 & 1028522880 & $(1,1,1)^\dagger$: 40,\ $(2,1,1)^\dagger$: 40,\ $(3,2,1)$: 2819040\\
43 & 43 & 1 & 6 & 1194064448$^\ast$ & 1304511264 & $(1,1,1)^\dagger$: 42,\ $(2,1,1)^\dagger$: 42,\ $(3,1,1)^\dagger$: 1764,\ $(6,1,1)^\dagger$: 1764\\

\end{longtable}


\section{$\PCN$-Polynome}

\tiny

\subsection{$\PCN$-Polynome für $n = 6$}


\begin{description}[leftmargin=0pt,labelindent=20pt,
  font=\normalsize]
  \item[$r=1$:] \textsf{\bfseries 2:} 0:\,$1$,\ 1:\,$1$,\ 4:\,$1$,\ 5:\,$1$, \textsf{\bfseries 3:} 0:\,$2$,\ 3:\,$1$,\ 5:\,$1$, \textsf{\bfseries 5:} $2$, \textsf{\bfseries 7:} 0:\,$3$,\ 3:\,$3$,\ 5:\,$3$, \textsf{\bfseries 11:} 0:\,$7$,\ 1:\,$1$,\ 5:\,$1$, \textsf{\bfseries 13:} 0:\,$11$,\ 2:\,$1$,\ 5:\,$1$, \textsf{\bfseries 17:} $3$, \textsf{\bfseries 19:} $15$, \textsf{\bfseries 23:} $7$, \textsf{\bfseries 29:} $11$, \textsf{\bfseries 31:} $13$, \textsf{\bfseries 37:} $18$, \textsf{\bfseries 41:} $35$, \textsf{\bfseries 43:} $3$, \textsf{\bfseries 47:} $23$, \textsf{\bfseries 53:} $14$, \textsf{\bfseries 59:} $10$, \textsf{\bfseries 61:} $6$, \textsf{\bfseries 67:} 0:\,$50$,\ 1:\,$1$,\ 5:\,$1$, \textsf{\bfseries 71:} $33$, \textsf{\bfseries 73:} $53$, \textsf{\bfseries 79:} $29$, \textsf{\bfseries 83:} $2$, \textsf{\bfseries 89:} $33$, \textsf{\bfseries 97:} $56$, \textsf{\bfseries 101:} $27$, \textsf{\bfseries 103:} $35$, \textsf{\bfseries 107:} $72$, \textsf{\bfseries 109:} $24$, \textsf{\bfseries 113:} $6$, \textsf{\bfseries 127:} $3$, \textsf{\bfseries 131:} $8$, \textsf{\bfseries 137:} $6$, \textsf{\bfseries 139:} $3$, \textsf{\bfseries 149:} $51$, \textsf{\bfseries 151:} $7$, \textsf{\bfseries 157:} $43$, \textsf{\bfseries 163:} $32$, \textsf{\bfseries 167:} $46$, \textsf{\bfseries 173:} $48$, \textsf{\bfseries 179:} $10$, \textsf{\bfseries 181:} $41$, \textsf{\bfseries 191:} $22$, \textsf{\bfseries 193:} $15$, \textsf{\bfseries 197:} $11$, \textsf{\bfseries 199:} $48$, \textsf{\bfseries 211:} $7$, \textsf{\bfseries 223:} $6$, \textsf{\bfseries 227:} $18$, \textsf{\bfseries 229:} $7$, \textsf{\bfseries 233:} $59$, \textsf{\bfseries 239:} $7$, \textsf{\bfseries 241:} $68$, \textsf{\bfseries 251:} $59$, \textsf{\bfseries 257:} $53$, \textsf{\bfseries 263:} $14$, \textsf{\bfseries 269:} $7$, \textsf{\bfseries 271:} $76$, \textsf{\bfseries 277:} $14$, \textsf{\bfseries 281:} $30$, \textsf{\bfseries 283:} $12$, \textsf{\bfseries 293:} $3$, \textsf{\bfseries 307:} $45$, \textsf{\bfseries 311:} $22$, \textsf{\bfseries 313:} $21$, \textsf{\bfseries 317:} $47$, \textsf{\bfseries 331:} $93$, \textsf{\bfseries 337:} $15$, \textsf{\bfseries 347:} $17$, \textsf{\bfseries 349:} $13$, \textsf{\bfseries 353:} $12$, \textsf{\bfseries 359:} $26$, \textsf{\bfseries 367:} $80$, \textsf{\bfseries 373:} $11$, \textsf{\bfseries 379:} $10$, \textsf{\bfseries 383:} $30$, \textsf{\bfseries 389:} $22$, \textsf{\bfseries 397:} $22$, \textsf{\bfseries 401:} $34$, \textsf{\bfseries 409:} $67$, \textsf{\bfseries 419:} $2$, \textsf{\bfseries 421:} $41$, \textsf{\bfseries 431:} $56$, \textsf{\bfseries 433:} $5$, \textsf{\bfseries 439:} $43$, \textsf{\bfseries 443:} $11$, \textsf{\bfseries 449:} $15$, \textsf{\bfseries 457:} $53$, \textsf{\bfseries 461:} $8$, \textsf{\bfseries 463:} $41$, \textsf{\bfseries 467:} $34$, \textsf{\bfseries 479:} $34$, \textsf{\bfseries 487:} $14$, \textsf{\bfseries 491:} $74$, \textsf{\bfseries 499:} $35$, \textsf{\bfseries 503:} $31$, \textsf{\bfseries 509:} $7$, \textsf{\bfseries 521:} $23$, \textsf{\bfseries 523:} $12$, \textsf{\bfseries 541:} $10$, \textsf{\bfseries 547:} $48$, 
\textsf{\bfseries 557:} $8$, \textsf{\bfseries 563:} $15$, \textsf{\bfseries 569:} $21$, \textsf{\bfseries 571:} $10$, \textsf{\bfseries 577:} $39$, \textsf{\bfseries 587:} $8$, \textsf{\bfseries 593:} $87$, \textsf{\bfseries 599:} $61$, \textsf{\bfseries 601:} $7$, \textsf{\bfseries 607:} $24$, \textsf{\bfseries 613:} $13$, \textsf{\bfseries 617:} $3$, \textsf{\bfseries 619:} $19$, \textsf{\bfseries 631:} $15$, \textsf{\bfseries 641:} $19$, \textsf{\bfseries 643:} $14$, \textsf{\bfseries 647:} $35$, \textsf{\bfseries 653:} $37$, \textsf{\bfseries 659:} $8$, \textsf{\bfseries 661:} $6$, \textsf{\bfseries 673:} $11$, \textsf{\bfseries 677:} $11$, \textsf{\bfseries 683:} $17$, \textsf{\bfseries 691:} $12$, \textsf{\bfseries 701:} $55$, \textsf{\bfseries 709:} $24$, \textsf{\bfseries 719:} $57$, \textsf{\bfseries 727:} $102$, \textsf{\bfseries 733:} $18$, \textsf{\bfseries 739:} $19$, \textsf{\bfseries 743:} $34$, \textsf{\bfseries 751:} $67$, \textsf{\bfseries 757:} $31$, \textsf{\bfseries 761:} $6$, \textsf{\bfseries 769:} $11$, \textsf{\bfseries 773:} $7$, \textsf{\bfseries 787:} $14$, \textsf{\bfseries 797:} $18$, \textsf{\bfseries 809:} $22$, \textsf{\bfseries 811:} $13$, \textsf{\bfseries 821:} $21$, \textsf{\bfseries 823:} $3$, \textsf{\bfseries 827:} $53$, \textsf{\bfseries 829:} $21$, \textsf{\bfseries 839:} $41$, \textsf{\bfseries 853:} $51$, \textsf{\bfseries 857:} $40$, \textsf{\bfseries 859:} $2$, \textsf{\bfseries 863:} $52$, \textsf{\bfseries 877:} $19$, \textsf{\bfseries 881:} $17$, \textsf{\bfseries 883:} $2$, \textsf{\bfseries 887:} $65$, \textsf{\bfseries 907:} $37$, \textsf{\bfseries 911:} $63$, \textsf{\bfseries 919:} $15$, \textsf{\bfseries 929:} $34$, \textsf{\bfseries 937:} $21$, \textsf{\bfseries 941:} $3$, \textsf{\bfseries 947:} $8$, \textsf{\bfseries 953:} $19$, \textsf{\bfseries 967:} $12$, \textsf{\bfseries 971:} $42$, \textsf{\bfseries 977:} $40$, \textsf{\bfseries 983:} $10$, \textsf{\bfseries 991:} $6$, \textsf{\bfseries 997:} $43$, \textsf{\bfseries 1009:} $11$, \textsf{\bfseries 1013:} $20$, \textsf{\bfseries 1019:} $66$, \textsf{\bfseries 1021:} $65$, \textsf{\bfseries 1031:} $91$, \textsf{\bfseries 1033:} $22$, \textsf{\bfseries 1039:} $3$, \textsf{\bfseries 1049:} $14$, \textsf{\bfseries 1051:} $17$, \textsf{\bfseries 1061:} $15$, \textsf{\bfseries 1063:} $10$, \textsf{\bfseries 1069:} $23$, \textsf{\bfseries 1087:} $3$, \textsf{\bfseries 1091:} $87$, \textsf{\bfseries 1093:} $19$, \textsf{\bfseries 1097:} $3$, \textsf{\bfseries 1103:} $11$, \textsf{\bfseries 1109:} $73$, \textsf{\bfseries 1117:} $17$, \textsf{\bfseries 1123:} $21$, \textsf{\bfseries 1129:} $17$, \textsf{\bfseries 1151:} $41$, \textsf{\bfseries 1153:} $5$, \textsf{\bfseries 1163:} $29$, \textsf{\bfseries 1171:} $10$, \textsf{\bfseries 1181:} $46$, \textsf{\bfseries 1187:} $7$, \textsf{\bfseries 1193:} $78$, \textsf{\bfseries 1201:} $22$, \textsf{\bfseries 1213:} $2$, \textsf{\bfseries 1217:} $22$, \textsf{\bfseries 1223:} $5$, \textsf{\bfseries 1229:} $3$, 
\textsf{\bfseries 1231:} $78$, \textsf{\bfseries 1237:} $6$, \textsf{\bfseries 1249:} $88$, \textsf{\bfseries 1259:} $11$, \textsf{\bfseries 1277:} $32$, \textsf{\bfseries 1279:} $44$, \textsf{\bfseries 1283:} $34$, \textsf{\bfseries 1289:} $85$, \textsf{\bfseries 1291:} $10$
  \item[$r=2$:] \textsf{\bfseries 2:} 0:\,$a$,\ 1:\,$1$,\ 5:\,$1$, \textsf{\bfseries 3:} 0:\,$a$,\ 1:\,$1$,\ 5:\,$1$, \textsf{\bfseries 5:} $3a$, \textsf{\bfseries 7:} 0:\,$3a + 3$,\ 1:\,$1$,\ 5:\,$1$, \textsf{\bfseries 11:} 0:\,$6a + 4$,\ 1:\,$1$,\ 5:\,$1$, \textsf{\bfseries 13:} $8a$, \textsf{\bfseries 17:} $3a$, \textsf{\bfseries 19:} $6a$, \textsf{\bfseries 23:} $2a$, \textsf{\bfseries 29:} $7a$, \textsf{\bfseries 31:} $21a$
  \item[$r=3$:] \textsf{\bfseries 2:} $a + 1$, \textsf{\bfseries 3:} $2a^2$, \textsf{\bfseries 5:} $3a^2 + 2a + 1$, \textsf{\bfseries 7:} $2a$
  \item[$r=4$:] \textsf{\bfseries 2:} 0:\,$a^2$,\ 1:\,$a^2 + a$,\ 5:\,$1$, \textsf{\bfseries 3:} 0:\,$a^2 + a$,\ 1:\,$1$,\ 5:\,$1$, \textsf{\bfseries 5:} $3a + 4$
  \item[$r=5$:] \textsf{\bfseries 2:} $a$, \textsf{\bfseries 3:} $2a^2$
  \item[$r=6$:] \textsf{\bfseries 2:} 0:\,$a^5$,\ 1:\,$1$,\ 5:\,$1$, \textsf{\bfseries 3:} 0:\,$2a^4 + 2a^2 + a + 1$,\ 1:\,$1$,\ 5:\,$1$
  \item[$r=7$:] \textsf{\bfseries 2:} $a^5 + 1$
  \item[$r=8$:] \textsf{\bfseries 2:} 0:\,$a$,\ 1:\,$1$,\ 5:\,$1$
  \item[$r=9$:] \textsf{\bfseries 2:} $a^7 + a$
  \item[$r=10$:] \textsf{\bfseries 2:} 0:\,$a^5 + 1$,\ 1:\,$1$,\ 5:\,$1$
\end{description}


\subsection{$\PCN$-Polynome für $n=10$}

\begin{description}[leftmargin=0pt,labelindent=20pt,
  font=\normalsize]
  \item[$r=1$:] \textsf{\bfseries 2:} 0:\,$1$,\ 1:\,$1$,\ 4:\,$1$,\ 9:\,$1$, \textsf{\bfseries 3:} 0:\,$2$,\ 7:\,$1$,\ 9:\,$1$, \textsf{\bfseries 5:} 0:\,$2$,\ 3:\,$2$,\ 9:\,$2$, \textsf{\bfseries 7:} 0:\,$3$,\ 1:\,$1$,\ 9:\,$1$, \textsf{\bfseries 11:} 0:\,$2$,\ 1:\,$3$,\ 9:\,$3$, \textsf{\bfseries 13:} 0:\,$2$,\ 1:\,$1$,\ 9:\,$1$, \textsf{\bfseries 17:} $7$, \textsf{\bfseries 19:} 0:\,$13$,\ 1:\,$4$,\ 9:\,$4$, \textsf{\bfseries 23:} $15$, \textsf{\bfseries 29:} 0:\,$8$,\ 1:\,$3$,\ 9:\,$3$, \textsf{\bfseries 31:} $12$, \textsf{\bfseries 37:} 0:\,$15$,\ 9:\,$2$, \textsf{\bfseries 41:} 0:\,$28$,\ 1:\,$1$,\ 9:\,$1$, \textsf{\bfseries 43:} $33$, \textsf{\bfseries 47:} 0:\,$26$,\ 1:\,$1$,\ 9:\,$1$, \textsf{\bfseries 53:} $12$, \textsf{\bfseries 59:} 0:\,$44$,\ 1:\,$1$,\ 9:\,$1$, \textsf{\bfseries 61:} 0:\,$10$,\ 1:\,$1$,\ 9:\,$1$, \textsf{\bfseries 67:} $18$, \textsf{\bfseries 71:} $22$, \textsf{\bfseries 73:} $20$, \textsf{\bfseries 79:} $34$, \textsf{\bfseries 83:} $19$, \textsf{\bfseries 89:} $31$, \textsf{\bfseries 97:} $26$, \textsf{\bfseries 101:} $3$, \textsf{\bfseries 103:} $74$, \textsf{\bfseries 107:} $31$, \textsf{\bfseries 109:} $69$, \textsf{\bfseries 113:} $19$, \textsf{\bfseries 127:} $55$, \textsf{\bfseries 131:} $50$, \textsf{\bfseries 137:} $6$, \textsf{\bfseries 139:} $56$, \textsf{\bfseries 149:} $52$, \textsf{\bfseries 151:} $51$, \textsf{\bfseries 157:} $26$, \textsf{\bfseries 163:} $44$, \textsf{\bfseries 167:} $70$, \textsf{\bfseries 173:} $2$, \textsf{\bfseries 179:} $21$, \textsf{\bfseries 181:} $103$, \textsf{\bfseries 191:} $95$, \textsf{\bfseries 193:} $40$, \textsf{\bfseries 197:} $12$, \textsf{\bfseries 199:} $3$, \textsf{\bfseries 211:} $29$, \textsf{\bfseries 223:} $79$, \textsf{\bfseries 227:} $60$, \textsf{\bfseries 229:} $24$, \textsf{\bfseries 233:} $40$, \textsf{\bfseries 239:} $47$, \textsf{\bfseries 241:} $74$, \textsf{\bfseries 251:} $72$, \textsf{\bfseries 257:} $3$, \textsf{\bfseries 263:} $21$, \textsf{\bfseries 269:} $71$, \textsf{\bfseries 271:} $42$, \textsf{\bfseries 277:} $17$, \textsf{\bfseries 281:} $44$, \textsf{\bfseries 283:} $3$, \textsf{\bfseries 293:} $27$, \textsf{\bfseries 307:} $56$, \textsf{\bfseries 311:} $97$, \textsf{\bfseries 313:} $41$, \textsf{\bfseries 317:} $50$, \textsf{\bfseries 331:} $99$, \textsf{\bfseries 337:} $31$, \textsf{\bfseries 347:} $5$, \textsf{\bfseries 349:} $30$, \textsf{\bfseries 353:} $3$, \textsf{\bfseries 359:} $14$, \textsf{\bfseries 367:} $12$, \textsf{\bfseries 373:} $61$, \textsf{\bfseries 379:} $15$, \textsf{\bfseries 383:} $20$, \textsf{\bfseries 389:} $40$, \textsf{\bfseries 397:} $13$, \textsf{\bfseries 401:} $27$, \textsf{\bfseries 409:} $57$, \textsf{\bfseries 419:} $57$, \textsf{\bfseries 421:} $145$, \textsf{\bfseries 431:} $51$, \textsf{\bfseries 433:} $58$, \textsf{\bfseries 439:} $31$, \textsf{\bfseries 443:} $57$, \textsf{\bfseries 449:} $139$, \textsf{\bfseries 457:} $62$, \textsf{\bfseries 461:} $27$, \textsf{\bfseries 463:} $11$, \textsf{\bfseries 467:} $5$, \textsf{\bfseries 479:} $19$, \textsf{\bfseries 487:} $11$, \textsf{\bfseries 491:} $26$, \textsf{\bfseries 499:} $7$, \textsf{\bfseries 503:} $51$, \textsf{\bfseries 509:} $72$, \textsf{\bfseries 521:} $66$, \textsf{\bfseries 523:} $2$, \textsf{\bfseries 541:} $158$, \textsf{\bfseries 547:} $2$, 
\textsf{\bfseries 557:} $34$, \textsf{\bfseries 563:} $18$, \textsf{\bfseries 569:} $39$, \textsf{\bfseries 571:} $17$, \textsf{\bfseries 577:} $5$, \textsf{\bfseries 587:} $97$, \textsf{\bfseries 593:} $12$, \textsf{\bfseries 599:} $262$, \textsf{\bfseries 601:} $44$, \textsf{\bfseries 607:} $12$, \textsf{\bfseries 613:} $45$, \textsf{\bfseries 617:} $26$, \textsf{\bfseries 619:} $2$, \textsf{\bfseries 631:} $95$, \textsf{\bfseries 641:} $3$, \textsf{\bfseries 643:} $13$, \textsf{\bfseries 647:} $20$, \textsf{\bfseries 653:} $37$, \textsf{\bfseries 659:} $7$, \textsf{\bfseries 661:} $79$, \textsf{\bfseries 673:} $11$, \textsf{\bfseries 677:} $82$, \textsf{\bfseries 683:} $39$, \textsf{\bfseries 691:} $98$, \textsf{\bfseries 701:} $18$, \textsf{\bfseries 709:} $139$, \textsf{\bfseries 719:} $11$, \textsf{\bfseries 727:} $21$, \textsf{\bfseries 733:} $62$, \textsf{\bfseries 739:} $51$, \textsf{\bfseries 743:} $89$, \textsf{\bfseries 751:} $17$, \textsf{\bfseries 757:} $45$, \textsf{\bfseries 761:} $7$, \textsf{\bfseries 769:} $59$, \textsf{\bfseries 773:} $3$, \textsf{\bfseries 787:} $21$, \textsf{\bfseries 797:} $8$, \textsf{\bfseries 809:} $24$, \textsf{\bfseries 811:} $13$, \textsf{\bfseries 821:} $74$, \textsf{\bfseries 823:} $7$, \textsf{\bfseries 827:} $17$, \textsf{\bfseries 829:} $40$, \textsf{\bfseries 839:} $153$, \textsf{\bfseries 853:} $87$, \textsf{\bfseries 857:} $28$, \textsf{\bfseries 859:} $70$, \textsf{\bfseries 863:} $33$, \textsf{\bfseries 877:} $42$, \textsf{\bfseries 881:} $35$, \textsf{\bfseries 883:} $39$, \textsf{\bfseries 887:} $29$, \textsf{\bfseries 907:} $29$, \textsf{\bfseries 911:} $17$, \textsf{\bfseries 919:} $15$, \textsf{\bfseries 929:} $69$, \textsf{\bfseries 937:} $11$, \textsf{\bfseries 941:} $63$, \textsf{\bfseries 947:} $6$, \textsf{\bfseries 953:} $54$, \textsf{\bfseries 967:} $5$, \textsf{\bfseries 971:} $29$, \textsf{\bfseries 977:} $21$, \textsf{\bfseries 983:} $5$, \textsf{\bfseries 991:} $7$, \textsf{\bfseries 997:} $43$, \textsf{\bfseries 1009:} $66$, \textsf{\bfseries 1013:} $5$, \textsf{\bfseries 1019:} $40$, \textsf{\bfseries 1021:} $77$, \textsf{\bfseries 1031:} $84$, \textsf{\bfseries 1033:} $13$, \textsf{\bfseries 1039:} $214$, \textsf{\bfseries 1049:} $221$, \textsf{\bfseries 1051:} $41$, \textsf{\bfseries 1061:} $72$, \textsf{\bfseries 1063:} $29$, \textsf{\bfseries 1069:} $7$, \textsf{\bfseries 1087:} $76$, \textsf{\bfseries 1091:} $68$, \textsf{\bfseries 1093:} $76$, \textsf{\bfseries 1097:} $14$, \textsf{\bfseries 1103:} $5$, \textsf{\bfseries 1109:} $10$, \textsf{\bfseries 1117:} $20$, \textsf{\bfseries 1123:} $95$, \textsf{\bfseries 1129:} $88$, \textsf{\bfseries 1151:} $17$, \textsf{\bfseries 1153:} $10$, \textsf{\bfseries 1163:} $19$, \textsf{\bfseries 1171:} $42$, \textsf{\bfseries 1181:} $65$, \textsf{\bfseries 1187:} $6$, \textsf{\bfseries 1193:} $20$, \textsf{\bfseries 1201:} $11$, \textsf{\bfseries 1213:} $24$, \textsf{\bfseries 1217:} $129$, \textsf{\bfseries 1223:} $10$, \textsf{\bfseries 1229:} $32$, 
\textsf{\bfseries 1231:} $75$, \textsf{\bfseries 1237:} $32$, \textsf{\bfseries 1249:} $11$, \textsf{\bfseries 1259:} $2$, \textsf{\bfseries 1277:} $50$, \textsf{\bfseries 1279:} $102$, \textsf{\bfseries 1283:} $28$, \textsf{\bfseries 1289:} $22$, \textsf{\bfseries 1291:} $72$, \textsf{\bfseries 1297:} $80$, \textsf{\bfseries 1301:} $38$, \textsf{\bfseries 1303:} $6$, \textsf{\bfseries 1307:} $39$, \textsf{\bfseries 1319:} $46$, \textsf{\bfseries 1321:} $19$, \textsf{\bfseries 1327:} $6$, \textsf{\bfseries 1361:} $6$, \textsf{\bfseries 1367:} $109$, \textsf{\bfseries 1373:} $20$, \textsf{\bfseries 1381:} $95$, \textsf{\bfseries 1399:} $52$, \textsf{\bfseries 1409:} $6$, \textsf{\bfseries 1423:} $71$, \textsf{\bfseries 1427:} $20$, \textsf{\bfseries 1429:} $50$, \textsf{\bfseries 1433:} $6$, \textsf{\bfseries 1439:} $11$, \textsf{\bfseries 1447:} $96$, \textsf{\bfseries 1451:} $6$, \textsf{\bfseries 1453:} $34$, \textsf{\bfseries 1459:} $12$, \textsf{\bfseries 1471:} $15$, \textsf{\bfseries 1481:} $6$, \textsf{\bfseries 1483:} $12$, \textsf{\bfseries 1487:} $19$, \textsf{\bfseries 1489:} $43$, \textsf{\bfseries 1493:} $83$, \textsf{\bfseries 1499:} $2$, \textsf{\bfseries 1511:} $66$, \textsf{\bfseries 1523:} $13$, \textsf{\bfseries 1531:} $10$, \textsf{\bfseries 1543:} $23$, \textsf{\bfseries 1549:} $154$, \textsf{\bfseries 1553:} $27$, \textsf{\bfseries 1559:} $19$, \textsf{\bfseries 1567:} $117$, \textsf{\bfseries 1571:} $14$, \textsf{\bfseries 1579:} $160$, \textsf{\bfseries 1583:} $30$, \textsf{\bfseries 1597:} $91$, \textsf{\bfseries 1601:} $56$, \textsf{\bfseries 1607:} $45$, \textsf{\bfseries 1609:} $21$, \textsf{\bfseries 1613:} $45$, \textsf{\bfseries 1619:} $58$, \textsf{\bfseries 1621:} $6$, \textsf{\bfseries 1627:} $12$, \textsf{\bfseries 1637:} $22$, \textsf{\bfseries 1657:} $15$, \textsf{\bfseries 1663:} $83$, \textsf{\bfseries 1667:} $20$, \textsf{\bfseries 1669:} $32$, \textsf{\bfseries 1693:} $5$, \textsf{\bfseries 1697:} $45$, \textsf{\bfseries 1699:} $21$, \textsf{\bfseries 1709:} $78$, \textsf{\bfseries 1721:} $102$, \textsf{\bfseries 1723:} $30$, \textsf{\bfseries 1733:} $11$, \textsf{\bfseries 1741:} $19$, \textsf{\bfseries 1747:} $78$, \textsf{\bfseries 1753:} $14$, \textsf{\bfseries 1759:} $30$, \textsf{\bfseries 1777:} $41$, \textsf{\bfseries 1783:} $42$, \textsf{\bfseries 1787:} $5$, \textsf{\bfseries 1789:} $6$, \textsf{\bfseries 1801:} $47$, \textsf{\bfseries 1811:} $141$, \textsf{\bfseries 1823:} $85$, \textsf{\bfseries 1831:} $30$, \textsf{\bfseries 1847:} $19$, \textsf{\bfseries 1861:} $6$, \textsf{\bfseries 1867:} $14$, \textsf{\bfseries 1871:} $55$, \textsf{\bfseries 1873:} $37$, \textsf{\bfseries 1877:} $66$, \textsf{\bfseries 1879:} $23$, \textsf{\bfseries 1889:} $11$, \textsf{\bfseries 1901:} $88$, \textsf{\bfseries 1907:} $8$, \textsf{\bfseries 1913:} $11$, \textsf{\bfseries 1931:} $2$, \textsf{\bfseries 1933:} $5$, \textsf{\bfseries 1949:} $47$, \textsf{\bfseries 1951:} $107$, \textsf{\bfseries 1973:} $48$, \textsf{\bfseries 1979:} $26$, \textsf{\bfseries 1987:} $45$, \textsf{\bfseries 1993:} $5$, 
\textsf{\bfseries 1997:} $20$, \textsf{\bfseries 1999:} $30$, \textsf{\bfseries 2003:} $15$, \textsf{\bfseries 2011:} $17$, \textsf{\bfseries 2017:} $67$, \textsf{\bfseries 2027:} $2$, \textsf{\bfseries 2029:} $6$, \textsf{\bfseries 2039:} $7$, \textsf{\bfseries 2053:} $32$, \textsf{\bfseries 2063:} $5$, \textsf{\bfseries 2069:} $113$, \textsf{\bfseries 2081:} $19$, \textsf{\bfseries 2083:} $2$, \textsf{\bfseries 2087:} $202$, \textsf{\bfseries 2089:} $185$, \textsf{\bfseries 2099:} $8$, \textsf{\bfseries 2111:} $63$, \textsf{\bfseries 2113:} $10$, \textsf{\bfseries 2129:} $27$, \textsf{\bfseries 2131:} $127$, \textsf{\bfseries 2137:} $23$, \textsf{\bfseries 2141:} $61$, \textsf{\bfseries 2143:} $14$, \textsf{\bfseries 2153:} $22$, \textsf{\bfseries 2161:} $23$, \textsf{\bfseries 2179:} $46$, \textsf{\bfseries 2203:} $84$, \textsf{\bfseries 2207:} $63$, \textsf{\bfseries 2213:} $18$, \textsf{\bfseries 2221:} $87$, \textsf{\bfseries 2237:} $35$, \textsf{\bfseries 2239:} $12$, \textsf{\bfseries 2243:} $98$, \textsf{\bfseries 2251:} $157$, \textsf{\bfseries 2267:} $32$, \textsf{\bfseries 2269:} $115$, \textsf{\bfseries 2273:} $76$, \textsf{\bfseries 2281:} $79$, \textsf{\bfseries 2287:} $106$, \textsf{\bfseries 2293:} $46$, \textsf{\bfseries 2297:} $10$, \textsf{\bfseries 2309:} $67$, \textsf{\bfseries 2311:} $75$, \textsf{\bfseries 2333:} $55$, \textsf{\bfseries 2339:} $22$, \textsf{\bfseries 2341:} $104$, \textsf{\bfseries 2347:} $43$, \textsf{\bfseries 2351:} $29$, \textsf{\bfseries 2357:} $71$, \textsf{\bfseries 2371:} $2$, \textsf{\bfseries 2377:} $124$, \textsf{\bfseries 2381:} $46$, \textsf{\bfseries 2383:} $5$, \textsf{\bfseries 2389:} $90$, \textsf{\bfseries 2393:} $171$, \textsf{\bfseries 2399:} $38$, \textsf{\bfseries 2411:} $177$, \textsf{\bfseries 2417:} $27$, \textsf{\bfseries 2423:} $15$, \textsf{\bfseries 2437:} $2$, \textsf{\bfseries 2441:} $12$, \textsf{\bfseries 2447:} $14$, \textsf{\bfseries 2459:} $17$, \textsf{\bfseries 2467:} $2$, \textsf{\bfseries 2473:} $29$, \textsf{\bfseries 2477:} $57$, \textsf{\bfseries 2503:} $5$, \textsf{\bfseries 2521:} $44$, \textsf{\bfseries 2531:} $8$, \textsf{\bfseries 2539:} $161$, \textsf{\bfseries 2543:} $7$, \textsf{\bfseries 2549:} $26$, \textsf{\bfseries 2551:} $89$, \textsf{\bfseries 2557:} $24$, \textsf{\bfseries 2579:} $17$, \textsf{\bfseries 2591:} $175$, \textsf{\bfseries 2593:} $13$, \textsf{\bfseries 2609:} $48$, \textsf{\bfseries 2617:} $10$, \textsf{\bfseries 2621:} $48$, \textsf{\bfseries 2633:} $31$, \textsf{\bfseries 2647:} $38$, \textsf{\bfseries 2657:} $3$, \textsf{\bfseries 2659:} $40$, \textsf{\bfseries 2663:} $35$, \textsf{\bfseries 2671:} $59$, \textsf{\bfseries 2677:} $22$, \textsf{\bfseries 2683:} $5$, \textsf{\bfseries 2687:} $11$, \textsf{\bfseries 2689:} $101$, \textsf{\bfseries 2693:} $45$, \textsf{\bfseries 2699:} $41$, \textsf{\bfseries 2707:} $21$, \textsf{\bfseries 2711:} $47$, \textsf{\bfseries 2713:} $41$, \textsf{\bfseries 2719:} $96$, \textsf{\bfseries 2729:} $53$, \textsf{\bfseries 2731:} $10$, \textsf{\bfseries 2741:} $91$, \textsf{\bfseries 2749:} $19$, 
\textsf{\bfseries 2753:} $35$, \textsf{\bfseries 2767:} $5$, \textsf{\bfseries 2777:} $77$, \textsf{\bfseries 2789:} $181$, \textsf{\bfseries 2791:} $68$, \textsf{\bfseries 2797:} $14$, \textsf{\bfseries 2801:} $3$, \textsf{\bfseries 2803:} $18$, \textsf{\bfseries 2819:} $14$, \textsf{\bfseries 2833:} $40$, \textsf{\bfseries 2837:} $44$, \textsf{\bfseries 2843:} $65$, \textsf{\bfseries 2851:} $165$, \textsf{\bfseries 2857:} $138$, \textsf{\bfseries 2861:} $3$, \textsf{\bfseries 2879:} $21$, \textsf{\bfseries 2887:} $39$, \textsf{\bfseries 2897:} $26$, \textsf{\bfseries 2903:} $15$, \textsf{\bfseries 2909:} $2$, \textsf{\bfseries 2917:} $57$, \textsf{\bfseries 2927:} $10$, \textsf{\bfseries 2939:} $10$, \textsf{\bfseries 2953:} $30$, \textsf{\bfseries 2957:} $63$, \textsf{\bfseries 2963:} $128$, \textsf{\bfseries 2969:} $12$, \textsf{\bfseries 2971:} $10$, \textsf{\bfseries 2999:} $79$, \textsf{\bfseries 3001:} $26$, \textsf{\bfseries 3011:} $194$, \textsf{\bfseries 3019:} $71$, \textsf{\bfseries 3023:} $35$, \textsf{\bfseries 3037:} $88$, \textsf{\bfseries 3041:} $60$, \textsf{\bfseries 3049:} $13$, \textsf{\bfseries 3061:} $74$, \textsf{\bfseries 3067:} $45$, \textsf{\bfseries 3079:} $105$, \textsf{\bfseries 3083:} $5$, \textsf{\bfseries 3089:} $17$, \textsf{\bfseries 3109:} $55$, \textsf{\bfseries 3119:} $179$, \textsf{\bfseries 3121:} $97$, \textsf{\bfseries 3137:} $74$, \textsf{\bfseries 3163:} $97$, \textsf{\bfseries 3167:} $35$, \textsf{\bfseries 3169:} $7$, \textsf{\bfseries 3181:} $131$, \textsf{\bfseries 3187:} $92$, \textsf{\bfseries 3191:} $91$, \textsf{\bfseries 3203:} $112$, \textsf{\bfseries 3209:} $63$, \textsf{\bfseries 3217:} $109$, \textsf{\bfseries 3221:} $19$, \textsf{\bfseries 3229:} $33$, \textsf{\bfseries 3251:} $26$, \textsf{\bfseries 3253:} $21$, \textsf{\bfseries 3257:} $17$, \textsf{\bfseries 3259:} $47$, \textsf{\bfseries 3271:} $174$, \textsf{\bfseries 3299:} $8$, \textsf{\bfseries 3301:} $71$, \textsf{\bfseries 3307:} $21$, \textsf{\bfseries 3313:} $10$, \textsf{\bfseries 3319:} $15$, \textsf{\bfseries 3323:} $5$, \textsf{\bfseries 3329:} $31$, \textsf{\bfseries 3331:} $66$, \textsf{\bfseries 3343:} $29$, \textsf{\bfseries 3347:} $11$, \textsf{\bfseries 3359:} $78$, \textsf{\bfseries 3361:} $55$, \textsf{\bfseries 3371:} $17$, \textsf{\bfseries 3373:} $15$, \textsf{\bfseries 3389:} $62$, \textsf{\bfseries 3391:} $11$, \textsf{\bfseries 3407:} $15$, \textsf{\bfseries 3413:} $8$, \textsf{\bfseries 3433:} $14$, \textsf{\bfseries 3449:} $110$, \textsf{\bfseries 3457:} $17$, \textsf{\bfseries 3461:} $60$, \textsf{\bfseries 3463:} $5$, \textsf{\bfseries 3467:} $24$, \textsf{\bfseries 3469:} $47$, \textsf{\bfseries 3491:} $128$, \textsf{\bfseries 3499:} $29$, \textsf{\bfseries 3511:} $179$, \textsf{\bfseries 3517:} $15$, \textsf{\bfseries 3527:} $45$, \textsf{\bfseries 3529:} $67$, \textsf{\bfseries 3533:} $5$, \textsf{\bfseries 3539:} $37$, \textsf{\bfseries 3541:} $35$, \textsf{\bfseries 3547:} $13$, \textsf{\bfseries 3557:} $35$, \textsf{\bfseries 3559:} $3$, \textsf{\bfseries 3571:} $34$, \textsf{\bfseries 3581:} $18$, 
\textsf{\bfseries 3583:} $20$, \textsf{\bfseries 3593:} $17$, \textsf{\bfseries 3607:} $20$, \textsf{\bfseries 3613:} $47$, \textsf{\bfseries 3617:} $5$, \textsf{\bfseries 3623:} $20$, \textsf{\bfseries 3631:} $15$, \textsf{\bfseries 3637:} $46$, \textsf{\bfseries 3643:} $21$, \textsf{\bfseries 3659:} $6$, \textsf{\bfseries 3671:} $43$, \textsf{\bfseries 3673:} $5$, \textsf{\bfseries 3677:} $52$, \textsf{\bfseries 3691:} $19$, \textsf{\bfseries 3697:} $46$, \textsf{\bfseries 3701:} $19$, \textsf{\bfseries 3709:} $33$, \textsf{\bfseries 3719:} $21$, \textsf{\bfseries 3727:} $47$, \textsf{\bfseries 3733:} $18$, \textsf{\bfseries 3739:} $50$, \textsf{\bfseries 3761:} $22$, \textsf{\bfseries 3767:} $29$, \textsf{\bfseries 3769:} $42$, \textsf{\bfseries 3779:} $88$, \textsf{\bfseries 3793:} $65$, \textsf{\bfseries 3797:} $5$, \textsf{\bfseries 3803:} $39$, \textsf{\bfseries 3821:} $141$, \textsf{\bfseries 3823:} $3$, \textsf{\bfseries 3833:} $23$, \textsf{\bfseries 3847:} $23$, \textsf{\bfseries 3851:} $33$, \textsf{\bfseries 3853:} $2$, \textsf{\bfseries 3863:} $97$, \textsf{\bfseries 3877:} $32$, \textsf{\bfseries 3881:} $89$, \textsf{\bfseries 3889:} $76$, \textsf{\bfseries 3907:} $17$, \textsf{\bfseries 3911:} $13$, \textsf{\bfseries 3917:} $37$, \textsf{\bfseries 3919:} $22$, \textsf{\bfseries 3923:} $14$, \textsf{\bfseries 3929:} $31$, \textsf{\bfseries 3931:} $7$, \textsf{\bfseries 3943:} $99$, \textsf{\bfseries 3947:} $15$, \textsf{\bfseries 3967:} $55$, \textsf{\bfseries 3989:} $58$, \textsf{\bfseries 4001:} $23$, \textsf{\bfseries 4003:} $33$, \textsf{\bfseries 4007:} $5$, \textsf{\bfseries 4013:} $65$, \textsf{\bfseries 4019:} $6$, \textsf{\bfseries 4021:} $11$, \textsf{\bfseries 4027:} $20$, \textsf{\bfseries 4049:} $54$, \textsf{\bfseries 4051:} $15$, \textsf{\bfseries 4057:} $10$, \textsf{\bfseries 4073:} $7$, \textsf{\bfseries 4079:} $92$, \textsf{\bfseries 4091:} $150$, \textsf{\bfseries 4093:} $53$, \textsf{\bfseries 4099:} $40$, \textsf{\bfseries 4111:} $54$, \textsf{\bfseries 4127:} $30$, \textsf{\bfseries 4129:} $26$, \textsf{\bfseries 4133:} $18$, \textsf{\bfseries 4139:} $84$, \textsf{\bfseries 4153:} $35$, \textsf{\bfseries 4157:} $23$, \textsf{\bfseries 4159:} $22$, \textsf{\bfseries 4177:} $7$, \textsf{\bfseries 4201:} $19$, \textsf{\bfseries 4211:} $38$, \textsf{\bfseries 4217:} $26$, \textsf{\bfseries 4219:} $62$, \textsf{\bfseries 4229:} $84$, \textsf{\bfseries 4231:} $43$, \textsf{\bfseries 4241:} $74$, \textsf{\bfseries 4243:} $3$, \textsf{\bfseries 4253:} $2$, \textsf{\bfseries 4259:} $136$, \textsf{\bfseries 4261:} $10$, \textsf{\bfseries 4271:} $7$, \textsf{\bfseries 4273:} $45$, \textsf{\bfseries 4283:} $59$, \textsf{\bfseries 4289:} $60$, \textsf{\bfseries 4297:} $63$, \textsf{\bfseries 4327:} $48$, \textsf{\bfseries 4337:} $3$, \textsf{\bfseries 4339:} $14$, \textsf{\bfseries 4349:} $12$, \textsf{\bfseries 4357:} $128$, \textsf{\bfseries 4363:} $61$, \textsf{\bfseries 4373:} $20$, \textsf{\bfseries 4391:} $63$, \textsf{\bfseries 4397:} $71$, \textsf{\bfseries 4409:} $14$, \textsf{\bfseries 4421:} $56$, 
\textsf{\bfseries 4423:} $69$, \textsf{\bfseries 4441:} $21$, \textsf{\bfseries 4447:} $34$, \textsf{\bfseries 4451:} $66$, \textsf{\bfseries 4457:} $14$, \textsf{\bfseries 4463:} $5$, \textsf{\bfseries 4481:} $59$, \textsf{\bfseries 4483:} $52$, \textsf{\bfseries 4493:} $33$, \textsf{\bfseries 4507:} $39$, \textsf{\bfseries 4513:} $116$, \textsf{\bfseries 4517:} $20$, \textsf{\bfseries 4519:} $13$, \textsf{\bfseries 4523:} $20$, \textsf{\bfseries 4547:} $58$, \textsf{\bfseries 4549:} $111$, \textsf{\bfseries 4561:} $66$, \textsf{\bfseries 4567:} $46$, \textsf{\bfseries 4583:} $69$, \textsf{\bfseries 4591:} $21$, \textsf{\bfseries 4597:} $11$, \textsf{\bfseries 4603:} $44$, \textsf{\bfseries 4621:} $103$, \textsf{\bfseries 4637:} $12$, \textsf{\bfseries 4639:} $24$, \textsf{\bfseries 4643:} $52$, \textsf{\bfseries 4649:} $3$, \textsf{\bfseries 4651:} $26$, \textsf{\bfseries 4657:} $30$, \textsf{\bfseries 4663:} $109$, \textsf{\bfseries 4673:} $21$, \textsf{\bfseries 4679:} $33$, \textsf{\bfseries 4691:} $21$, \textsf{\bfseries 4703:} $45$, \textsf{\bfseries 4721:} $7$, \textsf{\bfseries 4723:} $2$, \textsf{\bfseries 4729:} $51$, \textsf{\bfseries 4733:} $159$, \textsf{\bfseries 4751:} $69$, \textsf{\bfseries 4759:} $37$, \textsf{\bfseries 4783:} $14$, \textsf{\bfseries 4787:} $126$, \textsf{\bfseries 4789:} $24$, \textsf{\bfseries 4793:} $20$, \textsf{\bfseries 4799:} $13$, \textsf{\bfseries 4801:} $21$, \textsf{\bfseries 4813:} $73$, \textsf{\bfseries 4817:} $23$, \textsf{\bfseries 4831:} $54$, \textsf{\bfseries 4861:} $21$, \textsf{\bfseries 4871:} $38$, \textsf{\bfseries 4877:} $5$, \textsf{\bfseries 4889:} $63$, \textsf{\bfseries 4903:} $26$, \textsf{\bfseries 4909:} $13$, \textsf{\bfseries 4919:} $37$, \textsf{\bfseries 4931:} $218$, \textsf{\bfseries 4933:} $13$, \textsf{\bfseries 4937:} $33$, \textsf{\bfseries 4943:} $30$, \textsf{\bfseries 4951:} $30$, \textsf{\bfseries 4957:} $14$, \textsf{\bfseries 4967:} $7$, \textsf{\bfseries 4969:} $33$, \textsf{\bfseries 4973:} $8$, \textsf{\bfseries 4987:} $37$, \textsf{\bfseries 4993:} $5$, \textsf{\bfseries 4999:} $3$, \textsf{\bfseries 5003:} $18$, \textsf{\bfseries 5009:} $82$, \textsf{\bfseries 5011:} $12$, \textsf{\bfseries 5021:} $23$, \textsf{\bfseries 5023:} $3$, \textsf{\bfseries 5039:} $79$, \textsf{\bfseries 5051:} $2$, \textsf{\bfseries 5059:} $90$, \textsf{\bfseries 5077:} $18$, \textsf{\bfseries 5081:} $3$, \textsf{\bfseries 5087:} $46$, \textsf{\bfseries 5099:} $69$, \textsf{\bfseries 5101:} $21$, \textsf{\bfseries 5107:} $97$, \textsf{\bfseries 5113:} $57$, \textsf{\bfseries 5119:} $37$, \textsf{\bfseries 5147:} $5$, \textsf{\bfseries 5153:} $35$, \textsf{\bfseries 5167:} $79$, \textsf{\bfseries 5171:} $47$, \textsf{\bfseries 5179:} $12$, \textsf{\bfseries 5189:} $10$, \textsf{\bfseries 5197:} $89$, \textsf{\bfseries 5209:} $61$, \textsf{\bfseries 5227:} $38$, \textsf{\bfseries 5231:} $76$, \textsf{\bfseries 5233:} $43$, \textsf{\bfseries 5237:} $48$, \textsf{\bfseries 5261:} $2$, \textsf{\bfseries 5273:} $6$, \textsf{\bfseries 5279:} $155$, \textsf{\bfseries 5281:} $63$, 
\textsf{\bfseries 5297:} $38$, \textsf{\bfseries 5303:} $15$, \textsf{\bfseries 5309:} $13$, \textsf{\bfseries 5323:} $20$, \textsf{\bfseries 5333:} $48$, \textsf{\bfseries 5347:} $35$, \textsf{\bfseries 5351:} $33$, \textsf{\bfseries 5381:} $3$, \textsf{\bfseries 5387:} $28$, \textsf{\bfseries 5393:} $26$, \textsf{\bfseries 5399:} $7$, \textsf{\bfseries 5407:} $66$, \textsf{\bfseries 5413:} $52$, \textsf{\bfseries 5417:} $5$, \textsf{\bfseries 5419:} $50$, \textsf{\bfseries 5431:} $75$, \textsf{\bfseries 5437:} $7$, \textsf{\bfseries 5441:} $55$, \textsf{\bfseries 5443:} $2$, \textsf{\bfseries 5449:} $13$, \textsf{\bfseries 5471:} $21$, \textsf{\bfseries 5477:} $5$, \textsf{\bfseries 5479:} $3$, \textsf{\bfseries 5483:} $6$, \textsf{\bfseries 5501:} $140$, \textsf{\bfseries 5503:} $12$, \textsf{\bfseries 5507:} $24$, \textsf{\bfseries 5519:} $197$, \textsf{\bfseries 5521:} $11$, \textsf{\bfseries 5527:} $6$, \textsf{\bfseries 5531:} $79$, \textsf{\bfseries 5557:} $29$, \textsf{\bfseries 5563:} $26$, \textsf{\bfseries 5569:} $57$, \textsf{\bfseries 5573:} $30$, \textsf{\bfseries 5581:} $85$, \textsf{\bfseries 5591:} $95$, \textsf{\bfseries 5623:} $29$, \textsf{\bfseries 5639:} $71$, \textsf{\bfseries 5641:} $28$, \textsf{\bfseries 5647:} $46$, \textsf{\bfseries 5651:} $6$, \textsf{\bfseries 5653:} $20$, \textsf{\bfseries 5657:} $92$, \textsf{\bfseries 5659:} $44$, \textsf{\bfseries 5669:} $23$, \textsf{\bfseries 5683:} $38$, \textsf{\bfseries 5689:} $103$, \textsf{\bfseries 5693:} $51$, \textsf{\bfseries 5701:} $21$, \textsf{\bfseries 5711:} $67$, \textsf{\bfseries 5717:} $102$, \textsf{\bfseries 5737:} $89$, \textsf{\bfseries 5741:} $3$, \textsf{\bfseries 5743:} $156$, \textsf{\bfseries 5749:} $68$, \textsf{\bfseries 5779:} $10$, \textsf{\bfseries 5783:} $77$, \textsf{\bfseries 5791:} $110$, \textsf{\bfseries 5801:} $39$, \textsf{\bfseries 5807:} $41$, \textsf{\bfseries 5813:} $23$, \textsf{\bfseries 5821:} $10$, \textsf{\bfseries 5827:} $57$, \textsf{\bfseries 5839:} $373$, \textsf{\bfseries 5843:} $15$, \textsf{\bfseries 5849:} $48$, \textsf{\bfseries 5851:} $50$, \textsf{\bfseries 5857:} $13$, \textsf{\bfseries 5861:} $92$, \textsf{\bfseries 5867:} $61$, \textsf{\bfseries 5869:} $18$, \textsf{\bfseries 5879:} $22$, \textsf{\bfseries 5881:} $38$, \textsf{\bfseries 5897:} $5$, \textsf{\bfseries 5903:} $42$, \textsf{\bfseries 5923:} $42$, \textsf{\bfseries 5927:} $119$, \textsf{\bfseries 5939:} $164$, \textsf{\bfseries 5953:} $47$, \textsf{\bfseries 5981:} $15$, \textsf{\bfseries 5987:} $11$, \textsf{\bfseries 6007:} $59$, \textsf{\bfseries 6011:} $24$, \textsf{\bfseries 6029:} $261$, \textsf{\bfseries 6037:} $55$, \textsf{\bfseries 6043:} $7$, \textsf{\bfseries 6047:} $68$, \textsf{\bfseries 6053:} $72$, \textsf{\bfseries 6067:} $2$, \textsf{\bfseries 6073:} $19$, \textsf{\bfseries 6079:} $41$, \textsf{\bfseries 6089:} $28$, \textsf{\bfseries 6091:} $93$, \textsf{\bfseries 6101:} $34$, \textsf{\bfseries 6113:} $11$, \textsf{\bfseries 6121:} $21$, \textsf{\bfseries 6131:} $41$, \textsf{\bfseries 6133:} $19$, \textsf{\bfseries 6143:} $13$, 
\textsf{\bfseries 6151:} $6$, \textsf{\bfseries 6163:} $107$, \textsf{\bfseries 6173:} $13$, \textsf{\bfseries 6197:} $75$, \textsf{\bfseries 6199:} $13$, \textsf{\bfseries 6203:} $20$, \textsf{\bfseries 6211:} $2$, \textsf{\bfseries 6217:} $89$, \textsf{\bfseries 6221:} $52$, \textsf{\bfseries 6229:} $128$, \textsf{\bfseries 6247:} $10$, \textsf{\bfseries 6257:} $43$, \textsf{\bfseries 6263:} $22$, \textsf{\bfseries 6269:} $60$, \textsf{\bfseries 6271:} $11$, \textsf{\bfseries 6277:} $21$, \textsf{\bfseries 6287:} $26$, \textsf{\bfseries 6299:} $32$, \textsf{\bfseries 6301:} $51$, \textsf{\bfseries 6311:} $13$, \textsf{\bfseries 6317:} $17$, \textsf{\bfseries 6323:} $154$, \textsf{\bfseries 6329:} $65$, \textsf{\bfseries 6337:} $58$, \textsf{\bfseries 6343:} $66$, \textsf{\bfseries 6353:} $3$, \textsf{\bfseries 6359:} $74$, \textsf{\bfseries 6361:} $62$, \textsf{\bfseries 6367:} $54$, \textsf{\bfseries 6373:} $76$, \textsf{\bfseries 6379:} $58$, \textsf{\bfseries 6389:} $32$, \textsf{\bfseries 6397:} $189$, \textsf{\bfseries 6421:} $6$, \textsf{\bfseries 6427:} $74$, \textsf{\bfseries 6449:} $108$, \textsf{\bfseries 6451:} $18$, \textsf{\bfseries 6469:} $134$, \textsf{\bfseries 6473:} $61$, \textsf{\bfseries 6481:} $44$, \textsf{\bfseries 6491:} $7$, \textsf{\bfseries 6521:} $69$, \textsf{\bfseries 6529:} $7$, \textsf{\bfseries 6547:} $30$, \textsf{\bfseries 6551:} $17$, \textsf{\bfseries 6553:} $23$, \textsf{\bfseries 6563:} $102$, \textsf{\bfseries 6569:} $173$, \textsf{\bfseries 6571:} $23$, \textsf{\bfseries 6577:} $30$, \textsf{\bfseries 6581:} $14$, \textsf{\bfseries 6599:} $17$, \textsf{\bfseries 6607:} $45$, \textsf{\bfseries 6619:} $22$, \textsf{\bfseries 6637:} $2$, \textsf{\bfseries 6653:} $32$, \textsf{\bfseries 6659:} $24$, \textsf{\bfseries 6661:} $156$, \textsf{\bfseries 6673:} $5$, \textsf{\bfseries 6679:} $60$, \textsf{\bfseries 6689:} $15$, \textsf{\bfseries 6691:} $57$, \textsf{\bfseries 6701:} $46$, \textsf{\bfseries 6703:} $75$, \textsf{\bfseries 6709:} $2$, \textsf{\bfseries 6719:} $53$, \textsf{\bfseries 6733:} $31$, \textsf{\bfseries 6737:} $12$, \textsf{\bfseries 6761:} $272$, \textsf{\bfseries 6763:} $39$, \textsf{\bfseries 6779:} $50$, \textsf{\bfseries 6781:} $96$, \textsf{\bfseries 6791:} $110$, \textsf{\bfseries 6793:} $11$, \textsf{\bfseries 6803:} $43$, \textsf{\bfseries 6823:} $37$, \textsf{\bfseries 6827:} $6$, \textsf{\bfseries 6829:} $10$, \textsf{\bfseries 6833:} $27$, \textsf{\bfseries 6841:} $31$, \textsf{\bfseries 6857:} $24$, \textsf{\bfseries 6863:} $90$, \textsf{\bfseries 6869:} $8$, \textsf{\bfseries 6871:} $24$, \textsf{\bfseries 6883:} $12$, \textsf{\bfseries 6899:} $270$, \textsf{\bfseries 6907:} $21$, \textsf{\bfseries 6911:} $142$, \textsf{\bfseries 6917:} $8$, \textsf{\bfseries 6947:} $6$, \textsf{\bfseries 6949:} $28$, \textsf{\bfseries 6959:} $182$, \textsf{\bfseries 6961:} $69$, \textsf{\bfseries 6967:} $22$, \textsf{\bfseries 6971:} $6$, \textsf{\bfseries 6977:} $5$, \textsf{\bfseries 6983:} $15$, \textsf{\bfseries 6991:} $30$, \textsf{\bfseries 6997:} $93$, \textsf{\bfseries 7001:} $117$, 
\textsf{\bfseries 7013:} $61$, \textsf{\bfseries 7019:} $26$, \textsf{\bfseries 7027:} $11$, \textsf{\bfseries 7039:} $62$, \textsf{\bfseries 7043:} $28$, \textsf{\bfseries 7057:} $60$, \textsf{\bfseries 7069:} $7$, \textsf{\bfseries 7079:} $23$, \textsf{\bfseries 7103:} $10$, \textsf{\bfseries 7109:} $32$, \textsf{\bfseries 7121:} $23$, \textsf{\bfseries 7127:} $62$, \textsf{\bfseries 7129:} $184$, \textsf{\bfseries 7151:} $22$, \textsf{\bfseries 7159:} $75$, \textsf{\bfseries 7177:} $10$, \textsf{\bfseries 7187:} $39$, \textsf{\bfseries 7193:} $22$, \textsf{\bfseries 7207:} $93$, \textsf{\bfseries 7211:} $72$, \textsf{\bfseries 7213:} $51$, \textsf{\bfseries 7219:} $12$, \textsf{\bfseries 7229:} $34$, \textsf{\bfseries 7237:} $24$, \textsf{\bfseries 7243:} $41$, \textsf{\bfseries 7247:} $23$, \textsf{\bfseries 7253:} $71$, \textsf{\bfseries 7283:} $62$, \textsf{\bfseries 7297:} $45$, \textsf{\bfseries 7307:} $58$, \textsf{\bfseries 7309:} $31$, \textsf{\bfseries 7321:} $26$, \textsf{\bfseries 7331:} $6$, \textsf{\bfseries 7333:} $24$, \textsf{\bfseries 7349:} $26$, \textsf{\bfseries 7351:} $103$, \textsf{\bfseries 7369:} $39$, \textsf{\bfseries 7393:} $15$, \textsf{\bfseries 7411:} $21$, \textsf{\bfseries 7417:} $30$, \textsf{\bfseries 7433:} $5$, \textsf{\bfseries 7451:} $141$, \textsf{\bfseries 7457:} $12$, \textsf{\bfseries 7459:} $41$, \textsf{\bfseries 7477:} $19$, \textsf{\bfseries 7481:} $19$, \textsf{\bfseries 7487:} $7$, \textsf{\bfseries 7489:} $42$, \textsf{\bfseries 7499:} $8$, \textsf{\bfseries 7507:} $32$, \textsf{\bfseries 7517:} $59$, \textsf{\bfseries 7523:} $62$, \textsf{\bfseries 7529:} $29$, \textsf{\bfseries 7537:} $14$, \textsf{\bfseries 7541:} $66$, \textsf{\bfseries 7547:} $79$, \textsf{\bfseries 7549:} $30$, \textsf{\bfseries 7559:} $17$, \textsf{\bfseries 7561:} $47$, \textsf{\bfseries 7573:} $15$, \textsf{\bfseries 7577:} $19$, \textsf{\bfseries 7583:} $22$, \textsf{\bfseries 7589:} $167$, \textsf{\bfseries 7591:} $37$, \textsf{\bfseries 7603:} $52$, \textsf{\bfseries 7607:} $5$, \textsf{\bfseries 7621:} $24$, \textsf{\bfseries 7639:} $116$, \textsf{\bfseries 7643:} $19$, \textsf{\bfseries 7649:} $54$, \textsf{\bfseries 7669:} $11$, \textsf{\bfseries 7673:} $34$, \textsf{\bfseries 7681:} $143$, \textsf{\bfseries 7687:} $54$, \textsf{\bfseries 7691:} $10$, \textsf{\bfseries 7699:} $75$, \textsf{\bfseries 7703:} $11$, \textsf{\bfseries 7717:} $113$, \textsf{\bfseries 7723:} $59$, \textsf{\bfseries 7727:} $74$, \textsf{\bfseries 7741:} $30$, \textsf{\bfseries 7753:} $39$, \textsf{\bfseries 7757:} $8$, \textsf{\bfseries 7759:} $38$, \textsf{\bfseries 7789:} $18$, \textsf{\bfseries 7793:} $12$, \textsf{\bfseries 7817:} $20$, \textsf{\bfseries 7823:} $10$, \textsf{\bfseries 7829:} $43$, \textsf{\bfseries 7841:} $48$, \textsf{\bfseries 7853:} $18$, \textsf{\bfseries 7867:} $208$, \textsf{\bfseries 7873:} $78$, \textsf{\bfseries 7877:} $31$, \textsf{\bfseries 7879:} $112$, \textsf{\bfseries 7883:} $7$, \textsf{\bfseries 7901:} $22$, \textsf{\bfseries 7907:} $2$, \textsf{\bfseries 7919:} $29$, \textsf{\bfseries 7927:} $5$, 

              \textsf{\bfseries 7933:} $2$, \textsf{\bfseries 7937:} $38$, \textsf{\bfseries 7949:} $67$, \textsf{\bfseries 7951:} $171$, \textsf{\bfseries 7963:} $17$, \textsf{\bfseries 7993:} $39$, \textsf{\bfseries 8009:} $78$, \textsf{\bfseries 8011:} $57$, \textsf{\bfseries 8017:} $10$, \textsf{\bfseries 8039:} $127$, \textsf{\bfseries 8053:} $45$, \textsf{\bfseries 8059:} $10$, \textsf{\bfseries 8069:} $11$, \textsf{\bfseries 8081:} $135$, \textsf{\bfseries 8087:} $21$, \textsf{\bfseries 8089:} $43$, \textsf{\bfseries 8093:} $18$, \textsf{\bfseries 8101:} $131$, \textsf{\bfseries 8111:} $164$, \textsf{\bfseries 8117:} $5$, \textsf{\bfseries 8123:} $6$, \textsf{\bfseries 8147:} $78$, \textsf{\bfseries 8161:} $7$, \textsf{\bfseries 8167:} $34$, \textsf{\bfseries 8171:} $7$, \textsf{\bfseries 8179:} $118$, \textsf{\bfseries 8191:} $126$, \textsf{\bfseries 8209:} $56$, \textsf{\bfseries 8219:} $6$, \textsf{\bfseries 8221:} $18$, \textsf{\bfseries 8231:} $52$, \textsf{\bfseries 8233:} $59$, \textsf{\bfseries 8237:} $7$, \textsf{\bfseries 8243:} $18$, \textsf{\bfseries 8263:} $24$, \textsf{\bfseries 8269:} $10$, \textsf{\bfseries 8273:} $14$, \textsf{\bfseries 8287:} $6$, \textsf{\bfseries 8291:} $8$, \textsf{\bfseries 8293:} $46$, \textsf{\bfseries 8297:} $58$, \textsf{\bfseries 8311:} $73$, \textsf{\bfseries 8317:} $26$, \textsf{\bfseries 8329:} $140$, \textsf{\bfseries 8353:} $5$, \textsf{\bfseries 8363:} $5$, \textsf{\bfseries 8369:} $42$, \textsf{\bfseries 8377:} $59$, \textsf{\bfseries 8387:} $32$, \textsf{\bfseries 8389:} $6$, \textsf{\bfseries 8419:} $18$, \textsf{\bfseries 8423:} $86$, \textsf{\bfseries 8429:} $126$, \textsf{\bfseries 8431:} $3$, \textsf{\bfseries 8443:} $48$, \textsf{\bfseries 8447:} $58$, \textsf{\bfseries 8461:} $53$, \textsf{\bfseries 8467:} $32$, \textsf{\bfseries 8501:} $19$, \textsf{\bfseries 8513:} $101$, \textsf{\bfseries 8521:} $151$, \textsf{\bfseries 8527:} $213$, \textsf{\bfseries 8537:} $34$, \textsf{\bfseries 8539:} $11$, \textsf{\bfseries 8543:} $10$, \textsf{\bfseries 8563:} $7$, \textsf{\bfseries 8573:} $28$, \textsf{\bfseries 8581:} $19$, \textsf{\bfseries 8597:} $3$, \textsf{\bfseries 8599:} $6$, \textsf{\bfseries 8609:} $11$, \textsf{\bfseries 8623:} $40$, \textsf{\bfseries 8627:} $18$, \textsf{\bfseries 8629:} $51$, \textsf{\bfseries 8641:} $47$, \textsf{\bfseries 8647:} $3$, \textsf{\bfseries 8663:} $60$, \textsf{\bfseries 8669:} $52$, \textsf{\bfseries 8677:} $14$, \textsf{\bfseries 8681:} $47$, \textsf{\bfseries 8689:} $33$, \textsf{\bfseries 8693:} $7$, \textsf{\bfseries 8699:} $41$, \textsf{\bfseries 8707:} $58$, \textsf{\bfseries 8713:} $119$, \textsf{\bfseries 8719:} $19$, \textsf{\bfseries 8731:} $2$, \textsf{\bfseries 8737:} $37$, \textsf{\bfseries 8741:} $68$, \textsf{\bfseries 8747:} $5$, \textsf{\bfseries 8753:} $5$, \textsf{\bfseries 8761:} $148$, \textsf{\bfseries 8779:} $74$, \textsf{\bfseries 8783:} $76$, \textsf{\bfseries 8803:} $48$, \textsf{\bfseries 8807:} $118$, \textsf{\bfseries 8819:} $22$, \textsf{\bfseries 8821:} $44$, \textsf{\bfseries 8831:} $47$, \textsf{\bfseries 8837:} $37$, 
\textsf{\bfseries 8839:} $3$, \textsf{\bfseries 8849:} $59$, \textsf{\bfseries 8861:} $27$, \textsf{\bfseries 8863:} $5$, \textsf{\bfseries 8867:} $18$, \textsf{\bfseries 8887:} $28$, \textsf{\bfseries 8893:} $141$, \textsf{\bfseries 8923:} $41$, \textsf{\bfseries 8929:} $130$, \textsf{\bfseries 8933:} $78$, \textsf{\bfseries 8941:} $47$, \textsf{\bfseries 8951:} $146$, \textsf{\bfseries 8963:} $15$, \textsf{\bfseries 8969:} $83$, \textsf{\bfseries 8971:} $15$, \textsf{\bfseries 8999:} $21$, \textsf{\bfseries 9001:} $143$, \textsf{\bfseries 9007:} $58$, \textsf{\bfseries 9011:} $18$, \textsf{\bfseries 9013:} $242$, \textsf{\bfseries 9029:} $15$, \textsf{\bfseries 9041:} $34$, \textsf{\bfseries 9043:} $48$, \textsf{\bfseries 9049:} $92$, \textsf{\bfseries 9059:} $113$, \textsf{\bfseries 9067:} $99$, \textsf{\bfseries 9091:} $51$, \textsf{\bfseries 9103:} $6$, \textsf{\bfseries 9109:} $56$, \textsf{\bfseries 9127:} $189$, \textsf{\bfseries 9133:} $34$, \textsf{\bfseries 9137:} $5$, \textsf{\bfseries 9151:} $56$, \textsf{\bfseries 9157:} $35$, \textsf{\bfseries 9161:} $116$, \textsf{\bfseries 9173:} $20$, \textsf{\bfseries 9181:} $200$, \textsf{\bfseries 9187:} $3$, \textsf{\bfseries 9199:} $51$, \textsf{\bfseries 9203:} $168$, \textsf{\bfseries 9209:} $47$, \textsf{\bfseries 9221:} $70$, \textsf{\bfseries 9227:} $5$, \textsf{\bfseries 9239:} $71$, \textsf{\bfseries 9241:} $119$, \textsf{\bfseries 9257:} $10$, \textsf{\bfseries 9277:} $20$, \textsf{\bfseries 9281:} $59$, \textsf{\bfseries 9283:} $3$, \textsf{\bfseries 9293:} $88$, \textsf{\bfseries 9311:} $74$, \textsf{\bfseries 9319:} $209$, \textsf{\bfseries 9323:} $62$, \textsf{\bfseries 9337:} $46$, \textsf{\bfseries 9341:} $10$, \textsf{\bfseries 9343:} $39$, \textsf{\bfseries 9349:} $50$, \textsf{\bfseries 9371:} $2$, \textsf{\bfseries 9377:} $12$, \textsf{\bfseries 9391:} $70$, \textsf{\bfseries 9397:} $13$, \textsf{\bfseries 9403:} $11$, \textsf{\bfseries 9413:} $30$, \textsf{\bfseries 9419:} $8$, \textsf{\bfseries 9421:} $26$, \textsf{\bfseries 9431:} $57$, \textsf{\bfseries 9433:} $39$, \textsf{\bfseries 9437:} $67$, \textsf{\bfseries 9439:} $275$, \textsf{\bfseries 9461:} $48$, \textsf{\bfseries 9463:} $31$, \textsf{\bfseries 9467:} $80$, \textsf{\bfseries 9473:} $54$, \textsf{\bfseries 9479:} $65$, \textsf{\bfseries 9491:} $11$, \textsf{\bfseries 9497:} $34$, \textsf{\bfseries 9511:} $48$, \textsf{\bfseries 9521:} $23$, \textsf{\bfseries 9533:} $63$, \textsf{\bfseries 9539:} $62$, \textsf{\bfseries 9547:} $17$, \textsf{\bfseries 9551:} $74$, \textsf{\bfseries 9587:} $19$, \textsf{\bfseries 9601:} $43$, \textsf{\bfseries 9613:} $20$, \textsf{\bfseries 9619:} $2$, \textsf{\bfseries 9623:} $33$, \textsf{\bfseries 9629:} $19$, \textsf{\bfseries 9631:} $26$, \textsf{\bfseries 9643:} $51$, \textsf{\bfseries 9649:} $139$, \textsf{\bfseries 9661:} $61$, \textsf{\bfseries 9677:} $32$, \textsf{\bfseries 9679:} $259$, \textsf{\bfseries 9689:} $181$, \textsf{\bfseries 9697:} $20$, \textsf{\bfseries 9719:} $94$, \textsf{\bfseries 9721:} $33$, \textsf{\bfseries 9733:} $31$, \textsf{\bfseries 9739:} $112$, 
\textsf{\bfseries 9743:} $80$, \textsf{\bfseries 9749:} $59$, \textsf{\bfseries 9767:} $20$, \textsf{\bfseries 9769:} $71$, \textsf{\bfseries 9781:} $6$, \textsf{\bfseries 9787:} $29$, \textsf{\bfseries 9791:} $44$, \textsf{\bfseries 9803:} $104$, \textsf{\bfseries 9811:} $12$, \textsf{\bfseries 9817:} $53$, \textsf{\bfseries 9829:} $87$, \textsf{\bfseries 9833:} $47$, \textsf{\bfseries 9839:} $19$, \textsf{\bfseries 9851:} $8$, \textsf{\bfseries 9857:} $29$, \textsf{\bfseries 9859:} $56$, \textsf{\bfseries 9871:} $17$, \textsf{\bfseries 9883:} $80$, \textsf{\bfseries 9887:} $57$, \textsf{\bfseries 9901:} $22$, \textsf{\bfseries 9907:} $29$, \textsf{\bfseries 9923:} $7$, \textsf{\bfseries 9929:} $14$, \textsf{\bfseries 9931:} $119$, \textsf{\bfseries 9941:} $23$, \textsf{\bfseries 9949:} $123$, \textsf{\bfseries 9967:} $21$, \textsf{\bfseries 9973:} $13$ 
  \item[$r=2$:] \textsf{\bfseries 2:} 0:\,$a + 1$,\ 5:\,$a$,\ 9:\,$1$, \textsf{\bfseries 3:} $a$, \textsf{\bfseries 5:} 0:\,$3a$,\ 1:\,$2$,\ 9:\,$1$, \textsf{\bfseries 7:} $3a + 3$, \textsf{\bfseries 11:} $9a$, \textsf{\bfseries 13:} 0:\,$7a$,\ 9:\,$2$, \textsf{\bfseries 17:} $4a$, \textsf{\bfseries 19:} $8a$, \textsf{\bfseries 23:} $8a$, \textsf{\bfseries 29:} $27a$, \textsf{\bfseries 31:} $12a + 3$, \textsf{\bfseries 37:} $6a$, \textsf{\bfseries 41:} $24a$, \textsf{\bfseries 43:} $42a$, \textsf{\bfseries 47:} $5a$, \textsf{\bfseries 53:} $8a$, \textsf{\bfseries 59:} $15a$, \textsf{\bfseries 61:} $10a$, \textsf{\bfseries 67:} $25a$, \textsf{\bfseries 71:} $3a$, \textsf{\bfseries 73:} $13a$, \textsf{\bfseries 79:} $10a$, \textsf{\bfseries 83:} $2a$, \textsf{\bfseries 89:} $10a$, \textsf{\bfseries 97:} $9a$
  \item[$r=3$:] \textsf{\bfseries 2:} 0:\,$a$,\ 1:\,$a + 1$,\ 9:\,$1$, \textsf{\bfseries 3:} 0:\,$2a^2 + 2a$,\ 2:\,$2a$,\ 9:\,$1$, \textsf{\bfseries 5:} 0:\,$3a + 3$,\ 1:\,$1$,\ 9:\,$1$, \textsf{\bfseries 7:} $4a + 3$, \textsf{\bfseries 11:} $9a + 3$, \textsf{\bfseries 13:} $3a$, \textsf{\bfseries 17:} $10a + 3$, \textsf{\bfseries 19:} $14a + 1$
  \item[$r=4$:] \textsf{\bfseries 2:} 0:\,$a^2$,\ 1:\,$a^3$,\ 9:\,$1$, \textsf{\bfseries 3:} $2a + 1$, \textsf{\bfseries 5:} 0:\,$3a$,\ 1:\,$1$,\ 9:\,$1$, \textsf{\bfseries 7:} $6a$
  \item[$r=5$:] \textsf{\bfseries 2:} 0:\,$a^2 + a$,\ 1:\,$a$,\ 9:\,$1$, \textsf{\bfseries 3:} 0:\,$2a^2 + 2a + 2$,\ 2:\,$2$,\ 9:\,$1$
  \item[$r=6$:] \textsf{\bfseries 2:} 0:\,$a$,\ 1:\,$1$,\ 9:\,$1$, \textsf{\bfseries 3:} $a^2 + 1$
  %\item[$r=7$:] \textsf{\bfseries 2:} 0:\,$a^3 + 1$,\ 1:\,$1$,\ 9:\,$1$
  %\item[$r=8$:] \textsf{\bfseries 2:} 0:\,$a^7$,\ 1:\,$1$,\ 9:\,$1$
  %\item[$r=9$:] \input{./tables/pcns_10_9__0.tex}
\end{description}


\subsection{$\PCN$-Polynome für $n=12$}

\begin{description}[leftmargin=0pt,labelindent=20pt,
  font=\normalsize]
  \item[$r=1$:] \input{./tables/pcns_12_1__0.tex}
                \textsf{\bfseries 7933:} $24$, \textsf{\bfseries 7937:} $44$, \textsf{\bfseries 7949:} $11$, \textsf{\bfseries 7951:} $55$, \textsf{\bfseries 7963:} $22$, \textsf{\bfseries 7993:} $67$, \textsf{\bfseries 8009:} $243$, \textsf{\bfseries 8011:} $98$, \textsf{\bfseries 8017:} $194$, \textsf{\bfseries 8039:} $62$, \textsf{\bfseries 8053:} $13$, \textsf{\bfseries 8059:} $3$, \textsf{\bfseries 8069:} $71$, \textsf{\bfseries 8081:} $27$, \textsf{\bfseries 8087:} $65$, \textsf{\bfseries 8089:} $94$, \textsf{\bfseries 8093:} $163$, \textsf{\bfseries 8101:} $6$, \textsf{\bfseries 8111:} $37$, \textsf{\bfseries 8117:} $2$, \textsf{\bfseries 8123:} $32$, \textsf{\bfseries 8147:} $31$, \textsf{\bfseries 8161:} $22$, \textsf{\bfseries 8167:} $38$, \textsf{\bfseries 8171:} $59$, \textsf{\bfseries 8179:} $3$, \textsf{\bfseries 8191:} $297$, \textsf{\bfseries 8209:} $215$, \textsf{\bfseries 8219:} $28$, \textsf{\bfseries 8221:} $120$, \textsf{\bfseries 8231:} $93$, \textsf{\bfseries 8233:} $212$, \textsf{\bfseries 8237:} $123$, \textsf{\bfseries 8243:} $28$, \textsf{\bfseries 8263:} $106$, \textsf{\bfseries 8269:} $96$, \textsf{\bfseries 8273:} $48$, \textsf{\bfseries 8287:} $53$, \textsf{\bfseries 8291:} $2$, \textsf{\bfseries 8293:} $79$, \textsf{\bfseries 8297:} $5$, \textsf{\bfseries 8311:} $57$, \textsf{\bfseries 8317:} $26$, \textsf{\bfseries 8329:} $73$, \textsf{\bfseries 8353:} $68$, \textsf{\bfseries 8363:} $168$, \textsf{\bfseries 8369:} $127$, \textsf{\bfseries 8377:} $66$, \textsf{\bfseries 8387:} $6$, \textsf{\bfseries 8389:} $24$, \textsf{\bfseries 8419:} $138$, \textsf{\bfseries 8423:} $90$, \textsf{\bfseries 8429:} $41$, \textsf{\bfseries 8431:} $91$, \textsf{\bfseries 8443:} $164$, \textsf{\bfseries 8447:} $205$, \textsf{\bfseries 8461:} $55$, \textsf{\bfseries 8467:} $28$, \textsf{\bfseries 8501:} $139$, \textsf{\bfseries 8513:} $79$, \textsf{\bfseries 8521:} $158$, \textsf{\bfseries 8527:} $54$, \textsf{\bfseries 8537:} $95$, \textsf{\bfseries 8539:} $48$, \textsf{\bfseries 8543:} $143$, \textsf{\bfseries 8563:} $42$, \textsf{\bfseries 8573:} $70$, \textsf{\bfseries 8581:} $50$, \textsf{\bfseries 8597:} $2$, \textsf{\bfseries 8599:} $3$, \textsf{\bfseries 8609:} $24$, \textsf{\bfseries 8623:} $37$, \textsf{\bfseries 8627:} $39$, \textsf{\bfseries 8629:} $57$, \textsf{\bfseries 8641:} $21$, \textsf{\bfseries 8647:} $87$, \textsf{\bfseries 8663:} $41$, \textsf{\bfseries 8669:} $29$, \textsf{\bfseries 8677:} $105$, \textsf{\bfseries 8681:} $188$, \textsf{\bfseries 8689:} $47$, \textsf{\bfseries 8693:} $28$, \textsf{\bfseries 8699:} $44$, \textsf{\bfseries 8707:} $39$, \textsf{\bfseries 8713:} $73$, \textsf{\bfseries 8719:} $39$, \textsf{\bfseries 8731:} $74$, \textsf{\bfseries 8737:} $158$, \textsf{\bfseries 8741:} $135$, \textsf{\bfseries 8747:} $7$, \textsf{\bfseries 8753:} $45$, \textsf{\bfseries 8761:} $85$, \textsf{\bfseries 8779:} $197$, \textsf{\bfseries 8783:} $78$, \textsf{\bfseries 8803:} $41$, \textsf{\bfseries 8807:} $85$, \textsf{\bfseries 8819:} $8$, \textsf{\bfseries 8821:} $136$, \textsf{\bfseries 8831:} $113$, \textsf{\bfseries 8837:} $71$, 
\textsf{\bfseries 8839:} $3$, \textsf{\bfseries 8849:} $33$, \textsf{\bfseries 8861:} $52$, \textsf{\bfseries 8863:} $10$, \textsf{\bfseries 8867:} $128$, \textsf{\bfseries 8887:} $45$, \textsf{\bfseries 8893:} $199$, \textsf{\bfseries 8923:} $34$, \textsf{\bfseries 8929:} $165$, \textsf{\bfseries 8933:} $185$, \textsf{\bfseries 8941:} $96$, \textsf{\bfseries 8951:} $26$, \textsf{\bfseries 8963:} $124$, \textsf{\bfseries 8969:} $68$, \textsf{\bfseries 8971:} $197$, \textsf{\bfseries 8999:} $62$, \textsf{\bfseries 9001:} $58$, \textsf{\bfseries 9007:} $185$, \textsf{\bfseries 9011:} $28$, \textsf{\bfseries 9013:} $41$, \textsf{\bfseries 9029:} $10$, \textsf{\bfseries 9041:} $22$, \textsf{\bfseries 9043:} $57$, \textsf{\bfseries 9049:} $82$, \textsf{\bfseries 9059:} $22$, \textsf{\bfseries 9067:} $68$, \textsf{\bfseries 9091:} $11$, \textsf{\bfseries 9103:} $39$, \textsf{\bfseries 9109:} $168$, \textsf{\bfseries 9127:} $102$, \textsf{\bfseries 9133:} $85$, \textsf{\bfseries 9137:} $58$, \textsf{\bfseries 9151:} $12$, \textsf{\bfseries 9157:} $19$, \textsf{\bfseries 9161:} $6$, \textsf{\bfseries 9173:} $44$, \textsf{\bfseries 9181:} $56$, \textsf{\bfseries 9187:} $30$, \textsf{\bfseries 9199:} $12$, \textsf{\bfseries 9203:} $78$, \textsf{\bfseries 9209:} $11$, \textsf{\bfseries 9221:} $68$, \textsf{\bfseries 9227:} $403$, \textsf{\bfseries 9239:} $107$, \textsf{\bfseries 9241:} $159$, \textsf{\bfseries 9257:} $102$, \textsf{\bfseries 9277:} $29$, \textsf{\bfseries 9281:} $27$, \textsf{\bfseries 9283:} $70$, \textsf{\bfseries 9293:} $50$, \textsf{\bfseries 9311:} $11$, \textsf{\bfseries 9319:} $34$, \textsf{\bfseries 9323:} $6$, \textsf{\bfseries 9337:} $253$, \textsf{\bfseries 9341:} $50$, \textsf{\bfseries 9343:} $11$, \textsf{\bfseries 9349:} $77$, \textsf{\bfseries 9371:} $10$, \textsf{\bfseries 9377:} $73$, \textsf{\bfseries 9391:} $43$, \textsf{\bfseries 9397:} $15$, \textsf{\bfseries 9403:} $11$, \textsf{\bfseries 9413:} $48$, \textsf{\bfseries 9419:} $115$, \textsf{\bfseries 9421:} $18$, \textsf{\bfseries 9431:} $104$, \textsf{\bfseries 9433:} $47$, \textsf{\bfseries 9437:} $61$, \textsf{\bfseries 9439:} $204$, \textsf{\bfseries 9461:} $15$, \textsf{\bfseries 9463:} $90$, \textsf{\bfseries 9467:} $80$, \textsf{\bfseries 9473:} $12$, \textsf{\bfseries 9479:} $19$, \textsf{\bfseries 9491:} $40$, \textsf{\bfseries 9497:} $34$, \textsf{\bfseries 9511:} $60$, \textsf{\bfseries 9521:} $6$, \textsf{\bfseries 9533:} $5$, \textsf{\bfseries 9539:} $104$, \textsf{\bfseries 9547:} $12$, \textsf{\bfseries 9551:} $183$, \textsf{\bfseries 9587:} $28$, \textsf{\bfseries 9601:} $166$, \textsf{\bfseries 9613:} $23$, \textsf{\bfseries 9619:} $75$, \textsf{\bfseries 9623:} $5$, \textsf{\bfseries 9629:} $98$, \textsf{\bfseries 9631:} $191$, \textsf{\bfseries 9643:} $5$, \textsf{\bfseries 9649:} $29$, \textsf{\bfseries 9661:} $52$, \textsf{\bfseries 9677:} $58$, \textsf{\bfseries 9679:} $54$, \textsf{\bfseries 9689:} $135$, \textsf{\bfseries 9697:} $71$, \textsf{\bfseries 9719:} $194$, \textsf{\bfseries 9721:} $114$, \textsf{\bfseries 9733:} $45$, \textsf{\bfseries 9739:} $18$, 
\textsf{\bfseries 9743:} $158$, \textsf{\bfseries 9749:} $166$, \textsf{\bfseries 9767:} $31$, \textsf{\bfseries 9769:} $184$, \textsf{\bfseries 9781:} $33$, \textsf{\bfseries 9787:} $17$, \textsf{\bfseries 9791:} $53$, \textsf{\bfseries 9803:} $24$, \textsf{\bfseries 9811:} $18$, \textsf{\bfseries 9817:} $10$, \textsf{\bfseries 9829:} $14$, \textsf{\bfseries 9833:} $10$, \textsf{\bfseries 9839:} $43$, \textsf{\bfseries 9851:} $217$, \textsf{\bfseries 9857:} $39$, \textsf{\bfseries 9859:} $11$, \textsf{\bfseries 9871:} $29$, \textsf{\bfseries 9883:} $72$, \textsf{\bfseries 9887:} $142$, \textsf{\bfseries 9901:} $85$, \textsf{\bfseries 9907:} $41$, \textsf{\bfseries 9923:} $139$, \textsf{\bfseries 9929:} $12$, \textsf{\bfseries 9931:} $57$, \textsf{\bfseries 9941:} $8$, \textsf{\bfseries 9949:} $130$, \textsf{\bfseries 9967:} $10$, \textsf{\bfseries 9973:} $68$, \textsf{\bfseries 10007:} $5$, \textsf{\bfseries 10009:} $31$, \textsf{\bfseries 10037:} $43$, \textsf{\bfseries 10039:} $93$, \textsf{\bfseries 10061:} $60$, \textsf{\bfseries 10067:} $15$, \textsf{\bfseries 10069:} $35$, \textsf{\bfseries 10079:} $123$, \textsf{\bfseries 10091:} $370$, \textsf{\bfseries 10093:} $15$, \textsf{\bfseries 10099:} $220$, \textsf{\bfseries 10103:} $61$, \textsf{\bfseries 10111:} $166$, \textsf{\bfseries 10133:} $141$, \textsf{\bfseries 10139:} $14$, \textsf{\bfseries 10141:} $33$, \textsf{\bfseries 10151:} $26$, \textsf{\bfseries 10159:} $66$, \textsf{\bfseries 10163:} $46$, \textsf{\bfseries 10169:} $158$, \textsf{\bfseries 10177:} $86$, \textsf{\bfseries 10181:} $117$, \textsf{\bfseries 10193:} $275$, \textsf{\bfseries 10211:} $101$, \textsf{\bfseries 10223:} $34$, \textsf{\bfseries 10243:} $38$, \textsf{\bfseries 10247:} $10$, \textsf{\bfseries 10253:} $8$, \textsf{\bfseries 10259:} $30$, \textsf{\bfseries 10267:} $59$, \textsf{\bfseries 10271:} $37$, \textsf{\bfseries 10273:} $10$, \textsf{\bfseries 10289:} $77$, \textsf{\bfseries 10301:} $82$, \textsf{\bfseries 10303:} $53$, \textsf{\bfseries 10313:} $6$, \textsf{\bfseries 10321:} $116$, \textsf{\bfseries 10331:} $37$, \textsf{\bfseries 10333:} $52$, \textsf{\bfseries 10337:} $24$, \textsf{\bfseries 10343:} $30$, \textsf{\bfseries 10357:} $65$, \textsf{\bfseries 10369:} $23$, \textsf{\bfseries 10391:} $71$, \textsf{\bfseries 10399:} $55$, \textsf{\bfseries 10427:} $17$, \textsf{\bfseries 10429:} $29$, \textsf{\bfseries 10433:} $139$, \textsf{\bfseries 10453:} $24$, \textsf{\bfseries 10457:} $13$, \textsf{\bfseries 10459:} $42$, \textsf{\bfseries 10463:} $137$, \textsf{\bfseries 10477:} $320$, \textsf{\bfseries 10487:} $62$, \textsf{\bfseries 10499:} $8$, \textsf{\bfseries 10501:} $56$, \textsf{\bfseries 10513:} $61$, \textsf{\bfseries 10529:} $135$, \textsf{\bfseries 10531:} $10$, \textsf{\bfseries 10559:} $53$, \textsf{\bfseries 10567:} $29$, \textsf{\bfseries 10589:} $66$, \textsf{\bfseries 10597:} $80$, \textsf{\bfseries 10601:} $37$, \textsf{\bfseries 10607:} $67$, \textsf{\bfseries 10613:} $41$, \textsf{\bfseries 10627:} $13$, \textsf{\bfseries 10631:} $41$, \textsf{\bfseries 10639:} $87$, \textsf{\bfseries 10651:} $37$, \textsf{\bfseries 10657:} $61$, \textsf{\bfseries 10663:} $155$, 
\textsf{\bfseries 10667:} $60$, \textsf{\bfseries 10687:} $17$, \textsf{\bfseries 10691:} $128$, \textsf{\bfseries 10709:} $156$, \textsf{\bfseries 10711:} $35$, \textsf{\bfseries 10723:} $3$, \textsf{\bfseries 10729:} $14$, \textsf{\bfseries 10733:} $80$, \textsf{\bfseries 10739:} $68$, \textsf{\bfseries 10753:} $13$, \textsf{\bfseries 10771:} $201$, \textsf{\bfseries 10781:} $109$, \textsf{\bfseries 10789:} $101$, \textsf{\bfseries 10799:} $43$, \textsf{\bfseries 10831:} $12$, \textsf{\bfseries 10837:} $20$, \textsf{\bfseries 10847:} $10$, \textsf{\bfseries 10853:} $32$, \textsf{\bfseries 10859:} $146$, \textsf{\bfseries 10861:} $86$, \textsf{\bfseries 10867:} $46$, \textsf{\bfseries 10883:} $15$, \textsf{\bfseries 10889:} $22$, \textsf{\bfseries 10891:} $44$, \textsf{\bfseries 10903:} $150$, \textsf{\bfseries 10909:} $44$, \textsf{\bfseries 10937:} $3$, \textsf{\bfseries 10939:} $187$, \textsf{\bfseries 10949:} $10$, \textsf{\bfseries 10957:} $190$, \textsf{\bfseries 10973:} $2$, \textsf{\bfseries 10979:} $18$, \textsf{\bfseries 10987:} $19$, \textsf{\bfseries 10993:} $60$, \textsf{\bfseries 11003:} $73$, \textsf{\bfseries 11027:} $65$, \textsf{\bfseries 11047:} $11$, \textsf{\bfseries 11057:} $136$, \textsf{\bfseries 11059:} $15$, \textsf{\bfseries 11069:} $70$, \textsf{\bfseries 11071:} $30$, \textsf{\bfseries 11083:} $162$, \textsf{\bfseries 11087:} $165$, \textsf{\bfseries 11093:} $122$, \textsf{\bfseries 11113:} $39$, \textsf{\bfseries 11117:} $19$, \textsf{\bfseries 11119:} $344$, \textsf{\bfseries 11131:} $2$, \textsf{\bfseries 11149:} $123$, \textsf{\bfseries 11159:} $111$, \textsf{\bfseries 11161:} $111$, \textsf{\bfseries 11171:} $67$, \textsf{\bfseries 11173:} $106$, \textsf{\bfseries 11177:} $48$, \textsf{\bfseries 11197:} $143$, \textsf{\bfseries 11213:} $5$, \textsf{\bfseries 11239:} $208$, \textsf{\bfseries 11243:} $15$, \textsf{\bfseries 11251:} $98$, \textsf{\bfseries 11257:} $173$, \textsf{\bfseries 11261:} $61$, \textsf{\bfseries 11273:} $20$, \textsf{\bfseries 11279:} $46$, \textsf{\bfseries 11287:} $38$, \textsf{\bfseries 11299:} $75$, \textsf{\bfseries 11311:} $11$, \textsf{\bfseries 11317:} $88$, \textsf{\bfseries 11321:} $6$, \textsf{\bfseries 11329:} $305$, \textsf{\bfseries 11351:} $26$, \textsf{\bfseries 11353:} $34$, \textsf{\bfseries 11369:} $111$, \textsf{\bfseries 11383:} $13$, \textsf{\bfseries 11393:} $5$, \textsf{\bfseries 11399:} $44$, \textsf{\bfseries 11411:} $102$, \textsf{\bfseries 11423:} $67$, \textsf{\bfseries 11437:} $46$, \textsf{\bfseries 11443:} $31$, \textsf{\bfseries 11447:} $45$, \textsf{\bfseries 11467:} $30$, \textsf{\bfseries 11471:} $146$, \textsf{\bfseries 11483:} $15$, \textsf{\bfseries 11489:} $19$, \textsf{\bfseries 11491:} $193$, \textsf{\bfseries 11497:} $10$, \textsf{\bfseries 11503:} $3$, \textsf{\bfseries 11519:} $14$, \textsf{\bfseries 11527:} $110$, \textsf{\bfseries 11549:} $71$, \textsf{\bfseries 11551:} $28$, \textsf{\bfseries 11579:} $7$, \textsf{\bfseries 11587:} $28$, \textsf{\bfseries 11593:} $94$, \textsf{\bfseries 11597:} $26$, \textsf{\bfseries 11617:} $172$, \textsf{\bfseries 11621:} $8$, \textsf{\bfseries 11633:} $150$, \textsf{\bfseries 11657:} $20$, \textsf{\bfseries 11677:} $44$, 
\textsf{\bfseries 11681:} $101$, \textsf{\bfseries 11689:} $89$, \textsf{\bfseries 11699:} $51$, \textsf{\bfseries 11701:} $54$, \textsf{\bfseries 11717:} $51$, \textsf{\bfseries 11719:} $12$, \textsf{\bfseries 11731:} $174$, \textsf{\bfseries 11743:} $10$, \textsf{\bfseries 11777:} $29$, \textsf{\bfseries 11779:} $61$, \textsf{\bfseries 11783:} $7$, \textsf{\bfseries 11789:} $66$, \textsf{\bfseries 11801:} $38$, \textsf{\bfseries 11807:} $140$, \textsf{\bfseries 11813:} $80$, \textsf{\bfseries 11821:} $54$, \textsf{\bfseries 11827:} $199$, \textsf{\bfseries 11831:} $23$, \textsf{\bfseries 11833:} $181$, \textsf{\bfseries 11839:} $62$, \textsf{\bfseries 11863:} $59$, \textsf{\bfseries 11867:} $32$, \textsf{\bfseries 11887:} $14$, \textsf{\bfseries 11897:} $84$, \textsf{\bfseries 11903:} $151$, \textsf{\bfseries 11909:} $41$, \textsf{\bfseries 11923:} $57$, \textsf{\bfseries 11927:} $26$, \textsf{\bfseries 11933:} $22$, \textsf{\bfseries 11939:} $105$, \textsf{\bfseries 11941:} $13$, \textsf{\bfseries 11953:} $10$, \textsf{\bfseries 11959:} $66$, \textsf{\bfseries 11969:} $3$, \textsf{\bfseries 11971:} $99$, \textsf{\bfseries 11981:} $18$, \textsf{\bfseries 11987:} $47$, \textsf{\bfseries 12007:} $62$, \textsf{\bfseries 12011:} $101$, \textsf{\bfseries 12037:} $96$, \textsf{\bfseries 12041:} $75$, \textsf{\bfseries 12043:} $18$, \textsf{\bfseries 12049:} $143$, \textsf{\bfseries 12071:} $86$, \textsf{\bfseries 12073:} $44$, \textsf{\bfseries 12097:} $26$, \textsf{\bfseries 12101:} $42$, \textsf{\bfseries 12107:} $180$, \textsf{\bfseries 12109:} $149$, \textsf{\bfseries 12113:} $22$, \textsf{\bfseries 12119:} $95$, \textsf{\bfseries 12143:} $58$, \textsf{\bfseries 12149:} $113$, \textsf{\bfseries 12157:} $72$, \textsf{\bfseries 12161:} $176$, \textsf{\bfseries 12163:} $47$, \textsf{\bfseries 12197:} $55$, \textsf{\bfseries 12203:} $7$, \textsf{\bfseries 12211:} $128$, \textsf{\bfseries 12227:} $60$, \textsf{\bfseries 12239:} $59$, \textsf{\bfseries 12241:} $31$, \textsf{\bfseries 12251:} $73$, \textsf{\bfseries 12253:} $6$, \textsf{\bfseries 12263:} $62$, \textsf{\bfseries 12269:} $23$, \textsf{\bfseries 12277:} $22$, \textsf{\bfseries 12281:} $33$, \textsf{\bfseries 12289:} $23$, \textsf{\bfseries 12301:} $30$, \textsf{\bfseries 12323:} $93$, \textsf{\bfseries 12329:} $157$, \textsf{\bfseries 12343:} $134$, \textsf{\bfseries 12347:} $185$, \textsf{\bfseries 12373:} $68$, \textsf{\bfseries 12377:} $10$, \textsf{\bfseries 12379:} $46$, \textsf{\bfseries 12391:} $52$, \textsf{\bfseries 12401:} $54$, \textsf{\bfseries 12409:} $35$, \textsf{\bfseries 12413:} $272$, \textsf{\bfseries 12421:} $208$, \textsf{\bfseries 12433:} $97$, \textsf{\bfseries 12437:} $91$, \textsf{\bfseries 12451:} $105$, \textsf{\bfseries 12457:} $80$, \textsf{\bfseries 12473:} $111$, \textsf{\bfseries 12479:} $61$, \textsf{\bfseries 12487:} $47$, \textsf{\bfseries 12491:} $91$, \textsf{\bfseries 12497:} $515$, \textsf{\bfseries 12503:} $53$, \textsf{\bfseries 12511:} $55$, \textsf{\bfseries 12517:} $13$, \textsf{\bfseries 12527:} $28$, \textsf{\bfseries 12539:} $51$, \textsf{\bfseries 12541:} $127$, \textsf{\bfseries 12547:} $52$, \textsf{\bfseries 12553:} $31$, \textsf{\bfseries 12569:} $22$, 
\textsf{\bfseries 12577:} $145$, \textsf{\bfseries 12583:} $67$, \textsf{\bfseries 12589:} $22$, \textsf{\bfseries 12601:} $87$, \textsf{\bfseries 12611:} $26$, \textsf{\bfseries 12613:} $86$, \textsf{\bfseries 12619:} $48$, \textsf{\bfseries 12637:} $55$, \textsf{\bfseries 12641:} $23$, \textsf{\bfseries 12647:} $15$, \textsf{\bfseries 12653:} $111$, \textsf{\bfseries 12659:} $32$, \textsf{\bfseries 12671:} $152$, \textsf{\bfseries 12689:} $27$, \textsf{\bfseries 12697:} $60$, \textsf{\bfseries 12703:} $12$, \textsf{\bfseries 12713:} $24$, \textsf{\bfseries 12721:} $116$, \textsf{\bfseries 12739:} $146$, \textsf{\bfseries 12743:} $22$, \textsf{\bfseries 12757:} $11$, \textsf{\bfseries 12763:} $32$, \textsf{\bfseries 12781:} $89$, \textsf{\bfseries 12791:} $34$, \textsf{\bfseries 12799:} $57$, \textsf{\bfseries 12809:} $19$, \textsf{\bfseries 12821:} $14$, \textsf{\bfseries 12823:} $10$, \textsf{\bfseries 12829:} $18$, \textsf{\bfseries 12841:} $59$, \textsf{\bfseries 12853:} $22$, \textsf{\bfseries 12889:} $117$, \textsf{\bfseries 12893:} $132$, \textsf{\bfseries 12899:} $26$, \textsf{\bfseries 12907:} $35$, \textsf{\bfseries 12911:} $146$, \textsf{\bfseries 12917:} $68$, \textsf{\bfseries 12919:} $75$, \textsf{\bfseries 12923:} $91$, \textsf{\bfseries 12941:} $28$, \textsf{\bfseries 12953:} $13$, \textsf{\bfseries 12959:} $63$, \textsf{\bfseries 12967:} $40$, \textsf{\bfseries 12973:} $43$, \textsf{\bfseries 12979:} $162$, \textsf{\bfseries 12983:} $82$, \textsf{\bfseries 13001:} $44$, \textsf{\bfseries 13003:} $66$, \textsf{\bfseries 13007:} $86$, \textsf{\bfseries 13009:} $103$, \textsf{\bfseries 13033:} $90$, \textsf{\bfseries 13037:} $30$, \textsf{\bfseries 13043:} $15$, \textsf{\bfseries 13049:} $38$, \textsf{\bfseries 13063:} $57$, \textsf{\bfseries 13093:} $60$, \textsf{\bfseries 13099:} $3$, \textsf{\bfseries 13103:} $30$, \textsf{\bfseries 13109:} $10$, \textsf{\bfseries 13121:} $37$, \textsf{\bfseries 13127:} $74$, \textsf{\bfseries 13147:} $59$, \textsf{\bfseries 13151:} $26$, \textsf{\bfseries 13159:} $21$, \textsf{\bfseries 13163:} $2$, \textsf{\bfseries 13171:} $50$, \textsf{\bfseries 13177:} $10$, \textsf{\bfseries 13183:} $11$, \textsf{\bfseries 13187:} $68$, \textsf{\bfseries 13217:} $3$, \textsf{\bfseries 13219:} $21$, \textsf{\bfseries 13229:} $11$, \textsf{\bfseries 13241:} $22$, \textsf{\bfseries 13249:} $129$, \textsf{\bfseries 13259:} $30$, \textsf{\bfseries 13267:} $42$, \textsf{\bfseries 13291:} $66$, \textsf{\bfseries 13297:} $20$, \textsf{\bfseries 13309:} $24$, \textsf{\bfseries 13313:} $204$, \textsf{\bfseries 13327:} $51$, \textsf{\bfseries 13331:} $70$, \textsf{\bfseries 13337:} $38$, \textsf{\bfseries 13339:} $67$, \textsf{\bfseries 13367:} $21$, \textsf{\bfseries 13381:} $24$, \textsf{\bfseries 13397:} $85$, \textsf{\bfseries 13399:} $78$, \textsf{\bfseries 13411:} $3$, \textsf{\bfseries 13417:} $135$, \textsf{\bfseries 13421:} $56$, \textsf{\bfseries 13441:} $33$, \textsf{\bfseries 13451:} $46$, \textsf{\bfseries 13457:} $40$, \textsf{\bfseries 13463:} $40$, \textsf{\bfseries 13469:} $58$, \textsf{\bfseries 13477:} $31$, \textsf{\bfseries 13487:} $10$, \textsf{\bfseries 13499:} $14$, \textsf{\bfseries 13513:} $20$, 
\textsf{\bfseries 13523:} $168$, \textsf{\bfseries 13537:} $11$, \textsf{\bfseries 13553:} $71$, \textsf{\bfseries 13567:} $24$, \textsf{\bfseries 13577:} $52$, \textsf{\bfseries 13591:} $53$, \textsf{\bfseries 13597:} $51$, \textsf{\bfseries 13613:} $2$, \textsf{\bfseries 13619:} $44$, \textsf{\bfseries 13627:} $29$, \textsf{\bfseries 13633:} $85$, \textsf{\bfseries 13649:} $29$, \textsf{\bfseries 13669:} $30$, \textsf{\bfseries 13679:} $28$, \textsf{\bfseries 13681:} $46$, \textsf{\bfseries 13687:} $6$, \textsf{\bfseries 13691:} $2$, \textsf{\bfseries 13693:} $15$, \textsf{\bfseries 13697:} $147$, \textsf{\bfseries 13709:} $53$, \textsf{\bfseries 13711:} $11$, \textsf{\bfseries 13721:} $130$, \textsf{\bfseries 13723:} $19$, \textsf{\bfseries 13729:} $111$, \textsf{\bfseries 13751:} $62$, \textsf{\bfseries 13757:} $69$, \textsf{\bfseries 13759:} $30$, \textsf{\bfseries 13763:} $63$, \textsf{\bfseries 13781:} $75$, \textsf{\bfseries 13789:} $19$, \textsf{\bfseries 13799:} $26$, \textsf{\bfseries 13807:} $29$, \textsf{\bfseries 13829:} $101$, \textsf{\bfseries 13831:} $103$, \textsf{\bfseries 13841:} $48$, \textsf{\bfseries 13859:} $73$, \textsf{\bfseries 13873:} $80$, \textsf{\bfseries 13877:} $8$, \textsf{\bfseries 13879:} $48$, \textsf{\bfseries 13883:} $20$, \textsf{\bfseries 13901:} $44$, \textsf{\bfseries 13903:} $34$, \textsf{\bfseries 13907:} $128$, \textsf{\bfseries 13913:} $75$, \textsf{\bfseries 13921:} $126$, \textsf{\bfseries 13931:} $72$, \textsf{\bfseries 13933:} $44$, \textsf{\bfseries 13963:} $107$, \textsf{\bfseries 13967:} $5$, \textsf{\bfseries 13997:} $19$, \textsf{\bfseries 13999:} $158$, \textsf{\bfseries 14009:} $69$, \textsf{\bfseries 14011:} $23$, \textsf{\bfseries 14029:} $123$, \textsf{\bfseries 14033:} $12$, \textsf{\bfseries 14051:} $30$, \textsf{\bfseries 14057:} $21$, \textsf{\bfseries 14071:} $23$, \textsf{\bfseries 14081:} $38$, \textsf{\bfseries 14083:} $18$, \textsf{\bfseries 14087:} $17$, \textsf{\bfseries 14107:} $12$, \textsf{\bfseries 14143:} $54$, \textsf{\bfseries 14149:} $70$, \textsf{\bfseries 14153:} $96$, \textsf{\bfseries 14159:} $130$, \textsf{\bfseries 14173:} $6$, \textsf{\bfseries 14177:} $24$, \textsf{\bfseries 14197:} $42$, \textsf{\bfseries 14207:} $26$, \textsf{\bfseries 14221:} $130$, \textsf{\bfseries 14243:} $67$, \textsf{\bfseries 14249:} $210$, \textsf{\bfseries 14251:} $3$, \textsf{\bfseries 14281:} $82$, \textsf{\bfseries 14293:} $15$, \textsf{\bfseries 14303:} $21$, \textsf{\bfseries 14321:} $41$, \textsf{\bfseries 14323:} $107$, \textsf{\bfseries 14327:} $37$, \textsf{\bfseries 14341:} $46$, \textsf{\bfseries 14347:} $151$, \textsf{\bfseries 14369:} $66$, \textsf{\bfseries 14387:} $43$, \textsf{\bfseries 14389:} $40$, \textsf{\bfseries 14401:} $55$, \textsf{\bfseries 14407:} $94$, \textsf{\bfseries 14411:} $6$, \textsf{\bfseries 14419:} $18$, \textsf{\bfseries 14423:} $5$, \textsf{\bfseries 14431:} $159$, \textsf{\bfseries 14437:} $264$, \textsf{\bfseries 14447:} $15$, \textsf{\bfseries 14449:} $209$, \textsf{\bfseries 14461:} $7$, \textsf{\bfseries 14479:} $3$, \textsf{\bfseries 14489:} $99$, \textsf{\bfseries 14503:} $35$, \textsf{\bfseries 14519:} $74$, \textsf{\bfseries 14533:} $174$, 
\textsf{\bfseries 14537:} $12$, \textsf{\bfseries 14543:} $21$, \textsf{\bfseries 14549:} $28$, \textsf{\bfseries 14551:} $22$, \textsf{\bfseries 14557:} $128$, \textsf{\bfseries 14561:} $149$, \textsf{\bfseries 14563:} $22$, \textsf{\bfseries 14591:} $39$, \textsf{\bfseries 14593:} $37$, \textsf{\bfseries 14621:} $44$, \textsf{\bfseries 14627:} $22$, \textsf{\bfseries 14629:} $79$, \textsf{\bfseries 14633:} $123$, \textsf{\bfseries 14639:} $84$, \textsf{\bfseries 14653:} $45$, \textsf{\bfseries 14657:} $148$, \textsf{\bfseries 14669:} $98$, \textsf{\bfseries 14683:} $28$, \textsf{\bfseries 14699:} $32$, \textsf{\bfseries 14713:} $53$, \textsf{\bfseries 14717:} $11$, \textsf{\bfseries 14723:} $19$, \textsf{\bfseries 14731:} $52$, \textsf{\bfseries 14737:} $173$, \textsf{\bfseries 14741:} $7$, \textsf{\bfseries 14747:} $32$, \textsf{\bfseries 14753:} $5$, \textsf{\bfseries 14759:} $47$, \textsf{\bfseries 14767:} $6$, \textsf{\bfseries 14771:} $30$, \textsf{\bfseries 14779:} $35$, \textsf{\bfseries 14783:} $74$, \textsf{\bfseries 14797:} $7$, \textsf{\bfseries 14813:} $99$, \textsf{\bfseries 14821:} $6$, \textsf{\bfseries 14827:} $28$, \textsf{\bfseries 14831:} $13$, \textsf{\bfseries 14843:} $85$, \textsf{\bfseries 14851:} $112$, \textsf{\bfseries 14867:} $8$, \textsf{\bfseries 14869:} $50$, \textsf{\bfseries 14879:} $29$, \textsf{\bfseries 14887:} $124$, \textsf{\bfseries 14891:} $87$, \textsf{\bfseries 14897:} $43$, \textsf{\bfseries 14923:} $226$, \textsf{\bfseries 14929:} $296$, \textsf{\bfseries 14939:} $52$, \textsf{\bfseries 14947:} $45$, \textsf{\bfseries 14951:} $83$, \textsf{\bfseries 14957:} $27$, \textsf{\bfseries 14969:} $6$, \textsf{\bfseries 14983:} $29$, \textsf{\bfseries 15013:} $58$, \textsf{\bfseries 15017:} $3$, \textsf{\bfseries 15031:} $136$, \textsf{\bfseries 15053:} $59$, \textsf{\bfseries 15061:} $90$, \textsf{\bfseries 15073:} $15$, \textsf{\bfseries 15077:} $110$, \textsf{\bfseries 15083:} $19$, \textsf{\bfseries 15091:} $85$, \textsf{\bfseries 15101:} $15$, \textsf{\bfseries 15107:} $17$, \textsf{\bfseries 15121:} $43$, \textsf{\bfseries 15131:} $63$, \textsf{\bfseries 15137:} $17$, \textsf{\bfseries 15139:} $15$, \textsf{\bfseries 15149:} $71$, \textsf{\bfseries 15161:} $17$, \textsf{\bfseries 15173:} $14$, \textsf{\bfseries 15187:} $69$, \textsf{\bfseries 15193:} $20$, \textsf{\bfseries 15199:} $328$, \textsf{\bfseries 15217:} $39$, \textsf{\bfseries 15227:} $94$, \textsf{\bfseries 15233:} $61$, \textsf{\bfseries 15241:} $163$, \textsf{\bfseries 15259:} $124$, \textsf{\bfseries 15263:} $133$, \textsf{\bfseries 15269:} $53$, \textsf{\bfseries 15271:} $39$, \textsf{\bfseries 15277:} $18$, \textsf{\bfseries 15287:} $222$, \textsf{\bfseries 15289:} $55$, \textsf{\bfseries 15299:} $102$, \textsf{\bfseries 15307:} $62$, \textsf{\bfseries 15313:} $180$, \textsf{\bfseries 15319:} $23$, \textsf{\bfseries 15329:} $62$, \textsf{\bfseries 15331:} $35$, \textsf{\bfseries 15349:} $28$, \textsf{\bfseries 15359:} $35$, \textsf{\bfseries 15361:} $13$, \textsf{\bfseries 15373:} $6$, \textsf{\bfseries 15377:} $6$, \textsf{\bfseries 15383:} $15$, \textsf{\bfseries 15391:} $79$, \textsf{\bfseries 15401:} $111$, \textsf{\bfseries 15413:} $57$, 
\textsf{\bfseries 15427:} $12$, \textsf{\bfseries 15439:} $287$, \textsf{\bfseries 15443:} $17$, \textsf{\bfseries 15451:} $42$, \textsf{\bfseries 15461:} $26$, \textsf{\bfseries 15467:} $6$, \textsf{\bfseries 15473:} $11$, \textsf{\bfseries 15493:} $82$, \textsf{\bfseries 15497:} $34$, \textsf{\bfseries 15511:} $41$, \textsf{\bfseries 15527:} $76$, \textsf{\bfseries 15541:} $54$, \textsf{\bfseries 15551:} $23$, \textsf{\bfseries 15559:} $6$, \textsf{\bfseries 15569:} $12$, \textsf{\bfseries 15581:} $22$, \textsf{\bfseries 15583:} $54$, \textsf{\bfseries 15601:} $35$, \textsf{\bfseries 15607:} $26$, \textsf{\bfseries 15619:} $60$, \textsf{\bfseries 15629:} $73$, \textsf{\bfseries 15641:} $142$, \textsf{\bfseries 15643:} $19$, \textsf{\bfseries 15647:} $21$, \textsf{\bfseries 15649:} $166$, \textsf{\bfseries 15661:} $30$, \textsf{\bfseries 15667:} $154$, \textsf{\bfseries 15671:} $89$, \textsf{\bfseries 15679:} $99$, \textsf{\bfseries 15683:} $15$, \textsf{\bfseries 15727:} $41$, \textsf{\bfseries 15731:} $31$, \textsf{\bfseries 15733:} $38$, \textsf{\bfseries 15737:} $44$, \textsf{\bfseries 15739:} $126$, \textsf{\bfseries 15749:} $62$, \textsf{\bfseries 15761:} $13$, \textsf{\bfseries 15767:} $46$, \textsf{\bfseries 15773:} $98$, \textsf{\bfseries 15787:} $3$, \textsf{\bfseries 15791:} $259$, \textsf{\bfseries 15797:} $53$, \textsf{\bfseries 15803:} $20$, \textsf{\bfseries 15809:} $120$, \textsf{\bfseries 15817:} $51$, \textsf{\bfseries 15823:} $3$, \textsf{\bfseries 15859:} $94$, \textsf{\bfseries 15877:} $24$, \textsf{\bfseries 15881:} $6$, \textsf{\bfseries 15887:} $139$, \textsf{\bfseries 15889:} $61$, \textsf{\bfseries 15901:} $42$, \textsf{\bfseries 15907:} $211$, \textsf{\bfseries 15913:} $205$, \textsf{\bfseries 15919:} $17$, \textsf{\bfseries 15923:} $124$, \textsf{\bfseries 15937:} $7$, \textsf{\bfseries 15959:} $62$, \textsf{\bfseries 15971:} $10$, \textsf{\bfseries 15973:} $41$, \textsf{\bfseries 15991:} $15$, \textsf{\bfseries 16001:} $27$, \textsf{\bfseries 16007:} $37$, \textsf{\bfseries 16033:} $90$, \textsf{\bfseries 16057:} $129$, \textsf{\bfseries 16061:} $152$, \textsf{\bfseries 16063:} $79$, \textsf{\bfseries 16067:} $6$, \textsf{\bfseries 16069:} $66$, \textsf{\bfseries 16073:} $24$, \textsf{\bfseries 16087:} $96$, \textsf{\bfseries 16091:} $19$, \textsf{\bfseries 16097:} $20$, \textsf{\bfseries 16103:} $46$, \textsf{\bfseries 16111:} $187$, \textsf{\bfseries 16127:} $10$, \textsf{\bfseries 16139:} $63$, \textsf{\bfseries 16141:} $105$, \textsf{\bfseries 16183:} $3$, \textsf{\bfseries 16187:} $230$, \textsf{\bfseries 16189:} $73$, \textsf{\bfseries 16193:} $206$, \textsf{\bfseries 16217:} $75$, \textsf{\bfseries 16223:} $45$, \textsf{\bfseries 16229:} $41$, \textsf{\bfseries 16231:} $89$, \textsf{\bfseries 16249:} $191$, \textsf{\bfseries 16253:} $42$, \textsf{\bfseries 16267:} $48$, \textsf{\bfseries 16273:} $14$, \textsf{\bfseries 16301:} $72$, \textsf{\bfseries 16319:} $493$, \textsf{\bfseries 16333:} $66$, \textsf{\bfseries 16339:} $35$, \textsf{\bfseries 16349:} $22$, \textsf{\bfseries 16361:} $89$, \textsf{\bfseries 16363:} $53$, \textsf{\bfseries 16369:} $104$, \textsf{\bfseries 16381:} $94$, \textsf{\bfseries 16411:} $190$, 
\textsf{\bfseries 16417:} $330$, \textsf{\bfseries 16421:} $118$, \textsf{\bfseries 16427:} $39$, \textsf{\bfseries 16433:} $29$, \textsf{\bfseries 16447:} $57$, \textsf{\bfseries 16451:} $19$, \textsf{\bfseries 16453:} $5$, \textsf{\bfseries 16477:} $11$, \textsf{\bfseries 16481:} $87$, \textsf{\bfseries 16487:} $34$, \textsf{\bfseries 16493:} $47$, \textsf{\bfseries 16519:} $146$, \textsf{\bfseries 16529:} $62$, \textsf{\bfseries 16547:} $51$, \textsf{\bfseries 16553:} $10$, \textsf{\bfseries 16561:} $99$, \textsf{\bfseries 16567:} $235$, \textsf{\bfseries 16573:} $47$, \textsf{\bfseries 16603:} $52$, \textsf{\bfseries 16607:} $40$, \textsf{\bfseries 16619:} $53$, \textsf{\bfseries 16631:} $123$, \textsf{\bfseries 16633:} $637$, \textsf{\bfseries 16649:} $114$, \textsf{\bfseries 16651:} $14$, \textsf{\bfseries 16657:} $19$, \textsf{\bfseries 16661:} $59$, \textsf{\bfseries 16673:} $86$, \textsf{\bfseries 16691:} $17$, \textsf{\bfseries 16693:} $15$, \textsf{\bfseries 16699:} $42$, \textsf{\bfseries 16703:} $145$, \textsf{\bfseries 16729:} $42$, \textsf{\bfseries 16741:} $40$, \textsf{\bfseries 16747:} $34$, \textsf{\bfseries 16759:} $374$, \textsf{\bfseries 16763:} $55$, \textsf{\bfseries 16787:} $33$, \textsf{\bfseries 16811:} $102$, \textsf{\bfseries 16823:} $197$, \textsf{\bfseries 16829:} $87$, \textsf{\bfseries 16831:} $59$, \textsf{\bfseries 16843:} $28$, \textsf{\bfseries 16871:} $138$, \textsf{\bfseries 16879:} $43$, \textsf{\bfseries 16883:} $187$, \textsf{\bfseries 16889:} $24$, \textsf{\bfseries 16901:} $7$, \textsf{\bfseries 16903:} $59$, \textsf{\bfseries 16921:} $233$, \textsf{\bfseries 16927:} $22$, \textsf{\bfseries 16931:} $31$, \textsf{\bfseries 16937:} $21$, \textsf{\bfseries 16943:} $34$, \textsf{\bfseries 16963:} $37$, \textsf{\bfseries 16979:} $73$, \textsf{\bfseries 16981:} $78$, \textsf{\bfseries 16987:} $98$, \textsf{\bfseries 16993:} $67$, \textsf{\bfseries 17011:} $78$, \textsf{\bfseries 17021:} $70$, \textsf{\bfseries 17027:} $50$, \textsf{\bfseries 17029:} $68$, \textsf{\bfseries 17033:} $10$, \textsf{\bfseries 17041:} $22$, \textsf{\bfseries 17047:} $40$, \textsf{\bfseries 17053:} $32$, \textsf{\bfseries 17077:} $57$, \textsf{\bfseries 17093:} $72$, \textsf{\bfseries 17099:} $85$, \textsf{\bfseries 17107:} $20$, \textsf{\bfseries 17117:} $57$, \textsf{\bfseries 17123:} $6$, \textsf{\bfseries 17137:} $57$, \textsf{\bfseries 17159:} $168$, \textsf{\bfseries 17167:} $168$, \textsf{\bfseries 17183:} $111$, \textsf{\bfseries 17189:} $76$, \textsf{\bfseries 17191:} $6$, \textsf{\bfseries 17203:} $20$, \textsf{\bfseries 17207:} $106$, \textsf{\bfseries 17209:} $61$, \textsf{\bfseries 17231:} $267$, \textsf{\bfseries 17239:} $94$, \textsf{\bfseries 17257:} $15$, \textsf{\bfseries 17291:} $40$, \textsf{\bfseries 17293:} $137$, \textsf{\bfseries 17299:} $153$, \textsf{\bfseries 17317:} $58$, \textsf{\bfseries 17321:} $63$, \textsf{\bfseries 17327:} $86$, \textsf{\bfseries 17333:} $87$, \textsf{\bfseries 17341:} $73$, \textsf{\bfseries 17351:} $47$, \textsf{\bfseries 17359:} $6$, \textsf{\bfseries 17377:} $127$, \textsf{\bfseries 17383:} $165$, \textsf{\bfseries 17387:} $50$, \textsf{\bfseries 17389:} $66$, \textsf{\bfseries 17393:} $56$, 
 
                \textsf{\bfseries 17401:} $13$, \textsf{\bfseries 17417:} $40$, \textsf{\bfseries 17419:} $10$, \textsf{\bfseries 17431:} $68$, \textsf{\bfseries 17443:} $12$, \textsf{\bfseries 17449:} $74$, \textsf{\bfseries 17467:} $83$, \textsf{\bfseries 17471:} $178$, \textsf{\bfseries 17477:} $43$, \textsf{\bfseries 17483:} $146$, \textsf{\bfseries 17489:} $27$, \textsf{\bfseries 17491:} $12$, \textsf{\bfseries 17497:} $116$, \textsf{\bfseries 17509:} $58$, \textsf{\bfseries 17519:} $41$, \textsf{\bfseries 17539:} $152$, \textsf{\bfseries 17551:} $7$, \textsf{\bfseries 17569:} $23$, \textsf{\bfseries 17573:} $105$, \textsf{\bfseries 17579:} $7$, \textsf{\bfseries 17581:} $101$, \textsf{\bfseries 17597:} $5$, \textsf{\bfseries 17599:} $59$, \textsf{\bfseries 17609:} $207$, \textsf{\bfseries 17623:} $112$, \textsf{\bfseries 17627:} $20$, \textsf{\bfseries 17657:} $5$, \textsf{\bfseries 17659:} $31$, \textsf{\bfseries 17669:} $240$, \textsf{\bfseries 17681:} $30$, \textsf{\bfseries 17683:} $43$, \textsf{\bfseries 17707:} $17$, \textsf{\bfseries 17713:} $30$, \textsf{\bfseries 17729:} $19$, \textsf{\bfseries 17737:} $52$, \textsf{\bfseries 17747:} $199$, \textsf{\bfseries 17749:} $128$, \textsf{\bfseries 17761:} $23$, \textsf{\bfseries 17783:} $141$, \textsf{\bfseries 17789:} $12$, \textsf{\bfseries 17791:} $83$, \textsf{\bfseries 17807:} $123$, \textsf{\bfseries 17827:} $3$, \textsf{\bfseries 17837:} $279$, \textsf{\bfseries 17839:} $6$, \textsf{\bfseries 17851:} $13$, \textsf{\bfseries 17863:} $80$, \textsf{\bfseries 17881:} $189$, \textsf{\bfseries 17891:} $24$, \textsf{\bfseries 17903:} $30$, \textsf{\bfseries 17909:} $32$, \textsf{\bfseries 17911:} $24$, \textsf{\bfseries 17921:} $27$, \textsf{\bfseries 17923:} $50$, \textsf{\bfseries 17929:} $248$, \textsf{\bfseries 17939:} $62$, \textsf{\bfseries 17957:} $381$, \textsf{\bfseries 17959:} $30$, \textsf{\bfseries 17971:} $15$, \textsf{\bfseries 17977:} $20$, \textsf{\bfseries 17981:} $50$, \textsf{\bfseries 17987:} $143$, \textsf{\bfseries 17989:} $217$, \textsf{\bfseries 18013:} $24$, \textsf{\bfseries 18041:} $129$, \textsf{\bfseries 18043:} $105$, \textsf{\bfseries 18047:} $14$, \textsf{\bfseries 18049:} $59$, \textsf{\bfseries 18059:} $17$, \textsf{\bfseries 18061:} $11$, \textsf{\bfseries 18077:} $33$, \textsf{\bfseries 18089:} $24$, \textsf{\bfseries 18097:} $35$, \textsf{\bfseries 18119:} $61$, \textsf{\bfseries 18121:} $124$, \textsf{\bfseries 18127:} $3$, \textsf{\bfseries 18131:} $248$, \textsf{\bfseries 18133:} $114$, \textsf{\bfseries 18143:} $10$, \textsf{\bfseries 18149:} $15$, \textsf{\bfseries 18169:} $29$, \textsf{\bfseries 18181:} $2$, \textsf{\bfseries 18191:} $214$, \textsf{\bfseries 18199:} $42$, \textsf{\bfseries 18211:} $95$, \textsf{\bfseries 18217:} $388$, \textsf{\bfseries 18223:} $237$, \textsf{\bfseries 18229:} $258$, \textsf{\bfseries 18233:} $46$, \textsf{\bfseries 18251:} $78$, \textsf{\bfseries 18253:} $239$, \textsf{\bfseries 18257:} $26$, \textsf{\bfseries 18269:} $7$, \textsf{\bfseries 18287:} $206$, \textsf{\bfseries 18289:} $39$, \textsf{\bfseries 18301:} $21$, \textsf{\bfseries 18307:} $146$, \textsf{\bfseries 18311:} $83$, \textsf{\bfseries 18313:} $82$, \textsf{\bfseries 18329:} $107$, 
\textsf{\bfseries 18341:} $115$, \textsf{\bfseries 18353:} $159$, \textsf{\bfseries 18367:} $40$, \textsf{\bfseries 18371:} $52$, \textsf{\bfseries 18379:} $18$, \textsf{\bfseries 18397:} $14$, \textsf{\bfseries 18401:} $14$, \textsf{\bfseries 18413:} $53$, \textsf{\bfseries 18427:} $82$, \textsf{\bfseries 18433:} $40$, \textsf{\bfseries 18439:} $6$, \textsf{\bfseries 18443:} $55$, \textsf{\bfseries 18451:} $11$, \textsf{\bfseries 18457:} $67$, \textsf{\bfseries 18461:} $127$, \textsf{\bfseries 18481:} $104$, \textsf{\bfseries 18493:} $2$, \textsf{\bfseries 18503:} $99$, \textsf{\bfseries 18517:} $88$, \textsf{\bfseries 18521:} $88$, \textsf{\bfseries 18523:} $105$, \textsf{\bfseries 18539:} $10$, \textsf{\bfseries 18541:} $98$, \textsf{\bfseries 18553:} $84$, \textsf{\bfseries 18583:} $12$, \textsf{\bfseries 18587:} $23$, \textsf{\bfseries 18593:} $55$, \textsf{\bfseries 18617:} $45$, \textsf{\bfseries 18637:} $59$, \textsf{\bfseries 18661:} $13$, \textsf{\bfseries 18671:} $28$, \textsf{\bfseries 18679:} $48$, \textsf{\bfseries 18691:} $62$, \textsf{\bfseries 18701:} $3$, \textsf{\bfseries 18713:} $3$, \textsf{\bfseries 18719:} $177$, \textsf{\bfseries 18731:} $47$, \textsf{\bfseries 18743:} $55$, \textsf{\bfseries 18749:} $33$, \textsf{\bfseries 18757:} $60$, \textsf{\bfseries 18773:} $72$, \textsf{\bfseries 18787:} $69$, \textsf{\bfseries 18793:} $53$, \textsf{\bfseries 18797:} $21$, \textsf{\bfseries 18803:} $245$, \textsf{\bfseries 18839:} $37$, \textsf{\bfseries 18859:} $142$, \textsf{\bfseries 18869:} $12$, \textsf{\bfseries 18899:} $11$, \textsf{\bfseries 18911:} $14$, \textsf{\bfseries 18913:} $111$, \textsf{\bfseries 18917:} $20$, \textsf{\bfseries 18919:} $129$, \textsf{\bfseries 18947:} $62$, \textsf{\bfseries 18959:} $157$, \textsf{\bfseries 18973:} $39$, \textsf{\bfseries 18979:} $76$, \textsf{\bfseries 19001:} $467$, \textsf{\bfseries 19009:} $87$, \textsf{\bfseries 19013:} $45$, \textsf{\bfseries 19031:} $44$, \textsf{\bfseries 19037:} $13$, \textsf{\bfseries 19051:} $41$, \textsf{\bfseries 19069:} $85$, \textsf{\bfseries 19073:} $6$, \textsf{\bfseries 19079:} $89$, \textsf{\bfseries 19081:} $47$, \textsf{\bfseries 19087:} $35$, \textsf{\bfseries 19121:} $61$, \textsf{\bfseries 19139:} $79$, \textsf{\bfseries 19141:} $52$, \textsf{\bfseries 19157:} $32$, \textsf{\bfseries 19163:} $26$, \textsf{\bfseries 19181:} $19$, \textsf{\bfseries 19183:} $13$, \textsf{\bfseries 19207:} $62$, \textsf{\bfseries 19211:} $92$, \textsf{\bfseries 19213:} $20$, \textsf{\bfseries 19219:} $228$, \textsf{\bfseries 19231:} $23$, \textsf{\bfseries 19237:} $32$, \textsf{\bfseries 19249:} $51$, \textsf{\bfseries 19259:} $87$, \textsf{\bfseries 19267:} $108$, \textsf{\bfseries 19273:} $10$, \textsf{\bfseries 19289:} $3$, \textsf{\bfseries 19301:} $12$, \textsf{\bfseries 19309:} $96$, \textsf{\bfseries 19319:} $33$, \textsf{\bfseries 19333:} $122$, \textsf{\bfseries 19373:} $143$, \textsf{\bfseries 19379:} $94$, \textsf{\bfseries 19381:} $106$, \textsf{\bfseries 19387:} $18$, \textsf{\bfseries 19391:} $61$, \textsf{\bfseries 19403:} $8$, \textsf{\bfseries 19417:} $26$, \textsf{\bfseries 19421:} $399$, \textsf{\bfseries 19423:} $42$, \textsf{\bfseries 19427:} $309$, 
\textsf{\bfseries 19429:} $14$, \textsf{\bfseries 19433:} $26$, \textsf{\bfseries 19441:} $59$, \textsf{\bfseries 19447:} $29$, \textsf{\bfseries 19457:} $216$, \textsf{\bfseries 19463:} $60$, \textsf{\bfseries 19469:} $13$, \textsf{\bfseries 19471:} $30$, \textsf{\bfseries 19477:} $24$, \textsf{\bfseries 19483:} $42$, \textsf{\bfseries 19489:} $34$, \textsf{\bfseries 19501:} $133$, \textsf{\bfseries 19507:} $11$, \textsf{\bfseries 19531:} $73$, \textsf{\bfseries 19541:} $48$, \textsf{\bfseries 19543:} $180$, \textsf{\bfseries 19553:} $89$, \textsf{\bfseries 19559:} $73$, \textsf{\bfseries 19571:} $31$, \textsf{\bfseries 19577:} $11$, \textsf{\bfseries 19583:} $14$, \textsf{\bfseries 19597:} $46$, \textsf{\bfseries 19603:} $85$, \textsf{\bfseries 19609:} $44$, \textsf{\bfseries 19661:} $107$, \textsf{\bfseries 19681:} $33$, \textsf{\bfseries 19687:} $104$, \textsf{\bfseries 19697:} $20$, \textsf{\bfseries 19699:} $26$, \textsf{\bfseries 19709:} $11$, \textsf{\bfseries 19717:} $18$, \textsf{\bfseries 19727:} $19$, \textsf{\bfseries 19739:} $53$, \textsf{\bfseries 19751:} $89$, \textsf{\bfseries 19753:} $62$, \textsf{\bfseries 19759:} $79$, \textsf{\bfseries 19763:} $55$, \textsf{\bfseries 19777:} $111$, \textsf{\bfseries 19793:} $54$, \textsf{\bfseries 19801:} $62$, \textsf{\bfseries 19813:} $50$, \textsf{\bfseries 19819:} $89$, \textsf{\bfseries 19841:} $48$, \textsf{\bfseries 19843:} $269$, \textsf{\bfseries 19853:} $180$, \textsf{\bfseries 19861:} $31$, \textsf{\bfseries 19867:} $38$, \textsf{\bfseries 19889:} $60$, \textsf{\bfseries 19891:} $76$, \textsf{\bfseries 19913:} $33$, \textsf{\bfseries 19919:} $57$, \textsf{\bfseries 19927:} $46$, \textsf{\bfseries 19937:} $101$, \textsf{\bfseries 19949:} $42$, \textsf{\bfseries 19961:} $12$, \textsf{\bfseries 19963:} $52$, \textsf{\bfseries 19973:} $8$, \textsf{\bfseries 19979:} $33$, \textsf{\bfseries 19991:} $11$, \textsf{\bfseries 19993:} $124$, \textsf{\bfseries 19997:} $145$, \textsf{\bfseries 20011:} $84$, \textsf{\bfseries 20021:} $18$, \textsf{\bfseries 20023:} $54$, \textsf{\bfseries 20029:} $32$, \textsf{\bfseries 20047:} $96$, \textsf{\bfseries 20051:} $10$, \textsf{\bfseries 20063:} $123$, \textsf{\bfseries 20071:} $17$, \textsf{\bfseries 20089:} $43$, \textsf{\bfseries 20101:} $10$, \textsf{\bfseries 20107:} $132$, \textsf{\bfseries 20113:} $43$, \textsf{\bfseries 20117:} $18$, \textsf{\bfseries 20123:} $37$, \textsf{\bfseries 20129:} $52$, \textsf{\bfseries 20143:} $55$, \textsf{\bfseries 20147:} $19$, \textsf{\bfseries 20149:} $101$, \textsf{\bfseries 20161:} $13$, \textsf{\bfseries 20173:} $78$, \textsf{\bfseries 20177:} $80$, \textsf{\bfseries 20183:} $26$, \textsf{\bfseries 20201:} $41$, \textsf{\bfseries 20219:} $33$, \textsf{\bfseries 20231:} $70$, \textsf{\bfseries 20233:} $127$, \textsf{\bfseries 20249:} $73$, \textsf{\bfseries 20261:} $18$, \textsf{\bfseries 20269:} $11$, \textsf{\bfseries 20287:} $88$, \textsf{\bfseries 20297:} $183$, \textsf{\bfseries 20323:} $26$, \textsf{\bfseries 20327:} $89$, \textsf{\bfseries 20333:} $93$, \textsf{\bfseries 20341:} $40$, \textsf{\bfseries 20347:} $69$, \textsf{\bfseries 20353:} $5$, \textsf{\bfseries 20357:} $61$, \textsf{\bfseries 20359:} $21$, 
\textsf{\bfseries 20369:} $7$, \textsf{\bfseries 20389:} $124$, \textsf{\bfseries 20393:} $136$, \textsf{\bfseries 20399:} $13$, \textsf{\bfseries 20407:} $10$, \textsf{\bfseries 20411:} $187$, \textsf{\bfseries 20431:} $21$, \textsf{\bfseries 20441:} $21$, \textsf{\bfseries 20443:} $98$, \textsf{\bfseries 20477:} $14$, \textsf{\bfseries 20479:} $34$, \textsf{\bfseries 20483:} $15$, \textsf{\bfseries 20507:} $54$, \textsf{\bfseries 20509:} $195$, \textsf{\bfseries 20521:} $26$, \textsf{\bfseries 20533:} $33$, \textsf{\bfseries 20543:} $74$, \textsf{\bfseries 20549:} $157$, \textsf{\bfseries 20551:} $26$, \textsf{\bfseries 20563:} $12$, \textsf{\bfseries 20593:} $139$, \textsf{\bfseries 20599:} $127$, \textsf{\bfseries 20611:} $59$, \textsf{\bfseries 20627:} $35$, \textsf{\bfseries 20639:} $101$, \textsf{\bfseries 20641:} $113$, \textsf{\bfseries 20663:} $74$, \textsf{\bfseries 20681:} $228$, \textsf{\bfseries 20693:} $66$, \textsf{\bfseries 20707:} $62$, \textsf{\bfseries 20717:} $13$, \textsf{\bfseries 20719:} $24$, \textsf{\bfseries 20731:} $26$ 
  \item[$r=2$:] \input{./tables/pcns_12_2__0.tex}
  \item[$r=3$:] \input{./tables/pcns_12_3__0.tex}
  \item[$r=4$:] \input{./tables/pcns_12_4__0.tex}
  \item[$r=5$:] \input{./tables/pcns_12_5__0.tex}
  \item[$r=6$:] \input{./tables/pcns_12_6__0.tex}
  \item[$r=7$:] \input{./tables/pcns_12_7__0.tex}
  %\item[$r=8$:] \textsf{\bfseries 2:} 0:\,$a^7$,\ 1:\,$1$,\ 9:\,$1$
  %\item[$r=9$:] \input{./tables/pcns_10_9__0.tex}
\end{description}


\subsection{$\PCN$-Polynome für $n=14$}

\begin{description}[leftmargin=0pt,labelindent=20pt,
  font=\normalsize]
  \item[$r=1$:] \input{./tables/pcns_14_1__0.tex}
                \textsf{\bfseries 7933:} $42$, \textsf{\bfseries 7937:} $47$, \textsf{\bfseries 7949:} $18$, \textsf{\bfseries 7951:} $19$, \textsf{\bfseries 7963:} $28$, \textsf{\bfseries 7993:} $52$, \textsf{\bfseries 8009:} $84$, \textsf{\bfseries 8011:} $56$, \textsf{\bfseries 8017:} $10$, \textsf{\bfseries 8039:} $47$, \textsf{\bfseries 8053:} $96$, \textsf{\bfseries 8059:} $130$, \textsf{\bfseries 8069:} $153$, \textsf{\bfseries 8081:} $69$, \textsf{\bfseries 8087:} $107$, \textsf{\bfseries 8089:} $111$, \textsf{\bfseries 8093:} $11$, \textsf{\bfseries 8101:} $37$, \textsf{\bfseries 8111:} $159$, \textsf{\bfseries 8117:} $29$, \textsf{\bfseries 8123:} $31$, \textsf{\bfseries 8147:} $89$, \textsf{\bfseries 8161:} $197$, \textsf{\bfseries 8167:} $94$, \textsf{\bfseries 8171:} $43$, \textsf{\bfseries 8179:} $2$, \textsf{\bfseries 8191:} $118$, \textsf{\bfseries 8209:} $37$, \textsf{\bfseries 8219:} $26$, \textsf{\bfseries 8221:} $47$, \textsf{\bfseries 8231:} $208$, \textsf{\bfseries 8233:} $30$, \textsf{\bfseries 8237:} $31$, \textsf{\bfseries 8243:} $44$, \textsf{\bfseries 8263:} $41$, \textsf{\bfseries 8269:} $86$, \textsf{\bfseries 8273:} $106$, \textsf{\bfseries 8287:} $102$, \textsf{\bfseries 8291:} $22$, \textsf{\bfseries 8293:} $273$, \textsf{\bfseries 8297:} $55$, \textsf{\bfseries 8311:} $12$, \textsf{\bfseries 8317:} $232$, \textsf{\bfseries 8329:} $79$, \textsf{\bfseries 8353:} $68$, \textsf{\bfseries 8363:} $24$, \textsf{\bfseries 8369:} $6$, \textsf{\bfseries 8377:} $22$, \textsf{\bfseries 8387:} $50$, \textsf{\bfseries 8389:} $47$, \textsf{\bfseries 8419:} $31$, \textsf{\bfseries 8423:} $73$, \textsf{\bfseries 8429:} $14$, \textsf{\bfseries 8431:} $159$, \textsf{\bfseries 8443:} $2$, \textsf{\bfseries 8447:} $276$, \textsf{\bfseries 8461:} $31$, \textsf{\bfseries 8467:} $104$, \textsf{\bfseries 8501:} $10$, \textsf{\bfseries 8513:} $10$, \textsf{\bfseries 8521:} $23$, \textsf{\bfseries 8527:} $96$, \textsf{\bfseries 8537:} $17$, \textsf{\bfseries 8539:} $41$, \textsf{\bfseries 8543:} $47$, \textsf{\bfseries 8563:} $118$, \textsf{\bfseries 8573:} $2$, \textsf{\bfseries 8581:} $6$, \textsf{\bfseries 8597:} $59$, \textsf{\bfseries 8599:} $3$, \textsf{\bfseries 8609:} $28$, \textsf{\bfseries 8623:} $40$, \textsf{\bfseries 8627:} $33$, \textsf{\bfseries 8629:} $63$, \textsf{\bfseries 8641:} $102$, \textsf{\bfseries 8647:} $13$, \textsf{\bfseries 8663:} $29$, \textsf{\bfseries 8669:} $34$, \textsf{\bfseries 8677:} $35$, \textsf{\bfseries 8681:} $77$, \textsf{\bfseries 8689:} $141$, \textsf{\bfseries 8693:} $42$, \textsf{\bfseries 8699:} $8$, \textsf{\bfseries 8707:} $71$, \textsf{\bfseries 8713:} $60$, \textsf{\bfseries 8719:} $30$, \textsf{\bfseries 8731:} $53$, \textsf{\bfseries 8737:} $305$, \textsf{\bfseries 8741:} $68$, \textsf{\bfseries 8747:} $2$, \textsf{\bfseries 8753:} $6$, \textsf{\bfseries 8761:} $37$, \textsf{\bfseries 8779:} $86$, \textsf{\bfseries 8783:} $11$, \textsf{\bfseries 8803:} $42$, \textsf{\bfseries 8807:} $95$, \textsf{\bfseries 8819:} $103$, \textsf{\bfseries 8821:} $94$, \textsf{\bfseries 8831:} $87$, \textsf{\bfseries 8837:} $32$, 
\textsf{\bfseries 8839:} $38$, \textsf{\bfseries 8849:} $21$, \textsf{\bfseries 8861:} $42$, \textsf{\bfseries 8863:} $24$, \textsf{\bfseries 8867:} $2$, \textsf{\bfseries 8887:} $102$, \textsf{\bfseries 8893:} $22$, \textsf{\bfseries 8923:} $5$, \textsf{\bfseries 8929:} $114$, \textsf{\bfseries 8933:} $20$, \textsf{\bfseries 8941:} $89$, \textsf{\bfseries 8951:} $71$, \textsf{\bfseries 8963:} $11$, \textsf{\bfseries 8969:} $48$, \textsf{\bfseries 8971:} $28$, \textsf{\bfseries 8999:} $7$, \textsf{\bfseries 9001:} $193$, \textsf{\bfseries 9007:} $96$, \textsf{\bfseries 9011:} $37$, \textsf{\bfseries 9013:} $88$, \textsf{\bfseries 9029:} $69$, \textsf{\bfseries 9041:} $114$, \textsf{\bfseries 9043:} $3$, \textsf{\bfseries 9049:} $7$, \textsf{\bfseries 9059:} $28$, \textsf{\bfseries 9067:} $30$, \textsf{\bfseries 9091:} $19$, \textsf{\bfseries 9103:} $23$, \textsf{\bfseries 9109:} $87$, \textsf{\bfseries 9127:} $123$, \textsf{\bfseries 9133:} $13$, \textsf{\bfseries 9137:} $217$, \textsf{\bfseries 9151:} $12$, \textsf{\bfseries 9157:} $31$, \textsf{\bfseries 9161:} $77$, \textsf{\bfseries 9173:} $17$, \textsf{\bfseries 9181:} $73$, \textsf{\bfseries 9187:} $3$, \textsf{\bfseries 9199:} $138$, \textsf{\bfseries 9203:} $51$, \textsf{\bfseries 9209:} $52$, \textsf{\bfseries 9221:} $119$, \textsf{\bfseries 9227:} $15$, \textsf{\bfseries 9239:} $38$, \textsf{\bfseries 9241:} $79$, \textsf{\bfseries 9257:} $12$, \textsf{\bfseries 9277:} $13$, \textsf{\bfseries 9281:} $30$, \textsf{\bfseries 9283:} $3$, \textsf{\bfseries 9293:} $2$, \textsf{\bfseries 9311:} $91$, \textsf{\bfseries 9319:} $56$, \textsf{\bfseries 9323:} $23$, \textsf{\bfseries 9337:} $47$, \textsf{\bfseries 9341:} $7$, \textsf{\bfseries 9343:} $118$, \textsf{\bfseries 9349:} $161$, \textsf{\bfseries 9371:} $249$, \textsf{\bfseries 9377:} $41$, \textsf{\bfseries 9391:} $3$, \textsf{\bfseries 9397:} $22$, \textsf{\bfseries 9403:} $75$, \textsf{\bfseries 9413:} $69$, \textsf{\bfseries 9419:} $128$, \textsf{\bfseries 9421:} $2$, \textsf{\bfseries 9431:} $78$, \textsf{\bfseries 9433:} $85$, \textsf{\bfseries 9437:} $67$, \textsf{\bfseries 9439:} $99$, \textsf{\bfseries 9461:} $12$, \textsf{\bfseries 9463:} $42$, \textsf{\bfseries 9467:} $98$, \textsf{\bfseries 9473:} $23$, \textsf{\bfseries 9479:} $35$, \textsf{\bfseries 9491:} $101$, \textsf{\bfseries 9497:} $77$, \textsf{\bfseries 9511:} $19$, \textsf{\bfseries 9521:} $3$, \textsf{\bfseries 9533:} $45$, \textsf{\bfseries 9539:} $50$, \textsf{\bfseries 9547:} $14$, \textsf{\bfseries 9551:} $97$, \textsf{\bfseries 9587:} $43$, \textsf{\bfseries 9601:} $118$, \textsf{\bfseries 9613:} $15$, \textsf{\bfseries 9619:} $407$, \textsf{\bfseries 9623:} $19$, \textsf{\bfseries 9629:} $10$, \textsf{\bfseries 9631:} $48$, \textsf{\bfseries 9643:} $5$, \textsf{\bfseries 9649:} $175$, \textsf{\bfseries 9661:} $141$, \textsf{\bfseries 9677:} $28$, \textsf{\bfseries 9679:} $107$, \textsf{\bfseries 9689:} $101$, \textsf{\bfseries 9697:} $77$, \textsf{\bfseries 9719:} $172$, \textsf{\bfseries 9721:} $14$, \textsf{\bfseries 9733:} $86$, \textsf{\bfseries 9739:} $52$, 
\textsf{\bfseries 9743:} $35$, \textsf{\bfseries 9749:} $132$, \textsf{\bfseries 9767:} $173$, \textsf{\bfseries 9769:} $26$, \textsf{\bfseries 9781:} $106$, \textsf{\bfseries 9787:} $116$, \textsf{\bfseries 9791:} $99$, \textsf{\bfseries 9803:} $8$, \textsf{\bfseries 9811:} $26$, \textsf{\bfseries 9817:} $46$, \textsf{\bfseries 9829:} $73$, \textsf{\bfseries 9833:} $26$, \textsf{\bfseries 9839:} $19$, \textsf{\bfseries 9851:} $18$, \textsf{\bfseries 9857:} $45$, \textsf{\bfseries 9859:} $37$, \textsf{\bfseries 9871:} $37$, \textsf{\bfseries 9883:} $3$, \textsf{\bfseries 9887:} $70$, \textsf{\bfseries 9901:} $50$, \textsf{\bfseries 9907:} $29$, \textsf{\bfseries 9923:} $6$, \textsf{\bfseries 9929:} $35$, \textsf{\bfseries 9931:} $495$, \textsf{\bfseries 9941:} $18$, \textsf{\bfseries 9949:} $179$, \textsf{\bfseries 9967:} $46$, \textsf{\bfseries 9973:} $87$, \textsf{\bfseries 10007:} $37$, \textsf{\bfseries 10009:} $42$, \textsf{\bfseries 10037:} $17$, \textsf{\bfseries 10039:} $47$, \textsf{\bfseries 10061:} $72$, \textsf{\bfseries 10067:} $24$, \textsf{\bfseries 10069:} $41$, \textsf{\bfseries 10079:} $282$, \textsf{\bfseries 10091:} $158$, \textsf{\bfseries 10093:} $21$, \textsf{\bfseries 10099:} $40$, \textsf{\bfseries 10103:} $112$, \textsf{\bfseries 10111:} $31$, \textsf{\bfseries 10133:} $11$, \textsf{\bfseries 10139:} $54$, \textsf{\bfseries 10141:} $102$, \textsf{\bfseries 10151:} $39$, \textsf{\bfseries 10159:} $68$, \textsf{\bfseries 10163:} $15$, \textsf{\bfseries 10169:} $120$, \textsf{\bfseries 10177:} $26$, \textsf{\bfseries 10181:} $2$, \textsf{\bfseries 10193:} $3$, \textsf{\bfseries 10211:} $17$, \textsf{\bfseries 10223:} $19$, \textsf{\bfseries 10243:} $7$, \textsf{\bfseries 10247:} $120$, \textsf{\bfseries 10253:} $50$, \textsf{\bfseries 10259:} $28$, \textsf{\bfseries 10267:} $162$, \textsf{\bfseries 10271:} $14$, \textsf{\bfseries 10273:} $34$, \textsf{\bfseries 10289:} $43$, \textsf{\bfseries 10301:} $22$, \textsf{\bfseries 10303:} $48$, \textsf{\bfseries 10313:} $198$, \textsf{\bfseries 10321:} $23$, \textsf{\bfseries 10331:} $42$, \textsf{\bfseries 10333:} $5$, \textsf{\bfseries 10337:} $11$, \textsf{\bfseries 10343:} $10$, \textsf{\bfseries 10357:} $54$, \textsf{\bfseries 10369:} $55$, \textsf{\bfseries 10391:} $161$, \textsf{\bfseries 10399:} $19$, \textsf{\bfseries 10427:} $77$, \textsf{\bfseries 10429:} $98$, \textsf{\bfseries 10433:} $12$, \textsf{\bfseries 10453:} $141$, \textsf{\bfseries 10457:} $44$, \textsf{\bfseries 10459:} $111$, \textsf{\bfseries 10463:} $113$, \textsf{\bfseries 10477:} $76$, \textsf{\bfseries 10487:} $44$, \textsf{\bfseries 10499:} $62$, \textsf{\bfseries 10501:} $107$, \textsf{\bfseries 10513:} $136$, \textsf{\bfseries 10529:} $73$, \textsf{\bfseries 10531:} $21$, \textsf{\bfseries 10559:} $29$, \textsf{\bfseries 10567:} $7$, \textsf{\bfseries 10589:} $18$, \textsf{\bfseries 10597:} $187$, \textsf{\bfseries 10601:} $11$, \textsf{\bfseries 10607:} $31$, \textsf{\bfseries 10613:} $68$, \textsf{\bfseries 10627:} $58$, \textsf{\bfseries 10631:} $163$, \textsf{\bfseries 10639:} $42$, \textsf{\bfseries 10651:} $29$, \textsf{\bfseries 10657:} $56$, \textsf{\bfseries 10663:} $19$, 
\textsf{\bfseries 10667:} $5$, \textsf{\bfseries 10687:} $45$, \textsf{\bfseries 10691:} $122$, \textsf{\bfseries 10709:} $145$, \textsf{\bfseries 10711:} $119$, \textsf{\bfseries 10723:} $35$, \textsf{\bfseries 10729:} $53$, \textsf{\bfseries 10733:} $44$, \textsf{\bfseries 10739:} $10$, \textsf{\bfseries 10753:} $208$, \textsf{\bfseries 10771:} $118$, \textsf{\bfseries 10781:} $162$, \textsf{\bfseries 10789:} $73$, \textsf{\bfseries 10799:} $86$, \textsf{\bfseries 10831:} $119$, \textsf{\bfseries 10837:} $52$, \textsf{\bfseries 10847:} $5$, \textsf{\bfseries 10853:} $18$, \textsf{\bfseries 10859:} $8$, \textsf{\bfseries 10861:} $13$, \textsf{\bfseries 10867:} $174$, \textsf{\bfseries 10883:} $23$, \textsf{\bfseries 10889:} $3$, \textsf{\bfseries 10891:} $60$, \textsf{\bfseries 10903:} $3$, \textsf{\bfseries 10909:} $54$, \textsf{\bfseries 10937:} $63$, \textsf{\bfseries 10939:} $65$, \textsf{\bfseries 10949:} $101$, \textsf{\bfseries 10957:} $80$, \textsf{\bfseries 10973:} $71$, \textsf{\bfseries 10979:} $11$, \textsf{\bfseries 10987:} $13$, \textsf{\bfseries 10993:} $56$, \textsf{\bfseries 11003:} $11$, \textsf{\bfseries 11027:} $60$, \textsf{\bfseries 11047:} $44$, \textsf{\bfseries 11057:} $67$, \textsf{\bfseries 11059:} $29$, \textsf{\bfseries 11069:} $15$, \textsf{\bfseries 11071:} $46$, \textsf{\bfseries 11083:} $78$, \textsf{\bfseries 11087:} $83$, \textsf{\bfseries 11093:} $29$, \textsf{\bfseries 11113:} $43$, \textsf{\bfseries 11117:} $19$, \textsf{\bfseries 11119:} $54$, \textsf{\bfseries 11131:} $90$, \textsf{\bfseries 11149:} $13$, \textsf{\bfseries 11159:} $82$, \textsf{\bfseries 11161:} $118$, \textsf{\bfseries 11171:} $50$, \textsf{\bfseries 11173:} $191$, \textsf{\bfseries 11177:} $31$, \textsf{\bfseries 11197:} $2$, \textsf{\bfseries 11213:} $32$, \textsf{\bfseries 11239:} $3$, \textsf{\bfseries 11243:} $138$, \textsf{\bfseries 11251:} $13$, \textsf{\bfseries 11257:} $35$, \textsf{\bfseries 11261:} $11$, \textsf{\bfseries 11273:} $10$, \textsf{\bfseries 11279:} $65$, \textsf{\bfseries 11287:} $38$, \textsf{\bfseries 11299:} $15$, \textsf{\bfseries 11311:} $136$, \textsf{\bfseries 11317:} $2$, \textsf{\bfseries 11321:} $54$, \textsf{\bfseries 11329:} $170$, \textsf{\bfseries 11351:} $14$, \textsf{\bfseries 11353:} $23$, \textsf{\bfseries 11369:} $223$, \textsf{\bfseries 11383:} $147$, \textsf{\bfseries 11393:} $19$, \textsf{\bfseries 11399:} $88$, \textsf{\bfseries 11411:} $21$, \textsf{\bfseries 11423:} $58$, \textsf{\bfseries 11437:} $80$, \textsf{\bfseries 11443:} $53$, \textsf{\bfseries 11447:} $17$, \textsf{\bfseries 11467:} $163$, \textsf{\bfseries 11471:} $26$, \textsf{\bfseries 11483:} $83$, \textsf{\bfseries 11489:} $93$, \textsf{\bfseries 11491:} $163$, \textsf{\bfseries 11497:} $7$, \textsf{\bfseries 11503:} $3$, \textsf{\bfseries 11519:} $212$, \textsf{\bfseries 11527:} $33$, \textsf{\bfseries 11549:} $3$, \textsf{\bfseries 11551:} $7$, \textsf{\bfseries 11579:} $30$, \textsf{\bfseries 11587:} $3$, \textsf{\bfseries 11593:} $66$, \textsf{\bfseries 11597:} $50$, \textsf{\bfseries 11617:} $17$, \textsf{\bfseries 11621:} $76$, \textsf{\bfseries 11633:} $191$, \textsf{\bfseries 11657:} $35$, \textsf{\bfseries 11677:} $134$, 
\textsf{\bfseries 11681:} $149$, \textsf{\bfseries 11689:} $47$, \textsf{\bfseries 11699:} $50$, \textsf{\bfseries 11701:} $24$, \textsf{\bfseries 11717:} $118$, \textsf{\bfseries 11719:} $23$, \textsf{\bfseries 11731:} $14$, \textsf{\bfseries 11743:} $46$, \textsf{\bfseries 11777:} $12$, \textsf{\bfseries 11779:} $51$, \textsf{\bfseries 11783:} $14$, \textsf{\bfseries 11789:} $12$, \textsf{\bfseries 11801:} $41$, \textsf{\bfseries 11807:} $140$, \textsf{\bfseries 11813:} $32$, \textsf{\bfseries 11821:} $107$, \textsf{\bfseries 11827:} $59$, \textsf{\bfseries 11831:} $124$, \textsf{\bfseries 11833:} $99$, \textsf{\bfseries 11839:} $7$, \textsf{\bfseries 11863:} $89$, \textsf{\bfseries 11867:} $20$, \textsf{\bfseries 11887:} $58$, \textsf{\bfseries 11897:} $5$, \textsf{\bfseries 11903:} $30$, \textsf{\bfseries 11909:} $135$, \textsf{\bfseries 11923:} $30$, \textsf{\bfseries 11927:} $22$, \textsf{\bfseries 11933:} $23$, \textsf{\bfseries 11939:} $23$, \textsf{\bfseries 11941:} $11$, \textsf{\bfseries 11953:} $30$, \textsf{\bfseries 11959:} $51$, \textsf{\bfseries 11969:} $3$, \textsf{\bfseries 11971:} $141$, \textsf{\bfseries 11981:} $21$, \textsf{\bfseries 11987:} $18$, \textsf{\bfseries 12007:} $43$, \textsf{\bfseries 12011:} $70$, \textsf{\bfseries 12037:} $18$, \textsf{\bfseries 12041:} $54$, \textsf{\bfseries 12043:} $53$, \textsf{\bfseries 12049:} $69$, \textsf{\bfseries 12071:} $167$, \textsf{\bfseries 12073:} $30$, \textsf{\bfseries 12097:} $51$, \textsf{\bfseries 12101:} $31$, \textsf{\bfseries 12107:} $8$, \textsf{\bfseries 12109:} $158$, \textsf{\bfseries 12113:} $14$, \textsf{\bfseries 12119:} $95$, \textsf{\bfseries 12143:} $90$, \textsf{\bfseries 12149:} $27$, \textsf{\bfseries 12157:} $34$, \textsf{\bfseries 12161:} $29$, \textsf{\bfseries 12163:} $52$, \textsf{\bfseries 12197:} $7$, \textsf{\bfseries 12203:} $32$, \textsf{\bfseries 12211:} $14$, \textsf{\bfseries 12227:} $15$, \textsf{\bfseries 12239:} $92$, \textsf{\bfseries 12241:} $53$, \textsf{\bfseries 12251:} $270$, \textsf{\bfseries 12253:} $71$, \textsf{\bfseries 12263:} $120$, \textsf{\bfseries 12269:} $123$, \textsf{\bfseries 12277:} $57$, \textsf{\bfseries 12281:} $7$, \textsf{\bfseries 12289:} $66$, \textsf{\bfseries 12301:} $2$, \textsf{\bfseries 12323:} $162$, \textsf{\bfseries 12329:} $6$, \textsf{\bfseries 12343:} $28$, \textsf{\bfseries 12347:} $31$, \textsf{\bfseries 12373:} $23$, \textsf{\bfseries 12377:} $6$, \textsf{\bfseries 12379:} $2$, \textsf{\bfseries 12391:} $255$, \textsf{\bfseries 12401:} $113$, \textsf{\bfseries 12409:} $7$, \textsf{\bfseries 12413:} $45$, \textsf{\bfseries 12421:} $138$, \textsf{\bfseries 12433:} $90$, \textsf{\bfseries 12437:} $48$, \textsf{\bfseries 12451:} $103$, \textsf{\bfseries 12457:} $161$, \textsf{\bfseries 12473:} $274$, \textsf{\bfseries 12479:} $71$, \textsf{\bfseries 12487:} $54$, \textsf{\bfseries 12491:} $96$, \textsf{\bfseries 12497:} $42$, \textsf{\bfseries 12503:} $31$, \textsf{\bfseries 12511:} $94$, \textsf{\bfseries 12517:} $11$, \textsf{\bfseries 12527:} $59$, \textsf{\bfseries 12539:} $6$, \textsf{\bfseries 12541:} $72$, \textsf{\bfseries 12547:} $19$, \textsf{\bfseries 12553:} $39$, \textsf{\bfseries 12569:} $76$, 
\textsf{\bfseries 12577:} $80$, \textsf{\bfseries 12583:} $38$, \textsf{\bfseries 12589:} $47$, \textsf{\bfseries 12601:} $122$, \textsf{\bfseries 12611:} $38$, \textsf{\bfseries 12613:} $5$, \textsf{\bfseries 12619:} $60$, \textsf{\bfseries 12637:} $66$, \textsf{\bfseries 12641:} $24$, \textsf{\bfseries 12647:} $35$, \textsf{\bfseries 12653:} $67$, \textsf{\bfseries 12659:} $54$, \textsf{\bfseries 12671:} $57$, \textsf{\bfseries 12689:} $28$, \textsf{\bfseries 12697:} $322$, \textsf{\bfseries 12703:} $12$, \textsf{\bfseries 12713:} $92$, \textsf{\bfseries 12721:} $17$, \textsf{\bfseries 12739:} $191$, \textsf{\bfseries 12743:} $30$, \textsf{\bfseries 12757:} $15$, \textsf{\bfseries 12763:} $32$, \textsf{\bfseries 12781:} $46$, \textsf{\bfseries 12791:} $28$, \textsf{\bfseries 12799:} $141$, \textsf{\bfseries 12809:} $126$, \textsf{\bfseries 12821:} $12$, \textsf{\bfseries 12823:} $10$, \textsf{\bfseries 12829:} $31$, \textsf{\bfseries 12841:} $170$, \textsf{\bfseries 12853:} $58$, \textsf{\bfseries 12889:} $68$, \textsf{\bfseries 12893:} $37$, \textsf{\bfseries 12899:} $126$, \textsf{\bfseries 12907:} $56$, \textsf{\bfseries 12911:} $26$, \textsf{\bfseries 12917:} $33$, \textsf{\bfseries 12919:} $56$, \textsf{\bfseries 12923:} $87$, \textsf{\bfseries 12941:} $114$, \textsf{\bfseries 12953:} $29$, \textsf{\bfseries 12959:} $26$, \textsf{\bfseries 12967:} $20$, \textsf{\bfseries 12973:} $56$, \textsf{\bfseries 12979:} $51$, \textsf{\bfseries 12983:} $106$, \textsf{\bfseries 13001:} $105$, \textsf{\bfseries 13003:} $20$, \textsf{\bfseries 13007:} $90$, \textsf{\bfseries 13009:} $41$, \textsf{\bfseries 13033:} $126$, \textsf{\bfseries 13037:} $5$, \textsf{\bfseries 13043:} $34$, \textsf{\bfseries 13049:} $12$, \textsf{\bfseries 13063:} $109$, \textsf{\bfseries 13093:} $226$, \textsf{\bfseries 13099:} $62$, \textsf{\bfseries 13103:} $69$, \textsf{\bfseries 13109:} $10$, \textsf{\bfseries 13121:} $7$, \textsf{\bfseries 13127:} $41$, \textsf{\bfseries 13147:} $151$, \textsf{\bfseries 13151:} $134$, \textsf{\bfseries 13159:} $21$, \textsf{\bfseries 13163:} $50$, \textsf{\bfseries 13171:} $52$, \textsf{\bfseries 13177:} $5$, \textsf{\bfseries 13183:} $156$, \textsf{\bfseries 13187:} $106$, \textsf{\bfseries 13217:} $11$, \textsf{\bfseries 13219:} $33$, \textsf{\bfseries 13229:} $8$, \textsf{\bfseries 13241:} $27$, \textsf{\bfseries 13249:} $118$, \textsf{\bfseries 13259:} $24$, \textsf{\bfseries 13267:} $68$, \textsf{\bfseries 13291:} $277$, \textsf{\bfseries 13297:} $41$, \textsf{\bfseries 13309:} $6$, \textsf{\bfseries 13313:} $56$, \textsf{\bfseries 13327:} $95$, \textsf{\bfseries 13331:} $202$, \textsf{\bfseries 13337:} $73$, \textsf{\bfseries 13339:} $91$, \textsf{\bfseries 13367:} $217$, \textsf{\bfseries 13381:} $50$, \textsf{\bfseries 13397:} $32$, \textsf{\bfseries 13399:} $126$, \textsf{\bfseries 13411:} $48$, \textsf{\bfseries 13417:} $109$, \textsf{\bfseries 13421:} $78$, \textsf{\bfseries 13441:} $29$, \textsf{\bfseries 13451:} $11$, \textsf{\bfseries 13457:} $20$, \textsf{\bfseries 13463:} $26$, \textsf{\bfseries 13469:} $102$, \textsf{\bfseries 13477:} $2$, \textsf{\bfseries 13487:} $39$, \textsf{\bfseries 13499:} $14$, \textsf{\bfseries 13513:} $15$, 
\textsf{\bfseries 13523:} $137$, \textsf{\bfseries 13537:} $91$, \textsf{\bfseries 13553:} $51$, \textsf{\bfseries 13567:} $44$, \textsf{\bfseries 13577:} $57$, \textsf{\bfseries 13591:} $3$, \textsf{\bfseries 13597:} $86$, \textsf{\bfseries 13613:} $67$, \textsf{\bfseries 13619:} $32$, \textsf{\bfseries 13627:} $17$, \textsf{\bfseries 13633:} $53$, \textsf{\bfseries 13649:} $79$, \textsf{\bfseries 13669:} $18$, \textsf{\bfseries 13679:} $141$, \textsf{\bfseries 13681:} $52$, \textsf{\bfseries 13687:} $31$, \textsf{\bfseries 13691:} $179$, \textsf{\bfseries 13693:} $6$, \textsf{\bfseries 13697:} $13$, \textsf{\bfseries 13709:} $48$, \textsf{\bfseries 13711:} $114$, \textsf{\bfseries 13721:} $127$, \textsf{\bfseries 13723:} $80$, \textsf{\bfseries 13729:} $68$, \textsf{\bfseries 13751:} $67$, \textsf{\bfseries 13757:} $110$, \textsf{\bfseries 13759:} $83$, \textsf{\bfseries 13763:} $50$, \textsf{\bfseries 13781:} $50$, \textsf{\bfseries 13789:} $237$, \textsf{\bfseries 13799:} $84$, \textsf{\bfseries 13807:} $40$, \textsf{\bfseries 13829:} $11$, \textsf{\bfseries 13831:} $85$, \textsf{\bfseries 13841:} $103$, \textsf{\bfseries 13859:} $236$, \textsf{\bfseries 13873:} $19$, \textsf{\bfseries 13877:} $86$, \textsf{\bfseries 13879:} $123$, \textsf{\bfseries 13883:} $5$, \textsf{\bfseries 13901:} $160$, \textsf{\bfseries 13903:} $146$, \textsf{\bfseries 13907:} $128$, \textsf{\bfseries 13913:} $34$, \textsf{\bfseries 13921:} $211$, \textsf{\bfseries 13931:} $67$, \textsf{\bfseries 13933:} $5$, \textsf{\bfseries 13963:} $5$, \textsf{\bfseries 13967:} $58$, \textsf{\bfseries 13997:} $68$, \textsf{\bfseries 13999:} $86$, \textsf{\bfseries 14009:} $15$, \textsf{\bfseries 14011:} $39$, \textsf{\bfseries 14029:} $41$, \textsf{\bfseries 14033:} $20$, \textsf{\bfseries 14051:} $54$, \textsf{\bfseries 14057:} $27$, \textsf{\bfseries 14071:} $65$, \textsf{\bfseries 14081:} $48$, \textsf{\bfseries 14083:} $48$, \textsf{\bfseries 14087:} $80$, \textsf{\bfseries 14107:} $17$, \textsf{\bfseries 14143:} $3$, \textsf{\bfseries 14149:} $113$, \textsf{\bfseries 14153:} $95$, \textsf{\bfseries 14159:} $113$, \textsf{\bfseries 14173:} $112$, \textsf{\bfseries 14177:} $133$, \textsf{\bfseries 14197:} $197$, \textsf{\bfseries 14207:} $21$, \textsf{\bfseries 14221:} $136$, \textsf{\bfseries 14243:} $33$, \textsf{\bfseries 14249:} $17$, \textsf{\bfseries 14251:} $86$, \textsf{\bfseries 14281:} $78$, \textsf{\bfseries 14293:} $162$, \textsf{\bfseries 14303:} $270$, \textsf{\bfseries 14321:} $124$, \textsf{\bfseries 14323:} $12$, \textsf{\bfseries 14327:} $5$, \textsf{\bfseries 14341:} $21$, \textsf{\bfseries 14347:} $28$, \textsf{\bfseries 14369:} $27$, \textsf{\bfseries 14387:} $23$, \textsf{\bfseries 14389:} $42$, \textsf{\bfseries 14401:} $76$, \textsf{\bfseries 14407:} $97$, \textsf{\bfseries 14411:} $51$, \textsf{\bfseries 14419:} $70$, \textsf{\bfseries 14423:} $30$, \textsf{\bfseries 14431:} $119$, \textsf{\bfseries 14437:} $70$, \textsf{\bfseries 14447:} $74$, \textsf{\bfseries 14449:} $117$, \textsf{\bfseries 14461:} $40$, \textsf{\bfseries 14479:} $6$, \textsf{\bfseries 14489:} $6$, \textsf{\bfseries 14503:} $70$, \textsf{\bfseries 14519:} $70$, \textsf{\bfseries 14533:} $89$, 
\textsf{\bfseries 14537:} $28$, \textsf{\bfseries 14543:} $29$, \textsf{\bfseries 14549:} $83$, \textsf{\bfseries 14551:} $95$, \textsf{\bfseries 14557:} $56$, \textsf{\bfseries 14561:} $15$, \textsf{\bfseries 14563:} $21$, \textsf{\bfseries 14591:} $68$, \textsf{\bfseries 14593:} $124$, \textsf{\bfseries 14621:} $86$, \textsf{\bfseries 14627:} $19$, \textsf{\bfseries 14629:} $10$, \textsf{\bfseries 14633:} $134$, \textsf{\bfseries 14639:} $35$, \textsf{\bfseries 14653:} $20$, \textsf{\bfseries 14657:} $27$, \textsf{\bfseries 14669:} $58$, \textsf{\bfseries 14683:} $38$, \textsf{\bfseries 14699:} $71$, \textsf{\bfseries 14713:} $43$, \textsf{\bfseries 14717:} $42$, \textsf{\bfseries 14723:} $52$, \textsf{\bfseries 14731:} $87$, \textsf{\bfseries 14737:} $80$, \textsf{\bfseries 14741:} $135$, \textsf{\bfseries 14747:} $50$, \textsf{\bfseries 14753:} $80$, \textsf{\bfseries 14759:} $116$, \textsf{\bfseries 14767:} $24$, \textsf{\bfseries 14771:} $6$, \textsf{\bfseries 14779:} $143$, \textsf{\bfseries 14783:} $35$, \textsf{\bfseries 14797:} $5$, \textsf{\bfseries 14813:} $75$, \textsf{\bfseries 14821:} $6$, \textsf{\bfseries 14827:} $97$, \textsf{\bfseries 14831:} $11$, \textsf{\bfseries 14843:} $26$, \textsf{\bfseries 14851:} $52$, \textsf{\bfseries 14867:} $183$, \textsf{\bfseries 14869:} $14$, \textsf{\bfseries 14879:} $119$, \textsf{\bfseries 14887:} $44$, \textsf{\bfseries 14891:} $8$, \textsf{\bfseries 14897:} $174$, \textsf{\bfseries 14923:} $111$, \textsf{\bfseries 14929:} $74$, \textsf{\bfseries 14939:} $33$, \textsf{\bfseries 14947:} $30$, \textsf{\bfseries 14951:} $46$, \textsf{\bfseries 14957:} $48$, \textsf{\bfseries 14969:} $26$, \textsf{\bfseries 14983:} $24$, \textsf{\bfseries 15013:} $6$, \textsf{\bfseries 15017:} $5$, \textsf{\bfseries 15031:} $24$, \textsf{\bfseries 15053:} $12$, \textsf{\bfseries 15061:} $44$, \textsf{\bfseries 15073:} $71$, \textsf{\bfseries 15077:} $59$, \textsf{\bfseries 15083:} $190$, \textsf{\bfseries 15091:} $48$, \textsf{\bfseries 15101:} $154$, \textsf{\bfseries 15107:} $94$, \textsf{\bfseries 15121:} $22$, \textsf{\bfseries 15131:} $2$, \textsf{\bfseries 15137:} $12$, \textsf{\bfseries 15139:} $62$, \textsf{\bfseries 15149:} $11$, \textsf{\bfseries 15161:} $83$, \textsf{\bfseries 15173:} $22$, \textsf{\bfseries 15187:} $28$, \textsf{\bfseries 15193:} $22$, \textsf{\bfseries 15199:} $7$, \textsf{\bfseries 15217:} $61$, \textsf{\bfseries 15227:} $60$, \textsf{\bfseries 15233:} $45$, \textsf{\bfseries 15241:} $19$, \textsf{\bfseries 15259:} $163$, \textsf{\bfseries 15263:} $74$, \textsf{\bfseries 15269:} $8$, \textsf{\bfseries 15271:} $71$, \textsf{\bfseries 15277:} $58$, \textsf{\bfseries 15287:} $111$, \textsf{\bfseries 15289:} $113$, \textsf{\bfseries 15299:} $28$, \textsf{\bfseries 15307:} $68$, \textsf{\bfseries 15313:} $74$, \textsf{\bfseries 15319:} $117$, \textsf{\bfseries 15329:} $14$, \textsf{\bfseries 15331:} $63$, \textsf{\bfseries 15349:} $53$, \textsf{\bfseries 15359:} $46$, \textsf{\bfseries 15361:} $52$, \textsf{\bfseries 15373:} $122$, \textsf{\bfseries 15377:} $147$, \textsf{\bfseries 15383:} $22$, \textsf{\bfseries 15391:} $157$, \textsf{\bfseries 15401:} $12$, \textsf{\bfseries 15413:} $83$, 
\textsf{\bfseries 15427:} $71$, \textsf{\bfseries 15439:} $29$, \textsf{\bfseries 15443:} $38$, \textsf{\bfseries 15451:} $58$, \textsf{\bfseries 15461:} $18$, \textsf{\bfseries 15467:} $60$, \textsf{\bfseries 15473:} $6$, \textsf{\bfseries 15493:} $188$, \textsf{\bfseries 15497:} $180$, \textsf{\bfseries 15511:} $13$, \textsf{\bfseries 15527:} $15$, \textsf{\bfseries 15541:} $54$, \textsf{\bfseries 15551:} $56$, \textsf{\bfseries 15559:} $86$, \textsf{\bfseries 15569:} $15$, \textsf{\bfseries 15581:} $22$, \textsf{\bfseries 15583:} $101$, \textsf{\bfseries 15601:} $23$, \textsf{\bfseries 15607:} $48$, \textsf{\bfseries 15619:} $23$, \textsf{\bfseries 15629:} $12$, \textsf{\bfseries 15641:} $13$, \textsf{\bfseries 15643:} $37$, \textsf{\bfseries 15647:} $14$, \textsf{\bfseries 15649:} $166$, \textsf{\bfseries 15661:} $53$, \textsf{\bfseries 15667:} $32$, \textsf{\bfseries 15671:} $151$, \textsf{\bfseries 15679:} $116$, \textsf{\bfseries 15683:} $2$, \textsf{\bfseries 15727:} $57$, \textsf{\bfseries 15731:} $6$, \textsf{\bfseries 15733:} $109$, \textsf{\bfseries 15737:} $107$, \textsf{\bfseries 15739:} $14$, \textsf{\bfseries 15749:} $37$, \textsf{\bfseries 15761:} $232$, \textsf{\bfseries 15767:} $60$, \textsf{\bfseries 15773:} $110$, \textsf{\bfseries 15787:} $52$, \textsf{\bfseries 15791:} $212$, \textsf{\bfseries 15797:} $164$, \textsf{\bfseries 15803:} $24$, \textsf{\bfseries 15809:} $91$, \textsf{\bfseries 15817:} $116$, \textsf{\bfseries 15823:} $69$, \textsf{\bfseries 15859:} $32$, \textsf{\bfseries 15877:} $24$, \textsf{\bfseries 15881:} $99$, \textsf{\bfseries 15887:} $47$, \textsf{\bfseries 15889:} $84$, \textsf{\bfseries 15901:} $141$, \textsf{\bfseries 15907:} $46$, \textsf{\bfseries 15913:} $60$, \textsf{\bfseries 15919:} $13$, \textsf{\bfseries 15923:} $134$, \textsf{\bfseries 15937:} $57$, \textsf{\bfseries 15959:} $65$, \textsf{\bfseries 15971:} $233$, \textsf{\bfseries 15973:} $78$, \textsf{\bfseries 15991:} $15$, \textsf{\bfseries 16001:} $127$, \textsf{\bfseries 16007:} $5$, \textsf{\bfseries 16033:} $28$, \textsf{\bfseries 16057:} $129$, \textsf{\bfseries 16061:} $75$, \textsf{\bfseries 16063:} $26$, \textsf{\bfseries 16067:} $58$, \textsf{\bfseries 16069:} $6$, \textsf{\bfseries 16073:} $24$, \textsf{\bfseries 16087:} $45$, \textsf{\bfseries 16091:} $188$, \textsf{\bfseries 16097:} $46$, \textsf{\bfseries 16103:} $23$, \textsf{\bfseries 16111:} $136$, \textsf{\bfseries 16127:} $92$, \textsf{\bfseries 16139:} $10$, \textsf{\bfseries 16141:} $86$, \textsf{\bfseries 16183:} $45$, \textsf{\bfseries 16187:} $22$, \textsf{\bfseries 16189:} $37$, \textsf{\bfseries 16193:} $26$, \textsf{\bfseries 16217:} $55$, \textsf{\bfseries 16223:} $37$, \textsf{\bfseries 16229:} $90$, \textsf{\bfseries 16231:} $46$, \textsf{\bfseries 16249:} $111$, \textsf{\bfseries 16253:} $76$, \textsf{\bfseries 16267:} $20$, \textsf{\bfseries 16273:} $7$, \textsf{\bfseries 16301:} $97$, \textsf{\bfseries 16319:} $152$, \textsf{\bfseries 16333:} $52$, \textsf{\bfseries 16339:} $57$, \textsf{\bfseries 16349:} $91$, \textsf{\bfseries 16361:} $95$, \textsf{\bfseries 16363:} $28$, \textsf{\bfseries 16369:} $76$, \textsf{\bfseries 16381:} $11$, \textsf{\bfseries 16411:} $198$, 
\textsf{\bfseries 16417:} $78$, \textsf{\bfseries 16421:} $50$, \textsf{\bfseries 16427:} $47$, \textsf{\bfseries 16433:} $46$, \textsf{\bfseries 16447:} $3$, \textsf{\bfseries 16451:} $188$, \textsf{\bfseries 16453:} $17$, \textsf{\bfseries 16477:} $11$, \textsf{\bfseries 16481:} $94$, \textsf{\bfseries 16487:} $7$, \textsf{\bfseries 16493:} $95$, \textsf{\bfseries 16519:} $30$, \textsf{\bfseries 16529:} $88$, \textsf{\bfseries 16547:} $50$, \textsf{\bfseries 16553:} $10$, \textsf{\bfseries 16561:} $28$, \textsf{\bfseries 16567:} $3$, \textsf{\bfseries 16573:} $54$, \textsf{\bfseries 16603:} $45$, \textsf{\bfseries 16607:} $182$, \textsf{\bfseries 16619:} $40$, \textsf{\bfseries 16631:} $131$, \textsf{\bfseries 16633:} $123$, \textsf{\bfseries 16649:} $68$, \textsf{\bfseries 16651:} $123$, \textsf{\bfseries 16657:} $41$, \textsf{\bfseries 16661:} $283$, \textsf{\bfseries 16673:} $26$, \textsf{\bfseries 16691:} $19$, \textsf{\bfseries 16693:} $34$, \textsf{\bfseries 16699:} $60$, \textsf{\bfseries 16703:} $61$, \textsf{\bfseries 16729:} $134$, \textsf{\bfseries 16741:} $24$, \textsf{\bfseries 16747:} $29$, \textsf{\bfseries 16759:} $62$, \textsf{\bfseries 16763:} $35$, \textsf{\bfseries 16787:} $99$, \textsf{\bfseries 16811:} $31$, \textsf{\bfseries 16823:} $73$, \textsf{\bfseries 16829:} $12$, \textsf{\bfseries 16831:} $232$, \textsf{\bfseries 16843:} $72$, \textsf{\bfseries 16871:} $17$, \textsf{\bfseries 16879:} $75$, \textsf{\bfseries 16883:} $176$, \textsf{\bfseries 16889:} $165$, \textsf{\bfseries 16901:} $2$, \textsf{\bfseries 16903:} $86$, \textsf{\bfseries 16921:} $41$, \textsf{\bfseries 16927:} $14$, \textsf{\bfseries 16931:} $23$, \textsf{\bfseries 16937:} $22$, \textsf{\bfseries 16943:} $5$, \textsf{\bfseries 16963:} $30$, \textsf{\bfseries 16979:} $216$, \textsf{\bfseries 16981:} $93$, \textsf{\bfseries 16987:} $44$, \textsf{\bfseries 16993:} $51$, \textsf{\bfseries 17011:} $226$, \textsf{\bfseries 17021:} $69$, \textsf{\bfseries 17027:} $18$, \textsf{\bfseries 17029:} $10$, \textsf{\bfseries 17033:} $21$, \textsf{\bfseries 17041:} $85$, \textsf{\bfseries 17047:} $12$, \textsf{\bfseries 17053:} $32$, \textsf{\bfseries 17077:} $2$, \textsf{\bfseries 17093:} $29$, \textsf{\bfseries 17099:} $46$, \textsf{\bfseries 17107:} $12$, \textsf{\bfseries 17117:} $38$, \textsf{\bfseries 17123:} $8$, \textsf{\bfseries 17137:} $77$, \textsf{\bfseries 17159:} $21$, \textsf{\bfseries 17167:} $3$, \textsf{\bfseries 17183:} $22$, \textsf{\bfseries 17189:} $51$, \textsf{\bfseries 17191:} $77$, \textsf{\bfseries 17203:} $30$, \textsf{\bfseries 17207:} $237$, \textsf{\bfseries 17209:} $23$, \textsf{\bfseries 17231:} $13$, \textsf{\bfseries 17239:} $161$, \textsf{\bfseries 17257:} $78$, \textsf{\bfseries 17291:} $200$, \textsf{\bfseries 17293:} $98$, \textsf{\bfseries 17299:} $109$, \textsf{\bfseries 17317:} $135$, \textsf{\bfseries 17321:} $82$, \textsf{\bfseries 17327:} $232$, \textsf{\bfseries 17333:} $2$, \textsf{\bfseries 17341:} $198$, \textsf{\bfseries 17351:} $110$, \textsf{\bfseries 17359:} $192$, \textsf{\bfseries 17377:} $41$, \textsf{\bfseries 17383:} $26$, \textsf{\bfseries 17387:} $24$, \textsf{\bfseries 17389:} $24$, \textsf{\bfseries 17393:} $44$, 
 
                \textsf{\bfseries 17401:} $11$, \textsf{\bfseries 17417:} $264$, \textsf{\bfseries 17419:} $187$, \textsf{\bfseries 17431:} $59$, \textsf{\bfseries 17443:} $140$, \textsf{\bfseries 17449:} $91$, \textsf{\bfseries 17467:} $75$, \textsf{\bfseries 17471:} $211$, \textsf{\bfseries 17477:} $32$, \textsf{\bfseries 17483:} $70$, \textsf{\bfseries 17489:} $114$, \textsf{\bfseries 17491:} $12$, \textsf{\bfseries 17497:} $5$, \textsf{\bfseries 17509:} $30$, \textsf{\bfseries 17519:} $76$, \textsf{\bfseries 17539:} $23$, \textsf{\bfseries 17551:} $6$, \textsf{\bfseries 17569:} $57$, \textsf{\bfseries 17573:} $26$, \textsf{\bfseries 17579:} $93$, \textsf{\bfseries 17581:} $50$, \textsf{\bfseries 17597:} $97$, \textsf{\bfseries 17599:} $295$, \textsf{\bfseries 17609:} $27$, \textsf{\bfseries 17623:} $99$, \textsf{\bfseries 17627:} $70$, \textsf{\bfseries 17657:} $87$, \textsf{\bfseries 17659:} $13$, \textsf{\bfseries 17669:} $2$, \textsf{\bfseries 17681:} $73$, \textsf{\bfseries 17683:} $141$, \textsf{\bfseries 17707:} $5$, \textsf{\bfseries 17713:} $47$, \textsf{\bfseries 17729:} $187$, \textsf{\bfseries 17737:} $51$, \textsf{\bfseries 17747:} $53$, \textsf{\bfseries 17749:} $6$, \textsf{\bfseries 17761:} $69$, \textsf{\bfseries 17783:} $40$, \textsf{\bfseries 17789:} $17$, \textsf{\bfseries 17791:} $7$, \textsf{\bfseries 17807:} $89$, \textsf{\bfseries 17827:} $97$, \textsf{\bfseries 17837:} $2$, \textsf{\bfseries 17839:} $12$, \textsf{\bfseries 17851:} $40$, \textsf{\bfseries 17863:} $78$, \textsf{\bfseries 17881:} $46$, \textsf{\bfseries 17891:} $40$, \textsf{\bfseries 17903:} $7$, \textsf{\bfseries 17909:} $417$, \textsf{\bfseries 17911:} $93$, \textsf{\bfseries 17921:} $21$, \textsf{\bfseries 17923:} $26$, \textsf{\bfseries 17929:} $217$, \textsf{\bfseries 17939:} $32$, \textsf{\bfseries 17957:} $22$, \textsf{\bfseries 17959:} $28$, \textsf{\bfseries 17971:} $138$, \textsf{\bfseries 17977:} $5$, \textsf{\bfseries 17981:} $38$, \textsf{\bfseries 17987:} $94$, \textsf{\bfseries 17989:} $17$, \textsf{\bfseries 18013:} $15$, \textsf{\bfseries 18041:} $3$, \textsf{\bfseries 18043:} $128$, \textsf{\bfseries 18047:} $14$, \textsf{\bfseries 18049:} $104$, \textsf{\bfseries 18059:} $6$, \textsf{\bfseries 18061:} $10$, \textsf{\bfseries 18077:} $19$, \textsf{\bfseries 18089:} $73$, \textsf{\bfseries 18097:} $31$, \textsf{\bfseries 18119:} $199$, \textsf{\bfseries 18121:} $70$, \textsf{\bfseries 18127:} $29$, \textsf{\bfseries 18131:} $46$, \textsf{\bfseries 18133:} $13$, \textsf{\bfseries 18143:} $44$, \textsf{\bfseries 18149:} $63$, \textsf{\bfseries 18169:} $74$, \textsf{\bfseries 18181:} $30$, \textsf{\bfseries 18191:} $214$, \textsf{\bfseries 18199:} $135$, \textsf{\bfseries 18211:} $63$, \textsf{\bfseries 18217:} $14$, \textsf{\bfseries 18223:} $69$, \textsf{\bfseries 18229:} $280$, \textsf{\bfseries 18233:} $107$, \textsf{\bfseries 18251:} $6$, \textsf{\bfseries 18253:} $60$, \textsf{\bfseries 18257:} $61$, \textsf{\bfseries 18269:} $19$, \textsf{\bfseries 18287:} $66$, \textsf{\bfseries 18289:} $164$, \textsf{\bfseries 18301:} $21$, \textsf{\bfseries 18307:} $89$, \textsf{\bfseries 18311:} $208$, \textsf{\bfseries 18313:} $95$, \textsf{\bfseries 18329:} $74$, 
\textsf{\bfseries 18341:} $56$, \textsf{\bfseries 18353:} $73$, \textsf{\bfseries 18367:} $17$, \textsf{\bfseries 18371:} $14$, \textsf{\bfseries 18379:} $11$, \textsf{\bfseries 18397:} $42$, \textsf{\bfseries 18401:} $29$, \textsf{\bfseries 18413:} $51$, \textsf{\bfseries 18427:} $48$, \textsf{\bfseries 18433:} $17$, \textsf{\bfseries 18439:} $193$, \textsf{\bfseries 18443:} $18$, \textsf{\bfseries 18451:} $38$, \textsf{\bfseries 18457:} $37$, \textsf{\bfseries 18461:} $98$, \textsf{\bfseries 18481:} $139$, \textsf{\bfseries 18493:} $61$, \textsf{\bfseries 18503:} $7$, \textsf{\bfseries 18517:} $87$, \textsf{\bfseries 18521:} $22$, \textsf{\bfseries 18523:} $12$, \textsf{\bfseries 18539:} $6$, \textsf{\bfseries 18541:} $47$, \textsf{\bfseries 18553:} $34$, \textsf{\bfseries 18583:} $117$, \textsf{\bfseries 18587:} $22$, \textsf{\bfseries 18593:} $71$, \textsf{\bfseries 18617:} $33$, \textsf{\bfseries 18637:} $71$, \textsf{\bfseries 18661:} $31$, \textsf{\bfseries 18671:} $42$, \textsf{\bfseries 18679:} $83$, \textsf{\bfseries 18691:} $130$, \textsf{\bfseries 18701:} $82$, \textsf{\bfseries 18713:} $111$, \textsf{\bfseries 18719:} $14$, \textsf{\bfseries 18731:} $18$, \textsf{\bfseries 18743:} $91$, \textsf{\bfseries 18749:} $10$, \textsf{\bfseries 18757:} $2$, \textsf{\bfseries 18773:} $45$, \textsf{\bfseries 18787:} $51$, \textsf{\bfseries 18793:} $30$, \textsf{\bfseries 18797:} $79$, \textsf{\bfseries 18803:} $181$, \textsf{\bfseries 18839:} $92$, \textsf{\bfseries 18859:} $166$, \textsf{\bfseries 18869:} $56$, \textsf{\bfseries 18899:} $146$, \textsf{\bfseries 18911:} $436$, \textsf{\bfseries 18913:} $13$, \textsf{\bfseries 18917:} $20$, \textsf{\bfseries 18919:} $94$, \textsf{\bfseries 18947:} $60$, \textsf{\bfseries 18959:} $114$, \textsf{\bfseries 18973:} $23$, \textsf{\bfseries 18979:} $254$, \textsf{\bfseries 19001:} $69$, \textsf{\bfseries 19009:} $119$, \textsf{\bfseries 19013:} $29$, \textsf{\bfseries 19031:} $124$, \textsf{\bfseries 19037:} $3$, \textsf{\bfseries 19051:} $7$, \textsf{\bfseries 19069:} $50$, \textsf{\bfseries 19073:} $102$, \textsf{\bfseries 19079:} $21$, \textsf{\bfseries 19081:} $63$, \textsf{\bfseries 19087:} $19$, \textsf{\bfseries 19121:} $51$, \textsf{\bfseries 19139:} $54$, \textsf{\bfseries 19141:} $6$, \textsf{\bfseries 19157:} $243$, \textsf{\bfseries 19163:} $26$, \textsf{\bfseries 19181:} $83$, \textsf{\bfseries 19183:} $45$, \textsf{\bfseries 19207:} $77$, \textsf{\bfseries 19211:} $106$, \textsf{\bfseries 19213:} $67$, \textsf{\bfseries 19219:} $73$, \textsf{\bfseries 19231:} $6$, \textsf{\bfseries 19237:} $24$, \textsf{\bfseries 19249:} $70$, \textsf{\bfseries 19259:} $40$, \textsf{\bfseries 19267:} $20$, \textsf{\bfseries 19273:} $30$, \textsf{\bfseries 19289:} $55$, \textsf{\bfseries 19301:} $8$, \textsf{\bfseries 19309:} $46$, \textsf{\bfseries 19319:} $85$, \textsf{\bfseries 19333:} $21$, \textsf{\bfseries 19373:} $2$, \textsf{\bfseries 19379:} $14$, \textsf{\bfseries 19381:} $52$, \textsf{\bfseries 19387:} $7$, \textsf{\bfseries 19391:} $33$, \textsf{\bfseries 19403:} $19$, \textsf{\bfseries 19417:} $20$, \textsf{\bfseries 19421:} $19$, \textsf{\bfseries 19423:} $37$, \textsf{\bfseries 19427:} $21$, 
\textsf{\bfseries 19429:} $90$, \textsf{\bfseries 19433:} $27$, \textsf{\bfseries 19441:} $39$, \textsf{\bfseries 19447:} $192$, \textsf{\bfseries 19457:} $39$, \textsf{\bfseries 19463:} $170$, \textsf{\bfseries 19469:} $47$, \textsf{\bfseries 19471:} $91$, \textsf{\bfseries 19477:} $44$, \textsf{\bfseries 19483:} $3$, \textsf{\bfseries 19489:} $39$, \textsf{\bfseries 19501:} $6$, \textsf{\bfseries 19507:} $14$, \textsf{\bfseries 19531:} $77$, \textsf{\bfseries 19541:} $52$, \textsf{\bfseries 19543:} $40$, \textsf{\bfseries 19553:} $136$, \textsf{\bfseries 19559:} $14$, \textsf{\bfseries 19571:} $138$, \textsf{\bfseries 19577:} $6$, \textsf{\bfseries 19583:} $82$, \textsf{\bfseries 19597:} $56$, \textsf{\bfseries 19603:} $43$, \textsf{\bfseries 19609:} $66$, \textsf{\bfseries 19661:} $19$, \textsf{\bfseries 19681:} $11$, \textsf{\bfseries 19687:} $134$, \textsf{\bfseries 19697:} $17$, \textsf{\bfseries 19699:} $50$, \textsf{\bfseries 19709:} $10$, \textsf{\bfseries 19717:} $57$, \textsf{\bfseries 19727:} $14$, \textsf{\bfseries 19739:} $24$, \textsf{\bfseries 19751:} $29$, \textsf{\bfseries 19753:} $7$, \textsf{\bfseries 19759:} $127$, \textsf{\bfseries 19763:} $5$, \textsf{\bfseries 19777:} $184$, \textsf{\bfseries 19793:} $31$, \textsf{\bfseries 19801:} $62$, \textsf{\bfseries 19813:} $21$, \textsf{\bfseries 19819:} $10$, \textsf{\bfseries 19841:} $78$, \textsf{\bfseries 19843:} $87$, \textsf{\bfseries 19853:} $3$, \textsf{\bfseries 19861:} $46$, \textsf{\bfseries 19867:} $71$, \textsf{\bfseries 19889:} $291$, \textsf{\bfseries 19891:} $40$, \textsf{\bfseries 19913:} $34$, \textsf{\bfseries 19919:} $69$, \textsf{\bfseries 19927:} $55$, \textsf{\bfseries 19937:} $12$, \textsf{\bfseries 19949:} $40$, \textsf{\bfseries 19961:} $13$, \textsf{\bfseries 19963:} $2$, \textsf{\bfseries 19973:} $29$, \textsf{\bfseries 19979:} $34$, \textsf{\bfseries 19991:} $79$, \textsf{\bfseries 19993:} $30$, \textsf{\bfseries 19997:} $28$, \textsf{\bfseries 20011:} $18$, \textsf{\bfseries 20021:} $3$, \textsf{\bfseries 20023:} $62$, \textsf{\bfseries 20029:} $130$, \textsf{\bfseries 20047:} $39$, \textsf{\bfseries 20051:} $61$, \textsf{\bfseries 20063:} $17$, \textsf{\bfseries 20071:} $3$, \textsf{\bfseries 20089:} $222$, \textsf{\bfseries 20101:} $6$, \textsf{\bfseries 20107:} $5$, \textsf{\bfseries 20113:} $70$, \textsf{\bfseries 20117:} $75$, \textsf{\bfseries 20123:} $42$, \textsf{\bfseries 20129:} $27$, \textsf{\bfseries 20143:} $20$, \textsf{\bfseries 20147:} $19$, \textsf{\bfseries 20149:} $18$, \textsf{\bfseries 20161:} $65$, \textsf{\bfseries 20173:} $96$, \textsf{\bfseries 20177:} $73$, \textsf{\bfseries 20183:} $46$, \textsf{\bfseries 20201:} $362$, \textsf{\bfseries 20219:} $182$, \textsf{\bfseries 20231:} $145$, \textsf{\bfseries 20233:} $10$, \textsf{\bfseries 20249:} $215$, \textsf{\bfseries 20261:} $92$, \textsf{\bfseries 20269:} $160$, \textsf{\bfseries 20287:} $73$, \textsf{\bfseries 20297:} $48$, \textsf{\bfseries 20323:} $20$, \textsf{\bfseries 20327:} $26$, \textsf{\bfseries 20333:} $161$, \textsf{\bfseries 20341:} $24$, \textsf{\bfseries 20347:} $59$, \textsf{\bfseries 20353:} $13$, \textsf{\bfseries 20357:} $39$, \textsf{\bfseries 20359:} $114$, 
\textsf{\bfseries 20369:} $24$, \textsf{\bfseries 20389:} $21$, \textsf{\bfseries 20393:} $68$, \textsf{\bfseries 20399:} $70$, \textsf{\bfseries 20407:} $14$, \textsf{\bfseries 20411:} $90$, \textsf{\bfseries 20431:} $104$, \textsf{\bfseries 20441:} $27$, \textsf{\bfseries 20443:} $32$, \textsf{\bfseries 20477:} $215$, \textsf{\bfseries 20479:} $109$, \textsf{\bfseries 20483:} $76$, \textsf{\bfseries 20507:} $93$, \textsf{\bfseries 20509:} $63$, \textsf{\bfseries 20521:} $26$, \textsf{\bfseries 20533:} $14$, \textsf{\bfseries 20543:} $97$, \textsf{\bfseries 20549:} $8$, \textsf{\bfseries 20551:} $6$, \textsf{\bfseries 20563:} $23$, \textsf{\bfseries 20593:} $14$, \textsf{\bfseries 20599:} $26$, \textsf{\bfseries 20611:} $33$, \textsf{\bfseries 20627:} $6$, \textsf{\bfseries 20639:} $55$, \textsf{\bfseries 20641:} $23$, \textsf{\bfseries 20663:} $85$, \textsf{\bfseries 20681:} $57$, \textsf{\bfseries 20693:} $147$, \textsf{\bfseries 20707:} $112$, \textsf{\bfseries 20717:} $20$, \textsf{\bfseries 20719:} $69$, \textsf{\bfseries 20731:} $22$, \textsf{\bfseries 20743:} $34$, \textsf{\bfseries 20747:} $87$, \textsf{\bfseries 20749:} $37$, \textsf{\bfseries 20753:} $12$, \textsf{\bfseries 20759:} $7$, \textsf{\bfseries 20771:} $19$, \textsf{\bfseries 20773:} $35$, \textsf{\bfseries 20789:} $82$, \textsf{\bfseries 20807:} $10$, \textsf{\bfseries 20809:} $91$, \textsf{\bfseries 20849:} $6$, \textsf{\bfseries 20857:} $123$, \textsf{\bfseries 20873:} $24$, \textsf{\bfseries 20879:} $142$, \textsf{\bfseries 20887:} $19$, \textsf{\bfseries 20897:} $48$, \textsf{\bfseries 20899:} $168$, \textsf{\bfseries 20903:} $45$, \textsf{\bfseries 20921:} $142$, \textsf{\bfseries 20929:} $82$, \textsf{\bfseries 20939:} $98$, \textsf{\bfseries 20947:} $11$, \textsf{\bfseries 20959:} $28$, \textsf{\bfseries 20963:} $46$, \textsf{\bfseries 20981:} $130$, \textsf{\bfseries 20983:} $47$, \textsf{\bfseries 21001:} $52$, \textsf{\bfseries 21011:} $97$, \textsf{\bfseries 21013:} $19$, \textsf{\bfseries 21017:} $3$, \textsf{\bfseries 21019:} $2$, \textsf{\bfseries 21023:} $14$, \textsf{\bfseries 21031:} $12$, \textsf{\bfseries 21059:} $40$, \textsf{\bfseries 21061:} $111$, \textsf{\bfseries 21067:} $41$, \textsf{\bfseries 21089:} $15$, \textsf{\bfseries 21101:} $18$, \textsf{\bfseries 21107:} $33$, \textsf{\bfseries 21121:} $68$, \textsf{\bfseries 21139:} $14$, \textsf{\bfseries 21143:} $30$, \textsf{\bfseries 21149:} $78$, \textsf{\bfseries 21157:} $71$, \textsf{\bfseries 21163:} $32$, \textsf{\bfseries 21169:} $182$, \textsf{\bfseries 21179:} $33$, \textsf{\bfseries 21187:} $109$, \textsf{\bfseries 21191:} $86$, \textsf{\bfseries 21193:} $11$, \textsf{\bfseries 21211:} $3$, \textsf{\bfseries 21221:} $2$, \textsf{\bfseries 21227:} $5$, \textsf{\bfseries 21247:} $13$, \textsf{\bfseries 21269:} $18$, \textsf{\bfseries 21277:} $82$, \textsf{\bfseries 21283:} $38$, \textsf{\bfseries 21313:} $11$, \textsf{\bfseries 21317:} $18$, \textsf{\bfseries 21319:} $47$, \textsf{\bfseries 21323:} $15$, \textsf{\bfseries 21341:} $109$, \textsf{\bfseries 21347:} $41$, \textsf{\bfseries 21377:} $38$, \textsf{\bfseries 21379:} $18$, \textsf{\bfseries 21383:} $92$, \textsf{\bfseries 21391:} $151$, 
\textsf{\bfseries 21397:} $21$, \textsf{\bfseries 21401:} $21$, \textsf{\bfseries 21407:} $5$, \textsf{\bfseries 21419:} $101$, \textsf{\bfseries 21433:} $69$, \textsf{\bfseries 21467:} $8$, \textsf{\bfseries 21481:} $139$, \textsf{\bfseries 21487:} $20$, \textsf{\bfseries 21491:} $281$, \textsf{\bfseries 21493:} $78$, \textsf{\bfseries 21499:} $75$, \textsf{\bfseries 21503:} $33$, \textsf{\bfseries 21517:} $15$, \textsf{\bfseries 21521:} $14$, \textsf{\bfseries 21523:} $50$, \textsf{\bfseries 21529:} $171$, \textsf{\bfseries 21557:} $8$, \textsf{\bfseries 21559:} $43$, \textsf{\bfseries 21563:} $32$, \textsf{\bfseries 21569:} $27$, \textsf{\bfseries 21577:} $41$, \textsf{\bfseries 21587:} $2$, \textsf{\bfseries 21589:} $42$, \textsf{\bfseries 21599:} $63$, \textsf{\bfseries 21601:} $52$, \textsf{\bfseries 21611:} $13$, \textsf{\bfseries 21613:} $52$, \textsf{\bfseries 21617:} $21$, \textsf{\bfseries 21647:} $93$, \textsf{\bfseries 21649:} $101$, \textsf{\bfseries 21661:} $24$, \textsf{\bfseries 21673:} $61$, \textsf{\bfseries 21683:} $5$, \textsf{\bfseries 21701:} $114$, \textsf{\bfseries 21713:} $108$, \textsf{\bfseries 21727:} $74$, \textsf{\bfseries 21737:} $35$, \textsf{\bfseries 21739:} $142$, \textsf{\bfseries 21751:} $23$, \textsf{\bfseries 21757:} $17$, \textsf{\bfseries 21767:} $33$, \textsf{\bfseries 21773:} $78$, \textsf{\bfseries 21787:} $26$, \textsf{\bfseries 21799:} $117$, \textsf{\bfseries 21803:} $31$, \textsf{\bfseries 21817:} $67$, \textsf{\bfseries 21821:} $84$, \textsf{\bfseries 21839:} $85$, \textsf{\bfseries 21841:} $73$, \textsf{\bfseries 21851:} $10$, \textsf{\bfseries 21859:} $13$, \textsf{\bfseries 21863:} $5$, \textsf{\bfseries 21871:} $42$, \textsf{\bfseries 21881:} $142$, \textsf{\bfseries 21893:} $65$, \textsf{\bfseries 21911:} $181$, \textsf{\bfseries 21929:} $3$, \textsf{\bfseries 21937:} $26$, \textsf{\bfseries 21943:} $39$, \textsf{\bfseries 21961:} $134$, \textsf{\bfseries 21977:} $76$, \textsf{\bfseries 21991:} $68$, \textsf{\bfseries 21997:} $26$, \textsf{\bfseries 22003:} $29$, \textsf{\bfseries 22013:} $99$, \textsf{\bfseries 22027:} $117$, \textsf{\bfseries 22031:} $140$, \textsf{\bfseries 22037:} $14$, \textsf{\bfseries 22039:} $138$, \textsf{\bfseries 22051:} $12$, \textsf{\bfseries 22063:} $5$, \textsf{\bfseries 22067:} $29$, \textsf{\bfseries 22073:} $3$, \textsf{\bfseries 22079:} $318$, \textsf{\bfseries 22091:} $82$, \textsf{\bfseries 22093:} $24$, \textsf{\bfseries 22109:} $19$, \textsf{\bfseries 22111:} $13$, \textsf{\bfseries 22123:} $2$, \textsf{\bfseries 22129:} $41$, \textsf{\bfseries 22133:} $32$, \textsf{\bfseries 22147:} $38$, \textsf{\bfseries 22153:} $5$, \textsf{\bfseries 22157:} $41$, \textsf{\bfseries 22159:} $116$, \textsf{\bfseries 22171:} $52$, \textsf{\bfseries 22189:} $69$, \textsf{\bfseries 22193:} $6$, \textsf{\bfseries 22229:} $135$, \textsf{\bfseries 22247:} $10$, \textsf{\bfseries 22259:} $42$, \textsf{\bfseries 22271:} $332$, \textsf{\bfseries 22273:} $56$, \textsf{\bfseries 22277:} $87$, \textsf{\bfseries 22279:} $38$, \textsf{\bfseries 22283:} $77$, \textsf{\bfseries 22291:} $18$, \textsf{\bfseries 22303:} $7$, \textsf{\bfseries 22307:} $35$, \textsf{\bfseries 22343:} $30$, 
\textsf{\bfseries 22349:} $66$, \textsf{\bfseries 22367:} $30$, \textsf{\bfseries 22369:} $11$, \textsf{\bfseries 22381:} $18$, \textsf{\bfseries 22391:} $124$, \textsf{\bfseries 22397:} $117$, \textsf{\bfseries 22409:} $83$, \textsf{\bfseries 22433:} $75$, \textsf{\bfseries 22441:} $21$, \textsf{\bfseries 22447:} $87$, \textsf{\bfseries 22453:} $19$, \textsf{\bfseries 22469:} $62$, \textsf{\bfseries 22481:} $82$, \textsf{\bfseries 22483:} $48$, \textsf{\bfseries 22501:} $61$, \textsf{\bfseries 22511:} $74$, \textsf{\bfseries 22531:} $251$, \textsf{\bfseries 22541:} $199$, \textsf{\bfseries 22543:} $24$, \textsf{\bfseries 22549:} $2$, \textsf{\bfseries 22567:} $6$, \textsf{\bfseries 22571:} $58$, \textsf{\bfseries 22573:} $22$, \textsf{\bfseries 22613:} $5$, \textsf{\bfseries 22619:} $30$, \textsf{\bfseries 22621:} $85$, \textsf{\bfseries 22637:} $19$, \textsf{\bfseries 22639:} $6$, \textsf{\bfseries 22643:} $15$, \textsf{\bfseries 22651:} $60$, \textsf{\bfseries 22669:} $21$, \textsf{\bfseries 22679:} $217$, \textsf{\bfseries 22691:} $6$, \textsf{\bfseries 22697:} $24$, \textsf{\bfseries 22699:} $106$, \textsf{\bfseries 22709:} $69$, \textsf{\bfseries 22717:} $11$, \textsf{\bfseries 22721:} $15$, \textsf{\bfseries 22727:} $154$, \textsf{\bfseries 22739:} $65$, \textsf{\bfseries 22741:} $22$, \textsf{\bfseries 22751:} $69$, \textsf{\bfseries 22769:} $55$, \textsf{\bfseries 22777:} $57$, \textsf{\bfseries 22783:} $23$, \textsf{\bfseries 22787:} $46$, \textsf{\bfseries 22807:} $281$, \textsf{\bfseries 22811:} $29$, \textsf{\bfseries 22817:} $48$, \textsf{\bfseries 22853:} $148$, \textsf{\bfseries 22859:} $105$, \textsf{\bfseries 22861:} $84$, \textsf{\bfseries 22871:} $262$, \textsf{\bfseries 22877:} $75$, \textsf{\bfseries 22901:} $262$, \textsf{\bfseries 22907:} $44$, \textsf{\bfseries 22921:} $151$, \textsf{\bfseries 22937:} $6$, \textsf{\bfseries 22943:} $117$, \textsf{\bfseries 22961:} $311$, \textsf{\bfseries 22963:} $67$, \textsf{\bfseries 22973:} $17$, \textsf{\bfseries 22993:} $5$, \textsf{\bfseries 23003:} $13$, \textsf{\bfseries 23011:} $15$, \textsf{\bfseries 23017:} $35$, \textsf{\bfseries 23021:} $162$, \textsf{\bfseries 23027:} $32$, \textsf{\bfseries 23029:} $55$, \textsf{\bfseries 23039:} $47$, \textsf{\bfseries 23041:} $85$, \textsf{\bfseries 23053:} $11$, \textsf{\bfseries 23057:} $37$, \textsf{\bfseries 23059:} $15$, \textsf{\bfseries 23063:} $153$, \textsf{\bfseries 23071:} $183$, \textsf{\bfseries 23081:} $19$, \textsf{\bfseries 23087:} $56$, \textsf{\bfseries 23099:} $190$, \textsf{\bfseries 23117:} $157$, \textsf{\bfseries 23131:} $10$, \textsf{\bfseries 23143:} $136$, \textsf{\bfseries 23159:} $114$, \textsf{\bfseries 23167:} $40$, \textsf{\bfseries 23173:} $21$, \textsf{\bfseries 23189:} $75$, \textsf{\bfseries 23197:} $46$, \textsf{\bfseries 23201:} $35$, \textsf{\bfseries 23203:} $48$, \textsf{\bfseries 23209:} $302$, \textsf{\bfseries 23227:} $70$, \textsf{\bfseries 23251:} $46$, \textsf{\bfseries 23269:} $40$, \textsf{\bfseries 23279:} $43$, \textsf{\bfseries 23291:} $170$, \textsf{\bfseries 23293:} $23$, \textsf{\bfseries 23297:} $3$, \textsf{\bfseries 23311:} $19$, \textsf{\bfseries 23321:} $3$, \textsf{\bfseries 23327:} $80$, 
\textsf{\bfseries 23333:} $147$, \textsf{\bfseries 23339:} $91$, \textsf{\bfseries 23357:} $42$, \textsf{\bfseries 23369:} $34$, \textsf{\bfseries 23371:} $60$, \textsf{\bfseries 23399:} $127$, \textsf{\bfseries 23417:} $20$, \textsf{\bfseries 23431:} $38$, \textsf{\bfseries 23447:} $20$, \textsf{\bfseries 23459:} $105$, \textsf{\bfseries 23473:} $30$, \textsf{\bfseries 23497:} $10$, \textsf{\bfseries 23509:} $118$, \textsf{\bfseries 23531:} $35$, \textsf{\bfseries 23537:} $10$, \textsf{\bfseries 23539:} $32$, \textsf{\bfseries 23549:} $83$, \textsf{\bfseries 23557:} $65$, \textsf{\bfseries 23561:} $27$, \textsf{\bfseries 23563:} $2$, \textsf{\bfseries 23567:} $80$, \textsf{\bfseries 23581:} $18$, \textsf{\bfseries 23593:} $5$, \textsf{\bfseries 23599:} $23$, \textsf{\bfseries 23603:} $15$, \textsf{\bfseries 23609:} $15$, \textsf{\bfseries 23623:} $21$, \textsf{\bfseries 23627:} $63$, \textsf{\bfseries 23629:} $102$, \textsf{\bfseries 23633:} $5$, \textsf{\bfseries 23663:} $43$, \textsf{\bfseries 23669:} $12$, \textsf{\bfseries 23671:} $124$, \textsf{\bfseries 23677:} $15$, \textsf{\bfseries 23687:} $60$, \textsf{\bfseries 23689:} $206$, \textsf{\bfseries 23719:} $29$, \textsf{\bfseries 23741:} $31$, \textsf{\bfseries 23743:} $17$, \textsf{\bfseries 23747:} $34$, \textsf{\bfseries 23753:} $20$, \textsf{\bfseries 23761:} $51$, \textsf{\bfseries 23767:} $5$, \textsf{\bfseries 23773:} $180$, \textsf{\bfseries 23789:} $8$, \textsf{\bfseries 23801:} $86$, \textsf{\bfseries 23813:} $46$, \textsf{\bfseries 23819:} $29$, \textsf{\bfseries 23827:} $178$, \textsf{\bfseries 23831:} $73$, \textsf{\bfseries 23833:} $69$, \textsf{\bfseries 23857:} $15$, \textsf{\bfseries 23869:} $99$, \textsf{\bfseries 23873:} $10$, \textsf{\bfseries 23879:} $65$, \textsf{\bfseries 23887:} $51$, \textsf{\bfseries 23893:} $14$, \textsf{\bfseries 23899:} $2$, \textsf{\bfseries 23909:} $52$, \textsf{\bfseries 23911:} $6$, \textsf{\bfseries 23917:} $70$, \textsf{\bfseries 23929:} $70$, \textsf{\bfseries 23957:} $3$, \textsf{\bfseries 23971:} $106$, \textsf{\bfseries 23977:} $10$, \textsf{\bfseries 23981:} $29$, \textsf{\bfseries 23993:} $6$, \textsf{\bfseries 24001:} $43$, \textsf{\bfseries 24007:} $82$, \textsf{\bfseries 24019:} $142$, \textsf{\bfseries 24023:} $47$, \textsf{\bfseries 24029:} $12$, \textsf{\bfseries 24043:} $20$, \textsf{\bfseries 24049:} $69$, \textsf{\bfseries 24061:} $40$, \textsf{\bfseries 24071:} $78$, \textsf{\bfseries 24077:} $35$, \textsf{\bfseries 24083:} $8$, \textsf{\bfseries 24091:} $83$, \textsf{\bfseries 24097:} $33$, \textsf{\bfseries 24103:} $73$, \textsf{\bfseries 24107:} $29$, \textsf{\bfseries 24109:} $17$, \textsf{\bfseries 24113:} $14$, \textsf{\bfseries 24121:} $104$, \textsf{\bfseries 24133:} $17$, \textsf{\bfseries 24137:} $3$, \textsf{\bfseries 24151:} $127$, \textsf{\bfseries 24169:} $46$, \textsf{\bfseries 24179:} $19$, \textsf{\bfseries 24181:} $17$, \textsf{\bfseries 24197:} $18$, \textsf{\bfseries 24203:} $28$, \textsf{\bfseries 24223:} $39$, \textsf{\bfseries 24229:} $2$, \textsf{\bfseries 24239:} $211$, \textsf{\bfseries 24247:} $26$, \textsf{\bfseries 24251:} $52$, \textsf{\bfseries 24281:} $6$, \textsf{\bfseries 24317:} $80$, 
\textsf{\bfseries 24329:} $65$, \textsf{\bfseries 24337:} $20$, \textsf{\bfseries 24359:} $11$, \textsf{\bfseries 24371:} $8$, \textsf{\bfseries 24373:} $210$, \textsf{\bfseries 24379:} $50$, \textsf{\bfseries 24391:} $30$, \textsf{\bfseries 24407:} $11$, \textsf{\bfseries 24413:} $88$, \textsf{\bfseries 24419:} $6$, \textsf{\bfseries 24421:} $38$, \textsf{\bfseries 24439:} $86$, \textsf{\bfseries 24443:} $14$, \textsf{\bfseries 24469:} $17$, \textsf{\bfseries 24473:} $113$, \textsf{\bfseries 24481:} $194$, \textsf{\bfseries 24499:} $40$, \textsf{\bfseries 24509:} $23$, \textsf{\bfseries 24517:} $38$, \textsf{\bfseries 24527:} $57$, \textsf{\bfseries 24533:} $61$, \textsf{\bfseries 24547:} $2$, \textsf{\bfseries 24551:} $59$, \textsf{\bfseries 24571:} $35$, \textsf{\bfseries 24593:} $34$, \textsf{\bfseries 24611:} $42$, \textsf{\bfseries 24623:} $40$, \textsf{\bfseries 24631:} $103$, \textsf{\bfseries 24659:} $37$, \textsf{\bfseries 24671:} $148$, \textsf{\bfseries 24677:} $30$, \textsf{\bfseries 24683:} $63$, \textsf{\bfseries 24691:} $42$, \textsf{\bfseries 24697:} $37$, \textsf{\bfseries 24709:} $35$, \textsf{\bfseries 24733:} $130$, \textsf{\bfseries 24749:} $152$, \textsf{\bfseries 24763:} $33$, \textsf{\bfseries 24767:} $5$, \textsf{\bfseries 24781:} $2$, \textsf{\bfseries 24793:} $13$, \textsf{\bfseries 24799:} $6$, \textsf{\bfseries 24809:} $131$, \textsf{\bfseries 24821:} $39$, \textsf{\bfseries 24841:} $42$, \textsf{\bfseries 24847:} $10$, \textsf{\bfseries 24851:} $59$, \textsf{\bfseries 24859:} $139$, \textsf{\bfseries 24877:} $23$, \textsf{\bfseries 24889:} $133$, \textsf{\bfseries 24907:} $65$, \textsf{\bfseries 24917:} $11$, \textsf{\bfseries 24919:} $346$, \textsf{\bfseries 24923:} $24$, \textsf{\bfseries 24943:} $53$, \textsf{\bfseries 24953:} $12$, \textsf{\bfseries 24967:} $23$, \textsf{\bfseries 24971:} $145$, \textsf{\bfseries 24977:} $12$, \textsf{\bfseries 24979:} $52$, \textsf{\bfseries 24989:} $23$, \textsf{\bfseries 25013:} $73$, \textsf{\bfseries 25031:} $62$, \textsf{\bfseries 25033:} $5$, \textsf{\bfseries 25037:} $2$, \textsf{\bfseries 25057:} $11$, \textsf{\bfseries 25073:} $103$, \textsf{\bfseries 25087:} $3$, \textsf{\bfseries 25097:} $77$, \textsf{\bfseries 25111:} $44$, \textsf{\bfseries 25117:} $5$, \textsf{\bfseries 25121:} $62$, \textsf{\bfseries 25127:} $80$, \textsf{\bfseries 25147:} $47$, \textsf{\bfseries 25153:} $29$, \textsf{\bfseries 25163:} $72$, \textsf{\bfseries 25169:} $58$, \textsf{\bfseries 25171:} $115$, \textsf{\bfseries 25183:} $13$, \textsf{\bfseries 25189:} $197$, \textsf{\bfseries 25219:} $69$, \textsf{\bfseries 25229:} $92$, \textsf{\bfseries 25237:} $32$, \textsf{\bfseries 25243:} $44$, \textsf{\bfseries 25247:} $20$, \textsf{\bfseries 25253:} $18$, \textsf{\bfseries 25261:} $74$, \textsf{\bfseries 25301:} $18$, \textsf{\bfseries 25303:} $20$, \textsf{\bfseries 25307:} $15$, \textsf{\bfseries 25309:} $167$, \textsf{\bfseries 25321:} $44$, \textsf{\bfseries 25339:} $3$, \textsf{\bfseries 25343:} $160$, \textsf{\bfseries 25349:} $41$, \textsf{\bfseries 25357:} $20$, \textsf{\bfseries 25367:} $217$, \textsf{\bfseries 25373:} $65$, \textsf{\bfseries 25391:} $35$, \textsf{\bfseries 25409:} $54$, 
\textsf{\bfseries 25411:} $242$, \textsf{\bfseries 25423:} $10$, \textsf{\bfseries 25439:} $34$, \textsf{\bfseries 25447:} $23$, \textsf{\bfseries 25453:} $59$, \textsf{\bfseries 25457:} $65$, \textsf{\bfseries 25463:} $65$, \textsf{\bfseries 25469:} $60$, \textsf{\bfseries 25471:} $6$, \textsf{\bfseries 25523:} $68$, \textsf{\bfseries 25537:} $10$, \textsf{\bfseries 25541:} $43$, \textsf{\bfseries 25561:} $22$, \textsf{\bfseries 25577:} $27$, \textsf{\bfseries 25579:} $329$, \textsf{\bfseries 25583:} $170$, \textsf{\bfseries 25589:} $255$, \textsf{\bfseries 25601:} $33$, \textsf{\bfseries 25603:} $2$, \textsf{\bfseries 25609:} $103$, \textsf{\bfseries 25621:} $50$, \textsf{\bfseries 25633:} $45$, \textsf{\bfseries 25639:} $21$, \textsf{\bfseries 25643:} $15$, \textsf{\bfseries 25657:} $160$, \textsf{\bfseries 25667:} $39$, \textsf{\bfseries 25673:} $22$, \textsf{\bfseries 25679:} $99$, \textsf{\bfseries 25693:} $39$, \textsf{\bfseries 25703:} $23$, \textsf{\bfseries 25717:} $138$, \textsf{\bfseries 25733:} $22$, \textsf{\bfseries 25741:} $160$, \textsf{\bfseries 25747:} $77$, \textsf{\bfseries 25759:} $117$, \textsf{\bfseries 25763:} $195$, \textsf{\bfseries 25771:} $18$, \textsf{\bfseries 25793:} $54$, \textsf{\bfseries 25799:} $42$, \textsf{\bfseries 25801:} $182$, \textsf{\bfseries 25819:} $39$, \textsf{\bfseries 25841:} $27$, \textsf{\bfseries 25847:} $17$, \textsf{\bfseries 25849:} $87$, \textsf{\bfseries 25867:} $7$, \textsf{\bfseries 25873:} $179$, \textsf{\bfseries 25889:} $111$, \textsf{\bfseries 25903:} $13$, \textsf{\bfseries 25913:} $7$, \textsf{\bfseries 25919:} $17$, \textsf{\bfseries 25931:} $44$, \textsf{\bfseries 25933:} $2$, \textsf{\bfseries 25939:} $7$, \textsf{\bfseries 25943:} $10$, \textsf{\bfseries 25951:} $71$, \textsf{\bfseries 25969:} $14$, \textsf{\bfseries 25981:} $42$, \textsf{\bfseries 25997:} $22$, \textsf{\bfseries 25999:} $145$, \textsf{\bfseries 26003:} $69$, \textsf{\bfseries 26017:} $135$, \textsf{\bfseries 26021:} $40$, \textsf{\bfseries 26029:} $41$, \textsf{\bfseries 26041:} $131$, \textsf{\bfseries 26053:} $146$, \textsf{\bfseries 26083:} $28$, \textsf{\bfseries 26099:} $6$, \textsf{\bfseries 26107:} $23$, \textsf{\bfseries 26111:} $62$, \textsf{\bfseries 26113:} $21$, \textsf{\bfseries 26119:} $140$, \textsf{\bfseries 26141:} $185$, \textsf{\bfseries 26153:} $6$, \textsf{\bfseries 26161:} $87$, \textsf{\bfseries 26171:} $42$, \textsf{\bfseries 26177:} $62$, \textsf{\bfseries 26183:} $31$, \textsf{\bfseries 26189:} $56$, \textsf{\bfseries 26203:} $5$, \textsf{\bfseries 26209:} $88$, \textsf{\bfseries 26227:} $171$, \textsf{\bfseries 26237:} $107$, \textsf{\bfseries 26249:} $3$, \textsf{\bfseries 26251:} $37$, \textsf{\bfseries 26261:} $47$, \textsf{\bfseries 26263:} $204$, \textsf{\bfseries 26267:} $94$, \textsf{\bfseries 26293:} $24$, \textsf{\bfseries 26297:} $28$, \textsf{\bfseries 26309:} $114$, \textsf{\bfseries 26317:} $55$, \textsf{\bfseries 26321:} $79$, \textsf{\bfseries 26339:} $2$, \textsf{\bfseries 26347:} $18$, \textsf{\bfseries 26357:} $41$, \textsf{\bfseries 26371:} $15$, \textsf{\bfseries 26387:} $29$, \textsf{\bfseries 26393:} $6$, \textsf{\bfseries 26399:} $43$, \textsf{\bfseries 26407:} $23$, 
\textsf{\bfseries 26417:} $96$, \textsf{\bfseries 26423:} $5$, \textsf{\bfseries 26431:} $41$, \textsf{\bfseries 26437:} $20$, \textsf{\bfseries 26449:} $67$, \textsf{\bfseries 26459:} $32$, \textsf{\bfseries 26479:} $48$, \textsf{\bfseries 26489:} $30$, \textsf{\bfseries 26497:} $164$, \textsf{\bfseries 26501:} $2$, \textsf{\bfseries 26513:} $127$, \textsf{\bfseries 26539:} $197$, \textsf{\bfseries 26557:} $7$, \textsf{\bfseries 26561:} $60$, \textsf{\bfseries 26573:} $236$, \textsf{\bfseries 26591:} $132$, \textsf{\bfseries 26597:} $37$, \textsf{\bfseries 26627:} $18$, \textsf{\bfseries 26633:} $105$, \textsf{\bfseries 26641:} $56$, \textsf{\bfseries 26647:} $44$, \textsf{\bfseries 26669:} $32$, \textsf{\bfseries 26681:} $124$, \textsf{\bfseries 26683:} $29$, \textsf{\bfseries 26687:} $61$, \textsf{\bfseries 26693:} $45$, \textsf{\bfseries 26699:} $66$, \textsf{\bfseries 26701:} $87$, \textsf{\bfseries 26711:} $122$, \textsf{\bfseries 26713:} $38$, \textsf{\bfseries 26717:} $2$, \textsf{\bfseries 26723:} $31$, \textsf{\bfseries 26729:} $61$, \textsf{\bfseries 26731:} $69$, \textsf{\bfseries 26737:} $23$, \textsf{\bfseries 26759:} $29$, \textsf{\bfseries 26777:} $27$, \textsf{\bfseries 26783:} $53$, \textsf{\bfseries 26801:} $31$, \textsf{\bfseries 26813:} $108$, \textsf{\bfseries 26821:} $42$, \textsf{\bfseries 26833:} $20$, \textsf{\bfseries 26839:} $3$, \textsf{\bfseries 26849:} $48$, \textsf{\bfseries 26861:} $43$, \textsf{\bfseries 26863:} $14$, \textsf{\bfseries 26879:} $26$, \textsf{\bfseries 26881:} $44$, \textsf{\bfseries 26891:} $82$, \textsf{\bfseries 26893:} $80$, \textsf{\bfseries 26903:} $119$, \textsf{\bfseries 26921:} $68$, \textsf{\bfseries 26927:} $20$, \textsf{\bfseries 26947:} $118$, \textsf{\bfseries 26951:} $123$, \textsf{\bfseries 26953:} $37$, \textsf{\bfseries 26959:} $30$, \textsf{\bfseries 26981:} $10$, \textsf{\bfseries 26987:} $19$, \textsf{\bfseries 26993:} $74$, \textsf{\bfseries 27011:} $6$, \textsf{\bfseries 27017:} $85$, \textsf{\bfseries 27031:} $61$, \textsf{\bfseries 27043:} $73$, \textsf{\bfseries 27059:} $13$, \textsf{\bfseries 27061:} $7$, \textsf{\bfseries 27067:} $46$, \textsf{\bfseries 27073:} $131$, \textsf{\bfseries 27077:} $2$, \textsf{\bfseries 27091:} $219$, \textsf{\bfseries 27103:} $46$, \textsf{\bfseries 27107:} $11$, \textsf{\bfseries 27109:} $67$, \textsf{\bfseries 27127:} $5$, \textsf{\bfseries 27143:} $56$, \textsf{\bfseries 27179:} $193$, \textsf{\bfseries 27191:} $26$, \textsf{\bfseries 27197:} $68$, \textsf{\bfseries 27211:} $48$, \textsf{\bfseries 27239:} $28$, \textsf{\bfseries 27241:} $123$, \textsf{\bfseries 27253:} $32$, \textsf{\bfseries 27259:} $12$, \textsf{\bfseries 27271:} $15$, \textsf{\bfseries 27277:} $53$, \textsf{\bfseries 27281:} $145$, \textsf{\bfseries 27283:} $44$, \textsf{\bfseries 27299:} $42$, \textsf{\bfseries 27329:} $161$, \textsf{\bfseries 27337:} $37$, \textsf{\bfseries 27361:} $7$, \textsf{\bfseries 27367:} $24$, \textsf{\bfseries 27397:} $20$, \textsf{\bfseries 27407:} $30$, \textsf{\bfseries 27409:} $17$, \textsf{\bfseries 27427:} $53$, \textsf{\bfseries 27431:} $68$, \textsf{\bfseries 27437:} $14$, \textsf{\bfseries 27449:} $95$, \textsf{\bfseries 27457:} $20$, 
 
                \textsf{\bfseries 27479:} $34$, \textsf{\bfseries 27481:} $79$, \textsf{\bfseries 27487:} $82$, \textsf{\bfseries 27509:} $38$, \textsf{\bfseries 27527:} $33$, \textsf{\bfseries 27529:} $22$, \textsf{\bfseries 27539:} $23$, \textsf{\bfseries 27541:} $50$, \textsf{\bfseries 27551:} $17$, \textsf{\bfseries 27581:} $172$, \textsf{\bfseries 27583:} $109$, \textsf{\bfseries 27611:} $11$, \textsf{\bfseries 27617:} $23$, \textsf{\bfseries 27631:} $95$, \textsf{\bfseries 27647:} $125$, \textsf{\bfseries 27653:} $5$, \textsf{\bfseries 27673:} $34$, \textsf{\bfseries 27689:} $39$, \textsf{\bfseries 27691:} $69$, \textsf{\bfseries 27697:} $21$, \textsf{\bfseries 27701:} $95$, \textsf{\bfseries 27733:} $86$, \textsf{\bfseries 27737:} $10$, \textsf{\bfseries 27739:} $47$, \textsf{\bfseries 27743:} $179$, \textsf{\bfseries 27749:} $13$, \textsf{\bfseries 27751:} $73$, \textsf{\bfseries 27763:} $72$, \textsf{\bfseries 27767:} $29$, \textsf{\bfseries 27773:} $20$, \textsf{\bfseries 27779:} $55$, \textsf{\bfseries 27791:} $83$, \textsf{\bfseries 27793:} $14$, \textsf{\bfseries 27799:} $65$, \textsf{\bfseries 27803:} $8$, \textsf{\bfseries 27809:} $116$, \textsf{\bfseries 27817:} $101$, \textsf{\bfseries 27823:} $70$, \textsf{\bfseries 27827:} $13$, \textsf{\bfseries 27847:} $123$, \textsf{\bfseries 27851:} $10$, \textsf{\bfseries 27883:} $117$, \textsf{\bfseries 27893:} $23$, \textsf{\bfseries 27901:} $18$, \textsf{\bfseries 27917:} $13$, \textsf{\bfseries 27919:} $17$, \textsf{\bfseries 27941:} $98$, \textsf{\bfseries 27943:} $44$, \textsf{\bfseries 27947:} $31$, \textsf{\bfseries 27953:} $53$, \textsf{\bfseries 27961:} $297$, \textsf{\bfseries 27967:} $11$, \textsf{\bfseries 27983:} $26$, \textsf{\bfseries 27997:} $50$, \textsf{\bfseries 28001:} $108$, \textsf{\bfseries 28019:} $151$, \textsf{\bfseries 28027:} $3$, \textsf{\bfseries 28031:} $38$, \textsf{\bfseries 28051:} $46$, \textsf{\bfseries 28057:} $17$, \textsf{\bfseries 28069:} $30$, \textsf{\bfseries 28081:} $122$, \textsf{\bfseries 28087:} $19$, \textsf{\bfseries 28097:} $148$, \textsf{\bfseries 28099:} $40$, \textsf{\bfseries 28109:} $104$, \textsf{\bfseries 28111:} $65$, \textsf{\bfseries 28123:} $17$, \textsf{\bfseries 28151:} $104$, \textsf{\bfseries 28163:} $128$, \textsf{\bfseries 28181:} $2$, \textsf{\bfseries 28183:} $123$, \textsf{\bfseries 28201:} $14$, \textsf{\bfseries 28211:} $2$, \textsf{\bfseries 28219:} $39$, \textsf{\bfseries 28229:} $15$, \textsf{\bfseries 28277:} $21$, \textsf{\bfseries 28279:} $38$, \textsf{\bfseries 28283:} $13$, \textsf{\bfseries 28289:} $114$, \textsf{\bfseries 28297:} $10$, \textsf{\bfseries 28307:} $41$, \textsf{\bfseries 28309:} $77$, \textsf{\bfseries 28319:} $17$, \textsf{\bfseries 28349:} $128$, \textsf{\bfseries 28351:} $12$, \textsf{\bfseries 28387:} $45$, \textsf{\bfseries 28393:} $30$, \textsf{\bfseries 28403:} $15$, \textsf{\bfseries 28409:} $109$, \textsf{\bfseries 28411:} $40$, \textsf{\bfseries 28429:} $42$, \textsf{\bfseries 28433:} $119$, \textsf{\bfseries 28439:} $65$, \textsf{\bfseries 28447:} $61$, \textsf{\bfseries 28463:} $28$, \textsf{\bfseries 28477:} $22$, \textsf{\bfseries 28493:} $34$, \textsf{\bfseries 28499:} $44$, \textsf{\bfseries 28513:} $141$, 
\textsf{\bfseries 28517:} $5$, \textsf{\bfseries 28537:} $28$, \textsf{\bfseries 28541:} $82$, \textsf{\bfseries 28547:} $2$, \textsf{\bfseries 28549:} $7$, \textsf{\bfseries 28559:} $99$, \textsf{\bfseries 28571:} $11$, \textsf{\bfseries 28573:} $69$, \textsf{\bfseries 28579:} $11$, \textsf{\bfseries 28591:} $3$, \textsf{\bfseries 28597:} $109$, \textsf{\bfseries 28603:} $12$, \textsf{\bfseries 28607:} $45$, \textsf{\bfseries 28619:} $46$, \textsf{\bfseries 28621:} $13$, \textsf{\bfseries 28627:} $12$, \textsf{\bfseries 28631:} $19$, \textsf{\bfseries 28643:} $71$, \textsf{\bfseries 28649:} $60$, \textsf{\bfseries 28657:} $14$, \textsf{\bfseries 28661:} $175$, \textsf{\bfseries 28663:} $12$, \textsf{\bfseries 28669:} $23$, \textsf{\bfseries 28687:} $76$, \textsf{\bfseries 28697:} $31$, \textsf{\bfseries 28703:} $164$, \textsf{\bfseries 28711:} $3$, \textsf{\bfseries 28723:} $50$, \textsf{\bfseries 28729:} $365$, \textsf{\bfseries 28751:} $62$, \textsf{\bfseries 28753:} $179$, \textsf{\bfseries 28759:} $87$, \textsf{\bfseries 28771:} $46$, \textsf{\bfseries 28789:} $34$, \textsf{\bfseries 28793:} $67$, \textsf{\bfseries 28807:} $57$, \textsf{\bfseries 28813:} $43$, \textsf{\bfseries 28817:} $6$, \textsf{\bfseries 28837:} $62$, \textsf{\bfseries 28843:} $2$, \textsf{\bfseries 28859:} $8$, \textsf{\bfseries 28867:} $98$, \textsf{\bfseries 28871:} $26$, \textsf{\bfseries 28879:} $26$, \textsf{\bfseries 28901:} $15$, \textsf{\bfseries 28909:} $26$, \textsf{\bfseries 28921:} $69$, \textsf{\bfseries 28927:} $6$, \textsf{\bfseries 28933:} $54$, \textsf{\bfseries 28949:} $3$, \textsf{\bfseries 28961:} $58$, \textsf{\bfseries 28979:} $103$, \textsf{\bfseries 29009:} $82$, \textsf{\bfseries 29017:} $41$, \textsf{\bfseries 29021:} $172$, \textsf{\bfseries 29023:} $56$, \textsf{\bfseries 29027:} $82$, \textsf{\bfseries 29033:} $31$, \textsf{\bfseries 29059:} $48$, \textsf{\bfseries 29063:} $94$, \textsf{\bfseries 29077:} $200$, \textsf{\bfseries 29101:} $2$, \textsf{\bfseries 29123:} $18$, \textsf{\bfseries 29129:} $54$, \textsf{\bfseries 29131:} $145$, \textsf{\bfseries 29137:} $20$, \textsf{\bfseries 29147:} $5$, \textsf{\bfseries 29153:} $47$, \textsf{\bfseries 29167:} $181$, \textsf{\bfseries 29173:} $22$, \textsf{\bfseries 29179:} $110$, \textsf{\bfseries 29191:} $7$, \textsf{\bfseries 29201:} $123$, \textsf{\bfseries 29207:} $90$, \textsf{\bfseries 29209:} $34$, \textsf{\bfseries 29221:} $263$, \textsf{\bfseries 29231:} $71$, \textsf{\bfseries 29243:} $32$, \textsf{\bfseries 29251:} $186$, \textsf{\bfseries 29269:} $179$, \textsf{\bfseries 29287:} $37$, \textsf{\bfseries 29297:} $117$, \textsf{\bfseries 29303:} $28$, \textsf{\bfseries 29311:} $39$, \textsf{\bfseries 29327:} $33$, \textsf{\bfseries 29333:} $42$, \textsf{\bfseries 29339:} $94$, \textsf{\bfseries 29347:} $58$, \textsf{\bfseries 29363:} $128$, \textsf{\bfseries 29383:} $38$, \textsf{\bfseries 29387:} $55$, \textsf{\bfseries 29389:} $7$, \textsf{\bfseries 29399:} $119$, \textsf{\bfseries 29401:} $78$, \textsf{\bfseries 29411:} $39$, \textsf{\bfseries 29423:} $40$, \textsf{\bfseries 29429:} $83$, \textsf{\bfseries 29437:} $55$, \textsf{\bfseries 29443:} $33$, \textsf{\bfseries 29453:} $177$, 
\textsf{\bfseries 29473:} $82$, \textsf{\bfseries 29483:} $14$, \textsf{\bfseries 29501:} $3$, \textsf{\bfseries 29527:} $102$, \textsf{\bfseries 29531:} $50$, \textsf{\bfseries 29537:} $20$, \textsf{\bfseries 29567:} $45$, \textsf{\bfseries 29569:} $78$, \textsf{\bfseries 29573:} $20$, \textsf{\bfseries 29581:} $18$, \textsf{\bfseries 29587:} $17$, \textsf{\bfseries 29599:} $138$, \textsf{\bfseries 29611:} $48$, \textsf{\bfseries 29629:} $44$, \textsf{\bfseries 29633:} $83$, \textsf{\bfseries 29641:} $28$, \textsf{\bfseries 29663:} $55$, \textsf{\bfseries 29669:} $103$, \textsf{\bfseries 29671:} $52$, \textsf{\bfseries 29683:} $111$, \textsf{\bfseries 29717:} $47$, \textsf{\bfseries 29723:} $19$, \textsf{\bfseries 29741:} $108$, \textsf{\bfseries 29753:} $12$, \textsf{\bfseries 29759:} $79$, \textsf{\bfseries 29761:} $79$, \textsf{\bfseries 29789:} $43$, \textsf{\bfseries 29803:} $89$, \textsf{\bfseries 29819:} $78$, \textsf{\bfseries 29833:} $19$, \textsf{\bfseries 29837:} $30$, \textsf{\bfseries 29851:} $147$, \textsf{\bfseries 29863:} $10$, \textsf{\bfseries 29867:} $29$, \textsf{\bfseries 29873:} $45$, \textsf{\bfseries 29879:} $39$, \textsf{\bfseries 29881:} $171$, \textsf{\bfseries 29917:} $24$, \textsf{\bfseries 29921:} $19$, \textsf{\bfseries 29927:} $34$, \textsf{\bfseries 29947:} $158$, \textsf{\bfseries 29959:} $26$, \textsf{\bfseries 29983:} $33$, \textsf{\bfseries 29989:} $30$, \textsf{\bfseries 30011:} $55$, \textsf{\bfseries 30013:} $37$, \textsf{\bfseries 30029:} $48$, \textsf{\bfseries 30047:} $58$, \textsf{\bfseries 30059:} $8$, \textsf{\bfseries 30071:} $79$, \textsf{\bfseries 30089:} $30$, \textsf{\bfseries 30091:} $39$, \textsf{\bfseries 30097:} $140$, \textsf{\bfseries 30103:} $33$, \textsf{\bfseries 30109:} $41$, \textsf{\bfseries 30113:} $47$, \textsf{\bfseries 30119:} $22$, \textsf{\bfseries 30133:} $76$, \textsf{\bfseries 30137:} $27$, \textsf{\bfseries 30139:} $3$, \textsf{\bfseries 30161:} $150$, \textsf{\bfseries 30169:} $106$, \textsf{\bfseries 30181:} $10$, \textsf{\bfseries 30187:} $103$, \textsf{\bfseries 30197:} $17$, \textsf{\bfseries 30203:} $2$, \textsf{\bfseries 30211:} $93$, \textsf{\bfseries 30223:} $26$, \textsf{\bfseries 30241:} $41$, \textsf{\bfseries 30253:} $162$, \textsf{\bfseries 30259:} $46$, \textsf{\bfseries 30269:} $72$, \textsf{\bfseries 30271:} $82$, \textsf{\bfseries 30293:} $61$, \textsf{\bfseries 30307:} $20$, \textsf{\bfseries 30313:} $14$, \textsf{\bfseries 30319:} $35$, \textsf{\bfseries 30323:} $50$, \textsf{\bfseries 30341:} $151$, \textsf{\bfseries 30347:} $20$, \textsf{\bfseries 30367:} $67$, \textsf{\bfseries 30389:} $19$, \textsf{\bfseries 30391:} $22$, \textsf{\bfseries 30403:} $5$, \textsf{\bfseries 30427:} $80$, \textsf{\bfseries 30431:} $58$, \textsf{\bfseries 30449:} $47$, \textsf{\bfseries 30467:} $39$, \textsf{\bfseries 30469:} $24$, \textsf{\bfseries 30491:} $39$, \textsf{\bfseries 30493:} $13$, \textsf{\bfseries 30497:} $51$, \textsf{\bfseries 30509:} $142$, \textsf{\bfseries 30517:} $34$, \textsf{\bfseries 30529:} $65$, \textsf{\bfseries 30539:} $113$, \textsf{\bfseries 30553:} $42$, \textsf{\bfseries 30557:} $78$, \textsf{\bfseries 30559:} $15$, \textsf{\bfseries 30577:} $29$, 
\textsf{\bfseries 30593:} $71$, \textsf{\bfseries 30631:} $17$, \textsf{\bfseries 30637:} $158$, \textsf{\bfseries 30643:} $93$, \textsf{\bfseries 30649:} $52$, \textsf{\bfseries 30661:} $236$, \textsf{\bfseries 30671:} $111$, \textsf{\bfseries 30677:} $5$, \textsf{\bfseries 30689:} $73$, \textsf{\bfseries 30697:} $33$, \textsf{\bfseries 30703:} $29$, \textsf{\bfseries 30707:} $72$, \textsf{\bfseries 30713:} $42$, \textsf{\bfseries 30727:} $14$, \textsf{\bfseries 30757:} $122$, \textsf{\bfseries 30763:} $30$, \textsf{\bfseries 30773:} $111$, \textsf{\bfseries 30781:} $2$, \textsf{\bfseries 30803:} $101$, \textsf{\bfseries 30809:} $118$, \textsf{\bfseries 30817:} $85$, \textsf{\bfseries 30829:} $70$, \textsf{\bfseries 30839:} $296$, \textsf{\bfseries 30841:} $29$, \textsf{\bfseries 30851:} $262$, \textsf{\bfseries 30853:} $24$, \textsf{\bfseries 30859:} $67$, \textsf{\bfseries 30869:} $15$, \textsf{\bfseries 30871:} $58$, \textsf{\bfseries 30881:} $15$, \textsf{\bfseries 30893:} $128$, \textsf{\bfseries 30911:} $44$, \textsf{\bfseries 30931:} $266$, \textsf{\bfseries 30937:} $45$, \textsf{\bfseries 30941:} $108$, \textsf{\bfseries 30949:} $41$, \textsf{\bfseries 30971:} $51$, \textsf{\bfseries 30977:} $5$, \textsf{\bfseries 30983:} $91$, \textsf{\bfseries 31013:} $3$, \textsf{\bfseries 31019:} $32$, \textsf{\bfseries 31033:} $176$, \textsf{\bfseries 31039:} $7$, \textsf{\bfseries 31051:} $83$, \textsf{\bfseries 31063:} $104$, \textsf{\bfseries 31069:} $246$, \textsf{\bfseries 31079:} $99$, \textsf{\bfseries 31081:} $83$, \textsf{\bfseries 31091:} $39$, \textsf{\bfseries 31121:} $140$, \textsf{\bfseries 31123:} $11$, \textsf{\bfseries 31139:} $77$, \textsf{\bfseries 31147:} $48$, \textsf{\bfseries 31151:} $157$, \textsf{\bfseries 31153:} $122$, \textsf{\bfseries 31159:} $24$, \textsf{\bfseries 31177:} $7$, \textsf{\bfseries 31181:} $46$, \textsf{\bfseries 31183:} $3$, \textsf{\bfseries 31189:} $31$, \textsf{\bfseries 31193:} $10$, \textsf{\bfseries 31219:} $38$, \textsf{\bfseries 31223:} $51$, \textsf{\bfseries 31231:} $56$, \textsf{\bfseries 31237:} $73$, \textsf{\bfseries 31247:} $10$, \textsf{\bfseries 31249:} $423$, \textsf{\bfseries 31253:} $50$, \textsf{\bfseries 31259:} $19$, \textsf{\bfseries 31267:} $3$, \textsf{\bfseries 31271:} $114$, \textsf{\bfseries 31277:} $47$, \textsf{\bfseries 31307:} $8$, \textsf{\bfseries 31319:} $7$, \textsf{\bfseries 31321:} $7$, \textsf{\bfseries 31327:} $63$, \textsf{\bfseries 31333:} $19$, \textsf{\bfseries 31337:} $10$, \textsf{\bfseries 31357:} $86$, \textsf{\bfseries 31379:} $162$, \textsf{\bfseries 31387:} $46$, \textsf{\bfseries 31391:} $307$, \textsf{\bfseries 31393:} $29$, \textsf{\bfseries 31397:} $62$, \textsf{\bfseries 31469:} $57$, \textsf{\bfseries 31477:} $33$, \textsf{\bfseries 31481:} $114$, \textsf{\bfseries 31489:} $21$, \textsf{\bfseries 31511:} $175$, \textsf{\bfseries 31513:} $51$, \textsf{\bfseries 31517:} $19$, \textsf{\bfseries 31531:} $55$, \textsf{\bfseries 31541:} $27$, \textsf{\bfseries 31543:} $7$, \textsf{\bfseries 31547:} $58$, \textsf{\bfseries 31567:} $14$, \textsf{\bfseries 31573:} $166$, \textsf{\bfseries 31583:} $30$, \textsf{\bfseries 31601:} $141$, \textsf{\bfseries 31607:} $5$, 
\textsf{\bfseries 31627:} $107$, \textsf{\bfseries 31643:} $20$, \textsf{\bfseries 31649:} $53$, \textsf{\bfseries 31657:} $7$, \textsf{\bfseries 31663:} $19$, \textsf{\bfseries 31667:} $50$, \textsf{\bfseries 31687:} $10$, \textsf{\bfseries 31699:} $84$, \textsf{\bfseries 31721:} $43$, \textsf{\bfseries 31723:} $18$, \textsf{\bfseries 31727:} $41$, \textsf{\bfseries 31729:} $14$, \textsf{\bfseries 31741:} $106$, \textsf{\bfseries 31751:} $13$, \textsf{\bfseries 31769:} $73$, \textsf{\bfseries 31771:} $82$, \textsf{\bfseries 31793:} $48$, \textsf{\bfseries 31799:} $55$, \textsf{\bfseries 31817:} $140$, \textsf{\bfseries 31847:} $259$, \textsf{\bfseries 31849:} $14$, \textsf{\bfseries 31859:} $78$, \textsf{\bfseries 31873:} $71$, \textsf{\bfseries 31883:} $102$, \textsf{\bfseries 31891:} $18$, \textsf{\bfseries 31907:} $117$, \textsf{\bfseries 31957:} $33$, \textsf{\bfseries 31963:} $209$, \textsf{\bfseries 31973:} $8$, \textsf{\bfseries 31981:} $152$, \textsf{\bfseries 31991:} $103$, \textsf{\bfseries 32003:} $20$, \textsf{\bfseries 32009:} $14$, \textsf{\bfseries 32027:} $20$, \textsf{\bfseries 32029:} $18$, \textsf{\bfseries 32051:} $126$, \textsf{\bfseries 32057:} $113$, \textsf{\bfseries 32059:} $55$, \textsf{\bfseries 32063:} $45$, \textsf{\bfseries 32069:} $70$, \textsf{\bfseries 32077:} $5$, \textsf{\bfseries 32083:} $33$, \textsf{\bfseries 32089:} $34$, \textsf{\bfseries 32099:} $18$, \textsf{\bfseries 32117:} $2$, \textsf{\bfseries 32119:} $17$, \textsf{\bfseries 32141:} $19$, \textsf{\bfseries 32143:} $227$, \textsf{\bfseries 32159:} $21$, \textsf{\bfseries 32173:} $92$, \textsf{\bfseries 32183:} $213$, \textsf{\bfseries 32189:} $7$, \textsf{\bfseries 32191:} $41$, \textsf{\bfseries 32203:} $166$, \textsf{\bfseries 32213:} $22$, \textsf{\bfseries 32233:} $77$, \textsf{\bfseries 32237:} $27$, \textsf{\bfseries 32251:} $12$, \textsf{\bfseries 32257:} $70$, \textsf{\bfseries 32261:} $17$, \textsf{\bfseries 32297:} $46$, \textsf{\bfseries 32299:} $58$, \textsf{\bfseries 32303:} $78$, \textsf{\bfseries 32309:} $23$, \textsf{\bfseries 32321:} $42$, \textsf{\bfseries 32323:} $17$, \textsf{\bfseries 32327:} $14$, \textsf{\bfseries 32341:} $76$, \textsf{\bfseries 32353:} $30$, \textsf{\bfseries 32359:} $143$, \textsf{\bfseries 32363:} $51$, \textsf{\bfseries 32369:} $6$, \textsf{\bfseries 32371:} $14$, \textsf{\bfseries 32377:} $10$, \textsf{\bfseries 32381:} $60$, \textsf{\bfseries 32401:} $93$, \textsf{\bfseries 32411:} $122$, \textsf{\bfseries 32413:} $57$, \textsf{\bfseries 32423:} $90$, \textsf{\bfseries 32429:} $61$, \textsf{\bfseries 32441:} $26$, \textsf{\bfseries 32443:} $30$, \textsf{\bfseries 32467:} $3$, \textsf{\bfseries 32479:} $159$, \textsf{\bfseries 32491:} $93$, \textsf{\bfseries 32497:} $15$, \textsf{\bfseries 32503:} $39$, \textsf{\bfseries 32507:} $51$, \textsf{\bfseries 32531:} $105$, \textsf{\bfseries 32533:} $17$, \textsf{\bfseries 32537:} $95$, \textsf{\bfseries 32561:} $24$, \textsf{\bfseries 32563:} $13$, \textsf{\bfseries 32569:} $84$, \textsf{\bfseries 32573:} $27$, \textsf{\bfseries 32579:} $66$, \textsf{\bfseries 32587:} $32$, \textsf{\bfseries 32603:} $292$, \textsf{\bfseries 32609:} $41$, \textsf{\bfseries 32611:} $60$, 
\textsf{\bfseries 32621:} $376$, \textsf{\bfseries 32633:} $22$, \textsf{\bfseries 32647:} $59$, \textsf{\bfseries 32653:} $70$, \textsf{\bfseries 32687:} $69$, \textsf{\bfseries 32693:} $52$, \textsf{\bfseries 32707:} $18$, \textsf{\bfseries 32713:} $30$, \textsf{\bfseries 32717:} $159$, \textsf{\bfseries 32719:} $38$, \textsf{\bfseries 32749:} $33$, \textsf{\bfseries 32771:} $7$, \textsf{\bfseries 32779:} $67$, \textsf{\bfseries 32783:} $30$, \textsf{\bfseries 32789:} $14$, \textsf{\bfseries 32797:} $126$, \textsf{\bfseries 32801:} $3$, \textsf{\bfseries 32803:} $86$, \textsf{\bfseries 32831:} $105$, \textsf{\bfseries 32833:} $71$, \textsf{\bfseries 32839:} $60$, \textsf{\bfseries 32843:} $133$, \textsf{\bfseries 32869:} $97$, \textsf{\bfseries 32887:} $6$, \textsf{\bfseries 32909:} $229$, \textsf{\bfseries 32911:} $24$, \textsf{\bfseries 32917:} $94$, \textsf{\bfseries 32933:} $125$, \textsf{\bfseries 32939:} $7$, \textsf{\bfseries 32941:} $6$, \textsf{\bfseries 32957:} $45$, \textsf{\bfseries 32969:} $12$, \textsf{\bfseries 32971:} $44$, \textsf{\bfseries 32983:} $44$, \textsf{\bfseries 32987:} $43$, \textsf{\bfseries 32993:} $57$, \textsf{\bfseries 32999:} $65$, \textsf{\bfseries 33013:} $5$, \textsf{\bfseries 33023:} $80$, \textsf{\bfseries 33029:} $2$, \textsf{\bfseries 33037:} $51$, \textsf{\bfseries 33049:} $29$, \textsf{\bfseries 33053:} $93$, \textsf{\bfseries 33071:} $110$, \textsf{\bfseries 33073:} $82$, \textsf{\bfseries 33083:} $54$, \textsf{\bfseries 33091:} $29$, \textsf{\bfseries 33107:} $22$, \textsf{\bfseries 33113:} $122$, \textsf{\bfseries 33119:} $195$, \textsf{\bfseries 33149:} $108$, \textsf{\bfseries 33151:} $42$, \textsf{\bfseries 33161:} $52$, \textsf{\bfseries 33179:} $86$, \textsf{\bfseries 33181:} $123$, \textsf{\bfseries 33191:} $11$, \textsf{\bfseries 33199:} $54$, \textsf{\bfseries 33203:} $189$, \textsf{\bfseries 33211:} $50$, \textsf{\bfseries 33223:} $111$, \textsf{\bfseries 33247:} $40$, \textsf{\bfseries 33287:} $112$, \textsf{\bfseries 33289:} $93$, \textsf{\bfseries 33301:} $70$, \textsf{\bfseries 33311:} $149$, \textsf{\bfseries 33317:} $88$, \textsf{\bfseries 33329:} $11$, \textsf{\bfseries 33331:} $18$, \textsf{\bfseries 33343:} $3$, \textsf{\bfseries 33347:} $38$, \textsf{\bfseries 33349:} $89$, \textsf{\bfseries 33353:} $43$, \textsf{\bfseries 33359:} $28$, \textsf{\bfseries 33377:} $31$, \textsf{\bfseries 33391:} $26$, \textsf{\bfseries 33403:} $43$, \textsf{\bfseries 33409:} $47$, \textsf{\bfseries 33413:} $37$, \textsf{\bfseries 33427:} $127$, \textsf{\bfseries 33457:} $20$, \textsf{\bfseries 33461:} $3$, \textsf{\bfseries 33469:} $19$, \textsf{\bfseries 33479:} $183$, \textsf{\bfseries 33487:} $260$, \textsf{\bfseries 33493:} $23$, \textsf{\bfseries 33503:} $7$, \textsf{\bfseries 33521:} $85$, \textsf{\bfseries 33529:} $104$, \textsf{\bfseries 33533:} $87$, \textsf{\bfseries 33547:} $2$, \textsf{\bfseries 33563:} $17$, \textsf{\bfseries 33569:} $84$, \textsf{\bfseries 33577:} $20$, \textsf{\bfseries 33581:} $93$, \textsf{\bfseries 33587:} $13$, \textsf{\bfseries 33589:} $26$, \textsf{\bfseries 33599:} $116$, \textsf{\bfseries 33601:} $41$, \textsf{\bfseries 33613:} $19$, \textsf{\bfseries 33617:} $56$, 
\textsf{\bfseries 33619:} $73$, \textsf{\bfseries 33623:} $15$, \textsf{\bfseries 33629:} $11$, \textsf{\bfseries 33637:} $105$, \textsf{\bfseries 33641:} $3$, \textsf{\bfseries 33647:} $35$, \textsf{\bfseries 33679:} $43$, \textsf{\bfseries 33703:} $13$, \textsf{\bfseries 33713:} $33$, \textsf{\bfseries 33721:} $95$, \textsf{\bfseries 33739:} $48$, \textsf{\bfseries 33749:} $33$, \textsf{\bfseries 33751:} $12$, \textsf{\bfseries 33757:} $54$, \textsf{\bfseries 33767:} $29$, \textsf{\bfseries 33769:} $13$, \textsf{\bfseries 33773:} $97$, \textsf{\bfseries 33791:} $17$, \textsf{\bfseries 33797:} $19$, \textsf{\bfseries 33809:} $70$, \textsf{\bfseries 33811:} $71$, \textsf{\bfseries 33827:} $17$, \textsf{\bfseries 33829:} $63$, \textsf{\bfseries 33851:} $94$, \textsf{\bfseries 33857:} $22$, \textsf{\bfseries 33863:} $45$, \textsf{\bfseries 33871:} $69$, \textsf{\bfseries 33889:} $142$, \textsf{\bfseries 33893:} $67$, \textsf{\bfseries 33911:} $62$, \textsf{\bfseries 33923:} $28$, \textsf{\bfseries 33931:} $98$, \textsf{\bfseries 33937:} $5$, \textsf{\bfseries 33941:} $40$, \textsf{\bfseries 33961:} $19$, \textsf{\bfseries 33967:} $26$, \textsf{\bfseries 33997:} $68$, \textsf{\bfseries 34019:} $23$, \textsf{\bfseries 34031:} $47$, \textsf{\bfseries 34033:} $31$, \textsf{\bfseries 34039:} $11$, \textsf{\bfseries 34057:} $164$, \textsf{\bfseries 34061:} $85$, \textsf{\bfseries 34123:} $20$, \textsf{\bfseries 34127:} $28$, \textsf{\bfseries 34129:} $23$, \textsf{\bfseries 34141:} $40$, \textsf{\bfseries 34147:} $74$, \textsf{\bfseries 34157:} $20$, \textsf{\bfseries 34159:} $42$, \textsf{\bfseries 34171:} $31$, \textsf{\bfseries 34183:} $24$, \textsf{\bfseries 34211:} $8$, \textsf{\bfseries 34213:} $43$, \textsf{\bfseries 34217:} $69$, \textsf{\bfseries 34231:} $116$, \textsf{\bfseries 34253:} $78$, \textsf{\bfseries 34259:} $123$, \textsf{\bfseries 34261:} $68$, \textsf{\bfseries 34267:} $72$, \textsf{\bfseries 34273:} $5$, \textsf{\bfseries 34283:} $22$, \textsf{\bfseries 34297:} $83$, \textsf{\bfseries 34301:} $267$, \textsf{\bfseries 34303:} $53$, \textsf{\bfseries 34313:} $45$, \textsf{\bfseries 34319:} $38$, \textsf{\bfseries 34327:} $57$, \textsf{\bfseries 34337:} $101$, \textsf{\bfseries 34351:} $88$, \textsf{\bfseries 34361:} $88$, \textsf{\bfseries 34367:} $56$, \textsf{\bfseries 34369:} $93$, \textsf{\bfseries 34381:} $17$, \textsf{\bfseries 34403:} $35$, \textsf{\bfseries 34421:} $18$, \textsf{\bfseries 34429:} $149$, \textsf{\bfseries 34439:} $39$, \textsf{\bfseries 34457:} $5$, \textsf{\bfseries 34469:} $65$, \textsf{\bfseries 34471:} $60$, \textsf{\bfseries 34483:} $109$, \textsf{\bfseries 34487:} $10$, \textsf{\bfseries 34499:} $37$, \textsf{\bfseries 34501:} $7$, \textsf{\bfseries 34511:} $53$, \textsf{\bfseries 34513:} $11$, \textsf{\bfseries 34519:} $73$, \textsf{\bfseries 34537:} $31$, \textsf{\bfseries 34543:} $19$, \textsf{\bfseries 34549:} $117$, \textsf{\bfseries 34583:} $43$, \textsf{\bfseries 34589:} $18$, \textsf{\bfseries 34591:} $59$, \textsf{\bfseries 34603:} $42$, \textsf{\bfseries 34607:} $212$, \textsf{\bfseries 34613:} $13$, \textsf{\bfseries 34631:} $7$, \textsf{\bfseries 34649:} $12$, \textsf{\bfseries 34651:} $140$, 
\textsf{\bfseries 34667:} $102$, \textsf{\bfseries 34673:} $13$, \textsf{\bfseries 34679:} $113$, \textsf{\bfseries 34687:} $96$, \textsf{\bfseries 34693:} $56$, \textsf{\bfseries 34703:} $7$, \textsf{\bfseries 34721:} $23$, \textsf{\bfseries 34729:} $92$, \textsf{\bfseries 34739:} $171$, \textsf{\bfseries 34747:} $47$, \textsf{\bfseries 34757:} $3$, \textsf{\bfseries 34759:} $101$, \textsf{\bfseries 34763:} $8$, \textsf{\bfseries 34781:} $44$, \textsf{\bfseries 34807:} $47$, \textsf{\bfseries 34819:} $69$, \textsf{\bfseries 34841:} $29$, \textsf{\bfseries 34843:} $18$, \textsf{\bfseries 34847:} $189$, \textsf{\bfseries 34849:} $76$, \textsf{\bfseries 34871:} $46$, \textsf{\bfseries 34877:} $11$, \textsf{\bfseries 34883:} $32$, \textsf{\bfseries 34897:} $15$, \textsf{\bfseries 34913:} $17$, \textsf{\bfseries 34919:} $38$, \textsf{\bfseries 34939:} $53$, \textsf{\bfseries 34949:} $18$, \textsf{\bfseries 34961:} $75$, \textsf{\bfseries 34963:} $17$, \textsf{\bfseries 34981:} $39$, \textsf{\bfseries 35023:} $155$, \textsf{\bfseries 35027:} $2$, \textsf{\bfseries 35051:} $8$, \textsf{\bfseries 35053:} $45$, \textsf{\bfseries 35059:} $75$, \textsf{\bfseries 35069:} $51$, \textsf{\bfseries 35081:} $60$, \textsf{\bfseries 35083:} $41$, \textsf{\bfseries 35089:} $117$, \textsf{\bfseries 35099:} $183$, \textsf{\bfseries 35107:} $57$, \textsf{\bfseries 35111:} $23$, \textsf{\bfseries 35117:} $3$, \textsf{\bfseries 35129:} $12$, \textsf{\bfseries 35141:} $40$, \textsf{\bfseries 35149:} $6$, \textsf{\bfseries 35153:} $14$, \textsf{\bfseries 35159:} $34$, \textsf{\bfseries 35171:} $120$, \textsf{\bfseries 35201:} $43$, \textsf{\bfseries 35221:} $145$, \textsf{\bfseries 35227:} $65$, \textsf{\bfseries 35251:} $52$, \textsf{\bfseries 35257:} $202$, \textsf{\bfseries 35267:} $11$, \textsf{\bfseries 35279:} $137$, \textsf{\bfseries 35281:} $69$, \textsf{\bfseries 35291:} $71$, \textsf{\bfseries 35311:} $173$, \textsf{\bfseries 35317:} $131$, \textsf{\bfseries 35323:} $3$, \textsf{\bfseries 35327:} $129$, \textsf{\bfseries 35339:} $120$, \textsf{\bfseries 35353:} $30$, \textsf{\bfseries 35363:} $23$, \textsf{\bfseries 35381:} $119$, \textsf{\bfseries 35393:} $54$, \textsf{\bfseries 35401:} $83$, \textsf{\bfseries 35407:} $28$, \textsf{\bfseries 35419:} $33$, \textsf{\bfseries 35423:} $61$, \textsf{\bfseries 35437:} $44$, \textsf{\bfseries 35447:} $11$, \textsf{\bfseries 35449:} $13$, \textsf{\bfseries 35461:} $102$, \textsf{\bfseries 35491:} $11$, \textsf{\bfseries 35507:} $17$, \textsf{\bfseries 35509:} $43$, \textsf{\bfseries 35521:} $73$, \textsf{\bfseries 35527:} $3$, \textsf{\bfseries 35531:} $275$, \textsf{\bfseries 35533:} $111$, \textsf{\bfseries 35537:} $83$, \textsf{\bfseries 35543:} $118$, \textsf{\bfseries 35569:} $41$, \textsf{\bfseries 35573:} $111$, \textsf{\bfseries 35591:} $82$, \textsf{\bfseries 35593:} $5$, \textsf{\bfseries 35597:} $19$, \textsf{\bfseries 35603:} $77$, \textsf{\bfseries 35617:} $115$, \textsf{\bfseries 35671:} $83$, \textsf{\bfseries 35677:} $212$, \textsf{\bfseries 35729:} $37$, \textsf{\bfseries 35731:} $2$, \textsf{\bfseries 35747:} $88$, \textsf{\bfseries 35753:} $12$, \textsf{\bfseries 35759:} $97$, \textsf{\bfseries 35771:} $50$, 
\textsf{\bfseries 35797:} $22$, \textsf{\bfseries 35801:} $63$, \textsf{\bfseries 35803:} $83$, \textsf{\bfseries 35809:} $85$, \textsf{\bfseries 35831:} $88$, \textsf{\bfseries 35837:} $45$, \textsf{\bfseries 35839:} $103$, \textsf{\bfseries 35851:} $22$, \textsf{\bfseries 35863:} $12$, \textsf{\bfseries 35869:} $41$, \textsf{\bfseries 35879:} $172$, \textsf{\bfseries 35897:} $31$, \textsf{\bfseries 35899:} $40$, \textsf{\bfseries 35911:} $61$, \textsf{\bfseries 35923:} $31$, \textsf{\bfseries 35933:} $161$, \textsf{\bfseries 35951:} $218$, \textsf{\bfseries 35963:} $2$, \textsf{\bfseries 35969:} $108$, \textsf{\bfseries 35977:} $17$, \textsf{\bfseries 35983:} $71$, \textsf{\bfseries 35993:} $154$, \textsf{\bfseries 35999:} $204$, \textsf{\bfseries 36007:} $84$, \textsf{\bfseries 36011:} $171$, \textsf{\bfseries 36013:} $72$, \textsf{\bfseries 36017:} $12$, \textsf{\bfseries 36037:} $22$, \textsf{\bfseries 36061:} $62$, \textsf{\bfseries 36067:} $17$, \textsf{\bfseries 36073:} $82$, \textsf{\bfseries 36083:} $62$, \textsf{\bfseries 36097:} $33$, \textsf{\bfseries 36107:} $2$, \textsf{\bfseries 36109:} $84$, \textsf{\bfseries 36131:} $30$, \textsf{\bfseries 36137:} $54$, \textsf{\bfseries 36151:} $3$, \textsf{\bfseries 36161:} $24$, \textsf{\bfseries 36187:} $31$, \textsf{\bfseries 36191:} $62$, \textsf{\bfseries 36209:} $192$, \textsf{\bfseries 36217:} $19$, \textsf{\bfseries 36229:} $31$, \textsf{\bfseries 36241:} $51$, \textsf{\bfseries 36251:} $47$, \textsf{\bfseries 36263:} $60$, \textsf{\bfseries 36269:} $8$, \textsf{\bfseries 36277:} $95$, \textsf{\bfseries 36293:} $39$, \textsf{\bfseries 36299:} $41$, \textsf{\bfseries 36307:} $2$, \textsf{\bfseries 36313:} $5$, \textsf{\bfseries 36319:} $60$, \textsf{\bfseries 36341:} $10$, \textsf{\bfseries 36343:} $93$, \textsf{\bfseries 36353:} $33$, \textsf{\bfseries 36373:} $5$, \textsf{\bfseries 36383:} $28$, \textsf{\bfseries 36389:} $33$, \textsf{\bfseries 36433:} $155$, \textsf{\bfseries 36451:} $82$, \textsf{\bfseries 36457:} $239$, \textsf{\bfseries 36467:} $19$, \textsf{\bfseries 36469:} $31$, \textsf{\bfseries 36473:} $11$, \textsf{\bfseries 36479:} $34$, \textsf{\bfseries 36493:} $103$, \textsf{\bfseries 36497:} $83$, \textsf{\bfseries 36523:} $80$, \textsf{\bfseries 36527:} $166$, \textsf{\bfseries 36529:} $97$, \textsf{\bfseries 36541:} $14$, \textsf{\bfseries 36551:} $63$, \textsf{\bfseries 36559:} $115$, \textsf{\bfseries 36563:} $156$, \textsf{\bfseries 36571:} $67$, \textsf{\bfseries 36583:} $51$, \textsf{\bfseries 36587:} $24$, \textsf{\bfseries 36599:} $123$, \textsf{\bfseries 36607:} $55$, \textsf{\bfseries 36629:} $42$, \textsf{\bfseries 36637:} $38$, \textsf{\bfseries 36643:} $20$, \textsf{\bfseries 36653:} $117$, \textsf{\bfseries 36671:} $23$, \textsf{\bfseries 36677:} $73$, \textsf{\bfseries 36683:} $38$, \textsf{\bfseries 36691:} $110$, \textsf{\bfseries 36697:} $86$, \textsf{\bfseries 36709:} $17$, \textsf{\bfseries 36713:} $11$, \textsf{\bfseries 36721:} $59$, \textsf{\bfseries 36739:} $23$, \textsf{\bfseries 36749:} $76$, \textsf{\bfseries 36761:} $6$, \textsf{\bfseries 36767:} $105$, \textsf{\bfseries 36779:} $117$, \textsf{\bfseries 36781:} $24$, \textsf{\bfseries 36787:} $123$, 
\textsf{\bfseries 36791:} $218$, \textsf{\bfseries 36793:} $15$, \textsf{\bfseries 36809:} $97$, \textsf{\bfseries 36821:} $21$, \textsf{\bfseries 36833:} $23$, \textsf{\bfseries 36847:} $42$, \textsf{\bfseries 36857:} $48$, \textsf{\bfseries 36871:} $70$, \textsf{\bfseries 36877:} $2$, \textsf{\bfseries 36887:} $10$, \textsf{\bfseries 36899:} $54$, \textsf{\bfseries 36901:} $17$, \textsf{\bfseries 36913:} $31$, \textsf{\bfseries 36919:} $176$, \textsf{\bfseries 36923:} $22$, \textsf{\bfseries 36929:} $54$, \textsf{\bfseries 36931:} $61$, \textsf{\bfseries 36943:} $12$, \textsf{\bfseries 36947:} $202$, \textsf{\bfseries 36973:} $149$, \textsf{\bfseries 36979:} $13$, \textsf{\bfseries 36997:} $22$, \textsf{\bfseries 37003:} $175$, \textsf{\bfseries 37013:} $43$, \textsf{\bfseries 37019:} $86$, \textsf{\bfseries 37021:} $33$, \textsf{\bfseries 37039:} $15$, \textsf{\bfseries 37049:} $53$, \textsf{\bfseries 37057:} $80$, \textsf{\bfseries 37061:} $103$, \textsf{\bfseries 37087:} $40$, \textsf{\bfseries 37097:} $6$, \textsf{\bfseries 37117:} $41$, \textsf{\bfseries 37123:} $19$, \textsf{\bfseries 37139:} $44$, \textsf{\bfseries 37159:} $24$, \textsf{\bfseries 37171:} $15$, \textsf{\bfseries 37181:} $178$, \textsf{\bfseries 37189:} $51$, \textsf{\bfseries 37199:} $138$, \textsf{\bfseries 37201:} $22$, \textsf{\bfseries 37217:} $33$, \textsf{\bfseries 37223:} $187$, \textsf{\bfseries 37243:} $5$, \textsf{\bfseries 37253:} $11$, \textsf{\bfseries 37273:} $7$, \textsf{\bfseries 37277:} $207$, \textsf{\bfseries 37307:} $78$, \textsf{\bfseries 37309:} $46$, \textsf{\bfseries 37313:} $10$, \textsf{\bfseries 37321:} $23$, \textsf{\bfseries 37337:} $19$, \textsf{\bfseries 37339:} $37$, \textsf{\bfseries 37357:} $41$, \textsf{\bfseries 37361:} $3$, \textsf{\bfseries 37363:} $111$, \textsf{\bfseries 37369:} $129$, \textsf{\bfseries 37379:} $40$, \textsf{\bfseries 37397:} $3$, \textsf{\bfseries 37409:} $73$, \textsf{\bfseries 37423:} $61$, \textsf{\bfseries 37441:} $34$, \textsf{\bfseries 37447:} $26$, \textsf{\bfseries 37463:} $47$, \textsf{\bfseries 37483:} $3$, \textsf{\bfseries 37489:} $47$, \textsf{\bfseries 37493:} $57$, \textsf{\bfseries 37501:} $162$, \textsf{\bfseries 37507:} $68$, \textsf{\bfseries 37511:} $124$, \textsf{\bfseries 37517:} $48$, \textsf{\bfseries 37529:} $96$, \textsf{\bfseries 37537:} $19$, \textsf{\bfseries 37547:} $6$, \textsf{\bfseries 37549:} $132$, \textsf{\bfseries 37561:} $74$, \textsf{\bfseries 37567:} $65$, \textsf{\bfseries 37571:} $107$, \textsf{\bfseries 37573:} $150$, \textsf{\bfseries 37579:} $19$, \textsf{\bfseries 37589:} $392$, \textsf{\bfseries 37591:} $183$, \textsf{\bfseries 37607:} $105$, \textsf{\bfseries 37619:} $88$, \textsf{\bfseries 37633:} $22$, \textsf{\bfseries 37643:} $44$, \textsf{\bfseries 37649:} $62$, \textsf{\bfseries 37657:} $83$, \textsf{\bfseries 37663:} $12$, \textsf{\bfseries 37691:} $6$, \textsf{\bfseries 37693:} $34$, \textsf{\bfseries 37699:} $12$, \textsf{\bfseries 37717:} $55$, \textsf{\bfseries 37747:} $50$, \textsf{\bfseries 37781:} $14$, \textsf{\bfseries 37783:} $74$, \textsf{\bfseries 37799:} $181$, \textsf{\bfseries 37811:} $13$, \textsf{\bfseries 37813:} $199$, \textsf{\bfseries 37831:} $136$, 
 
                \textsf{\bfseries 37847:} $23$, \textsf{\bfseries 37853:} $14$, \textsf{\bfseries 37861:} $206$, \textsf{\bfseries 37871:} $140$, \textsf{\bfseries 37879:} $193$, \textsf{\bfseries 37889:} $69$, \textsf{\bfseries 37897:} $11$, \textsf{\bfseries 37907:} $6$, \textsf{\bfseries 37951:} $108$, \textsf{\bfseries 37957:} $15$, \textsf{\bfseries 37963:} $63$, \textsf{\bfseries 37967:} $5$, \textsf{\bfseries 37987:} $19$, \textsf{\bfseries 37991:} $46$, \textsf{\bfseries 37993:} $15$, \textsf{\bfseries 37997:} $13$, \textsf{\bfseries 38011:} $10$, \textsf{\bfseries 38039:} $91$, \textsf{\bfseries 38047:} $24$, \textsf{\bfseries 38053:} $18$, \textsf{\bfseries 38069:} $29$, \textsf{\bfseries 38083:} $126$, \textsf{\bfseries 38113:} $77$, \textsf{\bfseries 38119:} $14$, \textsf{\bfseries 38149:} $6$, \textsf{\bfseries 38153:} $6$, \textsf{\bfseries 38167:} $21$, \textsf{\bfseries 38177:} $40$, \textsf{\bfseries 38183:} $90$, \textsf{\bfseries 38189:} $43$, \textsf{\bfseries 38197:} $45$, \textsf{\bfseries 38201:} $54$, \textsf{\bfseries 38219:} $67$, \textsf{\bfseries 38231:} $379$, \textsf{\bfseries 38237:} $85$, \textsf{\bfseries 38239:} $58$, \textsf{\bfseries 38261:} $2$, \textsf{\bfseries 38273:} $3$, \textsf{\bfseries 38281:} $148$, \textsf{\bfseries 38287:} $17$, \textsf{\bfseries 38299:} $253$, \textsf{\bfseries 38303:} $39$, \textsf{\bfseries 38317:} $7$, \textsf{\bfseries 38321:} $7$, \textsf{\bfseries 38327:} $84$, \textsf{\bfseries 38329:} $151$, \textsf{\bfseries 38333:} $18$, \textsf{\bfseries 38351:} $31$, \textsf{\bfseries 38371:} $52$, \textsf{\bfseries 38377:} $37$, \textsf{\bfseries 38393:} $5$ 
  %\item[$r=2$:] \input{./tables/pcns_14_2__0.tex}
  %\item[$r=3$:] \input{./tables/pcns_14_3__0.tex}
  %\item[$r=4$:] \input{./tables/pcns_14_4__0.tex}
  %\item[$r=5$:] \input{./tables/pcns_14_5__0.tex}
  %\item[$r=6$:] \input{./tables/pcns_14_6__0.tex}
  %\item[$r=7$:] \input{./tables/pcns_14_7__0.tex}
  %\item[$r=8$:] \textsf{\bfseries 2:} 0:\,$a^7$,\ 1:\,$1$,\ 9:\,$1$
  %\item[$r=9$:] \input{./tables/pcns_10_9__0.tex}
\end{description}
