\section{$\PCN$-Polynome}

\newcommand{\insertPCNS}[3]{%
  \subsection{$\PCN$-Polynome für $n=#1$}
  \begin{description}[leftmargin=0pt,labelindent=20pt,font=\normalsize]
    \foreach \r/\max in {#2} {%
      \item[$r=\r$:]
        \foreach \num in {0,...,\max} {%
          \input{./tables/pcns_#1_\r__\num.tex}
        }
    }
    \foreach \r in {#3} {%
      \item[$r=\r$:] \input{./tables/pcns_#1_\r__0.tex}
    }
  \end{description}}



%\tiny
\fontsize{4}{5}\selectfont


\insertPCNS{6}{}{1,...,10}
\insertPCNS{10}{1/1}{2,...,8}
\insertPCNS{12}{1/2}{2,...,7}
\insertPCNS{14}{1/4}{2,...,11}
\insertPCNS{15}{1/5}{2,3}
\insertPCNS{18}{1/10}{2,...,8}
\insertPCNS{20}{1/14}{2,...,5}
\insertPCNS{21}{1/17}{2,...,5}
\insertPCNS{22}{1/20}{2,...,4}
\insertPCNS{24}{1/28}{2,...,4}
\insertPCNS{26}{1/36}{2,...,3}
\insertPCNS{28}{1/50}{2,...,9}
\insertPCNS{30}{1/64}{2,...,3}

