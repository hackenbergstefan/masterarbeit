%% Basierend auf einer TeXnicCenter-Vorlage von Mark Müller
%%%%%%%%%%%%%%%%%%%%%%%%%%%%%%%%%%%%%%%%%%%%%%%%%%%%%%%%%%%%%%%%%%%%%%%

% Wählen Sie die Optionen aus, indem Sie % vor der Option entfernen  
% Dokumentation des KOMA-Script-Packets: scrguide

%%%%%%%%%%%%%%%%%%%%%%%%%%%%%%%%%%%%%%%%%%%%%%%%%%%%%%%%%%%%%%%%%%%%%%%
%% Optionen zum Layout des Artikels                                  %%
%%%%%%%%%%%%%%%%%%%%%%%%%%%%%%%%%%%%%%%%%%%%%%%%%%%%%%%%%%%%%%%%%%%%%%%
\documentclass[%
%a5paper,             % alle weiteren Papierformat einstellbar
%landscape,           % Querformat
%10pt,                % Schriftgröße (12pt, 11pt (Standard))
%BCOR1cm,             % Bindekorrektur, bspw. 1 cm
%DIVcalc,             % führt die Satzspiegelberechnung neu aus
%                       s. scrguide 2.4
%twoside,             % Doppelseiten
%twocolumn,           % zweispaltiger Satz
halfparskip*,       % Absatzformatierung s. scrguide 3.1
%headsepline,         % Trennline zum Seitenkopf  
%footsepline,         % Trennline zum Seitenfuß
titlepage,            % Titelei auf eigener Seite
%normalheadings,      % Überschriften etwas kleiner (smallheadings)
%idxtotoc,            % Index im Inhaltsverzeichnis
%liststotoc,          % Abb.- und Tab.verzeichnis im Inhalt
bibtotoc,           % Literaturverzeichnis im Inhalt
%abstracton,          % Überschrift über der Zusammenfassung an 
%leqno,               % Nummerierung von Gleichungen links
%fleqn,               % Ausgabe von Gleichungen linksbündig
%draft                % überlangen Zeilen in Ausgabe gekennzeichnet
DIV = 15,
headsepline,
openany,
BCOR=0.9cm,
pointlessnumbers,        %keine Punkte nach Überschriften
chapterprefix=true,
openright
]
{scrbook}
\synctex=1


%% Deutsche Anpassungen %%%%%%%%%%%%%%%%%%%%%%%%%%%%%%%%%%%%%

\usepackage[ngerman]{babel}
\usepackage[T1]{fontenc}
\usepackage[utf8]{inputenc}
%\usepackage{anyfontsize}


%% UTF8 Anpassungen
\DeclareUnicodeCharacter{201E}{\glqq}
\DeclareUnicodeCharacter{201D}{\grqqnospace}


\usepackage{lmodern} %Type1-Schriftart für nicht-englische Texte

\usepackage{amsmath,amssymb,MnSymbol}
\usepackage{xcolor}


\usepackage[inline]{enumitem}
\setlist[enumerate]{label=(\arabic*)}

\usepackage{array}


%% Packages für Grafiken & Abbildungen %%%%%%%%%%%%%%%%%%%%%%
\usepackage{graphicx} %%Zum Laden von Grafiken
%\usepackage{subfig} %%Teilabbildungen in einer Abbildung
\usepackage{calc}

\usepackage{tikz}
\usetikzlibrary{calc,positioning,backgrounds}
\pgfdeclarelayer{background}
\pgfdeclarelayer{foreground}
\pgfsetlayers{background,main,foreground}
\usepackage{tikzpagenodes}
\usepackage{tikz-cd} %%PSTricks - nicht verwendbar mit pdfLaTeX
\usepackage[colorlinks=false, pdfborder={0 0 0}]{hyperref}
\usepackage[nameinlink,german]{cleveref}

\usepackage{listings}
\usepackage[automark]{scrpage2} % Headline styles 
\usepackage{xspace}

\usepackage{longtable}

%\usepackage{nomencl}
%\renewcommand{\nomname}{Liste der verwendeten Symbole}
%\makenomenclature


%% Listings setup %%%%%%%%%%%%%%%%
\lstset{
  mathescape = true,
  basicstyle = \small\normalfont\sffamily,
  frame = tb,
  framexleftmargin = 15pt,
% numbers = left,
  numberstyle = \tiny,
% numbersep = 5pt,
  breaklines = true,
  xleftmargin = 0.1\linewidth,
  xrightmargin = 0.1\linewidth,
  escapeinside = {(*}{*)},
  tabsize=3,
  morekeywords={if, and, or, is, then, else, endif, while, endwhile, for, from,
  to, do, endfor, Input, Output, Algorithmus, return},
  morecomment=[l]{//},
  columns=flexible
}


\lstnewenvironment{ccode}[1][]{%
  \lstset{ basicstyle = \small\normalfont\ttfamily,
  frame = tb,
  framexleftmargin = 15pt,
  numbers = left,
  numberstyle = \tiny,
  numbersep = 5pt,
  breaklines = true,
  xleftmargin = 26pt,
  xrightmargin = 15pt,
  escapeinside = {(*}{*)},
  tabsize=3,
  language=C,
  commentstyle=\color{gray}\small\itshape,
  keywordstyle={\color{mycol}},
  emphstyle={[1]\color{red}},
  emphstyle={[2]\color{blue}},
  emphstyle={[3]\color{green}},
  mathescape=true,#1}}{}

\lstnewenvironment{cexample}[1][]{%
  \lstset{ basicstyle = \small\normalfont\ttfamily,
  frame = none,
  %numbers = left,
  %numberstyle = \tiny,
  %numbersep = 5pt,
  breaklines = true,
  xleftmargin = 0pt,
  xrightmargin = 0pt,
  escapeinside = {(*}{*)},
  tabsize=3,
  language=C,
  commentstyle=\color{gray}\small\itshape,
  mathescape=true,#1}}{}

\lstnewenvironment{sagecode}[1][]{%
  \lstset{ basicstyle = \small\normalfont\ttfamily,
  frame = tb,
  framexleftmargin = 15pt,
  numbers = left,
  numberstyle = \tiny,
  numbersep = 5pt,
  breaklines = true,
  xleftmargin = 26pt,
  xrightmargin = 15pt,
  escapeinside = {$}{$},
  tabsize=3,
  language=Python,
  commentstyle=\color{gray}\small\itshape,
  keywordstyle={\color{mycol}},
  emphstyle={[1]\color{red}},
  emphstyle={[2]\color{blue}},
  emphstyle={[3]\color{green}},
  mathescape=true,#1}}{}

%%%%%%%%%%%%%%%%%%%%%%%%%%%%%%%%%%%%%%%%%%%%%%%%%%%%%%%%%%%%%%%%%%%%%%%%%%%%%%%
%% Style Anpassungen
%%%%%%%%%%%%%%%%%%%%%%%%%%%%%%%%%%%%%%%%%%%%%%%%%%%%%%%%%%%%%%%%%%%%%%%%%%%%%%%
%\usepackage[sc]{mathpazo}
%\renewcommand{\sfdefault}{fav}
%\setkomafont{disposition}{\sffamily}
%%%%%%%%%%%%%%%%%%%%%%%%%%%%%%%%%%%%%%%%%%%%%%%%%%%%%%%%%%%%%%%%%%%%%%%%%%%%%%%
%\colorlet{mycol}{red!50!black}
%% color correction for printing with Brother HL-3150CDW
\colorlet{mycol}{red!40!black} 

%% Theorems %%%%%%%%%%%%%%%%%%%%%%
\usepackage{tikztheorems}
\newtikztheorem[
  style=elegantbreak,
  color=mycol,
  font header=\normalfont\sffamily\bfseries,
  counter zero=chapter,
  postskip=20pt
  ]{satz}{Satz}

\newtikztheorem[
  style=elegantbreak,
  color=mycol,
  font header=\normalfont\sffamily\bfseries,
  counter parent=satz,
  postskip=20pt
  ]{problem}{Problem}

\newtikztheorem[
  style=elegantbreak,
  color=mycol,
  font header=\normalfont\sffamily\bfseries,
  counter parent=satz,
  postskip=20pt
  ]{vermutung}{Vermutung}

\newtikztheorem[
  style=elegantbreak,
  color=mycol,
  font header=\normalfont\sffamily\bfseries,
  counter parent=satz,
  postskip=20pt
  ]{proposition}{Proposition}
  
  
\newtikztheorem[
  style=elegantbreak,
  color=mycol,
  font header=\normalfont\sffamily\bfseries,
  font body=\normalfont,
  counter parent=satz,
  postskip=20pt
  ]{definition}{Definition}

\newtikztheorem[
  style=elegantbreak,
  color=mycol,
  font header=\normalfont\sffamily\bfseries,
  font body=\normalfont,
  counter parent=satz,
  postskip=20pt
  ]{invariante}{Invariante}

\theoremstyle{plain}
\theorembodyfont{\itshape}
\newtheorem{lemma}[satz]{Lemma}
\newtheorem{kor}[satz]{Korollar}
\newtheorem{prop}[satz]{Proposition}
\newtheorem{algorithmus}[satz]{Algorithmus}

%Custom theorems
\makeatletter
\newenvironment{plainthm}[1]{\let\plthm\@undefined
\newtheorem{plthm}[satz]{#1} \begin{plthm}}{\end{plthm}}
\makeatother

\theoremstyle{plain}
\theoremheaderfont{\normalfont\sffamily\itshape}
\theorembodyfont{\normalfont}
%\theoremsymbol{{\color{gray}$\triangle$}}
\theoremsymbol{%
  \tikz[baseline]{
    \path[fill=gray] (0,0)--(1.3ex,0)--(1.3ex,1.3ex)--cycle;
    }}
\newtheorem{bemerkung}[satz]{Bemerkung}
\newtheorem{beispiel}[satz]{Beispiel}
                                     
\theoremstyle{plain}
\theoremheaderfont{\normalfont\sffamily\bfseries}
\theorembodyfont{\normalfont}                

\newtheorem{notation}[satz]{Notation}



%% Autoref Names %%%%%%%%%%%%%%%%%
\crefname{lemma}{Lemma}{Lemmas}
\crefname{equation}{Gleichung}{Gleichungen}
\crefname{definition}{Definition}{Definitionen}
\crefname{algorithmus}{Algorithmus}{Algorithmen}
\crefname{kor}{Korollar}{Korollare}
\crefname{satz}{Satz}{Sätze}

%% Amsmath options %%%%%%%%%%%%%%%%%
\numberwithin{equation}{chapter}
\allowdisplaybreaks

%% Pagestyle %%%%%%%%%%%%%%%%%%%%%%%%%%
\usepackage{scrpage2}
\addtokomafont{pagenumber}{\sffamily\color{gray!50!black}}
\addtokomafont{pagehead}{\sffamily\upshape\color{gray!50!black}}
%\setheadwidth[0pt]{textwithmarginpar}}
\pagestyle{scrheadings}
\clearscrheadfoot
\setheadsepline{0pt}
\chead{\headmark}
\ohead[\myheadcmdempty]{\myheadcmd}
\ofoot{}

\def\myheadcmd{\tikz[remember picture]{
  \node[outer xsep=1pt, outer ysep=3pt, inner ysep=0pt,
    inner xsep=5pt, font=\sffamily] (a) {\pagemark};}%
    \tikz[remember picture, overlay]{
    %\ifthenelse{\isodd{\thepage}}{
    \ifthispageodd{
      \draw[gray,line width=1pt]
        ($(a.north west)+(0pt,0)$) |- ($(a.south east)+(-3pt,-3pt)$);
      \draw[gray] ($(a.south west)+(0pt,0pt)$)  -- 
        (a.south west -| current page text area.north west);
    }{%
      \draw[gray,line width=1pt]
        ($(a.north east)+(0pt,0)$) |- ($(a.south west)+(3pt,-3pt)$);
      \draw[gray] ($(a.south east)+(0pt,0pt)$)-- 
        (a.south east -| current page text area.north east);
    }%
  }%
}

\def\myheadcmdempty{\tikz[remember picture]{
  \node[outer xsep=1pt, outer ysep=3pt, inner ysep=0pt,
    inner xsep=5pt, font=\sffamily] (a) {\pagemark};}%
    \tikz[remember picture, overlay]{
    %\ifthenelse{\isodd{\thepage}}{
    \ifthispageodd{
      \draw[gray,line width=1pt]
        ($(a.north west)+(0pt,0)$) |- ($(a.south east)+(-3pt,-3pt)$);
    }{
      \draw[gray,line width=1pt]
        ($(a.north east)+(0pt,0)$) |- ($(a.south west)+(3pt,-3pt)$);
    }
  }
}

%% titlesec %%%%%%%%%%%%%%%%%%%%%%%%%%%%%%%%%%%%%%%%%%%%%%%%%
\usepackage{titlesec}
  
\titleformat{\chapter}[display]%
  {\normalfont\sffamily\huge\bfseries\color{mycol!95}}%
  {\color{gray}\chaptertitlename~\thechapter}{0ex}{}

\titleformat{\section}[hang]%
  {\normalfont\sffamily\huge\bfseries\color{mycol!95}}%
  {\color{gray}\thesection\hspace{1ex}\raisebox{-0.1\baselineskip}{\rule{1pt}{.8\baselineskip}}}{1ex}{}

\titleformat{\subsection}[hang]%
  {\normalfont\sffamily\Large\bfseries\color{mycol!95}}%
  {\color{gray}\thesubsection\hspace{1ex}\raisebox{-0.1\baselineskip}{\rule{0.5pt}{.8\baselineskip}}}{1ex}{}

  
\titleformat{\subsubsection}[hang]%
  {\normalfont\sffamily\large\bfseries\color{mycol!95}}%
  {\color{gray}\thesubsubsection\hspace{1ex}\raisebox{-0.1\baselineskip}{\rule{0.5pt}{.8\baselineskip}}}{1ex}{}


%% Makros %%%%%%%%%%%%%%%%%%%%%%
\newcommand{\A}{\ensuremath \mathbb{A}}
\newcommand{\R}{\ensuremath \mathcal{R}}
\newcommand{\N}{\ensuremath \mathbb{N}}
\newcommand{\Q}{\ensuremath \mathbb{Q}}
\newcommand{\Z}{\ensuremath \mathbb{Z}}
\newcommand{\C}{\ensuremath \mathcal{C}}
\newcommand{\CN}{\ensuremath \mathcal{CN}}
\newcommand{\PN}{\ensuremath \mathcal{PN}}
\newcommand{\PCN}{\ensuremath \mathcal{PCN}}
\newcommand{\G}{\ensuremath \mathcal{G}}
\newcommand{\F}{\ensuremath \mathbb{F}}
\newcommand{\K}{\ensuremath \mathbb{K}}
\renewcommand{\P}{\ensuremath \mathbb{P}}
\newcommand{\Kb}{\ensuremath \overline K}
\renewcommand{\O}{\ensuremath \mathcal{O}}
\renewcommand{\L}{\ensuremath \mathcal{L}}
\renewcommand{\l}{\ensuremath \ell}
\renewcommand{\S}{\ensuremath\mathcal{S}}
\newcommand{\m}{\ensuremath \mathfrak{m}}
\newcommand{\speq}[1]{\ #1\ }
\newcommand{\const}{\ensuremath \mathrm{const}}
\newcommand{\divp}[1]{\ensuremath [#1]} %Divisorpunkt
\newcommand{\probn}[1]{{\sffamily #1}} %Problem Name
\newcommand{\obda}{{\small oBdA\space}}
\newcommand{\inv}{^{-1}}
\newcommand{\kl}[1]{^{(#1)}}
\newcommand{\sage}{\texttt{Sage}\xspace}
\newcommand{\cython}{\texttt{Cython}\xspace}
\newcommand{\python}{\texttt{Python}\xspace}
\newcommand{\Clang}{\texttt{C}\xspace}
\let\cal\mathcal
\newcommand{\lbr}{[}
\newcommand{\rbr}{]}

\let\div\undefined
\DeclareMathOperator{\charak}{char}
\DeclareMathOperator{\div}{div}
\DeclareMathOperator{\ord}{ord}
\DeclareMathOperator{\summ}{sum}
\DeclareMathOperator{\comp}{\circ}
\DeclareMathOperator{\ggT}{ggT}
\DeclareMathOperator{\kgV}{kgV}
\DeclareMathOperator{\supp}{supp}
\DeclareMathOperator{\Div0}{Div^0}
\DeclareMathOperator{\Divv}{Div}
\DeclareMathOperator{\Pic0}{Pic^0}
\DeclareMathOperator{\Gal}{Gal}
\DeclareMathOperator{\End}{End}
\DeclareMathOperator{\Ord}{Ord}
\DeclareMathOperator{\im}{im}
\DeclareMathOperator{\id}{id}
\DeclareMathOperator{\Tr}{Tr}
\DeclareMathOperator{\Nm}{Nm}
\DeclareMathOperator{\spann}{span}
\DeclareMathOperator{\ideal}{\triangleleft}
\DeclareMathOperator{\Ann}{Ann}
\DeclareMathOperator{\cl}{cl}


\newcommand{\funcdef}[1]{%
  \begin{array}[t]{>{\displaystyle}r>{\displaystyle}c>{\displaystyle}l}%
  #1\end{array}}

\renewcommand{\vec}[1]{\begin{bmatrix}#1\end{bmatrix}}

\let\grqqnospace\grqq
\renewcommand{\grqq}{\grqqnospace\space}

%% Others  %%%%%%%%%%%%%%%%%%%%%%%
\newcommand{\?}{{\huge \color{red} ?}}
\newcommand{\TODO}{{\sffamily\bfseries\large \color{red} TODO}}

\newcommand{\overbox}[2]{\ensuremath\begin{array}[b]{c}%
\makebox[0pt]{\fbox{\scriptsize#2}}\\[-2pt]\text{\small$\downarrow$}\\[-3pt]%
{\displaystyle#1}\end{array}}%

\let\marginparold\marginpar
\renewcommand{\marginpar}[1]{%
  \marginparold[\raggedleft\scriptsize\sffamily #1]%
  {\raggedright\scriptsize\sffamily #1}}


%% Bibliographiestil %%%%%%%%%%%%%%%%%%%%%%%%%%%%%%%%%%%%%%%%%%%%%%%%%%
\usepackage[german=guillemets]{csquotes}
\usepackage[style=numeric,backend=biber,
  isbn=false,
  firstinits=true
  %articletitle=true,
  %sorting=nty
  ]{biblatex}
\addbibresource{bib.bib}
% makes volume of journal bold and adds colon
\DeclareFieldFormat[article]{volume}{\textbf{#1}}
\DeclareFieldFormat[article,incollection]{pages}{#1}
\DeclareFieldFormat[book]{publisher}{(#1)}
%\DeclareFieldFormat[article]{date}{\addcomma #1}
\DefineBibliographyStrings{ngerman}{%
  editors={(Hrsg.)\adddot},
  andothers={et al\adddot},
  byeditor={Hrsg.:}
}

\DeclareSourcemap{
  \maps[datatype=bibtex]{
    \map{
      \step[fieldsource=origvolume]
        \step[fieldset=usera,origfieldval]
    }
    \map{
      \step[fieldsource=origpages]
        \step[fieldset=userb,origfieldval]
    }
  }
}
\DeclareFieldFormat[customa]{usera}{\textbf{#1}}
\DeclareFieldFormat[customa]{origyear}{#1}
\DeclareFieldFormat[customa]{userb}{#1}
\DeclareFieldFormat[customa]{pages}{#1}


\DeclareBibliographyDriver{article}{%
  \usebibmacro{bibindex}%
  \usebibmacro{begentry}%
  \usebibmacro{author/translator+others}%
  \setunit{\labelnamepunct}\newblock
  \usebibmacro{title}%
  \newunit
  \printlist{language}%
  \newunit\newblock
  \usebibmacro{byauthor}%
  \newunit\newblock
  \usebibmacro{bytranslator+others}%
  \newunit\newblock
  \printfield{version}%
  \newunit\newblock
  %\usebibmacro{in:}%
  \usebibmacro{journal+issuetitle}%
  \newunit
  \usebibmacro{byeditor+others}%
  \newunit
  \usebibmacro{note+pages}%
  \newunit\newblock
  \iftoggle{bbx:isbn}
    {\printfield{issn}}
    {}%
  \newunit\newblock
  \usebibmacro{doi+eprint+url}%
  \newunit\newblock
  \usebibmacro{addendum+pubstate}%
  \setunit{\bibpagerefpunct}\newblock
  \usebibmacro{pageref}%
  \newunit\newblock
  \iftoggle{bbx:related}
    {\usebibmacro{related:init}%
     \usebibmacro{related}}
    {}%
  \usebibmacro{finentry}}

\DeclareBibliographyDriver{book}{%
  \usebibmacro{bibindex}%
  \usebibmacro{begentry}%
  \usebibmacro{author/editor+others/translator+others}%
  \setunit{\labelnamepunct}\newblock
  \usebibmacro{maintitle+title}%
  \newunit
  \printlist{language}%
  \newunit\newblock
  \usebibmacro{byauthor}%
  \newunit\newblock
  \usebibmacro{byeditor+others}%
  \newunit\newblock
  \printfield{edition}%
  \newunit
  \iffieldundef{maintitle}
    {\printfield{volume}%
     \printfield{part}}
    {}%
  \newunit
  \printfield{volumes}%
  \newunit\newblock
  \usebibmacro{series+number}%
  \newunit\newblock
  \printfield{note}%
  \newunit\newblock
  \usebibmacro{publisher+location+date}%
  \newunit\newblock
  \usebibmacro{chapter+pages}%
  \newunit
  \printfield{pagetotal}%
  \newunit\newblock
  \iftoggle{bbx:isbn}
    {\printfield{isbn}}
    {}%
  \newunit\newblock
  \usebibmacro{doi+eprint+url}%
  \newunit\newblock
  \usebibmacro{addendum+pubstate}%
  \setunit{\bibpagerefpunct}\newblock
  \usebibmacro{pageref}%
  \newunit\newblock
  \iftoggle{bbx:related}
    {\usebibmacro{related:init}%
     \usebibmacro{related}}
    {}%
  \usebibmacro{finentry}}

\renewbibmacro*{editor+others}{%
  \ifboolexpr{
    test \ifuseeditor
    and
    not test {\ifnameundef{editor}}
  }
    {\printnames{editor}%
     \setunit{\space}%
     \usebibmacro{editor+othersstrg}%
     \clearname{editor}}
    {}}


\DeclareBibliographyDriver{incollection}{%
  \usebibmacro{bibindex}%
  \usebibmacro{begentry}%
  \usebibmacro{author/translator+others}%
  \setunit{\labelnamepunct}\newblock
  \usebibmacro{title}%
  \newunit
  \printlist{language}%
  \newunit\newblock
  \usebibmacro{byauthor}%
  \newunit\newblock
  \usebibmacro{in:}%
  \usebibmacro{maintitle+booktitle}%
  \newunit\newblock
  \usebibmacro{byeditor+others}%
  \newunit\newblock
  \printfield{edition}%
  \newunit
  \iffieldundef{maintitle}
    {\printfield{volume}%
     \printfield{part}}
    {}%
  \newunit
  \printfield{volumes}%
  \newunit\newblock
  \usebibmacro{series+number}%
  \newunit\newblock
  \printfield{note}%
  \newunit\newblock
  \usebibmacro{publisher+location+date}%
  \newunit\newblock
  \usebibmacro{chapter+pages}%
  \newunit\newblock
  \iftoggle{bbx:isbn}
    {\printfield{isbn}}
    {}%
  \newunit\newblock
  \usebibmacro{doi+eprint+url}%
  \newunit\newblock
  \usebibmacro{addendum+pubstate}%
  \setunit{\bibpagerefpunct}\newblock
  \usebibmacro{pageref}%
  \newunit\newblock
  \iftoggle{bbx:related}
    {\usebibmacro{related:init}%
     \usebibmacro{related}}
    {}%
  \usebibmacro{finentry}}

\DeclareBibliographyDriver{customa}{%
  \usebibmacro{bibindex}%
  \usebibmacro{begentry}%
  \usebibmacro{author/translator+others}%
  \setunit{\labelnamepunct}\newblock
  \usebibmacro{title}%
  \newunit
  \printlist{language}%
  \newunit\newblock
  \usebibmacro{byauthor}%
  \newunit\newblock
  \usebibmacro{journal}
  \newunit
  \printfield{usera}
  \addcomma\addspace
  \printfield{origyear}
  \addcomma\addspace
  \printfield{userb}
  \newunit\newblock
  %\usebibmacro{in:}%
  \usebibmacro{maintitle+booktitle}%
  \newunit\newblock
  \usebibmacro{byeditor+others}%
  \newunit\newblock
  \printfield{edition}%
  \newunit
  \iffieldundef{maintitle}
    {\printfield{volume}%
     \printfield{part}}
    {}%
  \newunit
  \printfield{volumes}%
  \newunit\newblock
  \usebibmacro{series+number}%
  \newunit\newblock
  \printfield{note}%
  \newunit\newblock
  \usebibmacro{publisher+location+date}%
  \newunit\newblock
  \usebibmacro{chapter+pages}%
  \newunit\newblock
  \iftoggle{bbx:isbn}
    {\printfield{isbn}}
    {}%
  \newunit\newblock
  \usebibmacro{doi+eprint+url}%
  \newunit\newblock
  \usebibmacro{addendum+pubstate}%
  \setunit{\bibpagerefpunct}\newblock
  \usebibmacro{pageref}%
  \newunit\newblock
  \iftoggle{bbx:related}
    {\usebibmacro{related:init}%
     \usebibmacro{related}}
    {}%
  \usebibmacro{finentry}}

\newbibmacro*{publisher+location+date}{%
  %\printlist{location}%
  %\iflistundef{publisher}
    %{\setunit*{\addcomma\space}}
    %{\setunit*{\addcolon\space}}%
  \printlist{publisher}%
  \setunit*{\addcomma\space}%
  \printlist{location}
  \setunit*{\addcomma\space}%
  \usebibmacro{date}%
  \newunit}


\newbibmacro*{volume+number+eid}{%
  \printfield{volume}%
  \setunit*{\adddot}%
  \printfield{number}%
  \setunit{\addcomma\space}%
  \printfield{eid}}

\newbibmacro*{issue+date}{%
  %\printtext[parens]{%
    \iffieldundef{issue}
      {\setunit{\addcomma\space}%
       \usebibmacro{date}}
      {\printfield{issue}%
       \setunit*{\addspace}%
       \usebibmacro{date}}%
  %}%
  \newunit}


\newbibmacro{origvolume}{%
  \printfield{origvolume}%
  \setunit*{\adddot}%
  \printfield{orignumber}%
  \setunit{\addcomma\space}%
  \printfield{eid}}



\begin{document}

%% Trennungen %%%%%%%%%%%%%%%%%%%%%%%%%%%%%%%%%%%%%%%%%%%%%%%%%%%%%%%%%
%%%%%%%%%%%%%%%%%%%%%%%%%%%%%%%%%%%%%%%%%%%%%%%%%%%%%%%%%%%%%%%%%%%%%%%

\frontmatter


%%%%%%%%%%%%%%%%%%%%%%%%%%%%%%%%%%%%%%%%%%%%%%%%%%%%%%%%%%%%%%%%%%%%%%%
%% Ihr Artikel                                                       %%
%%%%%%%%%%%%%%%%%%%%%%%%%%%%%%%%%%%%%%%%%%%%%%%%%%%%%%%%%%%%%%%%%%%%%%%

%% eigene Titelseitengestaltung %%%%%%%%%%%%%%%%%%%%%%%%%%%%%%%%%%%%%%%    
\begin{titlepage}
\thispagestyle{empty}
\newcommand{\Rule}{\rule{\textwidth}{1mm}}
\begin{center}\sffamily\bfseries
\LARGE\textcolor{gray}{Masterarbeit\\ im Studiengang Mathematik}
\par\vspace*{2cm}
\tikz[baseline]{ \node[anchor=base, minimum width=\textwidth,
  text=mycol,
  inner xsep=5pt,
  inner ysep=10pt,
  align=center,
  font=\Huge]
  (main title) {Theoretische und experimentelle\\ Untersuchungen
    zu Normalbasen\\ für Erweiterungen endlicher Körper};
  \draw[overlay, line width=1mm, gray,
    line cap=round]
    (main title.south west)
    ++(0,-10pt) -- +(\textwidth,0)
    (main title.north west)
    ++(0,10pt) -- +(\textwidth,0);
}
\vfill
\normalfont\sffamily\large vorgelegt von\par
\bfseries\LARGE Stefan Hackenberg
\vfill
\normalfont\sffamily\large am\\
\bfseries\Large Institut für Mathematik\\
\normalfont\sffamily\large der\\
\bfseries\Large Universität Augsburg
\vfill
\normalfont\sffamily\large Erstprüfer \\
\bfseries\Large Prof. Dr. Dirk Hachenberger\par
\vspace*{0.3cm}
\normalfont\sffamily\large Zweitprüfer \\
\bfseries\Large Prof. Dr. Dieter Jungnickel\par
\vfill
\bfseries\Large Dezember 2014
\end{center}
\end{titlepage}


%% Angaben zur Standardformatierung des Titels %%%%%%%%%%%%%%%%%%%%%%%%
%\titlehead{Titelkopf }
%\and{Der Name des Co-Autoren}
%\thanks{Fußnote}     % entspr. \footnote im Fließtext
%\date{}              % falls anderes, als das aktuelle gewünscht

%% Widmungsseite %%%%%%%%%%%%%%%%%%%%%%%%%%%%%%%%%%%%%%%%%%%%%%%%%%%%%%

%\maketitle             % Titelei wird erzeugt

%% Zusammenfassung nach Titel, vor Inhaltsverzeichnis %%%%%%%%%%%%%%%%%
%\begin{abstract}
% Für eine kurze Zusammenfassung des folgenden Artikels.
% Für die Überschrift s. \documentclass[abstracton].
%\end{abstract}


\cleardoubleemptypage
\tableofcontents


%% Der Text %%%%%%%%%%%%%%%%%%%%%%%%%%%%%%%%%%%%%%%%%%%%%%%%%%%%%%%%%%%
\cleardoubleemptypage

\input{intro}


\mainmatter
%\printnomenclature

\addchap*{Liste verwendeter Symbole}

\renewcommand{\arraystretch}{1.2}

\subsubsection{Allgemeines}
\begin{longtable}[h]{>{\raggedright}p{4cm}@{\qquad}p{10cm}}
$\N$ & Menge der natürlichen Zahlen mit $0$\\
$\Z$ & Menge der ganzen Zahlen\\
$\Z_m$ & ganze Zahlen modulo $m$\\
$F[x]$ & univariater Polynomring über $F$\\
$F[x]_{<k}$ & Menge der univariaten Polynome über $F$ vom Grad kleiner $k$\\
$\varphi$ & Eulersche Phifunktion (\thref{def:euler_phi})\\
$M^\ast$ & Menge $M$ ohne das Element $0$\\
\end{longtable}

\subsubsection{Gruppentheoretisches}
\begin{longtable}[h]{>{\raggedright}p{4cm}@{\qquad}p{10cm}}
$\langle a \rangle$ & von $a$ erzeugte Gruppe\\
$\ord(u)$ & gruppentheoretische Ordnung des Elements $u$\\
\end{longtable}

\subsubsection{Ringe und Moduln}
\begin{longtable}[h]{>{\raggedright}p{4cm}@{\qquad}p{10cm}}
$(r)$ & von $r$ erzeugtes Ideal\\
$M^\times$ & Einheiten von $M$\\
$V_r$ & die von $r$ annihilierten Elemente
  (\thref{def:V_r})\\
$\Ann_R(S)$ & Annihilator von $S$ in $R$
  (\thref{def:annihilator})\\
$f(x)\cdot v$ & Multiplikation im $K[x]$-Modul bzgl. $\tau$
  (\thref{def:V_tau})\\
$\Ord_\tau(v)$ & $\tau$-Ordnung von $v$ 
  (\thref{def:tau_ordnung})\\
\end{longtable}

\subsubsection{Zahlentheoretisches}
\begin{longtable}[h]{>{\raggedright}p{4cm}@{\qquad}p{10cm}}
$\nu(n)$ & quadratfreier Teil von $n$ (\thref{def:quadratfreier_teil})\\
$\ord_n(q)$ & multiplikative Ordnung von $q$ modulo $n$
  (\thref{def:multiplikative_ordnung_mod})\\
$M_q(l\bmod m)$ & $:=\{l\,q^i\bmod m:\ i\in \N\}$ 
  (\thref{def:nebenklassen_mod_m})\\
$R_q(m)$ & vollständiges Repräsentantensystem von $M_q(1\bmod n)$
  (\thref{def:nebenklassen_mod_m})\\
$r_q(l\bmod m)$ & Länge der Bahn von $M_q(l\bmod m)$
  (\thref{def:nebenklassen_mod_m})\\
$\cl_r(n)$ & Abschluss von $r$ in $n$
  (\thref{def:closure})\\
$\Delta_q(m,d)$ & $:= \tfrac{\varphi(d) \ord_m(q)}{\ord_{md}(q)}$
  (\thref{satz:zerfall_f_x_s})\\
\end{longtable}

\subsubsection{Endliche Körper}
\begin{longtable}[h]{>{\raggedright}p{4cm}@{\qquad}p{10cm}}
$\F_q$ & Endlicher Körper mit $q$ Elementen\\
$\bar F$ & algebraischer Abschluss eines Körpers $F$\\
$\Tr_{E\mid F}$ & Spurfunktion von $E$ nach $F$\\
$\Nm_{E\mid F}$ & Normfunktion von $E$ nach $F$\\
$\charak(\F_q)$ & Charakteristik von $\F_q$\\
$\sigma$ & Frobenius-Endomorphismus
  (\thref{satz:frob_fix})\\
$\Gal(E\mid F)$ & Galoisgruppe einer Körpererweiterung $E$ über $F$\\
$K\kl n$ & $n$-ter Kreisteilungskörper
  (\thref{def:kreisteilungskorper})\\
$U\kl n$ & Menge der $n$-ten Einheitswurzeln
  (\thref{def:kreisteilungskorper})\\
$C\kl n$ & Menge der primitiven $n$-ten Einheitswurzeln
  (\thref{def:primitive_einheitswurzeln})\\
$\phi_q(f)$ & Polynomversion der Eulerschen Phifunktion
  (\thref{def:polynom_phi})\\
$\C_{k,t}$ & verallgemeinerter Kreisteilungsmodul
  (\thref{def:verallgemeinerter_kreisteilungsmodul})\\
$\tau(q,k)$ & $\tau$-Teiler
  (\thref{def:tau})\\
\end{longtable}


\subsubsection{Spezielle Polynome}
\begin{longtable}[h]{>{\raggedright}p{4cm}@{\qquad}p{10cm}}
$\Phi_n(x)$ & $n$-tes Kreisteilungspolynom 
  (\thref{def:kreisteilungspolynom})\\
$\Phi_{k,t}(x)$ & verallgemeinertes Kreisteilungspolynom
  (\thref{def:verallgemeinertes_kreisteilungspolynom})\\
$D_n(x,a)$ & Dickson-Polynom erster Art
  (\thref{def:dickson})\\
$E_n(x,a)$ & Dickson-Polynom zweiter Art
  (\thref{def:dickson})\\
$f^\ast(x)$ & reziprokes Polynom von $f(x)$
  (\thref{def:reziprokes_polynom})\\
\end{longtable}

\subsubsection{Spezielle Mengen}
\begin{longtable}[h]{>{\raggedright}p{4cm}@{\qquad}p{10cm}}
$\S_q$ & Menge der Grade stark regulärer Erweiterungen über $\F_q$
  (\thref{def:stark_regular})\\
$\cal N(q,n)$, $\CN(q,n)$, $\PN(q,n)$, $\PCN(q,n)$ &
  Anzahl normaler, vollständig normaler, primitiv normaler, 
  primitiv vollständig normaler Elemente von $\F_{q^n}$ über $\F_q$
  (\thref{def:anzahlen})\\
$\G$ & Menge der $n$, für die für alle Primzahlpotenzen $q$
  ein primitiv vollständig normales Element in $\F_{q^n}$ über $\F_q$ existiert\\
\end{longtable}

\renewcommand{\arraystretch}{1}

\chapter{Grundbegriffe}
\label{chap:grundbegriffe}

Wir eröffnen den Hauptteil der Arbeit mit dem Zusammentragen
einiger grundlegender Resultate, die dem Leser sicherlich bekannt sind. 
Daher werden wir die meisten Aussagen 
lediglich ohne Beweis zitieren. Wir beginnen dabei bei der Gruppentheorie und
insbesondere zyklische Gruppen. Diese werden uns später
helfen, die Untergruppe der (primitiven) Einheitswurzeln
in \autoref{chap:kreisteilungspolynome} zu verstehen. Im anschließenden Abschnitt
rekapitulieren wir ein wenig die Galoistheorie von endlichen Körpern.
Insbesondere wollen wir wiederholen, dass die 
Galoisgruppe endlicher Körper zyklisch ist und von einem
speziellen Automorphismus erzeugt wird.

\section{Ein wenig Gruppentheorie}

%Um später den Zerfall der Kreisteilungspolynome über endlichen Körpern zu
%verstehen, wiederholen wir zunächst ein paar Aussagen über zyklische Gruppen.

\autocite[Theorem 1.15]{lidl1997finite} fasst alle notwendigen Resultate
zusammen.

\begin{satz}
  \label{satz:zykl_gruppen}
  \begin{enumerate}
    \item Jede Untergruppe einer zyklischen Gruppe ist wieder zyklisch.
    \item Sei $\langle a \rangle$ eine zyklische Gruppe der Ordnung $m$,
      so erzeugt $a^k$ eine Untergruppe der Ordnung $\frac{m}{\ggT(m,k)}$.
    \item Sei $\langle a\rangle$ eine zyklische Gruppe der Ordnung $m$ und
      $d \mid m$, so enthält $\langle a \rangle$ genau eine Untergruppe der
      Ordnung $d$.
     \item Sei $f$ ein positiver Teiler der Gruppenordnung einer endlichen
        zyklischen Gruppe $\langle a \rangle$. Dann enthält $\langle a \rangle$
        genau $\varphi(f)$ Elemente der Ordnung $f$.
        ($\varphi$ bezeichne die Eulersche Phifunktion)
     \item Eine zyklische Gruppe der Ordnung $m$ enthält genau $\varphi(m)$
        Erzeuger. Ist $a$ ein Erzeuger, so sind alle Erzeuger der Form
        $a^r$ mit $\ggT(r,m) = 1$.
  \end{enumerate}
\end{satz}

Da wir später ein paar Eigenschaften benötigen
werden, wiederholen wir die wohlbekannte Definition der Eulerschen
Phifunktion und geben wir dann die wichtigsten Rechenregeln an.

\begin{definition}[Eulersche Phifunktion]
  %\nomenclature{$\varphi$}{Eulersche Phifunktion}
  Die Funktion
  \[ \varphi: \funcdef{\N^\ast &\to& \N^\ast,\\
    n &\mapsto& |\{ a \in \N:\ 1\leq a\leq n,\ \ggT(a,n)=1 \}|}\]
  heißt \emph{Eulersche $\varphi$-Funktion}.
\end{definition}

\begin{definition}[quadratfreier Teil]
  Sei $n \in \N$ und $n = p_1^{r_1}\cdot\ldots\cdot p_l^{r_l}$ seine
  Primfaktorzerlegung. Dann heißt
  \[ \nu(n) \speq{:=} p_1\cdot \ldots\cdot p_l\]
  \emph{quadratfreier Teil von $n$}.
\end{definition}


\begin{lemma}[Rechenregeln der Eulerschen Phifunktion]
  \label{lemma:rechenregeln_phifunktion}
  Seien $a,b\in\N^\ast$, so gilt
  \begin{enumerate}
    \item $\varphi(ab) = \varphi(a)\varphi(b)$, falls $\ggT(a,b) = 1$,
    \item $a = \sum_{d\mid a} \varphi(d)$ und
    \item $\varphi(a) = \tfrac{a}{\nu(a)}\varphi(\nu(a))$.
  \end{enumerate}
\end{lemma}


Zyklische Gruppen und endliche Körper hängen eng zusammen, da bekanntlich die
multiplikative Gruppe eines endlichen Körpers immer zyklisch ist. Dies können
wir nutzen, um Erzeugern (im Sinne der Gruppentheorie) der multiplikativen
Gruppe eines endlichen Körpers einen Namen zu geben.

\begin{satz}
  \label{satz:mult_gruppe_endl_korper_zyklisch}
  Die multiplikative Gruppe eines endlichen Körpers ist zyklisch.
\end{satz}
\begin{proof}
  \autocite[Theorem 2.8]{lidl1997finite}.
\end{proof}


\begin{definition}[primitiv]
  \label{def:primitiv}
  Sei $\F_q$ ein endlicher Körper. $u\in \F_q$ heißt \emph{primitiv} 
  (oder \emph{primitives Element}), falls $\langle u \rangle = \F_q^\ast$, 
  also $u$ ein Erzeuger der multiplikativen Gruppe $\F_q^\ast$ ist.
\end{definition}


\begin{bemerkung}
  Es ist klar, dass $u\in \F_q$ genau dann primitiv ist, wenn 
  $\ord(u) = q-1$, also seine gruppentheoretische Ordnung in $\F_q^\ast$ genau
  der Gruppenordnung entspricht.
\end{bemerkung}


\section{Automorphismen über endlichen Körpern}

\begin{satz}
  \label{satz:frob_auto}
  Seien $F = \F_q$ ein endlicher Körper der Charakteristik $p\neq 0$ und 
  $n\in \N^\ast$. Dann ist
  \[ \sigma_n: \funcdef{F &\to& F\\
    a &\mapsto& a^{p^n}}\]
  ein Automorphismus auf $F$.
\end{satz}
\begin{proof}
  \autocite[Corollary 3.18]{wan2003lectures}.
\end{proof}

\begin{bemerkung}
  Insbesondere gilt also für alle $a,b\in F$, $F$ wie oben:
  \[ (a\pm b)^{p^n} = a^{p^n} \pm b^{p^n}\,.\]
\end{bemerkung}

\begin{satz}
  \label{satz:frob_fix}
  Sei $q$ eine Primzahlpotenz und $n\in \N^\ast$. Der Automorphismus
  \[ \sigma: \funcdef{\F_{q^n} &\to& \F_{q^n}\\
    a &\mapsto& a^q}\]
  hält die Elemente von $\F_q$ fest, also 
  \[ \sigma |_{\F_q} = \id_{\F_q}\,.\]
  Ferner ist $\sigma^k \neq \id_{\F_{q^n}}$ für $k=1,\ldots,n-1$, alle
  $\sigma^k$s sind paarweise verschiedene Automorphismen und 
  $\sigma^n = \id_{\F_{q^n}}$.
  $\sigma$ heißt auch \emph{Frobenius-Endomorphismus} oder
  \emph{Frobenius-Automorphismus}.
\end{satz}


Bezeichne $\Gal(E \mid F)$ die Galoisgruppe einer Galoiserweiterung $E$ über
$F$, so können wir das folgende zentrale Resultat zitieren:

\begin{satz}
  \label{satz:frob_sind_alle_autos}
  Es gilt
  \[ \Gal(\F_{q^n}\mid \F_q) \speq= \langle \sigma\rangle\,.\]
  Das bedeutet, dass es neben $\sigma^0,\sigma,\ldots,\sigma^{n-1}$ keine weiteren
  Automorphismen von $\F_{q^n}$ gibt, die $\F_q$ fixieren.
\end{satz}
\begin{proof}
  \autocite[Theorem 7.3]{wan2003lectures}.
\end{proof}

Neben der Tatsache, dass der Frobenius-Automorphismus 
alle Elemente der Galoisgruppe erzeugt, können wir
auch zeigen, dass alle Potenzen des Frobenius von $\F_q^n$ über $\F_q$ linear
unabhängig sind. Dies gilt sogar in einem größeren Kontext:

\begin{satz}[Dedekindsches Lemma]
  \label{satz:dedekindsches_lemma}
  Seinen $K,L$ zwei Körper, $n \in \N$ und $\tau_1,\ldots,\tau_n: K\to L$
  verschiedene injektive Körperhomorphismen. Dann ist für jedes $x \in K$
  \[ \{\tau_1(x),\ldots,\tau_n(x) \}\]
  linear unabhängig über $L$.
\end{satz}
\begin{proof}
  \autocite[Satz 27.2]{karpfinger2010algebra}.
\end{proof}


Mit \thref{satz:frob_sind_alle_autos} wird klar, dass für ein
irreduzibles Polynom $f(x) \in \F_q[x]$, das in $\F_{q^n}$ eine Nullstelle 
$\alpha$ besitzt, auch $\sigma^i(\alpha)$ für alle $i=1,\ldots,n-1$
Nullstellen sind. Ferner kann man sich auch relativ leicht überlegen, dass auch
jedes Polynom $f(x) \in \F_q[x]$ vom Grad $n$ eine Nullstelle in 
$\F_{q^n}$ besitzt. Beides fasst nachstehender Satz zusammen.

\begin{satz}
  \label{satz:nst_irred_polys}
  Ist $f(x) \in \F_q[x]$ ein irreduzibles Polynom vom Grad $n$. Dann 
  existiert eine Nullstelle $\alpha$ von $f(x)$ in $\F_{q^n}$, alle 
  Nullstellen von $f(x)$ sind einfach und gegeben durch
  \[ \alpha, \alpha^q, \alpha^{q^2}, \ldots, \alpha^{q^{n-1}}\ \in \F_{q^n}\,.\]
\end{satz}
\begin{proof}
  \autocite[Theorem 2.14]{lidl1997finite}.
\end{proof}


\chapter{Der Zerfall von $x^n-1$ und die Kreisteilungspolynome}
\label{chap:kreisteilungspolynome}

Sei $K$ ein beliebiger Körper der Charakteristik $p$ 
und $\bar K$ ein fest gewählter algebraischer
Abschluss. Wir wollen nun untersuchen, wie das Polynom
$x^n-1 \in K[x]$ über $K$ zerfällt. Dazu orientieren wir uns 
an \autocite{lidl1997finite} und \autocite{wan2003lectures}.

\begin{definition}[Kreisteilungskörper, Einheitswurzeln]
  \label{def:kreisteilungskorper}
  Sei $n\in\N^\ast$. Der Zerfällungskörper von $x^n-1 \in K[x]$ heißt
  der \emph{$n$-te Kreisteilungskörper} und wird mit $K\kl n$ notiert.
  Die Nullstellen von $x^n-1$ in $K\kl n$ heißen \emph{$n$-te
  Einheitswurzeln} und die Menge derer wird mit $U\kl n$ bezeichnet.
\end{definition}


\begin{satz}
  Sei $n\in \N^\ast$.
  \begin{enumerate}
    \item Sei $p\nmid n$. Dann ist $U\kl n$ eine zyklische Gruppe (bzgl. der
      Multiplikation in $K\kl n$) der Ordnung $n$.
    \item Ist $p \mid n$ und schreibt man $n = p^e m$ 
      für positive ganze Zahlen $m$ und $e$ mit $p\nmid m$, so
      ist $K\kl n = K\kl m$ und $U\kl n= U\kl m$ und die Nullstellen von
      $x^n-1 \in K[x]$ sind gerade die Elemente in $U\kl m$ jedoch jeweils mit
      Multiplizität $p^e$.
  \end{enumerate}
\end{satz}
\begin{proof}
  \autocite[Theorem 2.42]{lidl1997finite}.
\end{proof}


\begin{definition}[primitive Einheitswurzeln]
  \label{def:primitive_einheitswurzeln}
  Sei $n\in \N^\ast$ und $p\nmid n$. Dann heißen die Erzeuger von 
  $U\kl n$ \emph{primitive $n$-te Einheitswurzeln}. 
  Die Menge der primitiven $n$-ten Einheitswurzeln wird mit
  $C\kl n$ bezeichnet.
\end{definition}


\begin{definition}[Kreisteilungspolynom]
  \label{def:kreisteilungspolynom}
  Seien $n \in \N^\ast$, $p\nmid n$. Das Polynom 
  \[ \Phi_n(x) \speq{:=} \prod_{\zeta \in C\kl n} (x - \zeta)\ \in 
    K\kl n[x]\]
  heißt \emph{$n$-tes Kreisteilungspolynom}.
\end{definition}

\begin{satz}
  \label{satz:zerfall_xn_1}
  Seien $K$ ein Körper der Charakteristik $p$ und $n \in \N^\ast$ mit $p\nmid
  n$. Dann gilt:
  \begin{enumerate}
    \item $x^n-1 = \prod_{d\mid n} \Phi_d(x)$.
    \item $\Phi_n(x) \in P[x]$, wobei $P$ den Primkörper von $K$ notiere.
  \end{enumerate}
\end{satz}
\begin{proof}
  \begin{enumerate}
    \item Dies ist eine einfache Folgerung aus \thref{satz:zykl_gruppen}.
    \item Lässt sich per Induktion recht einfach beweisen 
      (vgl. \autocite[Theorem 2.45 (ii)]{lidl1997finite}).
  \end{enumerate}
\end{proof}


\begin{definition}
  \label{def:multiplikative_ordnung_mod}
  Für zwei teilerfremde natürliche Zahlen $q,n$ größer Null sei
  \[ \ord_n(q) \speq{:=} \ord([q]_n)\]
  die \emph{multiplikative Ordnung von $q$ modulo $n$},
  wobei $[q]_n$ die Restklasse von $q$ in $\Z_n$ bezeichnet und 
  die Ordnung in der Einheitengruppe von $\Z_n$, notiert 
  durch $\Z_n^\times$, zu lesen ist.
\end{definition}

\begin{lemma}[Rechenregeln der multiplikation Ordnung modulo $n$]
  \label{lemma:rechenregeln_ordn}
  Seien $m,n,q \in \N^\ast$ mit \newline $\ggT(n,q)=1$,
  $\ggT(m,q) = 1$ und $\ggT(m,n) = 1$, so gilt
  \begin{enumerate}
    \item $\ord_n(q) \mid \varphi(n)$,
    \item $\ord_{mn}(q) = \kgV\{ \ord_m(q), \ord_n(q)\}$.
  \end{enumerate}
\end{lemma}
\begin{proof}
  \begin{enumerate}
    \item Klar, da $[q]_n$ in $\Z_n^\times$ eine Untergruppe der Ordnung
      $\ord_n(q)$ erzeugt. 
      Nach dem Satz von Lagrange teilt deren Ordnung die 
      Gruppenordnung $|\Z_n^\times| = \varphi(n)$.
    \item Nach dem Chinesischen Restsatz 
      (z.B. \autocite[Kapitel 2 Satz 12]{bosch2009algebra}) ist
      \[ f: \funcdef{ \Z_{nm} &\xrightarrow{\cong}& \Z_n \times \Z_m\,,\\{} 
          [x]_{nm} &\mapsto& ([x]_n, [x]_m)}\]
      ein Isomorphismus von Ringen, 
      da algebraisch $\Z_n$ ja nichts anderes ist, als
      $\Z\big/(n)$, wobei $(n)$ das von $n$ im Ring $\Z$ erzeugte Ideal meint.
      Dieser liefert einen Gruppenhomomorphismus auf
      den Einheiten:
      \[ f: \Z_{nm}^\times \to \Z_n^\times \times \Z_m^\times\,.\]
      Nun ist per definitionem von $\ord_{\bullet}(q)$ die Behauptung
      klar.
  \end{enumerate}
\end{proof}

Damit können wir nun zu einem zentralen Resultat dieses Abschnittes kommen, das
uns über die gesamte Arbeit hinweg begleiten wird.

\begin{satz}
  \label{satz:zerfall_kreisteilungspolys}
  Seien $q$ eine Primzahlpotenz und $n\in \N^\ast$ mit $\ggT(q,n)=1$. Dann
  zerfällt das $n$-te Kreisteilungspolynom $\Phi_n(x)$ über $\F_q$ in
  \[ \frac{\varphi(n)}{\ord_n(q)}\]
  irreduzible paarweise teilerfremde Polynome von jeweils Grad $\ord_n(q)$.
\end{satz}
\begin{proof}
  Sei $f(x) \mid \Phi_n(x)$ ein irreduzibler Teiler über $\F_q$. Ist dann
  $\zeta \in C\kl n$ eine Nullstelle von $f(x)$, so sind 
  nach \thref{satz:nst_irred_polys} auch
  \[ \zeta^q, \zeta^{q^2}, \ldots, \zeta^{q^{n-1}} \]
  Nullstellen von $f(x)$. Jedoch sind offenbar nur $\ord_n(q)$ dieser 
  verschieden und da $f$ als irreduzibles Polynom wieder nach 
  \thref{satz:nst_irred_polys} nur einfache Nullstellen besitzt,
  können wir folgern, dass $\deg f = \ord_n(q)$.
  Da $f(x)$ als beliebiger irreduzibler Teiler von $\Phi_n(x)$ gewählt wurde,
  folgt sofort die Behauptung, wenn man sich überlegt, dass der Grad des
  $n$-ten Kreisteilungspolynoms per Definition gerade $\varphi(n)$ ist.
\end{proof}


Im Beweis obigen Satzes haben wir gesehen, dass die Wirkung der Galoisgruppe
$\Gal(\F_{q^n}\mid \F_q)$ auf der Menge der primitiven $n$-ten Einheitswurzeln
$C\kl n$ (die Wirkung ist selbstredend durch Einsetzen gegeben) diese in 
Teilmengen der Mächtigkeit $\ord_n(q)$ zerlegt.
Dies lässt sich natürlich auf $U\kl n$ übertragen, da ja gerade 
nach \thref{satz:zykl_gruppen}
$U\kl n = \bigcupdot_{d\mid n} C\kl d$. 
Dies motiviert nachstehende Definition.


%\begin{lemma}
  %\label{lemma:uber-pi-m-1}
  
%\end{lemma}

%\begin{lemma}
  %\label{lemma:uber-pi-m-2}
  %Für natürliche Zahlen $m$, $k$ gilt
  %\[ k \speq= \cl_m(k)\cdot l \]
  %mit $\ggT(m \cl_m(k), l) = 1$.
%\end{lemma}
%\begin{proof}
  %Dies sieht man sehr leicht, wenn man sich die Primfaktorzerlegungen der
  %gegebenen Zahlen zu Gemüte führt: Sei also 
  %\begin{align*}
    %m \quad&=\quad p_1^{\nu_1} \cdot\ldots\cdot p_s^{\nu_s}\ \cdot\
      %r_1^{\eta_1}\cdot\ldots\cdot r_t^{\eta_t}\,,\\
    %k \quad&=\quad p_1^{\nu'_1} \cdot\ldots\cdot p_s^{\nu'_s}
      %\ \cdot\ {r'}_1^{{\eta'}_1} \cdot\ldots\cdot {r'}_{t'}^{{\eta'}_{t'}}\,,
  %\end{align*}
  %für geeignete $\nu_\cdot,\nu'_\cdot,\eta_\cdot,\eta'_\cdot > 0$, wobei
  %natürlich $\ggT(r_1\cdot\ldots\cdot r_t, r'_1\cdot\ldots\cdot r'_{t'}) = 1$.
  %Dann ist $\cl_m(k)$ der größte Teiler von $k$, dessen quadratfreier Teil, den
  %quadratfreien Teil von $m$ teilt, also
  %\[ \cl_m(k) \speq= p_1^{\nu'_1}\cdot\ldots\cdot p_s^{\nu'_s}\,.\]
  %Damit ist $l = {r'}_1^{{\eta'}_1}\cdot\ldots\cdot {r'}_{t'}^{{\eta'}_{t'}}$ 
  %und die Behauptung folgt sofort.
%\end{proof} 

\begin{definition}
  \label{def:nebenklassen_mod_m}
  Für $m,q\in\N$ mit $\ggT(m,q) = 1$ und $j \in \{0,\ldots,m-1\}$ definieren wir
  \[ M_q(j\bmod m) \speq{:=} \{ j\,q^i \bmod m:\ i\in\N\} \speq= 
    \{j,\ jq,\ jq^2,\ jq^3,\ldots\ \bmod m\}\,.\]
  Ein vollständiges Repräsentantensystem von Nebenklassen 
  der Untergruppe $M_q(1\bmod m)$
  in $\Z_m$ sei mit $R_q(m)$ bezeichnet. Für $l=1,\ldots,m-1$
  bezeichne ferner
  $r_q(l \bmod m) := |\{lq^i:\ i\in \N \}|$ die Länge der zugehörigen Bahn.
\end{definition}

\begin{bemerkung}
  Per Definition von $\ord_m(q)$ ist für $l\neq 0$
  \[ r_q(l\bmod m) = \ord_{\frac{m}{\ggT(m,l)}}(q)\,. \]
\end{bemerkung}


\begin{beispiel}
  \label{beispiel:zerfall_x21_1_1}
  Wollen wir den Zerfall von $x^{21}-1$ über $\F_2$ untersuchen, so berechnen
  wir erst ein Vertretersystem von Restklassen modulo 21:
  \[\begin{array}[t]{r|l}
    l \in R_2(21) & M_2(l \bmod{21}) \\\hline
    0 & 0 \\
    1 & 1, 2, 4, 8, 11, 16 \\
    3 & 3, 6, 12 \\
    5 & 5, 10, 13, 17, 19, 20 \\
    7 & 7, 14 \\
    9 & 9, 15, 18
  \end{array}\]
  Nun wissen wir aus \thref{satz:zerfall_xn_1}, dass 
  \[ x^{21} -1 = \Phi_1(x) \cdot \Phi_3(x) \cdot \Phi_7(x) \cdot
  \Phi_{21}(x)\,.\]
  Die Nullstellen von $\Phi_{21}(x)$ partitionieren sich gerade in
  diejenigen $M_2(l\bmod{21})$ für die $l=1,5$.
  Also haben wir 
  \[\footnotesize\setlength{\arraycolsep}{2pt}\everymath{\displaystyle}
    \begin{array}{rcccc} 
      \Phi_{21}(x) &=& (x^{6} + x^{4} + x^{2} + x + 1) 
        &\cdot& (x^{6} + x^{5} + x^{4} + x^{2} + 1) \\
      &=& (x-\zeta)(x-\zeta^2)(x-\zeta^4)(x-\zeta^8)(x-\zeta^{11})
        (x-\zeta^{16}) &\cdot&
        (x-\zeta^5)(x-\zeta^{10})(x-\zeta^{13})(x-\zeta^{17})
        (x-\zeta^{19})(x-\zeta^{20})\,
    \end{array}\]
  falls wir $\zeta \in C\kl{21}$ als Nullstelle von
  $x^6+x^4+x^2+x+1$ setzen.
  Analog erhalten wir den Zerfall von $\Phi_7(x)$ durch Betrachtung der
  $M_2(l\bmod{21})$ für $l=3,9$. 
  \[\footnotesize\setlength{\arraycolsep}{2pt}\everymath{\displaystyle}
    \begin{array}{rcccc} 
      \Phi_7(x)&=& (x^{3} + x + 1) &\cdot& (x^{3} + x^{2} + 1) \\
      &=& (x-\zeta^3)(x-\zeta^{3\cdot 2})(x-\zeta^{3\cdot 4}) &\cdot&
        (x-\zeta^{3\cdot 3})(x-\zeta^{3\cdot 5})(x-\zeta^{3\cdot 6}) 
    \end{array}\]
  Sammeln wir den Rest auf, erhalten wir die Partitionierung für
  $\Phi_3(x)$ und den trivialen Fall $\Phi_1(x)$.
  \[ \begin{array}{rcc} 
    \Phi_3(x) &=& x^2 + x + 1\\
              &=& (x-\zeta^7)(x-\zeta^{14})\,,\\[10pt]
    \Phi_1(x) &=& x-1\\
             &=& x-\zeta^0\,.
    \end{array}\]
\end{beispiel}


Nun können wir uns überlegen, ob und wie unterschiedliche Kreisteilungspolynome
zusammenhängen und kommen dabei auf die bekannten Resultate, die z.B. in 
\autocite[Proposition 10.6, 10.7]{hachenberger1997finite} zu finden sind. Um
diese anzugeben, benötigen wir jedoch noch eine Definition und zitieren
einige Eigenschaften.

\begin{definition}
  \label{def:closure}
  Seien $r,n\in \N$, so definiere
  \[ \cl_r(n) \speq{:=} \max\{k \in \N^\ast:\ k \mid n,\ \nu(k) \mid \nu(r)
  \}\,.\]
\end{definition}


\begin{lemma}
  \label{lemma:cl_1}
  Seien $q>1$ eine ganze Zahl, $n\in \N^\ast$ und $r$ ein Primteiler von $q-1$.
  Dann gilt:
  \begin{enumerate}
    \item Ist $r\neq 2$ oder $q\equiv 1 \bmod 4$, so gilt
      \[ \cl_r(q^{r^n}-1) \speq= r^n\, \cl_r(q-1) \,.\]
    \item Ist $q \equiv 3 \bmod 4$, so gilt
      \[ \cl_2(q^{2^n}-1) \speq= 2^{n-1}\, \cl_2(q^2-1)\,.\]
  \end{enumerate}
\end{lemma}
\begin{proof}
  \autocite[Lemma 19.4]{hachenberger1997finite}.
\end{proof}


\begin{lemma}
  \label{lemma:cl_2}
  Seien $q,m,k > 1$ ganze Zahlen mit $\nu(k) \mid \nu(m)\mid q-1$. Dann gilt
  \begin{enumerate}
    \item Ist $m$ ungerade oder $q \equiv 1 \bmod 4$ oder $k$ ungerade, so
      gilt
      \[ \cl_m(q^k-1) \speq= k\,\cl_m(q-1)\,. \]
    \item Ist $m$ gerade, $q \equiv 3 \bmod 4$ und $k$ gerade, so ist
      \[ \cl_m(q^k-1) \speq= \tfrac k 2\, \cl_m(q^2-1)\,.\]
  \end{enumerate}
\end{lemma}
\begin{proof}
  \autocite[Lemma 19.5]{hachenberger1997finite}.
\end{proof}


\begin{satz}
  \label{satz:zusammenhang_unterschiedlicher_kreisteilungspolys}
  Seien $t,k \in \N^\ast$ und $K$ ein Körper der Charakteristik $p$.
  \begin{enumerate}
    \item Ist $\nu(t) \mid k$, so gilt
      \[ \Phi_k(x^t) \speq= \Phi_{kt}(x) \ \in K[x]\,.\]
    \item Sind $t$ und $k$ teilerfremd, so gilt
      \[ \Phi_k(x^t) \speq= \prod_{d\mid t} \Phi_{kd}(x)\ \in K[x]\,.\]
    \item Insbesondere gilt: Seien $q = p^r$ eine Primzahlpotenz,
      $t,k\in\N^\ast$ mit $p\nmid t,k$ und $\pi$ eine Potenz von $p$. Sei ferner
      $t = \cl_k(t)\cdot \bar t$, so gilt
      \[ \Phi_k(x^{t\pi}) \speq= 
        \left(\prod_{d\mid \bar t} \Phi_{k\,d\,\cl_k(t)} (x)\right)^\pi
        \ \in \F_q[x]\,. \]
  \end{enumerate}
\end{satz}
\begin{proof}
  Dass sich Potenzen von $p$ aus dem Argument herausziehen lassen, ist klar,
  da $\id_P = (.)^\pi: P \to P$ für den Primkörper $P \subset K$ nach 
  \thref{satz:frob_auto} eine lineare Abbildung ist. 
  Ferner haben nach \thref{satz:zerfall_xn_1} die
  Kreisteilungspolynome nur Koeffizienten in $P$.

  Der Kern des Beweises des Rests liegt in der Betrachtung des 
  Gruppenhomomorphismus
  \[ \psi_n:\ \bar K^\ast \to \bar K^\ast,\ x \mapsto x^n\]
  für $p\nmid n$. Denn nun ist offensichtlich, dass die Nullstellen von 
  $\Phi_k(x^t)$ gerade alle Elemente in $\bar K^\ast$, deren $t$-te Potenz
  eine primitive $k$-te Einheitswurzel ist, sind, also $\psi_t\inv(C\kl k)$.
  Ergo formulieren sich die Aussagen wie folgt um:
  \begin{enumerate}[label=(\arabic*')]
    \item Ist $\nu(t) \mid k$, so gilt
      $\psi_t\inv(C\kl k) \speq= C\kl{kt}$.
    \item Ist $\ggT(t,k)=1$, so gilt
      $\psi_t\inv(C\kl k) \speq= \bigcupdot_{d\mid t} C\kl{kd}$.
    \item Ist $k,t\in \N^\ast$ mit $p\nmid t,k$ und $k = \cl_k(t) \bar t$,
      so gilt
      \[ \psi_t\inv(C\kl k) \speq= 
        \bigcupdot_{d\mid \bar t} C\kl{k\,d\,\cl_k(t)}\]
  \end{enumerate}
  Nun ist offensichtlich, dass es reicht (3') zu zeigen. 
  Dazu notiere $t_0 := \cl_k(t)$ und seien $d\mid \bar t$ 
  und $\zeta \in C\kl{kd t_0}$ beliebig. Dann ist 
  \[ \ord(\zeta^t) = \ord((\zeta^{t_0d})^{\frac{\bar t}{d}})
    = k\,,\]
  da per Definition von $\cl_k(t)$ gerade $\ggT(\bar t, kt_0) = 1$.
  Also gilt $\psi_t(C\kl{kdt_0}) \subseteq C\kl k$
  und damit
  \[ \bigcupdot_{d\mid \bar t} C\kl{kdt_0} \speq\subseteq 
    \psi_t\inv \psi_t ( \cupdot_{d\mid \bar t} C\kl{kdt_0}) 
    \speq\subseteq \psi_t\inv(C\kl k)\]
  Die Gleichheit folgt mit einem Zählargument:
  Auf der einen Seite ist
  \[ \big|\bigcupdot_{d\mid \bar t} C\kl{kdt_0}\big| = 
    \sum_{d\mid \bar t} \varphi(kdt_0) = 
    \varphi(kt_0)\sum_{d\mid \bar t}\varphi(d) = \varphi(kt_0) \cdot \bar t
     = \varphi(k)t\,,\]
  wobei an \thref{lemma:rechenregeln_phifunktion} erinnert sei.
  Auf der anderen Seite haben wir
  \[ \big| \psi_t\inv(C \kl k)\big| \speq= t |C\kl k| \speq= t \varphi(k)\,,\]
  was den Beweis abschließt.
\end{proof}

Bevor wir den Zerfall der Kreisteilungspolynome noch genauer untersuchen,
kann man als einfache Folgerung angeben, wann genau ein
Binom $x^n-\beta \in \F_q[x]$ irreduzibel ist.

\begin{satz}
  \label{satz:binom_irreduzibel}
  Seien $\beta \in \F_q^\ast$ und $n\in \N$. Es gilt:
  $x^n -\beta \in \F_q[x]$ ist genau
  dann irreduzibel, wenn 
  \begin{enumerate}
    \item $p := \charak(\F_q) \nmid n$,
    \item $\nu(n) \mid e := \ord(\beta)$ und
    \item $\ord_{ne}(q) = n$.
  \end{enumerate}
\end{satz}
\begin{proof}
  Zunächst ist klar, dass $p \nmid n$ erfüllt sein muss, da ansonsten 
  $\beta' \in \F_q^\ast$ existiert mit $\beta'^p = \beta$ 
  (vgl. \thref{satz:frob_auto}), also wäre
  $x^n-\beta = (x^\frac n p - \beta')^p$ eine Faktorisierung.
  Nun sei $u$ eine Nullstelle von $x^n-\beta$, so lässt sich beobachten, dass
  (in Notation des Beweises von
  \thref{satz:zusammenhang_unterschiedlicher_kreisteilungspolys})
  $u \in \psi_n\inv(C\kl{e})$. Damit gilt nach
  \thref{satz:zusammenhang_unterschiedlicher_kreisteilungspolys} (3)
  \[ x^n-\beta \speq= \prod_{d\mid \bar n} \ggT(x^n-\beta, \Phi_{en_0d}(x))\]
  für $n = \cl_{e}(n) \bar n$ und diese Zerlegung ist, wie man sich analog zum
  Beweis von \thref{satz:zerfall_kreisteilungspolys} überlegen kann, 
  nicht trivial (vgl. \autocite[Proposition 5.3.5]{hachenberger2015}). 
  Damit ist (2) der Behauptung
  klar, so dass $x^n-\beta \mid \Phi_{ne}(x)$. Ferner zerfällt $\Phi_{ne}(x)$
  nach \thref{satz:zerfall_kreisteilungspolys} in 
  $\frac{\varphi(ne)}{\ord_{ne}(q)}$ irreduzible Faktoren von jeweils Grad
  $\ord_{ne}(q)$. Damit wird auch (3) der Behauptung augenblicklich klar.
\end{proof}

Für die letzte Bedingung in obigem Satz existieren noch verschiedene weitere
äquivalente Charakterisierungen, die nachstehend zu finden sind.

\begin{satz}
  \label{satz:binom_irreduzibel_aquiv}
  Seien $p$ eine Primzahl, $q$ eine Potenz von $p$ und $n,e\in \N^\ast$ 
  mit $p\nmid n$ und $\nu(n)\mid e\mid q-1$. Dann sind äquivalent:
  \begin{enumerate}
    \item $\ord_{ne}(q)  = n$,
    \item $\ggT(\tfrac{q-1}{e},n)=1$ und $q\equiv 1 \bmod 4$, falls $4\mid n$,
      und
    \item $\cl_n(q-1) \mid e$ und $q\equiv 1 \bmod 4$, falls $4\mid n$.
  \end{enumerate}
\end{satz}
\begin{proof}
  \autocite[Theorem 5.3.7]{hachenberger2015} und 
  \autocite[Corollary 5.3.8]{hachenberger2015}.
\end{proof}


Wir haben nun erkannt, wann genau Binome über einem endlichen Körper
irreduzibel sind. Doch wenn man dem Titel dieses Kapitels Glauben schenken mag,
interessieren wir uns hier vorangig für den Zerfall der Kreisteilungspolynome
über endlichen Körpern. Diese sind im Allgemeinen keine Binome, 
aber genau das Wissen über die Irreduzibiltät von
Binomen lässt uns Bedingungen formulieren, die dazu führen, dass ein
Kreisteilungspolynom über einem gegebenen endlichen Körper in irreduzible
Binome zerfällt.
Später (\autoref{sec:stark_regulare_erweiterungen}) werden wir diese
Bedingungen \emph{stark regulär} (\thref{def:stark_regular}) nennen und
einsehen, dass sie eine wesentliche Rolle bei der expliziten Konstruktion 
von Normalbasen spielen. Nachstehender Satz hat seinen Ursprung in 
\autocite[Lemma 22.2]{hachenberger1997finite}, jedoch mit anderem
Beweis.

\begin{satz}
  \label{satz:kreisteilungspolynome_binome}
  Seien $\F_q$ ein endlicher Körper von Charakteristik $p$ und $m \in \N^\ast$.
  Es gelte $p\nmid m$, $\nu(m)\mid q-1$ und $4\mid q-1$, falls $2\mid m$.
  Setze $l := \cl_m(q-1)$, $a := \ggT(l,m)$ und $I_a := \{ j\in \N^\ast:\ 
  j\leq a,\ \ggT(j,a)=1\}$. Ist $\zeta \in \F_q^\ast$ eine primitive 
  $a$-te Einheitswurzel, so ist
  \[ \Phi_m(x) \speq= \prod_{j\in I_a} \big( x^\frac m a - \zeta^j\big) \]
  die vollständige Faktorisierung des $m$-ten Kreisteilungspolynoms über
  $\F_q$.
\end{satz}
\begin{proof}
  Wir stellen fest, dass $\F_q$ in der Tat $a$-te Einheitswurzeln enthält, da 
  $\ord_a(q) = 1$. Dies ist klar, da $l$ per definitionem $q-1$ teilt und 
  $a = \ggT(l,m)$.
  Nun wollen wir uns klar werden, dass beide Seiten obiger Gleichung auch
  identisch sind: Für $j \in I_a$ durchläuft $\zeta^j$ alle primitiven $a$-ten
  Einheitswurzeln und damit sind die Nullstellen der rechten Seite der
  Gleichung gerade alle primitiven $m$-ten Einheitswurzeln.
  Bleibt die Irreduzbilität von $x^\frac m a - \zeta^j$ zu zeigen, wobei wir
  ohne Einschränkung $j=1$ wählen können: Klar ist, dass $p\nmid \frac m a$, da
  $p \nmid m$ nach Voraussetzungen. Ferner ist $\nu(l) = \nu(m)$, da wegen
  $\nu(m) \mid q-1$ gilt:
  \[ \cl_m(q-1) = \max\{k\in \N^\ast:\ k\mid q-1,\ \nu(k)= \nu(m) \}\,. \]
  Also ist auch $\nu(a) = \nu(\ggT(l,m)) = \nu(m)$ und damit folgt
  $\nu(\tfrac m a) \mid \nu(m) = \nu(a) \mid a$, womit auch (2) in 
  \thref{satz:binom_irreduzibel} erfüllt wäre.
  Da $\nu(m)\mid q-1$ ist $\cl_{\frac m a}(q-1) \mid m$, also auch
  $\cl_{\frac m a}(q-1)\mid a$. Damit wäre durch die Bedingung $q\equiv 1 \bmod
  4$, falls $2\mid m$, auch (3) in \thref{satz:binom_irreduzibel_aquiv}
  erfüllt.
\end{proof}

\begin{bemerkung}
  Man hätte obigen Beweis auch ohne das Wissen über irreduzible Binome führen
  können, in dem man sich \thref{satz:zerfall_kreisteilungspolys} bedient. So
  findet man dies auch in \autocite[Lemma 22.2]{hachenberger1997finite}.
\end{bemerkung}



Erinnert man sich nun erneut an \thref{satz:zerfall_kreisteilungspolys}, so
kann man sich die Frage stellen, ob man den
Zusammenhang unterschiedlicher Kreisteilungspolynome aus 
\thref{satz:zusammenhang_unterschiedlicher_kreisteilungspolys}
in dem Sinne verfeinern kann, dass man sich nicht für das gesamte 
Kreisteilungspolynom interessiert, sondern lediglich für einen irreduziblen
Teiler. Diese Frage beantwortet nachstehender Satz.


\begin{satz}
  \label{satz:zerfall_f_x_s}
  Seien $q=p^r$ eine Primzahlpotenz und
  $m,t\in \N$ mit $p\nmid m$, $p\nmid t$ und $\ggT(m,t) = 1$.
  Definieren wir für $d\mid t$ ferner
  \[ \Delta_q(m,d) \speq{:=} \frac{\varphi(d) \ord_m(q)}{\ord_{md}(q)}\,,\]
  so gilt:
  \begin{enumerate}
    \item Ist $f(x) \mid \Phi_m(x)$ ein über $\F_q$ irreduzibler monischer Teiler des
      $m$-ten Kreisteilungspolynoms, so gilt
      \[ f(x^t) \speq= \prod_{d\mid t} 
        \prod_{i=1}^{\Delta_q(m,d)} f_{d,i}(x)\,,\]
      wobei für alle $i=1,\ldots,\Delta_q(m,d)$ 
      \[ f_{d,i}\in \F_q[x] 
        \text{ monisch, irreduzibel und } f_{d,i}(x) \mid \Phi_{md}(x)\,.\]
      Ferner sind alle $f_{d,i}(x)$ paarweise teilerfremd.
    \item Sind $f(x) \mid \Phi_m(x)$ und $g(x) \mid \Phi_m(x)$ zwei
      teilerfremde, monische, über $\F_q$ irreduzible Teiler des $m$-ten
      Kreisteilungspolynoms, so sind auch $f(x^t)$ und $g(x^t)$ teilerfremd.
  \end{enumerate}
\end{satz}
\begin{proof}
  Wie schon im Beweis von
  \thref{satz:zusammenhang_unterschiedlicher_kreisteilungspolys} betrachten wir
  den Gruppenhomomorphismus $\psi_t$, diesmal eingeschränkt auf $U\kl{mt}$:
  \[ \psi_t:\ \funcdef{U\kl{mt} & \to& U\kl m\,,\\
    x &\mapsto& x^t\,,}\]
  was offenbar ein wohldefinierter Gruppenhomomorphismus bleibt.
  Offensichtlich ist $\ker\psi_t = U\kl t$. Da $\ggT(m,t) = 1$, also 
  $U\kl{mt} = U\kl m \odot U\kl t$ als leichte Folgerung aus
  \thref{satz:zykl_gruppen}, ist $\psi_t$ auch surjektiv.

  Ist nun $\alpha \in C\kl m$ eine Nullstelle von $f(x)$, 
  so existiert -- wiederum
  weil $m$ und $t$ teilerfremd sind -- genau ein $\beta \in C\kl m$ mit
  $\beta^t = \alpha$. Damit ist also
  \[ \psi_t\inv(\{\alpha\}) \speq= \beta\, U\kl t \speq=
    \bigcupdot_{d\mid t}\ \beta\, C\kl d \,.\]
  Notiert wieder $\sigma$ der Frobenius von
  $\F_q$, so sind nach \thref{satz:nst_irred_polys}
  $\sigma^j(\alpha)$, $j=0,\ldots,\delta-1$ für 
  $\delta=\ord_q(m)$ die Nullstellen von $f(x)$. Da $p\nmid t$ bleibt
  die Menge der $t$-ten Einheitswurzeln invariant unter $\sigma $ und 
  damit ist die Menge der Nullstellen von $f(x^t)$ gerade
  \begin{equation}
    \bigcupdot_{j=0}^{\delta-1} \sigma^j(\beta)\, U\kl t \speq=
    \bigcupdot_{j=0}^{\delta-1}\ \bigcupdot_{d\mid t}\ \beta^{q^j}\, C\kl d
    \speq=
    \bigcupdot_{d\mid t}\ \bigcupdot_{j=0}^{\delta-1}\ \beta^{q^j}\, C\kl d
    \speq{=:} \bigcupdot_{d\mid t} N_d\,. 
  \end{equation}
  %Wir bemerken kurz, dass $\alpha$ nicht in obiger Nullstellenmenge liegt, da
  %$q^i \neq t$ für alle $i=1,\ldots,\delta-1$ und damit 
  %$\beta^{q^i} \neq \alpha$ für alle $i=1,\ldots,\delta-1$, was
  %$f(x)\nmid f(x^t)$ zeigt.
  Wollen wir nun einsehen, wie $f(x^t)$ über $\F_q$ zerfällt, so müssen wir
  überlegen, wie obige Nullstellenmenge in $\sigma$-invariante Teilmengen
  zerfällt. Für jedes $d\mid t$ und jedes $j\in\{0,\ldots,\delta -1\}$ ist
  $\zeta \in \beta^{q^j}\,C\kl d$ ein Element mit $\ord(\zeta) = md$, also
  Nullstelle von $\Phi_{md}(x)$. Ferner gilt
  offenbar $\forall d\mid t:\ |N_d| = \delta\varphi(d)$
  und wir können folgern, dass $N_d$ in genau
  \[ \frac{\delta\varphi(d)}{\ord_{md}(q)} \speq= \frac{\ord_m(q)\,
    \varphi(d)}{\ord_{md}(q)} \speq=
    \Delta_q(m,d)\]
  $\sigma$-invariante Teilmengen zerfällt. $\Delta_q(m,d)$ ist in der Tat eine
  natürliche Zahl größer 0, da nach \thref{lemma:rechenregeln_ordn} (2)
  \[ \frac{\ord_m(q)\,\varphi(d)}{\ord_{md}(q)} \speq=
    \frac{\varphi(d)\, \ggT(\ord_m(q),\ord_d(q))}{\ord_d(q)}\]
  und $\ord_d(q) \mid \varphi(d)$ nach \thref{lemma:rechenregeln_ordn} (1).
  Damit ist alles in (1) gezeigt.
  Der Zusatz (2) folgt sofort, denn ist $\alpha_f \in C\kl m$ 
  bzw. $\alpha_g \in C\kl m$ Nullstelle von $f$ bzw. $g$, so gehören diese zu 
  verschiedenen $\sigma$-invarianten Teilmengen von $C\kl m$ (vgl. 
  auch \thref{beispiel:zerfall_x21_1_1}) und folglich gehören auch
  $\beta_f\in C\kl m$ bzw. $\beta_g\in C\kl m$ mit $\beta_f^t = \alpha_f$ bzw.
  $\beta_g^t = \alpha_g$ zu verschiedenen und damit disjunkten
  $\sigma$-invarianten Teilmengen von $C\kl m$.
\end{proof}

%\begin{satz}
  %\label{satz:f_x_s_ist_teiler_von_phimd}
  %Seien $q=p^r$ eine Primzahlpotenz, $F = \F_q$ ein endlicher Körper,
  %$m$ eine natürliche Zahl und $s = \ord_{\nu(m)}(q)$ mit 
  %$\ggT(s,m) = 1$. Weiter sei $s = \bar s p^\beta$ mit 
  %$\ggT(\bar s, p) = 1$. Ist ferner $f(x) \mid \Phi_m(x)$ ein über $\F_q$
  %irreduzibler monischer Teiler des $m$-ten Kreisteilungspolynoms, so gilt:
  %\[ f(x^s) \speq= \prod_{i=1}^? f_{i}(x)^{p^\beta}\,,\]
  %so dass $f_i(x)$ ein irreduzibler monischer Teiler von $\Phi_{md(i)}(x)$ ist
  %für $d:\ \{1,\ldots,?\} \to \{\text{Teiler von } \bar s\},\ 
  %i\mapsto d(i)$ surjektiv und monoton. Ferner gilt sogar 
  %$d(1) < d(2)$.
%\end{satz}
%\marginpar{Leider habe ich keine Ahnung, wie $i\mapsto d(i)$ aussieht und
  %was $?$ ist. Insbesondere gelingt es mir nicht, die Surjektivität zu
  %beweisen.}
%\begin{proof}
  %Zunächst ist klar, dass wir \obda annehmen können, dass $p \nmid s$. Ist
  %nämlich $s = \bar s p^\beta$ wie oben, so gilt
  %\[ f(x^s) = f(x^{\bar s})^{p^\beta}\,,\]
  %da $\F_q \to \F_q, x \mapsto x^p$ bekanntlich ein Ringhomomorphismus ist.
  %Ist dann $\zeta \in E$ eine primitive $(ms)$-te Einheitswurzel für
  %$E = F_e$, $e = \ord_{ms}(q)$, so ist $\ord(\zeta^s) = m$ und es gilt nach
  %\cref{} \marginpar{Reference!}
  %\[ \Phi_m(x) \speq= \prod_{j \in R_q(m)\atop \ggT(j,m) = 1}
    %\prod_{i\in M_q(j\bmod m)} \ (x - \zeta^{si} ) \]
  %Also haben wir
  %\[ f(x) \speq= \prod_{i\in M_q(j_0 \bmod m)} \ (x - \zeta^{si})\]
  %für ein $j_0\in R_q(m)$.
  %Betrachten wir nun einen Linearfaktor von $f(x^s)$, so wollen wir 
  %zeigen, dass
  %\[g(x) := (x^s - \zeta^s) \speq= \prod_{i=0}^{s-1} ( x - \zeta^{im +1} )\,.\]
  %Dies ist aber nicht schwer zu sehen, da für $i \in \{0,\ldots,s-1\}$
  %\[ g(\zeta^{in+1}) \speq= \zeta^{ism + s} - \zeta^s \speq= 0 \,.\]
  %Da $\zeta$ primitive $(ms)$-te Einheitswurzel ist, haben wir $s$ paarweise
  %verschiedene Nullstellen von $g$ gefunden und obige Behauptung gezeigt.
  %Nun betrachten wir die Aufteilung der Kreisteilungspolynome:
  %Nach \cref{} haben wir
  %\[ \Phi_m(x^s) \speq= \prod_{d \mid s} \Phi_{md}(x)\,. \]
  %Nun wollen wir einsehen, dass jedes $\zeta^{in+1}$ ein 
  %$\Phi_{md}(x)$ für $d\mid s$ \glqq trifft\grqq: \TODO
  %Sei nun $I \subseteq \{0,\ldots,s-1\}$, so dass $\forall d\mid s$ genau ein
  %$i \in I$ existiert mit $(x-\zeta^{im+1}) \mid \Phi_{md}(x)$. Dann sind wir
  %jedoch fertig, da
  %\[ f(x^s) \speq= \prod_{i \in I} \prod_{j\in M_q(i \bmod md_i)}
    %\ (x - \zeta^{(im+1)j}) \, \]
  %wobei $d_i$ gerade der zu $i \in I$ korrespondierende Teiler $d_i\mid s$ sei,
  %das kleinste Produkt ist, das $f(x^s)$ zu einem Polynom über $F$ macht.
%\end{proof}


\begin{beispiel}
  Greifen wir noch einmal \thref{beispiel:zerfall_x21_1_1} auf und betrachten
  einen irreduziblen Teiler $f(x)$ von $\Phi_7(x)$ über $\F_2$, sagen wir
  \[ f(x) \speq{:=} x^3+ x +1\,.\]
  Sei $t := 3$.
  Nun wissen wir nach \thref{satz:zerfall_f_x_s}, dass $f(x^3)$ wie folgt über
  $\F_2$ zerfällt:
  \[ f(x^3) \speq= 
    \tikz[baseline]{\node[anchor=base,rounded corners,fill=gray!5]
      (n)
      {$\displaystyle\prod_{i=1}^{\Delta_2(7,1)} f_{1,i}(x)$};
      \node[above=0pt of n, font=\scriptsize, text=gray]{$d=1 \mid 3$};}
    \cdot
    \tikz[baseline]{\node[anchor=base,rounded corners,fill=gray!5]
      (n)
      {$\displaystyle\prod_{i=1}^{\Delta_2(7,3)} f_{3,i}(x)$};
      \node[above=0pt of n, font=\scriptsize, text=gray]{$d=3 \mid 3$};}
    \speq=
    \tikz[baseline]{\node[anchor=base,rounded corners,fill=gray!5]
      (n)
      {$\displaystyle f_{1,1}(x)$};}
    \cdot
    \tikz[baseline]{\node[anchor=base,rounded corners,fill=gray!5]
      (n)
      {$\displaystyle f_{3,1}(x)$};}
    \]
    da 
    \begin{alignat*}{6}
      \Delta_2(7,1) &\speq=& \frac{\varphi(1)\ord_7(2)}{\ord_{7}(2)} &\speq=&
        \frac{1\cdot 3}{3} &\speq=& 1\,, \\
      \Delta_2(7,3) &\speq=& \frac{\varphi(3)\ord_7(2)}{\ord_{21}(2)} &\speq=&
        \frac{2\cdot 3}{6} &\speq=& 1\,. 
    \end{alignat*}
    Wir wollen nun herausfinden, welche Teiler $f_{1,1}(x)$ und $f_{3,1}(x)$ 
    von $\Phi_7(x)$ und $\Phi_{21}(x)$ sind. Wir übernehmen den Zerfall
    der Kreisteilungspolynome aus \thref{beispiel:zerfall_x21_1_1}
    und können einsehen, dass
    \[ f_{1,1}(x) \speq= x^3+x^2+1\,,\qquad
      f_{3,1}(x) \speq= x^6+x^5+x^4+x^2+1\,.\]
\end{beispiel}

%\begin{beispiel}
  %Sei $q = p = 2$ und $m = 7$, so haben wir über $\F_2$
  %\begin{align*}
    %\Phi_m(x) = \Phi_7(x) &= 
      %x^{6} + x^{5} + x^{4} + x^{3} + x^{2} + x + 1 \\
    %&= (x^{3} + x + 1) \cdot (x^{3} + x^{2} + 1)
  %\end{align*}
  %Hieraus wählen wir einen irreduziblen Teiler, sagen wir
  %\[ f(x) = x^3 + x +1\,.\]
  %Nun wollen wir den Beweis von \cref{satz:f_x_s_ist_teiler_von_phimd}
  %nachvollziehen. Dazu berechnen wir erst $\ord_7(2) = 3  =: s$. 
  %Sei dann $\zeta$ eine 
  %$(ms) = 21$-te Einheitswurzel. Um $\Phi_7$ in Termen von $\zeta$ darstellen
  %zu können, brauchen wir ein Vertretersystem von Restklassen ${}\bmod 21$
  %\[\begin{array}[t]{r|l}
    %l \in R_2(21) & M_2(l \bmod 21) \\\hline
    %0 & 0 \\
    %1 & 1, 2, 4, 8, 11, 16 \\
    %3 & 3, 6, 12 \\
    %5 & 5, 10, 13, 17, 19, 20 \\
    %7 & 7, 14 \\
    %9 & 9, 15, 18
  %\end{array}\]
  %Damit haben wir 
  %\[ \setlength{\arraycolsep}{2pt}\begin{array}{rcccc}
    %\Phi_7(x) &=& (x^{3} + x + 1) &\cdot& (x^{3} + x^{2} + 1) \\
      %&=& (x-\zeta^3)(x-\zeta^{3\cdot 2})(x-\zeta^{3\cdot 4}) &\cdot&
        %(x-\zeta^{3\cdot 3})(x-\zeta^{3\cdot 5})(x-\zeta^{3\cdot 6})
  %\end{array}\]
  %Nun folgt 
  %\[\Phi_7(x^3) 
        %= \prod_{d \mid \bar s} \Phi_{md}(x) = \Phi_7(x) \cdot \Phi_{21}(x)\]
  %für 
  %\[\footnotesize\setlength{\arraycolsep}{2pt}\everymath{\displaystyle}
    %\begin{array}{rcccc} 
      %\Phi_7(x)&=& (x^{3} + x + 1) &\cdot& (x^{3} + x^{2} + 1) \\
      %&=& (x-\zeta^3)(x-\zeta^{3\cdot 2})(x-\zeta^{3\cdot 4}) &\cdot&
        %(x-\zeta^{3\cdot 3})(x-\zeta^{3\cdot 5})(x-\zeta^{3\cdot 6}) \\[10pt]
      %\Phi_{21}(x) &=& (x^{6} + x^{4} + x^{2} + x + 1) 
        %&\cdot& (x^{6} + x^{5} + x^{4} + x^{2} + 1) \\
      %&=& (x-\zeta)(x-\zeta^2)(x-\zeta^4)(x-\zeta^8)(x-\zeta^{11})
        %(x-\zeta^{16}) &\cdot&
        %(x-\zeta^5)(x-\zeta^{10})(x-\zeta^{13})(x-\zeta^{17})
        %(x-\zeta^{19})(x-\zeta^{20}) 
    %\end{array}\]
  %Ferner ist 
  %\begin{align*}
    %f(x^s) = f(x^3) &= x^{9} + x^{3} + 1\\
    %&= (x^{3} + x^{2} + 1) \cdot (x^{6} + x^{5} + x^{4} + x^{2} + 1)
  %\end{align*}
  %und wir erkennen, dass bereits alles durch
  %\[ (x-\zeta^3)(x^3) \speq= (x-\zeta) (x-\zeta^8) (x-\zeta^{15})\]
  %festgelegt ist.
  %Hier wäre also $f(x^s) = f_1(x)f_2(x)$ mit 
  %\[ d:\ \{1,2\} \to \{1,3\},\ 1\mapsto 1,\ 2\mapsto 3\,,\]
  %was in diesem Fall sogar bijektiv ist.
%\end{beispiel}

\begin{beispiel}
  Als zweites Beispiel wollen wir uns einen Fall betrachten, 
  in dem $\Delta_q(d,m)$ nicht immer $1$ ist.
  Sei $p=q=3$, $m=5$ und $t= 4$.
  Also müssen wir ein Vertretersystem von Restklassen modulo $20$ betrachten:
  \[\begin{array}[t]{r|l}
    l \in R_3(20) & M_2(l \bmod 20) \\\hline
    0 & 0 \\
    1 & 1, 3, 7, 9 \\
    2 & 2, 6, 14, 18 \\
    4 & 4, 8, 12, 16 \\
    5 & 5, 15 \\
    10& 10 \\
    11& 11, 13, 17, 19 
    \end{array}\]
  Wir sehen, dass $\Phi_{20}(x)$ für $l=1,11$ in 
  2 Polynome von jeweils Grad $4$ zerfällt:
  \[\setlength{\arraycolsep}{2pt}\everymath{\displaystyle}
    \begin{array}{rcccc} 
      \Phi_{20}(x) &=& 
        (x^{4} + x^{3} + 2 x + 1) &\cdot& (x^{4} + 2 x^{3} + x + 1) \\
      &=& (x-\zeta^{11})(x-\zeta^{13})(x-\zeta^{17})(x-\zeta^{19}) &\cdot&
        (x-\zeta)(x-\zeta^{3})(x-\zeta^{7})(x-\zeta^{9})\,,
    \end{array}\]
  wobei wir $\zeta \in C\kl{20}$ mit Minimalpolynom $x^4+2x^3+x+1$ gewählt
  haben.
  Nun können wir den Zerfall von $\Phi_5(x)$ und $\Phi_{10}(x)$ in Termen von
  $\zeta$ anhand der Restklassen modulo $20$ beschreiben:
  \[\setlength{\arraycolsep}{2pt}\everymath{\displaystyle}
    \begin{array}{rcccc} 
      \Phi_5(x)&=& x^4 + x^3 + x^2 + x + 1\\
      &=& (x-\zeta^4)(x-\zeta^{8})(x-\zeta^{12})(x-\zeta^{16})\,, \\[10pt]
      \Phi_{10}(x) &=& x^{4} + 2 x^{3} + x^{2} + 2 x + 1 \\
      &=& (x-\zeta^2)(x-\zeta^6)(x-\zeta^{14})(x-\zeta^{18})\,.
    \end{array}\]
  Die Restklassen für $l=0,5,10$ gehören zu den Kreisteilungspolynomen 
  $\Phi_1(x), \Phi_4(x)$ und $\Phi_2(x)$, die wir für ein Beispiel zu
  \thref{satz:zerfall_f_x_s} nicht benötigen.
  Nun brauchen wir wieder einen irreduziblen monischen Teiler von
  $\Phi_m(x)$ und setzen daher $f(x) = \Phi_m(x)$.
  Wir berechnen wie oben
  \begin{alignat*}{6}
    \Delta_3(5,1) &\speq=& \frac{\varphi(1)\ord_5(3)}{\ord_{5}(3)} &\speq=&
      \frac{1\cdot 4}{4} &\speq=& 1\,, \\
    \Delta_3(5,2) &\speq=& \frac{\varphi(1)\ord_5(3)}{\ord_{10}(3)} &\speq=&
      \frac{1\cdot 4}{4} &\speq=& 1\,, \\
    \Delta_3(5,4) &\speq=& \frac{\varphi(4)\ord_5(3)}{\ord_{20}(3)} &\speq=&
      \frac{2\cdot 4}{4} &\speq=& 2\,. 
  \end{alignat*}
  Nun ist klar, wie $f(x^t)$ über $\F_3$ zerfällt:
  \[\small f(x^4) \speq= \big(
    \tikz[baseline]{\node[anchor=base,rounded corners,fill=gray!5]
      (n)
      {$\displaystyle x^4+x^3+x^2+x+1$};
      \node[above=0pt of n, font=\scriptsize, text=gray]{$d=1 \mid 4$};}
    \big)\cdot\big(
    \tikz[baseline]{\node[anchor=base,rounded corners,fill=gray!5]
      (n)
      {$\displaystyle x^4+2x^3+x^2+2x+1$};
      \node[above=0pt of n, font=\scriptsize, text=gray]{$d=2 \mid 4$};}
    \big)\cdot\big(
    \tikz[baseline]{\node[anchor=base,rounded corners,fill=gray!5]
      (n)
      {$\displaystyle (x^4+x^3+2x+1)(x^4+2x^3+x+1)$};
      \node[above=0pt of n, font=\scriptsize, text=gray]{$d=4 \mid 4$};}
    \big) \]
\end{beispiel}

%\begin{beispiel}
  %Sei $q = p = 3$ und $m = 22$, so haben wir über $\F_q$
  %\begin{align*} 
    %\Phi_m(x) = \Phi_{22}(x) \speq{&=} 
      %x^{10} + 2 x^{9} + x^{8} + 2 x^{7} + x^{6} + 2 x^{5} + x^{4} + 
      %2 x^{3} + x^{2} + 2 x + 1 \\
    %\speq{&=}
      %(x^{5} + 2 x^{3} + 2 x^{2} + 2 x + 1) \cdot 
      %(x^{5} + 2 x^{4} + 2 x^{3} + 2 x^{2} + 1) \,.
  %\end{align*}
  %Es ist ferner $s = \ord_{\nu(m)}(q) = \ord_{22}(3) = 5 = \bar s$. Sei
  %\[ f(x) \speq= x^{5} + 2 x^{3} + 2 x^{2} + 2 x + 1 \quad\in\F_q[x]\]
  %ein irreduzibler Teiler von $\Phi_m(x)$ in $\F_q[x]$.
  
  %Wählen wir nun wie im Beweis von \cref{satz:f_x_s_ist_teiler_von_phimd} eine
  %$(ms) = 100$-te Einheitswurzel $\zeta$, so müssen wir ein Vertretersystem von
  %Restklassen ${}\bmod m$ berechnen, um die passenden Zerlegungen in
  %Linearfaktoren angeben zu können:
  %\begin{center}
    %\[\begin{array}{r|l}
      %l \in R_q & M_q(l \bmod m) \\\hline
      %0 & 0 \\
      %1 & 1, 3, 5, 9, 15    \\
      %2 & 2, 6, 8, 10, 18   \\
      %4 & 4, 12, 14, 16, 20 \\
      %7 & 7, 13, 17, 19, 21 \\
      %11 & 11
    %\end{array}\]
  %\end{center}
  %Da nur $1$ und $7$ teilerfremd zu $22$ sind, ist also
  %\[\setlength{\arraycolsep}{3pt}\begin{array}{rcccc}
    %\Phi_{22}(x) &=& (x^{5} + 2 x^{3} + 2 x^{2} + 2 x + 1) &\cdot& 
      %(x^{5} + 2 x^{4} + 2 x^{3} + 2 x^{2} + 1) \\
    %&=& (x - \zeta)(x-\zeta^3)(x-\zeta^5)(x-\zeta^9)(x-\zeta^{15}) &\cdot&
      %(x-\zeta^7)(x-\zeta^{13})(x-\zeta^{17})(x-\zeta^{19})(x-\zeta^{21})
  %\end{array}\]
  %Gehen wir nun über $f(x^s)$ zu betrachten, so erhalten wir über $\F_q$
  %\begin{align*}
    %f(x^s) &= x^{25} + 2 x^{15} + 2 x^{10} + 2 x^{5} + 1\\
    %&= (x^{5} + 2 x^{3} + 2 x^{2} + 2 x + 1) \\
    %&\ \  \cdot (x^{20} + x^{18} + x^{17} + 2 x^{16} + x^{15} + 
      %x^{14} + 2 x^{10} + 2 x^{9} + 2 x^{8} + x^{7} + 2 x^{5} + x^{4} + x^{3} + 
      %2 x^{2} + x + 1)\,.
  %\end{align*}

  %Dies teilt $\Phi_{m}(x^s) = \Phi_{22}(x^5)$, was über $\F_q$ wie folgt
  %zerfällt:
  %\[\begin{array}{rcc}
    %\Phi_{22}(x^5) &=& \prod_{d\mid \bar s} \Phi_{md}(x)
      %=  \Phi_{22}(x) \cdot \Phi_{110}(x) \\
    %&=&
  %\end{array}\]
  %Wie $\Phi_{22}$ in Termen von $\zeta$ zerfällt haben wir oben bereits
  %gesehen,

%\end{beispiel}

\chapter{Moduln}

Nähern wir uns der Situation von Normalbasen in möglichst allgemeiner Form, so
beginnt die Reise bei der Betrachtung folgender Situation:
\begin{definition}[$(V,\tau)$]
  Sei $\K$ ein Körper und $V$ ein $\K$-Vektorraum und 
  $\tau \in \End_\K(V)$, so können wir $V$ als $\K[x]$-Modul auffassen:
  \[ f(x) \cdot v \speq{:=} f(\tau)(v)\]
  für alle $f(x) \in \K[x]$ und $v\in V$.
  Nenne das Paar $(V,\tau)$ \emph{$\K[x]$-Modul bzgl. $\tau$}.
\end{definition}

\begin{notation}
  Sei $\tau\in \End_\K(V)$.
  \begin{itemize}
  \item Es bezeichne $\mu_\tau$ das Minimalpolynom von 
    $\tau$, also das normierte Polynom kleinsten Grades $f(x)\in \K[x]$ mit 
    $f(\tau) = 0$.
  \item Ferner schreibe $\chi_\tau$ für das charakteristische Polynom von 
    $\tau$, also $\chi_\tau(x) := \det(x \id_V - \tau) \in \K[x]$.
  \end{itemize}
\end{notation}


\begin{bemerkung}
  Ist $\K  = F :=\F_q$ ein endlicher Körper, 
  $V = E := \F_{q^n}$ eine Körpererweiterung
  von Grad $n$ und 
  \[\tau = \sigma: \funcdef{E & \to & E\\
    v &\mapsto & v^q}\]
  der Frobenius von $E$, so ist
  \[ \mu_\tau(x) \ =\ \chi_\tau(x) \ =\ x^n - 1\,,\]
  denn: Es ist klar, dass $n = \deg \chi_\tau$ und da nach dem Satz von
  Cayley-Hamilton ist $\sigma$ Nullstelle von $\chi_\tau$. Daher teilt
  $\mu_\tau$ das charakteristische Polynom. Jedoch kennen wir das
  Minimalpolynom von $\tau$: Nach Dedekinds-Unabhängigkeitslemma ist 
  $\id_E,\sigma,\ldots,\sigma^{n-1}$ linear unabhänig über $E$, also insbesondere
  über $F$, und $\sigma^n = \id_E$.\marginpar{References!}
\end{bemerkung}


\begin{definition}[$\tau$-Ordnung, Teilmodul]
  Sei $(V,\tau)$ ein $\K[x]$-Modul. Zu jedem $v \in V$ betrachte den
  $\K[x]$-Modulhomomorphismus
  \[ \psi_w: \funcdef{\K[x] & \to & V \\
    f(x) & \mapsto & f(x)\cdot v }  \]
  Sei ferner $\dim V < \infty$.
  \begin{enumerate}
    \item Ist $\ker\psi_v = (g(x))$ für $g(x)\in \K[x]$ normiert, so heißt
      $g(x)$ \emph{$\tau$-Ordnung von $v$}\@. Ferner ist $g(x)$ eindeutig.
      Schreibe $\Ord_\tau(v) := g(x)$.
    \item $\K[\tau]\cdot v := \im{\psi_v}$ heißt der von \emph{$v$ erzeugte
      $\K[x]$-Teilmodul von $V$}.
  \end{enumerate}
\end{definition}
\marginpar{Eindeutigkeit!}


\begin{notation}
  Für $\K = \F_q$ einen endlichen Körper, $V = E \mid \F_q$ eine 
  Körpererweiterung und $\tau = \sigma$ den Frobenius-Endomorphismus schreibe
  \[ \Ord_q := \Ord_\tau \]
  und bezeichne $\Ord_q$ mit \emph{$q$-Ordnung}.
\end{notation}

\begin{lemma}
  \label{lemma:eigenschaften-tau-ordnung}
  Sei $(V,\tau)$ ein $\K[x]$-Modul. Ferner seien
  $u,v\in V$ mit $g(x) := \Ord_\tau(u)$, $h(x) := \Ord_\tau(v)$ und 
  $f(x) \in \K[x]$. Dann gilt
  \begin{enumerate}
    \item $\Ord_\tau(f(x)\cdot u) = \frac{g(x)}{\ggT(f(x),g(x))}$.
    \item $\Ord_\tau(u+v) = g(x)h(x)$, falls $\ggT(g,h) = 1$.
  \end{enumerate}
\end{lemma}
\begin{proof}
  \begin{enumerate}
    \item \TODO
    \item \TODO
  \end{enumerate}
\end{proof}

\begin{lemma}
  Sei $(V,\tau)$ ein $\K[x]$-Modul. Sei $v\in V$. Dann gilt:
  \[ \dim_\K( \K[x]\cdot v ) \speq= \deg( \Ord_\tau(v) )\,.\]
\end{lemma}
\begin{proof}
  Nach dem Homomorphiesatz gilt: \marginpar{References!}
  $ \im\psi_v \cong \K[x] \big/ \ker \psi_v$.
\end{proof}


\begin{definition}[zyklischer Modul]
  $(V,\tau)$ heißt \emph{zyklischer $\K[x]$-Modul bzgl. $w$}, falls es ein 
  $w\in \K$ gibt, sodass $K[\tau]\cdot w = V$.
\end{definition}


\begin{satz}
  Es gilt:
  \[ (V,\tau) \text{ ist ein zyklischer Modul} \quad\Leftrightarrow\quad
    \mu_\tau = \chi_\tau\]
\end{satz}
\begin{proof}
  Fassen wir zunächst ein paar einfache Tatsachen zusammen:
  Ist $u \in V$, so haben wir 
  \[ \dim(\K[x]\cdot v) = \deg( \Ord_\tau(v) ) \speq\leq 
    \deg\mu_\tau \speq\leq \deg \chi_\tau \]
  und 
  \[ \Ord_\tau (v) \speq\mid \mu_\tau \speq\mid \chi_\tau \,,\]
  wobei die erste Teilbarkeitsrelation per definitionem erfüllt ist und die
  zweite gerade der Satz von Cayley-Hamilton ist.
  Damit kommen wir zum direkten Beweis:
  \begin{description}
    \item["`$\Rightarrow$"'] Sei $V$ also zyklisch bzgl. $w$, so ist dies nach
      obigem äquivalent zu $\deg(\Ord_\tau(w)) = n$. Daraus folgt aber sofort
      $\mu_t = \chi_\tau$, da beide normiert sind.
    \item["`$\Leftarrow$"'] Zunächst sei behauptet, dass es stets ein 
      $w \in V$ gibt mit $\Ord_\tau(w) = \mu_\tau$. Sei dazu 
      $\mu_\tau(x) = \prod_{i=1}^r p_i(x)^{a_i}$ die Zerlegung in irreduzible
      Faktoren über $\K[x]$, so existieren $w_i \in V$ mit
      $\Ord_\tau(w_i) = p_i^{a_i}$. Andernfalls hätten wir einen Widerspruch 
      zum Minimalpolynom von $\tau$!
      Nach \autoref{lemma:eigenschaften-tau-ordnung} ist dann aber 
      $w := \sum_{i=1}^r w_i$ ein Element in $V$ mit $\tau$-Ordnung $\mu_\tau$.

      Ist dann also $\mu_\tau = \chi_\tau$, so hat obiges $w$ genau
      $\tau$-Ordnung $\chi_\tau$; erzeugt also $V$ als $\K[x]$-Modul.
  \end{description}
\end{proof}

Nun wollen wir spezielle Untermoduln von $V$ betrachten, welche uns guten
Aufschluss über die Struktur von $V$ geben können:

\begin{notation}
  Seien $(V,\tau)$ ein $\K[x]$-Modul und $g(x) \in \K[x]$.
  Definiere
  \[ V_g \speq{:=} \{ v \in V \mid g(x)\cdot v = 0 \}\,.\]
\end{notation}

Zunächst ist klar, dass $V_g \neq 0$ nur für $g$ Teiler von $\mu_\tau$ gelten
kann. Damit können wir folgende "`Rechenregeln"' formulieren:

\begin{lemma}
  Seien $g(x), h(x) \in \K[x]$ mit $g,h \mid \mu_\tau$. Dann gilt:
  \begin{enumerate}
    \item $V_g \cap V_h \speq= V_{\ggT(g,h)}$
    \item $V_g + V_h \speq= V_{\kgV(g,h)}$
  \end{enumerate}
\end{lemma}
\begin{proof}
  Per definitionem ist klar, dass für $v \in V_g$ gerade
  $\Ord_\tau(v) \mid g$. Also können wir $V_g$ auch wie folgt auffassen:
  \[ V_g \speq= \{ v\in V:\ \Ord_\tau(v) \mid g\} \,,\]
  Damit sind die Behautungen nach \cref{lemma:eigenschaften-tau-ordnung} klar,
  denn für $v \in V$ gilt:
  \[ v\in V_g \cap V_h \speq\Leftrightarrow 
    \Ord_\tau(v) \mid g \land \Ord_\tau(v) \mid h \speq\Leftrightarrow
    \Ord_\tau(v) \mid \ggT(g,h) \speq\Leftrightarrow v \in V_{\ggT(g,h)}\]
  und ebenso
  \[ v \in V_g + V_h \speq\Leftrightarrow 
    \Ord_\tau(v) \mid g \lor \Ord_\tau(v) \mid h \speq\Leftrightarrow
    \Ord_\tau(v) \mid \kgV(g,h) \speq\Leftrightarrow v \in V_{\kgV(g,h)}\,.\]
\end{proof}


\begin{satz}
  Sei $(V,\tau)$ ein zyklischer Modul mit $\dim(V) = n$. Sei ferner 
  $g(x)\in \K[x]$ normiert mit $g\mid \mu_\tau$. Dann gilt:
  \begin{enumerate}
    \item $V_g$ ist ein $\K[x]$-Teilmodul von $V$.
    \item Alle $\K[x]$-Teilmoduln von $V$ sind von dieser Form.
    \item $V_g$ ist zyklisch bzgl. $\tau$ mit Minimalpolynom $g(x)$.
      Ferner ist $\dim(V_g) = \deg(g)$.
    \item Die Erzeuger von $V_g$ sind genau die Elemente $v\in V$ mit 
      $\Ord_\tau(v) = g$.
  \end{enumerate}
\end{satz}
\begin{proof}
  \begin{enumerate}
    \item 
    \item
    \item
    \item
  \end{enumerate}
\end{proof}

\chapter{Normalbasen -- Ein Überblick}

Seien wieder $F := \F_q$ ein endlicher Körper von Charakteristik $p$ und 
$E := \F_{q^n} \mid F$ eine Körpererweiterung.
Wir wiederholen kurz die Definition einer \emph{Normalbasis}

\begin{definition}[normales Element, normales Polynom, Normalbasis]
  Sei $F$ ein Körper und $E \mid F$ eine endliche Galoiserweiterung von Grad
  $n$. Sei ferner $w\in E$ mit $F(w) = E$. $w$ heißt \emph{normal über $F$},
  falls
  \[ \{ \gamma(w) \mid \gamma \in G\}\]
  eine $F$-Basis von $E$ ist. 
  $\{ \gamma(w) \mid \gamma \in G\}$ heißt entsprechend \emph{Normalbasis} und
  $g(x) \in F[x]$ mit 
  \[ g(x) = \prod_{\gamma \in G}(x - \gamma(w))\]
  heißt \emph{normales Polynom}.
\end{definition}

Um effizient normale Elemente in $E\mid F$ zu finden, betrachten wir 
$(E,\sigma)$ als $F[x]$-Modul und nutzen die Aussagen aus
\autoref{chap:moduln}.

\begin{satz}
  \begin{enumerate}
    \item Die Erzeuger von $(E,\sigma)$ als $F[x]$-Modul sind genau die 
      normalen Elemente in $E\mid F$.
    \item Man hat eine Bijektion von Mengen
      \[ \{V_g :\ g(x) \in F[x] \text{ monisch mit } g(x) \mid x^n-1\}
        \overset{1-1}{\longleftrightarrow}
        \{F[x] \text{-Teilmoduln von }E\}\]
        wobei $V_g := \{v \in E : g(x)\cdot v = 0\} = \ker(g(\sigma))$.
    \item Jedes $V_g$ ist ein zyklischer Modul und es gilt
      \[u \text{ erzeugt } V_g \quad\Leftrightarrow \quad
        \Ord_q(u) = g(x).\]
        Insbesondere sind die Erzeuger von $V$ genau die Elemente $v \in V$ mit 
        $\Ord_\tau(v) = x^n-1$.
  \end{enumerate}
\end{satz}
\begin{proof}
  Alles schon in \TODO~gezeigt.
\end{proof}

Dies liefert uns die grundlegende Idee für das Auffinden von normalen
Elementen:
\begin{lemma}
  Sei $x^n-1 = \prod_{i=1}^s r_i(x)$ eine Zerlegung in paarweise teilerfremde
  Polynome, so gilt:
  \[ E \speq= \bigoplus_{i=1}^s V_{r_i} \,.\]
\end{lemma}
\begin{proof}
  \TODO.
\end{proof}

\begin{kor}
  Sei $x^n-1 = \prod_{i=1}^s r_i(x)$ eine Zerlegung in paarweise teilerfremde
  Polynome. Seien ferner $u_i \in V_{r_i}$ Elemente mit 
  $\Ord_q(u_i) = r_i(x)$ $\forall i=1,\ldots,s$. Dann ist
  \[ u \speq= u_1 + u_2 + \ldots + u_s\]
  normal in $E \mid F$.
\end{kor}
\begin{proof}
  Da obige Zerlegung von $x^n-1$ paarweise teilerfremd ist, folgt nach
  \cref{lemma:eigenschaften-tau-ordnung}
  $\Ord_q(u) = \prod_{i=1}^s \Ord_q(u_i) = \prod_{i=1}^s r_i(x) = x^n -1$.
\end{proof}

\begin{beispiel}
  
\end{beispiel}


An diesem Punkt stellt sich natürlich die Frage, wie wir dies nutzbar machen
können. Ist nämlich $p\nmid n$, so kennen wir eine Faktorisierung von $x^n-1$:
\[ x^n-1 \speq= \prod_{d\mid n} \Phi_d(x)\,.\]
Des Weiteren können wir eine Zerlegung von $\Phi_d(x)$ sogar genauer angeben:

\chapter{Vollständige Normalbasen}
\label{chap:vollst_normalbasen}

In den vorherigen Kapiteln haben wir einige Resultate zu Normalbasen kennen
gelernt. Es stellt sich jedoch ganz natürlich die Frage, ob dieser Begriff
nicht erweitert werden kann: Für eine Körpererweiterung $E$ über $F$ existieren
im Allgemeinen Zwischenkörper $E \mid K \mid F$, so wollen wir untersuchen,
ob ein Element $w \in E$, welches normal über $F$ ist, auch normal über allen
Zwischenkörpern bleibt. Solche Elemente wollen wir \emph{vollständig normal}
nennen. Ähnlich zu normalen Elementen kann man mit Hilfe der Modulstrukturen
von Körpererweiterungen eine Theorie aufbauen, die es erlaubt, vollständig
normale Elemente in vollständige Erzeuger (\thref{def:vollst_erzeuger}) zu
zerlegen, wie es für normale Elemente aus \thref{kor:summe_erzeuger_normal}
kennt. Hachenberger konnte in \autocite{hachenberger1997finite} ausarbeiten, wie
die simultan auftretenden Modulstrukturen zu behandeln sind. Wir wollen hier
die zentralen Resultate lediglich ohne Beweise zitieren. Eine ebenfalls gute
Übersicht dazu findet man in \autocite[Section 5.4]{mullen2013handbook}.
Wir beginnen bei der grundlegendsten Definition, die sich ja bereits in der
Kapitelüberschrift wiederfinden lässt.

\begin{definition}[vollständig normal]
  Sei $E \mid F$ eine Körpererweiterung endlicher Körper $E$ über $F$.
  $w\in E$ heißt \emph{vollständig normal}, falls $w$ normal über jedem
  Zwischenkörper $E \mid K\mid F$ ist.

  Die Begriffe \emph{vollständige Normalbasis}, \emph{vollständig normales
  Polynom} sind analog zu \thref{def:normal} zu setzen.
\end{definition}

Ein besonders trivialer Fall würde auftreten, wenn bereits alle normalen
Elemente eine Körpererweiterung auch vollständig normal wären. Man kann zeigen,
dass dies in der Tat unter gewissen Bedingungen auftreten kann und verleiht
dieser Konstellation den Namen \emph{einfach}.

\begin{definition}[einfach]
  \label{def:einfach}
  Eine Körpererweiterung $E \mid F$ endlicher Körper $F$ und $E$ heißt 
  \emph{einfach}, falls jedes normale Element von $E$ über $F$ bereits
  vollständig normal ist.
\end{definition}


\begin{satz}
  \label{satz:einfache_erweiterungen}
  Sei $\F_{q^m} \mid \F_q$ eine Erweiterung endlicher Körper
  von Charakteristik $p$. Dann sind äquivalent:
  \begin{enumerate}
    \item $\F_{q^m} \mid \F_q$ ist einfach.
    \item Für jeden Primteiler $r \mid m$ ist jedes normale Element in 
      $\F_{q^m}$ über $\F_q$ auch normal in $\F_{q^m}$ über $\F_{q^r}$.
    \item Für jeden Primteiler $r \mid m$ teilt $r$ nicht
      $\ord_{(\frac m r)'}(q)$.
  \end{enumerate}
  Dabei ist $\tfrac m r = (\tfrac m r)'\,p^b$ mit $\ggT((\tfrac m r)',p) = 1$.
\end{satz}
\begin{proof}
  \autocite[Corollary 15.8]{hachenberger1997finite}.
\end{proof}

\begin{kor}
  \label{kor:einfache_erweiterungen}
  Insbesondere ist $\F_{q^m}$ über $\F_q$ einfach, falls
  \begin{enumerate}
    \item $m = r$ oder $m=r^2$ für eine Primzahl $r$.
    \item $m' \mid (q-1)$, wobei $m=m'p^b$ mit $\ggT(m',p) = 1$.
    \item $m = p^b$ für $b\geq 0$.
  \end{enumerate}
\end{kor}
\begin{proof}
  (1) ist klar. Für (2) sei auf \autocite[Theorem 15.9]{hachenberger1997finite}
  verwiesen und (3) ist eine Folgerung aus (2).
\end{proof}

Im Abschnitt über normale Elemente konnten wir herausarbeiten, dass die
Zerlegung von $x^n-1$ in Kreisteilungspolynome eine guter Startpunkt ist, um
normale Elemente zu konstruieren und die Modulstrukturen zu beschreiben. Jedoch
zeigt es sich, dass im Allgemeinen ein Element, dessen $q$-Ordnung einem
Kreisteilungspolynom entspricht, eine $q^d$-Ordnung für einen Teiler $d$ von
$n$ besitzt, die kein reines Kreisteilungspolynom mehr ist. 
Also muss eine passende Klasse von Polynomen gefunden werden, um die
verschiedenen simultan auftauchenden $q^d$-Ordnungen zu erfassen: 
\emph{verallgemeinerte Kreisteilungspolynome}.

\begin{definition}[verallgemeinertes Kreisteilungspolynom]
  Sei $F$ ein endlicher Körper. Seien $k,t\geq 1$ natürliche Zahlen und 
  $k$ teilerfremd zu $\charak F$, so heißt
  \[ \Phi_{k,t}(x) \speq{:=} \Phi_k(x^t) \ \in F[x]\]
  \emph{verallgemeinertes Kreisteilungspolynom}.
\end{definition}


\begin{definition}[verallgemeinerter Kreisteilungsmodul, Modulcharakter]
  \label{def:verallgemeinerter_kreisteilungsmodul}
  Sei $\Phi_{k,t}$ ein verallgemeinertes Kreisteilungspolynom über einem
  endlichen Körper $F$. Notiere ferner $\sigma: \bar F\to \bar F$ 
  den Frobenius von $F$,so heißt
  \[ \C_{k,t} \speq{:=} \{ w \in \bar F:\ \Phi_{k,t}(\sigma)(w) = 0 \}\]
  \emph{verallgemeinerter Kreisteilungsmodul}.

  Der \emph{Modulcharakter} von $\C_{k,t}$ ist $\frac{k\,t}{\nu(k)}$.
\end{definition}

\begin{definition}[vollständiger Erzeuger]
  \label{def:vollst_erzeuger}
  Sei $\C_{k,t}$ ein verallgemeinerter Kreisteilungsmodul über $\F_q$.
  $w \in \bar F$ heißt 
  \emph{vollständiger Erzeuger von $\C_{k,t}$}, falls
  $w$ ein Erzeuger von $\C_{k,t}$ als $\F_{q^d}[x]$-Modul 
  für alle Teiler $d$ des Modulcharakters $\frac{kt}{\nu(k)}$ ist.
\end{definition}

\begin{definition}[Zerlegung in verallgemeinerte Kreisteilungsmoduln]
  Sei $\Phi_{k,t}$ ein verallgemeinertes Kreisteilungspolynom über $F$.
  $\Delta \subseteq F[x]$ heißt eine 
  \emph{Zerlegung von $\Phi_{k,t}$ in verallgemeinerte Kreisteilungspolynome}, 
  falls $\Delta$ nur verallgemeinerte Kreisteilungspolynome enthält, diese
  paarweise teilerfremd sind und
  \[ \Phi_{k,t}(x) \speq= \prod_{\Psi \in \Delta} \Psi(x)\,. \]
  Definiere ferner
  \[ i(\Delta) \speq{:=} \{ (l,s) \in \N^2:\ 
    \Phi_{l,s} \in \Delta\}\,.\]
\end{definition}


\begin{definition}[verträgliche Zerlegung]
  \label{def:vertraeglich}
  Sei $\Delta$ eine Zerlegung von $\Phi_{k,t}$ in verallgemeinerte
  Kreisteilungspolynome über $F$.
  Dann heißt $\Delta$ 
  \emph{verträgliche Zerlegung} falls gilt: Für jedes 
  $(l,s) \in i(\Delta)$ sei
  $w_{l,s} \in \bar F$ ein vollständiger Erzeuger von 
  $\C_{l,s}$ über $F$,
  so ist 
  \[ w = \sum_{(l,s) \in i(\Delta)} w_{l,s} \]
  ein vollständiger Erzeuger von $\C_{k,t}$ über $F$.
\end{definition}


Nun können wir den einen zentralen Satz formulieren, der eine passende
Zerlegung eines erweiterten Kreisteilungspolynoms herstellt, so dass sich ein
vollständiger Erzeuger als Summe von vollständigen Erzeugern der entsprechenden
Teilmoduln zusammensetzen lässt. Man bemerke an dieser Stelle, dass das Problem
der vollständigen Erzeuger (und damit der vollständigen Normalbasen) ungleich
schwerer ist, als das der normalen Elemente, da sich dort Elemente 
mit teilerfremden $q$-Ordnungen \emph{immer} zu einem Element summieren, dessen
$q$-Ordnung gerade das Produkt der $q$-Ordnungen ist 
(vgl. \thref{satz:zerlegungssatz_zykl_vektorraume}); mit anderen Worten 
also die Summe von Erzeugern disjunkter Teilmoduln stets wieder einen Erzeuger
liefert. Dies ist bei vollständigen Erzeugern nur bedingt gegeben, wie
nachstehender Zerlegungssatz beschreibt.

\begin{satz}[Zerlegungssatz für verallgemeinerte Kreisteilungsmoduln]
  \label{satz:zerlegungssatz}
  Sei $\Phi_{k,t}$ ein verallgemeinertes Kreisteilungspolynom über einem
  endlichen Körper $\F_q$ mit Charakteristik $p$. Sei $r$ eine Primzahl
  mit

  \begin{itemize*}[itemjoin={\qquad}]
    \item $r \mid t$,
    \item $r \neq p$,
    \item $r \nmid k$.
  \end{itemize*}

  Dann ist 
  \[ \Delta_r \speq{:=} \{ \Phi_{k,\frac{t}{r}},\ \Phi_{kr, \frac{t}{r}}\}\]
  eine Zerlegung von $\Phi_{k,t}$ in verallgemeinerte Kreisteilungspolynome und
  diese ist verträglich genau dann, wenn
  \[ r^a \nmid \ord_{\nu(kt')}(q) \]
  mit $a = \max\{ b\in \N: r^b \mid t\}$.
\end{satz}
\begin{proof}
  \autocite[Decomposition Theorem, Section 19]{hachenberger1997finite}.
\end{proof}


Sicherlich kann man sich nun fragen, in welchen Fällen die kanonische Zerlegung
von eines erweiterten Kreisteilungspolynoms in Kreisteilungspolynome noch
verträglich ist. Nach \autocite[Theorem 19.10]{hachenberger1997finite} 
ist die kanonische Zerlegung von $\Phi_{k,t}(x)^\pi$ verträglich über 
$\F_q$, falls $\ord_{\nu(kt')}(q)$ und $t'$ teilerfremd sind. Dies motiviert
dieser Klasse von Kreisteilungsmoduln einen eigenen Namen zu geben:

\begin{definition}[regulär]
  \label{def:regulaer}
  Ein verallgemeinerter Kreisteilungsmodul $\C_{k,t}$ 
  mit $\ggT(k,t)=1$ heißt \emph{regulär} über $\F_q$,
  falls $\ord_{\nu(k\,t')}(q)$ und $k\,t$ teilerfremd sind.

  Eine Körpererweiterung $\F_{q^m} \mid \F_q$ heißt \emph{regulär}, falls
  $\C_{1,m}$ regulär ist.
\end{definition}


\begin{definition}[ausfallend]
  \label{def:ausfallend}
  Sei $\C_{k, p^b}$ ein regulärer verallgemeinerter Kreisteilungsmodul 
  über $\F_q$ mit $\charak \F_q = p$. Schreibe $k = 2^c \cdot \bar k$ mit $\bar
  k$ ungerade. Dann heißt $\C_{k,p^b}$ \emph{ausfallend}, falls gilt:
  \begin{itemize}
    \item $q \equiv 3 \bmod 4$,
    \item $c \geq 3$ und 
    \item $\ord_{2^c}(q) = 2$.
  \end{itemize}
\end{definition}


Hachenberger war es nun möglich, für reguläre Kreisteilungsmoduln zu beweisen,
dass alle auftretenden Zwischenkörper, deren Betrachtung bei der Suche nach
vollständigen Erzeugern notwendig ist, von einem einzigen 
Zwischenkörper (oder zwei Zwischenkörpern) dominiert werden. Das bedeutet, dass
ein Element eines regulären Kreisteilungsmoduls maximal zwei bestimmte
$q^\bullet$-Ordnungen besitzen muss, um bereits den Kreisteilungsmodul
vollständig zu erzeugen. Die
geforderten $q^\bullet$-Ordnungen werden durch nachstehende Definition gegeben
und wir schließen dieses Kapitel mit der Angabe dieses wahrlich beachtlichen
Resultats.

\begin{definition}[$\tau$-Teiler]
  \label{def:tau}
  Sei $\C_{k,p^b}$ ein regulärer verallgemeinerter Kreisteilungsmodul über
  $\F_q$. Schreibe
  \[ \ord_k(q) \speq= \ord_{\nu(k)}(q) \ \prod_{r \in \pi(k)} r^{\alpha_r}\,,\]
  wobei $\pi(k)$ die Primteiler von $k$ bezeichnen.
  Dann heißt
  \[ \tau \speq{:=} \tau(q,k) \speq{:=} \prod_{r\in \pi(k)} 
    r^{\lfloor \frac{\alpha_r}{2}\rfloor}\]
  der \emph{$\tau$-Teiler von $\C_{k,p^b}$}.
\end{definition}

\begin{satz}[Über reguläre Erweiterungen]
  \label{satz:regulare_erweiterungen}
  Sei $\F_q$ ein endlicher Körper von Charakteristik $p$. 
  Seien $k$ eine positive ganze Zahl teilerfremd zu $q$ und 
  $\C_{k,p^b}$ ein regulärer verallgemeinerter Kreisteilungsmodul. Dann gilt:
  \begin{enumerate}
    \item Ist $\C_{k,p^b}$ nicht ausfallend, so ist $u \in \bar F$ genau dann
      ein vollständiger Erzeuger von $\C_{k,p^b}$, falls
      \[ \Ord_{q^\tau}(u) \speq= \Phi_{\frac k \tau,\, p^b} \,.\]
    \item Ist $\C_{k,p^b}$ ausfallend, so ist $u\in \bar F$ genau dann
      ein vollständiger Erzeuger von $\C_{k,p^b}$, falls
      \[ \Ord_{q^\tau}(u) \speq= \Phi_{\frac k \tau,\, p^b} \quad
        \text{und}\quad 
        \Ord_{q^{2\tau}}(u) \speq= \Phi_{\frac{k}{2\tau},\, p^b}\,.\]
  \end{enumerate}
\end{satz}
\begin{proof}
  \autocite[Theorem 20.3]{hachenberger1997finite}.
\end{proof}

\chapter{Enumeration primitiv vollständig normaler Elemente}

In den vorangegangenen Kapiteln haben wir lediglich normale und vollständig
normale Elemente in Erweiterungen endlicher Körper betrachtet. Es existiert
jedoch eine weitere besondere Eigenschaft, die gerade in der Anwendung von
großem Interesse ist: Primitivität (\thref{def:primitiv}). Zusammen haben wir
nun drei Eigenschaften eines Elements $u \in E$ einer Erweiterung endlicher
Körper $E\mid F$ kennengelernt, die von Interesse sind. Daher ist es nur
sinnvoll sich der Frage zu widmen, \emph{wie viele} Elemente mit den jeweiligen
Eigenschaften es gibt. Mit \thref{satz:zykl_gruppen} und 
\thref{satz:mult_gruppe_endl_korper_zyklisch} ist sofort klar, dass es in
$\F_q$ genau $\varphi(q-1)$ primitive Elemente gibt! Daher ist die
Fragestellung nach der Anzahl primitiver Elemente schnell gelöst. Darüber
hinaus wollen wir die folgenden Notationen treffen.

\begin{definition}
  Seien $\F_q$ ein endlicher Körper und $n \in \N^\ast$, so bezeichne
  $\cal N(q,n)$, $\CN(q,n)$, $\PN(q,n)$ bzw. $\PCN(q,n)$ die Anzahl der 
  normalen, vollständig normalen, primitiv normalen bzw. primitiv vollständig
  normalen Elemente in $\F_{q^n}$ über $\F_q$, d.h.
  \begin{align*}
    \cal N(q,n) &\speq{:=} 
      |\{ u \in \F_{q^n}:\ u\text{ ist normal über }\F_q\}| \\
    \CN(q,n) &\speq{:=} 
      |\{ u \in \F_{q^n}:\ u\text{ ist vollständig normal über }\F_q\}| \\
    \PN(q,n) &\speq{:=} 
      |\{ u \in \F_{q^n}:\ u\text{ ist primitiv und normal über }\F_q\}| \\
    \PCN(q,n) &\speq{:=} 
      |\{ u \in \F_{q^n}:\ u\text{ ist primitiv und vollständig 
      normal über }\F_q\}| \\
    \G &\speq{:=} 
      \{ n\in \N^\ast, n\geq 2:\ 
      \forall q\text{ Primzahlpotenz gilt } \PCN(q,n) > 0 \}
  \end{align*}
\end{definition}

Vielleicht erscheint die Definition von $\G$ etwas überraschend, da für jedes
Element in $\G$ schließlich \emph{unendlich viele} Körpererweiterungen auf die
Existenz eines $\PCN$-Elements getestet werden müssen. Doch es sei an dieser
Stelle vorweg genommen, dass wir in der Lage sind mit Hilfe eines
asymptoptischen Resultats und der konkreten Angabe von endlich vielen
$\PCN$-Elementen zu zeigen, dass $\G$ nicht leer ist!

Nun können wir folgende Probleme definieren:

\begin{problem}[$\cal N(q,n)=?$]
  \label{prob:n=}
  Seien $q$ eine Primzahlpotenz und $n\in \N^\ast$. Was ist
  $\cal N(q,n)$?
\end{problem}
\begin{problem}[$\CN(q,n)=?$]
  \label{prob:cn=}
  Seien $q$ eine Primzahlpotenz und $n\in \N^\ast$. Was ist
  $\CN(q,n)$?
\end{problem}
\begin{problem}[$\PN(q,n)=?$]
  \label{prob:pn=}
  Seien $q$ eine Primzahlpotenz und $n\in \N^\ast$. Was ist
  $\PN(q,n)$?
\end{problem}
\begin{problem}[$\PCN(q,n)=?$]
  \label{prob:pcn=}
  Seien $q$ eine Primzahlpotenz und $n\in \N^\ast$. Was ist
  $\PCN(q,n)$?
\end{problem}

Offensichtlich können wir die obigen Problemstellungen leicht abschwächen und
uns zunächst fragen, ob überhaupt Elemente mit den geforderten Eigenschaften
existieren. Auch dazu wollen wir passende Probleme formulieren.

\begin{problem}[$\cal N(q,n)>0?$]
  \label{prob:n>0}
  Seien $q$ eine Primzahlpotenz und $n\in \N^\ast$. Ist
  $\cal N(q,n)>0$?
\end{problem}
\begin{problem}[$\CN(q,n)>0?$]
  \label{prob:cn>0}
  Seien $q$ eine Primzahlpotenz und $n\in \N^\ast$. Ist
  $\CN(q,n)>0$?
\end{problem}
\begin{problem}[$\PN(q,n)>0?$]
  \label{prob:pn>0}
  Seien $q$ eine Primzahlpotenz und $n\in \N^\ast$. Ist
  $\PN(q,n)>0$?
\end{problem}
\begin{problem}[$\PCN(q,n)>0?$]
  \label{prob:pcn>0}
  Seien $q$ eine Primzahlpotenz und $n\in \N^\ast$. Ist
  $\PCN(q,n)>0$?
\end{problem}


Zuletzt wollen wir natürlich auch für $\G$ eine Problemstellung 
zu formulieren:

\begin{problem}[$n\in \G ?$]
  Finde möglichst viele $n\in \N^\ast$, $n\geq 2$ mit $n\in \G$.
\end{problem}

Bisher haben wir all diese Probleme nicht ausreichend geklärt. Doch im
folgenden Abschnitt wollen wir uns jenen Fragestellungen zunächst theoretisch
widmen, um in den darauffolgenden gezielte Enumerationen auf Basis der
theoretischen Resultate, die über (vollständig) normale Elemente im bisherigen
Verlauf erarbeitet wurden, durchzuführen, um für die offen bleibenden Fragen 


\section{Theoretische Enumerationen}


Wir starten mit einem wohlbekannten Resultat, das eine Antwort auf die Frage
nach der Existenz von normalen Elementen (\thref{prob:n>0}) gibt:

\begin{satz}[Satz von der Normalbasis]
  Zu jedem endlichen Körper $F$ und jeder endlichen Erweiterung $E$ von $F$
  existiert eine Normalbasis von $E$ über $F$.
\end{satz}
\begin{proof}
  \autocite[Theorem 2.35]{lidl1997finite}.
\end{proof}

Selbige Aussage können wir auch für vollständig normale Elemente treffen, was
zuerst \citeyear{blessenohl1986} von \citeauthor{blessenohl1986}
\autocite{blessenohl1986} bewiesen wurde.

\begin{satz}[Verschärfung des Satzes von der Normalbasis]
  Zu jedem endlichen Körper $F$ und jeder endlichen Erweiterung $E$ von $F$
  existiert eine vollständige Normalbasis von $E$ über $F$.
\end{satz}
\begin{proof}
  \autocite[Satz 1.2]{blessenohl1986}.
\end{proof}


Damit wäre auch \thref{prob:cn>0} beantwortet! Von den Existenzfragen bleibt
damit noch die Existenz von primitiv normalen und primitiv vollständig normalen
Elementen in beliebigen Erweiterungen offen. Erstere beantwortete
\citeauthor{lenstra1987} \citeyear{lenstra1987} \autocite{lenstra1987}
nach den Vorarbeiten von Carlitz und Davenport.

\begin{satz}[Satz von der primitiven Normalbasis]
  Zu jedem endlichen Körper $F$ und jeder endlichen Erweiterung $E$ von $F$
  existiert eine primitive Normalbasis von $E$ über $F$.
\end{satz}
\begin{proof}
  \autocite{lenstra1987}.
\end{proof}


Bleibt also nur noch die Frage nach der Existenz primitiver vollständig
normaler Elemente. Auch wenn Hachenberger 
\citeyear{hachenberger2001} \autocite{hachenberger2001} und 
\citeyear{hachenberger2014} \autocite{hachenberger2014} die beiden
nachstehenden bedeutsamen Resultate beweisen konnte, bleibt die Suche nach 
$\PCN$-Elementen weiterhin ein offenes Problem, dem wir uns im weiteren Verlauf
experimentell widmen wollen.

\begin{satz}
  Seien $q$ eine Primzahlpotenz und $n \in \N^\ast$, so dass
  $\F_{q^n}$ über $\F_q$ eine reguläre Erweiterung ist. Sei ferner
  $4\mid (q-1)$, falls $q$ ungerade und $n$ gerade ist. Dann existiert ein
  primitives Element in $\F_{q^n}$, das vollständig normal über $\F_q$ ist.
\end{satz}
\begin{proof}
  \autocite[Theorem 1.4]{hachenberger2001}.
\end{proof}


\begin{satz}
  Sei $n\in \N^\ast$ mit $n\geq 2$. Dann gilt:
  Für Primzahlpotenzen $q$ mit $q \geq n^4$ existiert ein primitives Element in 
  $\F_{q^n}$, das vollständig normal über $\F_q$ ist.
\end{satz}
\begin{proof}
  \autocite[Theorem 2]{hachenberger2014}.
\end{proof}

Wir können nun zusammenfassen, dass von obigen Existenzproblemen lediglich
\thref{prob:pcn>0} überlebt hat und alle anderen durch theoretische Resultate
abgedeckt werden konnten. Nun können wir versuchen die Zählprobleme anzugehen
und starten mit einem allgemein bekannten Resultat.

\begin{definition}
  Sei $f(x) \in \F_q[x]$ ein Polynom über einem endlichen Körper. Definiere
  \[ \phi_q(f) \speq{:=} |\{ g(x) \in \F_q[x]:\ 
    \deg g < \deg f,\ \ggT(f,g) = 1\}|\,.\]
\end{definition}

\begin{bemerkung}
  $\phi_q(f)$ ist das Analogon zur Eulerschen Phifunktion für Polynome, da
   $\phi_q(f)$ gerade die Anzahl der Einheiten im Ring $\F_q[x]\big/(f(x))$
   angibt.
\end{bemerkung}

\begin{satz}
  Seien $q$ eine Primzahlpotenz und $n\in \N^\ast$ mit $n\geq 2$, so existieren
  in $\F_{q^n}$ genau 
  \[ \phi_q(x^n-1) \speq= q^{n'(\pi-1)}\,\prod_{d\mid n}
    \left( q^{\ord_d(q)} -1 \right)^{\frac{\varphi(d)}{\ord_d(q)}}\]
  Elemente, die normal über $\F_q$ sind, wobei $n = n'\pi$ mit $\ggT(n',q) = 1$.
\end{satz}
\begin{proof}
  \autocite[Theorem 3.73]{lidl1997finite} oder 
  \autocite[Theorem 10.5]{hachenberger1997finite}.
\end{proof}

Obiger Satz beantwortet also \thref{prob:n=} vollständig.

\lstMakeShortInline[
  basicstyle = \small\normalfont\ttfamily,
  frame = none,
% numbers = left,
  numberstyle = \tiny,
% numbersep = 5pt,
  breaklines = true,
  %xleftmargin = 0.1\linewidth,
  %xrightmargin = 0.1\linewidth,
  escapeinside = {(*}{*)},
  tabsize=3,
  language=C,
  mathescape=true,
  breaklines=true]@

\lstMakeShortInline[
  basicstyle = \small\normalfont\ttfamily,
  frame = none,
% numbers = left,
  numberstyle = \tiny,
% numbersep = 5pt,
  breaklines = true,
  %xleftmargin = 0.1\linewidth,
  %xrightmargin = 0.1\linewidth,
  escapeinside = {(*}{*)},
  tabsize=3,
  language=Python,
  mathescape=true,
  breaklines=true]"

\newcommand{\ttgray}{\color{gray}\ttfamily}


\section{Implementierung endlicher Körper und Körpererweiterungen}
\label{sec:impl_endl_körper}
Grundsätzlich wurde zur konkreten Suche und Enumeration primitiver und
vollständig normaler Elemente das Computeralgebrasystem \sage verwendet.
\sage bietet bereits die Möglichkeit in endlichen Körpern zu rechnen. Jedoch
hat sich herausgestellt, dass die zugrunde liegenden \Clang-Bibliotheken 
(im allgemeinen Fall ist dies das \texttt{Pari C library}%
\footnote{vgl. \url{http://www.sagemath.org/doc/reference/%
rings_standard/sage/rings/finite_rings/constructor.html}}) 
zu langsam sind. Dies ist sicherlich auf die Allgemeinheit ihrer
Anwendungsgebiete zurückzuführen. Beispielsweise arbeitet die 
\texttt{Pari}-Bibliothek stets mit Ganzzahlen beliebiger Größe. Deren
Arithmetik ist selbstredend aufwendiger und langsamer, als maschineninterne
\texttt{Integer}-Arithmetik. Daher haben wir uns entschlossen eigene 
\Clang-Bibliotheken anzulegen, die auf einfacher (jedoch begrenzter) 
\texttt{Integer}-Arithmetik basieren.

\subsection{Beschreibung von Elementen endlicher Körper}
\label{sub:beschreibung_endliche_koerper}
Die Implementierung von Primkörpern ist freilich kanonisch. Daher brauchen wir
an dieser Stelle nicht viele Worte verlieren, da wir auf der Suche nach
primitiv und vollständig normalen Elementen ohnehin nur in Erweiterungen von
Graden größer $1$ zu rechnen haben.

Sei also $\F_q$ ein endlicher Körper von Charakteristik $p$ und $q = p^r$
für $r>1$.
Wie auch in \sage üblich, haben wir uns entschieden, bei der programmatischen
Beschreibung die Isomorphie
\[ \F_q \speq\cong \F_p[x] \big/ (f(x))\]
mit $f(x) \in F_p[x]$ irreduzibel, monisch von Grad $r$ zu nutzen. 
Also wird ein Element $w \in \F_q$ als Array der Länge $r+1$ beschrieben,
wobei die nullte Stelle des Arrays auch den Koeffizienten von $x^0$ meint, und
alle Berechnungen (insb. Multiplikation) modulo $f(x)$ ausgeführt werden.

Es hat sich herausgestellt, dass es von Vorteil ist, neben dem Koeffizienten
tragenden Array ein weiteres Array mitzuführen, welches die
Indizes speichert, deren zugehörige Koeffizienten nicht verschwinden. Letztlich
fehlt noch, wie es in \Clang üblich und notwendig ist, die Länge des
Indexarrays zu speichern und wir erhalten den Datentyp @struct FFElem@.

\begin{ccode}[caption={[\texttt{struct FFElem} aus 
 \url{../Sage/enumeratePCNs.c}]Aus \url{../Sage/enumeratePCNs.c}}]
/**
 * Finite Field Element. 
 * 
 * !! idcs must be in desc order !!
 *
 * Uses int arrays, i.e. you must not consider 
 * PrimeFields of order p with  (p-1)*(p-1) > INT_MAX
 */
struct FFElem{
    int *el;
    int *idcs;
    int len;
};
\end{ccode}

@len@ gibt immer die Länge von @idcs@ an. Zusätzlich fordern wir noch folgende
Eigenschaften, die den Umgang mit @struct FFElem@ erleichtern.

\begin{invariante}
  \label{invariante:desc_order}
  Für das Indexarray @idcs@ eines @struct FFElem@ sei sichergestellt, 
  dass die Werte stets in absteigender Reihenfolge sortiert sind. 
\end{invariante}

\begin{invariante}
  \label{invariante:array_len}
  Bei der Benutzung von @struct FFElem@ sei sichergestellt, 
  dass die Länge aller auftretenden Arrays dem Grade der 
  Körpererweiterung über dem jeweiligen Primkörper entspricht.
\end{invariante}

\thref{invariante:desc_order} erleichtert den Zugriff auf
den Grad des Elements (also seinen Grad als Polynom in
$\F_p[x]\big/(f(x))$). Letztere Invariante stellt sicher, dass durch
Veränderung eines @struct FFElem@ (beispielsweise Arithmetik) kein 
Speicherzugriffsfehler auftritt.

\begin{beispiel}
  Wollen wir das Element 
  \[ w := x^8 + 2x^6 + x^2 + 2 \in \F_3[x]\]
  des endlichen Körpers $\F_{3^{10}}$ 
  (wir verzichten auf Angabe eines Minimalpolynoms, da es hier keine
  Rolle spielt) in obiger Darstellung beschreiben, so müssen wir \Clang-üblich
  Speicher allokieren und die Arrays in passender Länge anlegen:
  \begin{cexample}
    struct FFElem *w = malloc(sizeof(struct FFElem));
    w->el = (int[]) {2, (*\ttgray 0*), 1, (*\ttgray 0, 0, 0,*) 2, (*\ttgray 0,*) 1, (*\ttgray 0*)};
    w->idcs = (int[]) {8, 6, 2, 0, (*\ttgray 0, 0, 0, 0, 0, 0*)};
    w->len = 4;
  \end{cexample}
  Der besseren Lesbarkeit zu Gute haben wir die ungenutzten Indizes und die 
  verschwindenden Koeffizienten mit @0@ aufgefüllt und ausgegraut. 
  Man überlege sich jedoch,
  dass lediglich eine einzige @0@ notwendig ist und alle anderen 
  beliebig ersetzt werden könnten. Beispielsweise ist
  \begin{cexample}
    struct FFElem *w = malloc(sizeof(struct FFElem));
    w->el = (int[]) {2, (*\ttgray -10*), 1, (*\ttgray 100, -2, -3,*) 2, (*\ttgray -4,*) 1, (*\ttgray -8*)};
    w->idcs = (int[]) {8, 6, 2, 0, (*\ttgray -3, -2, -5, -1, -1, -1*)};
    w->len = 4;
  \end{cexample}
  mit obiger Beschreibung identisch.
\end{beispiel}


\subsubsection{Hilfsfunktionen zum Anlegen und Löschen}

Da \Clang ohne \emph{Garbage-Collection} (d.h. Verwaltung nicht mehr genutzter
Variablen) auskommt, muss man selbst für die
entsprechende Speicherverwaltung sorgen. Dies erleichtern die Funktionen
@mallocFFElem@ und @freeFFElem@.

\begin{ccode}[caption={[\texttt{FFElem *mallocFFElem} aus 
 \url{../Sage/enumeratePCNs.c}]Aus \url{../Sage/enumeratePCNs.c}}]
inline struct FFElem *mallocFFElem(int m){
    struct FFElem *ff = malloc(sizeof(struct FFElem));
    ff->el = malloc(m*sizeof(int));
    ff->idcs = malloc(m*sizeof(int));
    ff->len = 0;
    return ff;
}
\end{ccode}

\begin{ccode}[caption={[\texttt{void freeFFElem} aus 
 \url{../Sage/enumeratePCNs.c}]Aus \url{../Sage/enumeratePCNs.c}}]
inline void freeFFElem(struct FFElem *ff){
    free(ff->el);
    free(ff->idcs);
    free(ff);
}
\end{ccode}

Schließlich führen wir noch eine Funktion ein, die den Inhalt eines
@struct FFElem@s in ein neues kopiert. Dieses muss aber bereits allokiert sein!


\begin{ccode}[caption={[\texttt{void copyFFElem} aus 
 \url{../Sage/enumeratePCNs.c}]Aus \url{../Sage/enumeratePCNs.c}}]
/**
 * Copies the content of ff1 into ff2
 *
 * !! ff2 must be malloced!
 */
inline void copyFFElem(struct FFElem *ff1, struct FFElem *ff2){
    if(ff1 == ff2) return;
    int i;
    for(i=0;i < ff1->len;i++){
        ff2->idcs[i] = ff1->idcs[i];
        ff2->el[ ff1->idcs[i] ] = ff1->el[ ff1->idcs[i] ];
    }
    ff2->len = ff1->len;
}
\end{ccode}

\subsection{Arithmetik in endlichen Körpern}
\label{sub:arithmetik_in_endlichen_körpern}


\subsubsection{Additions- und Multiplikationstabellen}
Will man Arithmetik mit @struct FFElems@ betreiben, so stellt sich sicherlich
am Anfang die Frage, wie die Arithmetik im Primkörper 
$\F_p = \{0,1,\ldots,p-1\}$
aussehen möge. Da
die @FFElem@s auf @int@-Arrays basieren liegt es nahe, die Addition bzw.
Multiplikation zweier Elemente $a,b\in \F_p$ durch die integrierten Funktionen
@($a$+$b$) % $p$@ 
und @($a$*$b$) % $p$@
zu implementieren. Es hat sich jedoch herausgestellt, dass dies vergleichsweise
langsam ist. Insbesondere bei kleinen Primzahlen 
(die hier betrachteten Primzahlen waren kleiner gleich 43) 
hat sich das Anlegen einer
Additions- und einer Multiplikationstabelle bewährt. Dies sind @int@-Arrays,
sodass die @($a$+$b$)@-te Stelle der Additions- und die 
@($a$*$b$)@-te Stelle der Multiplikationstabelle gerade das Ergebnis der
jeweiligen Rechnung in $\F_p$ liefert.

\begin{bemerkung}
  Um sich nicht um vorzeichenbehaftete Werte kümmern zu müssen, überdecken die
  Tabellen auch negative Bereiche und daher ist eine Additionstabelle in $\F_p$
  stets von Länge $4(p-1)+1$ und eine Multiplikationstabelle von
  Länge $2(p-1)^2+1$.
\end{bemerkung}

\begin{beispiel}
  Betreiben wir Arithmetik in $\F_3$, so legen wir eine Additions- bzw.
  Multiplikationstabelle wie folgt an und stellen durch eine Verschiebung des
  Pointers sicher, dass auch vorzeichenbehaftete Rechnungen richtig erfasst
  werden können.
  \begin{cexample}
    int addTableRaw[] = {2, 0, 1, 2, 0, 1, 2, 0, 1};
    int initialAddShift = 4;
    int *addTable = addTableRaw+initialAddShift;
    int multTableRaw[] = {2, 0, 1, 2, 0, 1, 2, 0, 1};
    int initialMultShift = 4;
    int *multTable = multTableRaw+initialMultShift;
  \end{cexample}
  Führen wir nun Rechnungen durch können wir diese nutzen:
  \begin{cexample}
    addTable[ 2+1 ]  // == 0 
    addTable[ 0-2 ]  // == 1
    multTable[ 2*2 ]  // == 1
  \end{cexample}
\end{beispiel}


\subsubsection{Addition}
Aufgrund der effizienteren Darstellung der Elemente endlicher Körper durch
Speicherung ihrer Indizes, ist die Addition nicht lediglich gegeben durch
komponentenweise Betrachtung, sondern erfordert etwas mehr Aufwand.

\begin{ccode}[caption={[\texttt{void addFFElem} aus 
 \url{../Sage/enumeratePCNs.c}]Aus \url{../Sage/enumeratePCNs.c}}]
/**
 * Adds two FFElems.
 *
 * !! ff1 may be same as ret !!
 * !! ff2 must not be same as ret !!
 */
inline void addFFElem(struct FFElem *ff1, struct FFElem *ff2,
        struct FFElem *ret,
        int *tmp,
        int *multTable, int *addTable){
    int i=0,j=0,k=0, i2;
    bool end = false;
    //handle trivial cases
    if(ff1->len == 0){
        copyFFElem(ff2,ret);
        return;
    }
    if(ff2->len == 0){
        copyFFElem(ff1,ret);
        return;
    }
    copyArray(ff1->idcs,tmp,ff1->len);
    while( end == false ){
        while( tmp[i] != ff2->idcs[j] ){
            if( tmp[i] > ff2->idcs[j] ){
                ret->el[ tmp[i] ] = ff1->el[ tmp[i] ];
                ret->idcs[k] = tmp[i];
                i++; k++;
            }else if( tmp[i] < ff2->idcs[j] ){
                ret->el[ ff2->idcs[j] ] = ff2->el[ ff2->idcs[j] ];
                ret->idcs[k] = ff2->idcs[j];
                j++; k++;
            }
            if(i == ff1->len || j == ff2->len){
                end = true;
                break;
            }

        }
        if(end == true) break;
        //tmp[i] == ff2->idcs[j]
        i2 = tmp[i];
        ret->el[i2] = addTable[ ff1->el[i2] + ff2->el[i2] ];
        if(ret->el[i2] != 0){
            ret->idcs[k] = i2;
            k++;
        }
        i++; j++;
        if(i == ff1->len || j == ff2->len) end = true;
    }
    //add rest of ff1 or ff2
    if(i != ff1->len ){
        while(i<ff1->len){
            ret->el[ tmp[i] ] = ff1->el[ tmp[i] ];
            ret->idcs[k] = tmp[i];
            i++; k++;
        }
    }else if(j != ff2->len){
        while(j<ff2->len){
            ret->el[ ff2->idcs[j] ] = ff2->el[ ff2->idcs[j] ];
            ret->idcs[k] = ff2->idcs[j];
            j++; k++;
        }
    }
    ret->len = k;
}
\end{ccode}  

Wie später aus der Beschreibung anderer Algorithmen hervorgeht, ist es von
Vorteil, wenn das Ergebnis einer Addition bereits eines der beiden addierten
Elemente ist. Auf diese Weise spart man sich das Anlegen unnötiger Hilfs-@FFElem@s.
Wie man schnell einsieht, werden jeweils nur die beiden Indexarrays durchlaufen
und lediglich wenn der Wert der jeweiligen Stellen übereinstimmt, muss eine
Addition ausgeführt werden; ansonsten reicht es, den jeweiligen Koeffizienten 
zu übernehmen.


\subsubsection{Multiplikation}
Wir haben uns entschieden, keine speziellen Multiplikationsalgorithmen 
(wie Karatsuba oder FFT-basierte Algorithmen) zu implementieren, da
die hier betrachteten Erweiterungen nicht von Graden sind, in denen jene
Algorithmen ihre Vorteile ausspielen könnten.

\begin{ccode}[caption={[\texttt{void multiplyFFElem} aus 
 \url{../Sage/enumeratePCNs.c}]Aus \url{../Sage/enumeratePCNs.c}}]
/**
 * Multiplies two FFElems and reduces the result by mipo.
 *
 * !! tmp must have at least length m!
 * !! ret must be malloced!
 */
inline void multiplyFFElem(struct FFElem *ff1, struct FFElem *ff2, 
        struct FFElem *ret, 
        struct FFElem *mipo, int *tmp, int m,
        int *multTable, int *addTable){
    /* 
     * catch trivial cases
     */
    if(ff1->len == 0 || ff2->len == 0){
        ret->len = 0;
        return;
    }
    if(ff1->len == 1 && ff1->idcs[0] == 0 && ff1->el[0] == 1){
        copyFFElem(ff2, ret);
        return;
    }
    if(ff2->len == 1 && ff2->idcs[0] == 0 && ff2->el[0] == 1){
        copyFFElem(ff1, ret);
        return;
    }

    /*
     * Do multiplication
     */
    int maxlen = ff1->idcs[0] + ff2->idcs[0] + 1;
    int i,j,i2,j2,k;
    int max2 = maxlen;
    if( maxlen > m ){
        max2 = m;
        initPoly(tmp,maxlen-m);
    }
    initPoly(ret->el,max2);
    //multiply 
    for(i=0;i<(ff1->len);i++){
        for(j=0;j<(ff2->len);j++){
            i2 = ff1->idcs[i];
            j2 = ff2->idcs[j];
            k = i2+j2;
            if(k<m){
                ret->el[k] = addTable[ ret->el[k] + 
                    multTable[ ff1->el[i2] * ff2->el[j2] ] ];
            }else{
                tmp[k-m] = addTable[ tmp[k-m] +
                    multTable[ ff1->el[i2] * ff2->el[j2] ] ];
            }
        }
    }
    
    /*
     * Reduce mod mipo
     */
    if(maxlen > m){
        int quo;
        for(i=maxlen-m-1;i>=0;i--){
            quo = tmp[i];
            if(quo == 0) continue;
            for(j=0;j<(mipo->len); j++){
                j2 = mipo->idcs[j];
                k = i+j2;
                if(k>=m){
                    tmp[k-m] = addTable[ tmp[k-m] - 
                        multTable[ mipo->el[j2]*quo ] ];
                }else{
                    ret->el[k] = addTable[ ret->el[k] - 
                        multTable[ mipo->el[j2]*quo ] ];
                }
            }
        }
    }

    /*
     * Recalc indices
     */
    i2 = 0;
    for(i=max2-1;i>=0;i--){
        if(ret->el[i] != 0){
            ret->idcs[i2] = i;
            i2++;
        }
    }
    ret->len = i2;
}
\end{ccode}
Außer den beiden zu multiplizierenden @FFElem@s muss man natürlich das
Minimalpolynom des zu Grunde liegenden Körpers und dessen Grad über dem
Primkörper -- hier mit @int m@ bezeichnet -- mit übergeben.
Leider war es an dieser Stelle im Gegensatz zur Addition nicht 
möglich, die Indizes des Produkts direkt zu berechnen, da es sich bei den
Koeffizienten des Produkts ja um Summen von Produkten von Koeffizienten der beiden
Faktoren handelt. Daher muss nach der
Reduktion modulo des Minimalpolynoms eine Neuberechnung des Index-Arrays erfolgen.

\subsubsection{Quadratur}
Im Hinblick auf das Testen von @struct FFElem@s auf Primitivität und dem damit
verbundenen Potenzieren, existiert eine separate Funktion zur Quadrierung eines 
@FFElem@s. Es ist klar, dass beim Quadrieren weniger Produkte und Summen
berechnet werden müssen als bei einer allgemeinen Multiplikation.
\begin{ccode}[caption={[\texttt{void squareFFElem} aus 
 \url{../Sage/enumeratePCNs.c}]Aus \url{../Sage/enumeratePCNs.c}}]
/**
 * Squares an FFElem
 *
 * !! ff is not modified !!
 * !! tmp must have at least length m !!
 */
inline void squareFFElem(struct FFElem *ff, struct FFElem *mipo,
        struct FFElem *ret, int *tmp, int m,
        int *multTable, int *addTable){
    /* 
     * catch trivial cases
     */
    if(ff->len == 0){
        copyFFElem(ff,ret);
        return;
    }
    if(ff->len == 1 && ff->idcs[0] == 0 && ff->el[0] == 1){
        copyFFElem(ff,ret);
        return;
    }

    /*
     * Do multiplication
     */
    int maxlen = 2*ff->idcs[0] + 1;
    int i,j,i2,j2,k;
    int max2 = maxlen;
    if( maxlen > m ){
        max2 = m;
        initPoly(tmp,maxlen-m);
    }
    initPoly(ret->el,max2);
    for(i=0;i<(ff->len);i++){
        // same index must be squared
        i2 = ff->idcs[i];
        k = 2*i2;
        if(k<m){
            ret->el[k] = addTable[ ret->el[k] + 
                multTable[ ff->el[i2]*ff->el[i2] ] ];
        }else{
            tmp[k-m] = addTable[ tmp[k-m] +
                multTable[ ff->el[i2]*ff->el[i2] ] ];
        }
        // other indices only multipied and doubled
        for(j=i+1;j<(ff->len);j++){
            i2 = ff->idcs[i];
            j2 = ff->idcs[j];
            k = i2+j2;
            if(k<m){
                ret->el[k] = addTable[ ret->el[k] + 
                    multTable[ 2 * multTable[ ff->el[i2] * ff->el[j2] ] ] ];
            }else{
                tmp[k-m] = addTable[ tmp[k-m] +
                    multTable[ 2 * multTable[ ff->el[i2] * ff->el[j2] ] ]];
            }
        }
    }
    /*
     * Reduce mod mipo
     */
    if(maxlen > m){
        int quo;
        for(i=maxlen-m-1;i>=0;i--){
            quo = tmp[i];
            if(quo == 0) continue;
            for(j=0;j<(mipo->len); j++){
                j2 = mipo->idcs[j];
                k = i+j2;
                if(k>=m){
                    tmp[k-m] = addTable[ tmp[k-m] - 
                        multTable[ mipo->el[j2]*quo ] ];
                }else{
                    ret->el[k] = addTable[ ret->el[k] - 
                        multTable[ mipo->el[j2]*quo ] ];
                }
            }
        }
    }

    /*
     * Recalc indices
     */
    i2 = 0;
    for(i=max2-1;i>=0;i--){
        if(ret->el[i] != 0){
            ret->idcs[i2] = i;
            i2++;
        }
    }
    ret->len = i2;
}
\end{ccode}


\subsection{Matrizen und Polynome über endlichen Körpern}

\subsubsection{Matrizen und Matrixmultiplikation}

Nach \thref{satz:frob_auto} ist das Potenzieren mit der Charakteristik in
endlichen Körpern eine lineare Abbildung. Dies wollen wir Nutzen und haben
daher eine Darstellung für Matrizen über endlichen Körpern -- naheliegenderweise
ein Array aus @FFElem@s -- implementiert.

\begin{ccode}[caption={[\texttt{void matmul} aus 
 \url{../Sage/enumeratePCNs.c}]Aus \url{../Sage/enumeratePCNs.c}}]
/**
 * Matrix multiplication
 *
 * !! tmp must have at least length m!
 */
inline void matmul(struct FFElem **mat, struct FFElem *ff,
        struct FFElem *ret, 
        int m, int *multTable, int *addTable){
    int i,j,i2, row;
    bool end;
    for(row=0;row<m;row++){
        ret->el[row] = 0;
        i=0; j=0;
        end = false;
        while(end == false){
            while(ff->idcs[i] != mat[row]->idcs[j]){
                if(ff->idcs[i] > mat[row]->idcs[j]) i++;
                else if(ff->idcs[i] < mat[row]->idcs[j]) j++;
                if(i == ff->len || j == mat[row]->len){
                    end = true;
                    break;
                }
            }
            if(end == true) break;
            i2 = ff->idcs[i]; // == mat[row]->idcs[j]
            ret->el[row] = addTable[ ret->el[row] 
                + multTable[ mat[row]->el[i2]*ff->el[i2] ] ];
            i++;
            j++;
            if(i==ff->len || j==mat[row]->len) end = true;
        }
    }
    i2 = 0;
    for(i=m-1;i>=0;i--){
        if(ret->el[i] != 0){
            ret->idcs[i2] = i;
            i2++;
        }
    }
    ret->len = i2;
}
\end{ccode}

Hier wird -- anders als bei der Addition -- nur nach den gemeinsamen Indizes
gesucht (alle anderen Produkte sind schließlich 0). 
\thref{invariante:desc_order} stellt dabei wiederum sicher, dass das hier
aufgeführte Verfahren funktioniert.

Ferner existiert eine Funktion, die das Freigeben von Matrizen erleichtert.

\begin{ccode}[caption={[\texttt{void freeFFElemMatrix} aus 
 \url{../Sage/enumeratePCNs.c}]Aus \url{../Sage/enumeratePCNs.c}}]
inline void freeFFElemMatrix(struct FFElem **mat, int len){
    if(mat==0) return;
    int i;
    for(i=0;i<len;i++) freeFFElem(mat[i]);
    free(mat);
}
\end{ccode}


\subsubsection{Polynome}
Im Hinblick auf das Testen von @FFElem@s auf vollständige Normalität 
(bzw. vollständige Erzeuger-Eigenschaft) müssen wir einen Weg wählen, Polynome
über endlichen Körpern darzustellen; also Polynome deren Koeffizienten 
@FFElem@s sind. Dazu führen wir ein eigenes @struct@ ein.

\begin{ccode}[caption={[\texttt{struct FFPoly} aus 
 \url{../Sage/enumeratePCNs.c}]Aus \url{../Sage/enumeratePCNs.c}}]
struct FFPoly{
    struct FFElem **poly;
    int lenPoly;
};
\end{ccode}


\begin{ccode}[caption={[\texttt{FFPoly* mallocFFPoly} aus 
 \url{../Sage/enumeratePCNs.c}]Aus \url{../Sage/enumeratePCNs.c}}]
inline struct FFPoly *mallocFFPoly(int m, int lenPoly){
    struct FFPoly *poly = malloc(lenPoly*sizeof(struct FFElem*));
    poly->lenPoly = lenPoly;
    int i;
    for(i=0;i<lenPoly;i++) poly->poly[i] = mallocFFElem(m);
    return poly;
}
\end{ccode}  

\begin{ccode}[caption={[\texttt{void freeFFPoly} aus 
 \url{../Sage/enumeratePCNs.c}]Aus \url{../Sage/enumeratePCNs.c}}]
inline void freeFFPoly(struct FFPoly *poly){
    int i;
    for(i=0;i<poly->lenPoly;i++) freeFFElem(poly->poly[i]);
    free(poly->poly);
    free(poly);
}
\end{ccode}  


\section{Potenzieren und Primitivitätstest}

\subsection{Potenzieren}
Für das Potenzieren von @FFElem@s wurde stets ein Square-and-Multiply-Ansatz
verwendet. Da in endlichen Körpern jedoch das Potenzieren mit der
Charakteristik eine lineare Abbildung darstellt, ist es a priori nicht unklug
eine $p$-adische Square-and-Multiply-Variante zu wählen. Es hat sich jedoch
herausgestellt, dass in den meisten Fällen normales Square-and-Multiply
schneller ist als sein $p$-adisches Pendant. Dies veranschaulicht auch
nachstehendes Beispiel.

\begin{beispiel}
  Sei $u \in E := \F_{3^4}$ und zu berechnen sei $u^{16}$, so stellen wir zunächst
  $16$ binär und $3$-adisch dar:
  \[ 16 \speq= 10000_2 \speq= 121_3 \,.\]
  Damit gilt
  \[ u^{16} \speq= ((u^2)^2)^2)^2 \speq= (u^3\cdot u\cdot u)^3\cdot u\,.\]
  In einer Implementierung sehen wir also, dass die binäre Exponentiation 
  $4$ Quadrierungen „kostet“, die $3$-adische Version 
  hingegen $2$ Matrixmultiplikationen und $3$ Multiplikationen.
  Da in der Regel allgemeine Multiplikationen teuer sind, wäre in diesem Fall
  die binäre Variante wohl die bessere Wahl.\\
  Wollen wir $u^{10}$ berechnen, so sehen wir aus
  \[ 10 \speq= 1010_2 \speq= 101_3\,,\]
  dass in diesem Fall die binäre Exponentiation $3$ Quadierungen und eine
  allgemeine Multiplikation erfordert, die $3$-adische Variante jedoch nur 
  $2$ Matrixmultiplikationen und $1$ allgemeine Multiplikation. Letzteres lässt
  sich sogar auf eine Matrixmultiplikation reduzieren, berechnet man die
  Darstellungsmatrix der linearen Abbildung $E\to E,\ x\mapsto x^9$ bereits
  vorher! Die beiden Varianten der Berechnung würden 
  in diesem Fall also wie folgt von Statten gehen:
  \[ u^{10} \speq= ((u^2)^2\cdot u)^2 \speq= u^9\cdot u\,.\]
\end{beispiel}

Nachstehend werden nun die beiden Varianten der Implementierung der Potenzierung
aufgeführt. Wir beginnen mit $p$-adischem Square-and-Multiply. Zu bemerken ist,
dass die Potenz bereits in $p$-adischer Darstellung als @int@-Array übergeben
werden muss. Zudem werden vermeidbare Matrixmultiplikationen (vgl. obiges
Beispiel) nicht durchgeführt und es ist sicherzustellen, 
dass @struct FFElem **matCharac@ als @struct FFElem*@-Array
von Länge $(l+1)m$ ist, wobei $l$ die Länge
des maximal auftretenden $0$-Intervalls in der $p$-adischen Darstellung meint 
(in obigem Beispiel bei $u^{10}$ wäre $l=1$).


\begin{ccode}[caption={[\texttt{void powerFFElem} aus 
 \url{../Sage/enumeratePCNs.c}]Aus \url{../Sage/enumeratePCNs.c}},
  label=lst:powerffelem]
/**
 * Square and multiply in charac
 * mat is powering by charac
 *
 * !! ff is modified !!
 */
inline void powerFFElem(struct FFElem *ff, struct FFElem *mipo,
        struct FFElem *ret, 
        int m, int *power, int powerLen,
        struct FFElem **matCharac, int *tmp, struct FFElem *ffTmp,
        int *multTable, int *addTable){
    int i,j,k;
    int lenCurGap = 0;
    struct FFElem *ffSwitch = 0;
    struct FFElem *ffRetInt = ret;
    // init ret to 1
    ffRetInt->el[0] = 1; ffRetInt->idcs[0] = 0; ffRetInt->len = 1;
    for(j=powerLen-1;j>=0;j--){
        for(k=0;k<power[j];k++){
            multiplyFFElem(ffRetInt,ff,ffTmp, mipo,tmp,m,multTable,addTable);
            ffSwitch = ffRetInt; ffRetInt = ffTmp; ffTmp = ffSwitch;
        }
        if(j==0 || power[j-1] == 0){
            lenCurGap++;
            continue;
        }
        matmul(matCharac+lenCurGap*m, ff, ffTmp, m, multTable,addTable);
        ffSwitch = ff; ff = ffTmp; ffTmp = ffSwitch;
        lenCurGap = 0;
    }
    copyFFElem(ffRetInt,ret);
}
\end{ccode}


Als nächstes folgt die standardmäßige binäre Exponentiation. Auch hier wird die
Potenz bereits in Binärdarstellung erwartet.

\begin{ccode}[caption={[\texttt{void powerFFElemSqM} aus 
 \url{../Sage/enumeratePCNs.c}]Aus \url{../Sage/enumeratePCNs.c}},
  label=lst:powerffelemsqm]
/**
 * Square and multiply
 *
 * !! ff is modified !!
 */
inline void powerFFElemSqM(struct FFElem *ff, struct FFElem *mipo,
        struct FFElem *ret, 
        int m, int *power, int powerLen,
        int *tmp, struct FFElem *ffTmp,
        int *multTable, int *addTable){
    int i,j,k;
    int lenCurGap = 0;
    struct FFElem *ffSwitch = 0;
    struct FFElem *ffRetInt = ret;
    // init ret to 1
    ffRetInt->el[0] = 1; ffRetInt->idcs[0] = 0; ffRetInt->len = 1;
    for(j=powerLen-1;j>=0;j--){
        if(power[j] == 1){
            multiplyFFElem(ffRetInt,ff,ffTmp, mipo,tmp,m,multTable,addTable);
            //switch ffTmp and ffRetInt
            ffSwitch = ffRetInt; ffRetInt = ffTmp; ffTmp = ffSwitch;
        }
        if(j>0){
            squareFFElem(ff,mipo,ffTmp,tmp,m,multTable,addTable);
            //switch ffTmp and ff
            ffSwitch = ff; ff = ffTmp; ffTmp = ffSwitch;
        }
    }
    copyFFElem(ffRetInt,ret);
}
\end{ccode}


\subsection{Primitivitätstest}
\label{subsub:primitivitaetstest}

Beim Testen eines Elements eines endlichen Körpers auf Primitivität bedienen
wir uns des wohlbekannten Satzes von Lagrange aus der Gruppentheorie und geben
zunächst ein kleines Lemma an, auf dem der dann folgende Algorithmus basiert.

\begin{lemma}
  Sei $u \in \F_q$ und
  \[ q-1 \speq= p_1^{\nu_1}\cdot\ldots\cdot p_r^{\nu_r}\]
  die Primfaktorzerlegung von $q-1$. Definiere
  für alle $i=1,\dots,r$
  \[ \bar p_i \speq{:=}  \frac{q-1}{p_i} \speq=
    p_1^{\nu_1}\cdot \ldots\cdot p_{i-1}^{\nu_{i-1}} \cdot
    p_i^{\nu_i-1}\cdot p_{i+1}^{\nu_{i+1}}\cdot\ldots\cdot p_r^{\nu_r}\,.\]
  Dann gilt: $u$ ist primitiv genau dann, wenn
  \[ u^{\bar p_i} \speq\neq 1\quad\forall i=1,\ldots,r \,.\]
\end{lemma}
\begin{proof}
  Klar.
\end{proof}

Es bleibt jedoch immer noch offen diese $r$ Potenzierungen möglichst gut
zu organisieren. 
%Nehmen wir an, die Primzahlen sind in der Primfaktorzerlegung
%\[ q-1 \speq= p_1^{\nu_1}\cdot\ldots\cdot p_r^{\nu_r}\]
%aufsteigend sortiert, also $p_1<p_2<\ldots < p_r$, so
%hat sich als besonders hilfreich erwiesen, die Potenzen $\bar n_i$ auf
%Basis der Potenzen
%\begin{itemize}
  %\item $d := \ggT\{ \bar n_i:\ i=1,\ldots,r\}$ und 
  %\item $d' := \ggT\{ \frac{\bar n_i}{d}:\ i=1,\ldots,r-1\}$
%\end{itemize}
%durchzuführen und die bereits berechneten Potenzen zu nutzen, 
%wie nachstehendes Beispiel veranschaulicht.
Dazu formulieren wir folgendes Lemma, das den Aufwand der Berechnung
in vielen Fällen deutlich verringert hat.

\begin{lemma}
  Sei $q-1 = p_1^{\nu_1}\cdot\ldots\cdot p_r^{\nu_r}$ die 
  absteigend sortierte Primfaktorzerlegung
  von $q-1$, d.h. $p_1>p_2>\ldots>p_r$. Notiere
  \begin{itemize}
    \item $\bar p_i := \tfrac{q-1}{p_i}$,
    \item $d := \ggT\{ \bar p_i:\ i=1,\ldots,r\}$, \quad
      $d' := \ggT\{ \frac{\bar p_i}{d}:\ i=2,\ldots,r\}$, 
    \item $v := u^d$, \quad $w := v^{d'}$,
    \item $\bar n_1 := \tfrac{\bar p_1}{d}$, \quad
      $\bar n_i := \tfrac{\bar p_i}{d\, d'}$ für $i=2,\ldots,r$,
    \item $u_2 := w^{\bar n_2}$ und 
      $u_i := w^{\bar n_i - \bar n_{i-1}}$ für $i=3,\ldots,r$.
  \end{itemize}
  Es gilt: $u \in \F_q$ ist genau dann nicht primitiv, falls eine der
  nachstehenden Bedingungen erfüllt ist:
  \begin{enumerate}
    \item $v \speq= 1$.
    \item $v^{\bar n_1} \speq= 1$.
    \item $w \speq= 1$.
    \item $u_2 \speq= 1$.
    \item $u_i\cdot u_{i-1} \speq= 1$ für ein $i=3,\ldots r$.
  \end{enumerate}
  Ferner gilt: Die Differenzen $\bar n_i-\bar n_{i-1}$ für 
  $i=3,\ldots,r$ sind alle größer $0$.
\end{lemma}
\begin{proof}
  Nach vorausgehendem Lemma müssen wir also zeigen, dass für alle
  $i=1,\ldots,r$ überprüft wird, ob $u^{\bar p_i} \neq 1$. Genau dann ist
  $u$ primitiv. Zunächst erkennen wir, dass mit 
  \[ v^{\bar n_1} \speq= u^{\bar p_1}\]
  und 
  \[ u_2 \speq= w^{\bar n_2} \speq= u^{\bar p_2} \]
  die ersten beiden Kofaktoren abgedeckt wären.
  Ist $v = 1$ bzw. $w = 1$ hat $u$ Ordnung $d$ bzw. $d\,d'$, ist also nicht
  primitiv. Sei also $i \in \{3,\ldots,r\}$, so folgern wir
  \[ u_i \cdot u_{i-1} \speq= 
    w^{\bar n_i - \bar n_{i-1}} \cdot u_{i-1} \speq=
    w^{\bar n_i - \bar n_{i-1} + \bar n_{i-1}} \speq= 
    w^{\bar n_i} \speq= u^{\bar p_i}\,.\]
  Zuletzt erkennen wir, dass aufgrund der absteigenden Sortierung der $p_i$s,
  die $\bar p_i$s und damit die $\bar n_i$s aufsteigend sortiert sind, also
  $\bar n_i - \bar n_{i-1} > 0$ für alle $i=3,\ldots,r$.
\end{proof}


\begin{beispiel}
  Sei $u \in \F_{3^{10}}$. Da
  \[ 3^{10}-1 \speq= 61 \cdot 11^{2} \cdot 2^{3} \,,\]
  sind folgende Potenzen von $u$ zu berechnen:
  \[ \begin{array}{l@{\ =\ }l@{}l@{}l@{\ =\ }l}
    \bar p_1 & 2^3 \cdot & 11^2 && 968\,, \\
    \bar p_2 & 2^3 \cdot & 11 & \cdot 61 & 5368\,,\\
    \bar p_3 & 2^2 \cdot & 11^2 & \cdot 61 & 29524\,.
    \end{array}\]
  Wir sehen jedoch dass die Potenzen
  \[ d \speq{:=} \ggT\{ \bar p_1,\bar p_2,\bar p_3\} 
    \speq= 2^2\cdot 11 \speq= 44 \]
  und 
  \[ d' \speq{:=} \ggT\{ \tfrac{\bar p_2}{d},\tfrac{\bar p_3}{d}\} \speq= 61\]
  uns die Arbeit erheblich erleichtern können:
  Berechnen wir $v := u^d = u^{44}$ und 
  $w := v^{d'} = v^{61}$ separat, so schreiben sich die restlichen Potenzen 
  für $\bar n_2 := 2$ und $\bar n_3 := 11$ wie folgt:
  \[ \begin{array}{l@{\ =\ }l@{}l}
    u^{\bar p_1} & v^2\,, \\
    u^{\bar p_2} & u_2 &\speq= w^2\,, \\
    u^{\bar p_3} & u_3\cdot u_2 &\speq= w^9 \cdot u_2 \,.
  \end{array} \]
  Es ist klar, dass in diesem Beispiel obiges Vorgehen eine erhebliche
  Verkleinerung der zu berechnenden Potenzen liefert, die jedoch nicht in allen
  Fällen erwartet werden kann.
\end{beispiel}


In nachstehender Implementierung sind die Potenzen $d$
mit @commonBarFactor@ und $d'$ mit \\
@commonBiggestBarFactor@ bezeichnet und
werden in $p$-adischer bzw. binärer Darstellung erwartet.  
Alle anderen zu berechnenden Potenzen, d.h. nach obigem Lemma die Liste
\[ [\bar n_1,\ \bar n_2,\ \bar n_3-\bar n_2,\ldots,\ 
  \bar n_r - \bar n_{r-1}]\,, \]
werden in 
@barFactors@ wiederum in $p$-adischer bzw. binärer Darstellung übergeben.
@barFactors@ ist dabei ein einziges @int@-Array,
wobei die jeweilige Länge der einzelnen Faktoren in dem @int@-Array
@lenBarFactors@ zu hinterlegen ist. Wie im Quelltext bemerkt, wird 
die Exponentiation $p$-adisch durchgeführt (siehe \autoref{lst:powerffelem}),
falls @matCharac@ ungleich @0@ ist, ansonsten binär 
(siehe \autoref{lst:powerffelemsqm}), wobei natürlich sicherzustellen ist, dass
die Potenzen in passender Darstellung vorliegen.

\begin{ccode}[caption={[\texttt{bool isPrimitive} aus 
 \url{../Sage/enumeratePCNs.c}]Aus \url{../Sage/enumeratePCNs.c}},
  label=lst:isPrimitive]
/**
 * Test if element is primitive.
 *
 * !! if matCharac is Zero, all powers are assumed as binary arrays !!
 *
 * !! fff,ffTmp,ffTmp2,ffTmp3,ffRet must be malloced !!
 * !! x is NOT modified !!
 */
inline bool isPrimitive(struct FFElem *ff, struct FFElem *mipo,
        int m,
        int *barFactors, int *lenBarFactors, int countBarFactors,
        int *commonBarFactor, int lenCommonBarFactor,
        int *commonBiggestBarFactor, int lenCommonBiggestBarFactor,
        struct FFElem **matCharac, 
        struct FFElem *fff, struct FFElem *ffff, struct FFElem *ffTmp,
        struct FFElem *ffTmp2, struct FFElem *ffRet, 
        int *tmp, int *multTable, int *addTable){
    int i;
    int curPos = 0;
    bool binarySqM = (matCharac == 0);
    struct FFElem *ffSwitch = 0;

    copyFFElem(ff,fff);
    // all barFactors are power of commonBarFactor
    if(binarySqM)
        powerFFElemSqM(fff,mipo,ffTmp,
                m,commonBarFactor,lenCommonBarFactor,
                tmp,ffTmp2,
                multTable,addTable);
    else 
        powerFFElem(fff,mipo,ffTmp,
                m,commonBarFactor,lenCommonBarFactor,
                matCharac,tmp,ffTmp2,
                multTable,addTable);
    if(isOne(ffTmp)) return false;
    //switch ffTmp and fff
    ffSwitch = fff; fff = ffTmp; ffTmp = ffSwitch;
    copyFFElem(fff,ffff);
    //test first barFactor
    if(binarySqM)
        powerFFElemSqM(ffff,mipo,ffTmp,
                m,barFactors,lenBarFactors[0],
                tmp,ffTmp2,
                multTable,addTable);
    else
        powerFFElem(ffff,mipo,ffTmp,
                m,barFactors,lenBarFactors[0],
                matCharac,tmp,ffTmp2,
                multTable,addTable);
    if(isOne(ffTmp)) return false;
    curPos += lenBarFactors[0];
    //test further factors which are powers of commonBiggestBarFactor
    //so first, calc y^commonBiggestBarFactor
    copyFFElem(fff, ffff);
    if(binarySqM)
        powerFFElemSqM(ffff,mipo,ffTmp,
                m,commonBiggestBarFactor,lenCommonBiggestBarFactor,
                tmp,ffTmp2,
                multTable,addTable);
    else
        powerFFElem(ffff,mipo,ffTmp,
                m,commonBiggestBarFactor,lenCommonBiggestBarFactor,
                matCharac,tmp,ffTmp2,
                multTable,addTable);
    if(isOne(ffTmp)) return false;
    ffSwitch = fff; fff = ffTmp; ffTmp = ffSwitch;
    for(i=1;i<countBarFactors;i++){
        // copy z (fff) to ffff
        copyFFElem(fff,ffff);
        // *** ffff == fff == y^commonBiggestBarFactor
        if(binarySqM)
            powerFFElemSqM(ffff, mipo, ffTmp, 
                    m,barFactors+curPos, lenBarFactors[i],
                    tmp,ffTmp2,
                    multTable,addTable);
        else
            powerFFElem(ffff, mipo, ffTmp, 
                    m,barFactors+curPos, lenBarFactors[i],
                    matCharac,tmp,ffTmp2,
                    multTable,addTable);

        if(i>1){
            multiplyFFElem(ffRet,ffTmp,ffTmp2,mipo,
                    tmp,m,multTable,addTable);
        }else{
            ffSwitch = ffTmp2; ffTmp2 = ffTmp; ffTmp = ffSwitch;
        }
        if(isOne(ffTmp2)) return false;
        curPos += lenBarFactors[i];
        ffSwitch = ffRet; ffRet = ffTmp2; ffTmp2 = ffSwitch;
    }
    return true;
}
\end{ccode}

\section{Frobenius-Auswertung und Test auf vollständige
  Erzeuger-Eigenschaft}

\subsection{Frobenius-Auswertung}

Sei wie immer $F := \F_q$ ein endlicher Körper und $E := \F_{q^m}$ eine
Körpererweiterung. Seien $\C_{k,t}$ ein verallgemeinerter
Kreisteilungsmodul über $F$ (vgl.
\thref{def:verallgemeinerter_kreisteilungsmodul}) und 
$u\in E$ ein Element, das wir als vollständigen Erzeuger 
in Betracht ziehen (vgl. \thref{def:vollst_erzeuger}). Nach 
\thref{satz:moduln_ueber_v_g} (3) ist $u$ genau dann ein vollständiger 
Erzeuger von $\C_{k,t}$, wenn
\[ \Ord_{q^d}(u) \speq= \Phi_{\nu(k),\, \frac{kt}{\nu(k)d}} \qquad
  \forall d \mid \tfrac{k\,t}{\nu(k)}\,.\]
Folglich müssen wir, um $q$-Ordnungen berechnen zu können, in der Lage sein,
für beliebige Zwischenkörper $E\mid K\mid F$ von Grad $d$ über $F$ und beliebige 
$f(x) \in K[x]$ 
\[ f(\sigma^d)(u) \in E\]
auswerten zu können, wobei wieder $\sigma: \bar F\to \bar F, x \mapsto x^q$ den
Frobenius von $F$ bezeichne. Bleibt die Frage, wie die verschiedenen
Zwischenkörper mit Hilfe der @FFElem@s gelesen werden können. Da wir $E$ jedoch
stets als Erweiterung über dem zu Grunde liegenden Primkörper betrachten 
(vgl. \autoref{sub:beschreibung_endliche_koerper}) ist dies völlig unklar.
Daher umgehen wir dieses Problem und betrachten eine beliebige Einbettung von
$K$ in $E$. Die Frage, ob man damit immer noch $q$-Ordnungen berechnen kann,
beantwortet nachstehendes Lemma.

\begin{lemma}
  \label{lemma:einbettung_egal}
  Seien $F := \F_p[y] \big/(m_F(y))$, $K := \F_p[y]\big/(m_K(y))$ und
  $E := \F_p[y]\big/(m_E(y))$ drei endliche Körper mit 
  $\deg(m_F) \mid \deg(m_K) \mid \deg(m_E)$. Notiere
  $q := p^{\deg(m_F)}$, $d := \tfrac{\deg(m_K)}{\deg(m_F)}$ und
  $\sigma: E \to E,\ v\mapsto v^q$ den Frobenius von $F$.
  Sind ferner
  \[ \alpha_1: \funcdef{K &\to& E,\\{} [y] &\mapsto& \alpha_1(y)} 
    \quad\text{und}\quad
    \alpha_2: \funcdef{K &\to& E,\\{} [y] &\mapsto& \alpha_2(y)} \]
  zwei injektive Körperhomomorphismen 
  ($[y]$ bezeichne dabei die Restklasse von $y$ in
  $\F_p[y]\big/(m_K(y))$), so gilt für $u\in E$:
  Entweder ist
  \[ \alpha_1(f)(\sigma^d)(u) \speq= 0 \quad\text{und}\quad
    \alpha_2(f)(\sigma^d)(u) \speq=0 \]
  oder
  \[ \alpha_1(f)(\sigma^d)(u) \speq\neq 0 \quad\text{und}\quad
    \alpha_2(f)(\sigma^d)(u) \speq\neq 0\,. \]
  Mit anderen Worten hängt also die Frage, ob eine Frobenius-Auswertung $0$ 
  ist oder nicht, nicht von der Wahl der konkreten Einbettung ab.
\end{lemma}
\begin{proof}
  Wir bemerken, dass sich $\alpha_1(y) \in \F_p[y]\big/(m_E(y))$ durch
  ein Polynom $\beta_1(y) \in \F_p[y]$ mit 
  $\deg(\beta_1) < \deg(m_E)$ repräsentieren lässt. Analog
  sei $\beta_2(y) \in \F_p[y]$ jenes für $\alpha_2(y)$. Da es sich bei
  $\alpha_1$ und $\alpha_2$ um injektive Körperhomomorphismen handelt, sind
  $\im \alpha_1$ und $\im\alpha_2$ wieder endliche Körper von gleicher Ordnung
  und $\beta_1(y)$ bzw. $\beta_2(y)$ sind über $\F_p$ irreduzible Polynome von
  Grad $\deg(m_K)$. Wegen der Eindeutigkeit endlicher Körper sind
  damit die Bilder von $\alpha_1$ und $\alpha_2$ als Körper isomorph, was
  obige Behauptung zeigt.
\end{proof}

%\begin{lemma}
  %\label{lemma:einbettung_egal}
  %Seien $F \mid K\mid E$ ein Turm endlicher Körper mit $[K:F] = d$ und 
  %$\sigma: \bar F\to \bar F$ der Frobenius von $F$. 
  %Sei $f(x) \in K[x]$ ein Polynom und $u\in E$. Für je zwei
  %injektive Körperhomomorphismen
  %$g,h: K\to E$ ist entweder
  %\[ h(f)(\sigma^d)(u) \speq= 0 \quad\text{und}\quad 
    %g(f)(\sigma^d)(u) \speq= 0\]
  %oder 
  %\[ h(f)(\sigma^d)(u) \speq\neq 0 \quad\text{und}\quad 
    %g(f)(\sigma^d)(u) \speq\neq 0\,,\]
  %wobei $h(f) \in E[x]$ koeffizientenweise zu lesen ist.\\
  %Mit anderen Worten hängt also die Frage, ob eine Frobenius-Auswertung $0$ 
  %ist oder nicht, nicht von der Wahl der konkreten Einbettung ab.
%\end{lemma}
%\begin{proof}
  %\marginpar{Müsste man hier ausführlicher argumentieren?}
  %Aufgrund der Eindeutigkeit endlicher Körper (z.B.
  %\thref{satz:eindeutigkeit_endlicher_koerper}) unterscheiden sich zwei
  %Einbettungen $g,h: K\to E$ lediglich um einen Automorphismus $a: E\to E$,
  %also $h = a \circ g$. Dies beweist aber bereits die Behauptung.
%\end{proof}


Damit können wir erst einmal davon ausgehen, dass die zu betrachtenden Polynome
bereits in $E[x]$ liegen; also vom Typ @FFPoly@ sind. Analog zu 
\autoref{lst:powerffelem} wird auch hier das Potenzieren durch
Matrixmultiplikation beschrieben, wobei sicherzustellen ist, dass die maximal
auftretende Matrixpotenz vorhanden ist, d.h. übergibt man ein Polynom @poly@
vom Grad $k$, so muss @mats@ als Array bestehend aus @FFElem*@ von Länge
$m\cdot k$ sein, wobei $m$ wiederum den Grad der Erweiterung von $E$ über dem
Primkörper meint. Das bedeutet insbesondere, dass die erste Matrix in @mats@
die Darstellungsmatrix zu $\sigma^1$ ist und der Fall $\sigma^0 = \id$ separat
betrachtet werden muss (vgl. Zeile \texttt{19} in \autoref{lst:applyFrob}).

Im Hinblick auf das Berechnen von $q$-Ordnungen, wo ein Körperelement meist
mehr als einmal einer Frobenius-Auswertung unterzogen werden muss, haben wir
die Möglichkeit bereitgestellt, bereits durchgeführte Matrixmultiplikationen in
@matmulCache@ zu speichern. Das Array @matmulCacheCalced@ gibt dabei an, welche
Stellen in @matmulCache@ bereits berechnet wurden. Selbstredend wird dieser 
Zwischenspeicher durch die Ausführung von @applyFrob@ fortwährend aktualisiert.

\begin{ccode}[caption={[\texttt{void applyFrob} aus 
 \url{../Sage/enumeratePCNs.c}]Aus \url{../Sage/enumeratePCNs.c}},
  label=lst:applyFrob]
/*
 * calculates g(sigma^frobPower)(x) where g is a polynomial 
 * and sigma the frobenius
 * application of frobenius is given by mats
 */
inline void applyFrob(struct FFElem *ff, struct FFElem *mipo,
        struct FFPoly *poly,
        struct FFElem **mats,
        int frobPower, struct FFElem *ret, 
        int m, int *tmp, struct FFElem *ffTmp, struct FFElem *ffTmp2,
        struct FFElem **matmulCache, bool *matmulCacheCalced,
        int *multTable, int *addTable){
    int i,j;
        
    ret->len = 0;
    for(i=0;i<poly->lenPoly;i++){
        if(poly->poly[i]->len == 0) continue;
        j = i*frobPower-1;
        if(i>0 && matmulCacheCalced[j] == true){
            multiplyFFElem(matmulCache[j],poly->poly[i],
                    ffTmp, mipo,
                    tmp,m,multTable,addTable);
            addFFElem(ret,ffTmp,ret,tmp,multTable,addTable);
        }else{
            if(i>0){
                matmul(mats+j*m, ff, ffTmp, m, multTable,addTable);
                //update matmulCache
                copyFFElem(ffTmp, matmulCache[j]);
                matmulCacheCalced[j] = true;
            }else{
                copyFFElem(ff,ffTmp);
            }
            //go on and multiply ffTmp with current coefficient
            multiplyFFElem(ffTmp, poly->poly[i],
                    ffTmp2, mipo,
                    tmp,m,multTable,addTable);
            addFFElem(ret,ffTmp2,ret,tmp,multTable,addTable);
        }
    }
}
\end{ccode}

Falls bereits klar ist, dass für ein gegebenes Element nur eine
Frobenius-Auswertung vollzogen wird, so ist der @matmulCache@ überflüssig und
führt zur Variante @applyFrob_noCache@, die ansonsten identisch zu Obigem ist.

\begin{ccode}[caption={[\texttt{void applyFrob\_noCache} aus 
 \url{../Sage/enumeratePCNs.c}]Aus \url{../Sage/enumeratePCNs.c}},
  label=lst:applyFrob_noCache]
/*
 * calculates g(sigma^frobPower)(x) where g is a polynomial 
 * and sigma the frobenius
 * application of frobenius is given by mats
 */
inline void applyFrob_noCache(struct FFElem *ff, struct FFElem *mipo,
        struct FFPoly *poly,
        struct FFElem **mats,
        int frobPower, struct FFElem *ret, 
        int m, int *tmp, struct FFElem *ffTmp, struct FFElem *ffTmp2,
        int *multTable, int *addTable){
    int i,j;
    ret->len = 0;
    
    for(i=0;i<poly->lenPoly;i++){
        if(poly->poly[i]->len == 0) continue;
        if(i>0){
            j = i*frobPower-1;
            matmul(mats+j*m, ff, ffTmp, m, multTable,addTable);
        }else{
            copyFFElem(ff,ffTmp);
        }
        multiplyFFElem(ffTmp, poly->poly[i],
                ffTmp2, mipo,
                tmp,m,multTable,addTable);
        addFFElem(ret,ffTmp2,ret,tmp,multTable,addTable);
    }
}
\end{ccode}


\subsection{Testen von vollständigen Erzeugern}

Wie bereits erwähnt, ist $u\in \F_{q^m}$ über $\F_q$ genau dann ein vollständiger
Erzeuger eines verallgemeinerten Kreisteilungsmoduls $\C_{k,t}$, wenn
\[ \Ord_{q^d}(u) \speq= \Phi_{\nu(k),\, \frac{kt}{\nu(k)d}} \qquad
  \forall d\mid \tfrac{k\,t}{\nu(k)}\,.\]
Analog zum Primitivitätstest reicht es, lediglich maximale Kofaktoren des
jeweiligen verallgemeinerten Kreisteilungspolynoms zu testen. 
Dies ist in nachstehendem Lemma für \emph{einen} Teiler
formuliert, wobei wir uns ohne Einschränkung
auf den Fall $d=1$ beschränken können; andernfalls vollführe man lediglich eine
Änderung der Notation von $q$, $m$, $k$ und $t$.

\begin{lemma}
  Seien $u\in \F_{q^m}$ und $\Phi_{k,t}(x) \in \F_q[x]$ ein verallgemeinertes
  Kreisteilungspolynom. Sei ferner
  \[ \Phi_{k,t}(x)  \speq= f_1(x)^{\nu_1} \cdot \ldots \cdot
    f_r(x)^{\nu_r}\quad \in \F_q[x]\]
  die vollständige Faktorisierung von $\Phi_{k,t}$ über $\F_q$ und
  bezeichne $F_i(x) := \tfrac{\Phi_{k,t}(x)}{f_i(x)}$ den jeweiligen maximalen
  Kofaktor von $f_i$ in $\Phi_{k,t}$ für alle $i=1,\ldots,r$.
  Seien zuletzt $h: \F_q \to \F_{q^m}$ ein injektiver Körperhomomorphismus und 
  $\sigma:\bar \F_q \to \bar \F_q, x\mapsto x^q$ der Frobenius von $\F_q$, 
  so ist $\Ord_q(u) = \Phi_{k,t}$ genau dann, wenn
  \[h(\Phi_{k,t})(\sigma)(u) \speq= 0 
    \qquad\text{und}\qquad 
    h(F_i)(\sigma)(u) \speq\neq 0 \quad \forall i=1,\ldots,r \,.\]
\end{lemma}
\begin{proof}
  Klar per Definition der $q$-Ordnung und \thref{lemma:einbettung_egal}.
\end{proof}


Nun können wir auf diese Weise leicht eine Implementierung eines Tests auf
vollständige Erzeuger-Eigenschaft angeben, wenn wir davon ausgehen, dass die
Berechnung der maximalen Kofaktoren bereits geschehen ist. In
\autoref{lst:testSubmod} ist also sicherzustellen, dass in dem Array
@polys@ sowohl das verallgemeinerte Kreisteilungspolynom, als auch alle 
maximalen Kofaktoren auftauchen. Das Array @evalToZero@ gibt dabei an, ob 
bei Vorliegen eines vollständigen Erzeugers die Auswertung am jeweiligen
Polynom $0$ ergibt (@true@) oder nicht (@false@). Der Rückgabewert der Funktion
ist selbstredend ein @bool@ mit der Information, ob das getestete Element @ff@
ein vollständiger Erzeuger dieses Kreisteilungsmoduls ist (@true@) oder nicht
(@false@).

\begin{ccode}[caption={[\texttt{bool testSubmod} aus 
 \url{../Sage/enumeratePCNs.c}]Aus \url{../Sage/enumeratePCNs.c}},
  label=lst:testSubmod]
inline bool testSubmod(struct FFElem *ff, struct FFElem *mipo, 
        struct FFPoly **polys,
        int polysCount, bool *evalToZero, 
        struct FFElem **mats, int *frobPowers,
        int m, int *tmp, 
        struct FFElem *ffTmp, struct FFElem *ffTmp2, struct FFElem *ffTmp3, 
        struct FFElem **matmulCache, bool *matmulCacheCalced,
        int *multTable, int *addTable){
    int i;
    int goodCounter = 0;
    for(i=0;i<polysCount;i++){
        applyFrob(ff,mipo,
                polys[i],
                mats,frobPowers[i], ffTmp,
                m,tmp,ffTmp2,ffTmp3,
                matmulCache,matmulCacheCalced,
                multTable,addTable);
        if( isZero(ffTmp) == evalToZero[i] ){
            goodCounter++;
        }else{
            return false;
        }
    }
    if(goodCounter == polysCount){
        return true;
    }
    return false;
}
\end{ccode}


Ferner bieten wir die Möglichkeit ein Element auf vollständige
Erzeuger-Eigenschaft für mehrere verallgemeinerte Kreisteilungsmoduln zu
testen, wie \autoref{lst:testAllSubmods} zeigt. @decompCount@ ist dabei die
Anzahl der zu testenden verallgemeinerten Kreisteilungsmoduln und das Array
\newline
@polysCountPerDecomp@ gibt die Anzahl der Polynome für den jeweiligen
Kreisteilungsmodul an. Das Array @bool *toTestIndicator@ legt fest, welche
Kreisteilungsmodule getestet werden. Der Rückgabewert -- anders als in 
\autoref{lst:testSubmod} -- ist ein @int@, der die Werte $-1$, falls @ff@ kein
vollständiger Erzeuger der getesteten Kreisteilungsmoduln ist, oder $i$, falls
@ff@ gerade vollständiger Erzeuger des $i$-ten getesteten Kreisteilungsmoduls
ist, annimmt. Ferner bricht die Funktion ab, falls @ff@ ein vollständiger
Erzeuger ist, da es klar sein sollte, dass diese Eigenschaft lediglich für
\emph{einen} verallgemeinerten Kreisteilungsmodul zutreffen kann.

\begin{ccode}[caption={[\texttt{int testAllSubmods} aus 
 \url{../Sage/enumeratePCNs.c}]Aus \url{../Sage/enumeratePCNs.c}},
  label=lst:testAllSubmods]
inline int testAllSubmods(struct FFElem *ff, struct FFElem *mipo, 
        int decompCount, struct FFPoly **polys,
        int *polysCountPerDecomp, bool *evalToZero, 
        struct FFElem **mats, int *frobPowers, bool *toTestIndicator,
        int m, int *tmp, 
        struct FFElem *ffTmp, struct FFElem *ffTmp2, struct FFElem *ffTmp3,
        struct FFElem **matmulCache, bool *matmulCacheCalced,
        int *multTable, int *addTable){
    if(ff->len == 0) return -1;
    int i,j,k;
    int goodCounter = 0;
    int curDecompPosition = 0;
    for(i=0;i<decompCount;i++){
        if(toTestIndicator != 0 && toTestIndicator[i] == false){
            curDecompPosition += polysCountPerDecomp[i];
            continue;
        }
        goodCounter = 0;
        for(j=0;j<polysCountPerDecomp[i];j++){
            applyFrob(ff,mipo,
                    polys[curDecompPosition+j],
                    mats,frobPowers[curDecompPosition+j], ffTmp,
                    m,tmp,ffTmp2,ffTmp3,
                    matmulCache,matmulCacheCalced,
                    multTable,addTable);
            if( isZero(ffTmp) == evalToZero[curDecompPosition+j] ){
                goodCounter++;
            }else break;
        }
        if(goodCounter == polysCountPerDecomp[i]){
            return i;
        }
        curDecompPosition += polysCountPerDecomp[i];
    }
    return -1;
}
\end{ccode}


\section{Implementierung der gezielten Enumeration}

\subsection{Enumeration eines verallgemeinerten Kreisteilungsmoduls}
\label{sub:enumeration_eines_verallg_kreisteilungsmoduls}

Sei $E := \F_{q^m}$ über $F := \F_q$ eine Körpererweiterung endlicher Körper. Die Frage
nach einer Enumeration aller vollständig normaler Elemente dieser Erweiterung
lässt sich nach dem Zerlegungssatz (\thref{satz:zerlegungssatz}) auf die
separate Enumeration von verallgemeinerten Kreisteilungsmoduln zurückführen.
Daher starten wir mit einem verallgemeinerten Kreisteilungsmodul $\C_{k,t}$
über $\F_q$. Sicherlich könnte man alle $q^m$ Elemente von $E$ testen, ob sie
vollständige Erzeuger von $\C_{k,t}$ sind, was jedoch einen unnötig großen
Aufwand darstellen würde. Sei nämlich $u\in E$ ein vollständiger Erzeuger von
$\C_{k,t}$, so erhalten wir alle weiteren Elemente dieses Kreisteilungsmoduls
durch Anwendung von \thref{kor:moduln_ueber_v_g}, was wir hier in passender
Notation noch einmal formulieren möchten.

\begin{lemma}
  \label{lemma:moduln_durch_polys}
  Sei $u\in E$ ein vollständiger Erzeuger von $\C_{k,t}$ über $F$. 
  Dann gilt
  \[ \C_{k,t} \speq= \big\{ f(\sigma)(u):\ f(x) \in F[x]_{<\varphi(k)t}\big\} \,,\]
  wobei wiederum $\sigma$ den Frobenius von $F$ und 
  $\varphi$ die Eulersche Phifunktion notieren.
\end{lemma}
\begin{proof}
  \thref{kor:moduln_ueber_v_g} mit der Erkenntnis, dass
  $\deg(\Phi_{k,t}) = \varphi(k)\,t$.
\end{proof}

Nun ist klar, wie wir ausgehend von einem Erzeuger alle weiteren generieren
können: Sei $u \in E$ ein vollständiger Erzeuger von $\C_{k,t}$ über $F$, so 
berechnen wir iterativ $v := f(\sigma)(u)$ für alle $f \in F[x]_{<\varphi(k)t}$
mit @applyFrob_noCache@ (\autoref{lst:applyFrob_noCache}) 
und testen anschließend $v$ auf vollständige Erzeuger-Eigenschaft mit 
@testSubmod@ (\autoref{lst:testSubmod}). 
Die Generierung der $f$s erfolgt direkt in \autoref{lst:calcSubmoduleElements},
wobei die Elemente aus $F$ wieder mittels eines injektiven Körperhomomorphismus 
$F\to E$ als @FFElem*@-Array namens @elementsF@ übergeben werden.

Wie erwähnt, müssen wir dieses Verfahren natürlich mit einem vollständigen
Erzeuger starten. Es ist sicherzustellen, dass sich dieser am aktuellen Knoten 
der Liste @struct Node *root@ befindet.

\begin{ccode}[caption={[\texttt{void calcSubmoduleElements} aus 
 \url{../Sage/enumeratePCNs.c}]Aus \url{../Sage/enumeratePCNs.c}},
  label=lst:calcSubmoduleElements]
inline void calcSubmoduleElements(struct Node *root,
        struct FFElem *mipo,
        int maxLenPoly, 
        unsigned long long *genCounts, int curGen,
        struct FFPoly **polys, int polysCount, bool *evalToZero,
        struct FFElem **mats, int matLen, int *frobPowers,
        struct FFElem **elementsF,
        int m, int q, int *tmp,
        struct FFElem *ffTmp, struct FFElem *ffTmp2, struct FFElem *ffTmp3, 
        struct FFElem *ffTmp4,
        struct FFElem **matmulCache, bool *matmulCacheCalced,
        int *multTable, int *addTable){
    int i,j;
    struct Node *curRoot = root;
    struct FFElem *ff = root->ff;
    int *curPoly = malloc( maxLenPoly*sizeof(int) );
    struct FFPoly *curFPoly = malloc( sizeof(struct FFPoly) );
    curFPoly->poly = malloc( maxLenPoly*sizeof(struct FFElem*) );
    curFPoly->lenPoly = 0;
    
    initPoly(curPoly,maxLenPoly);
    curPoly[0] = 2;
    int curLenPoly = 1;
    if( q == 2 && maxLenPoly > 1){
        curLenPoly = 2;
        curPoly[0] = 0;
        curPoly[1] = 1;
    }
    if(q != 2 || maxLenPoly > 1){
        while(true){
            //setup curFPoly
            for(i=0;i<curLenPoly;i++)
                curFPoly->poly[i] = elementsF[curPoly[i]];
            curFPoly->lenPoly = curLenPoly;
            //apply Frobenius
            applyFrob_noCache(ff,mipo,
                    curFPoly,
                    mats,1, ffTmp, //return value
                    m,tmp,ffTmp2,ffTmp3,
                    multTable,addTable);
            //test generated element
            for(i=0;i<matLen;i++) matmulCacheCalced[i] = false;
            if(testSubmod(ffTmp, mipo,
                    polys,polysCount,evalToZero,
                    mats,frobPowers,m,tmp,
                    ffTmp2,ffTmp3,ffTmp4,
                    matmulCache,matmulCacheCalced, multTable,addTable)){
                curRoot = appendToEnd(curRoot,ffTmp,m);
                genCounts[curGen]++;
            }
            //generate next element
            curPoly[0] += 1;
            if( curPoly[0] == q ){
                for(i=0;i<maxLenPoly-1 && curPoly[i]==q;i++){
                    curPoly[i] = 0;
                    curPoly[i+1] += 1;
                }
                if(i+1>curLenPoly)
                    curLenPoly = i+1;
                if( curPoly[maxLenPoly-1]==q){
                    break;
                }
            }
        }
    }
    free(curPoly);
    free(curFPoly->poly);
    free(curFPoly);
}
\end{ccode}

Wie man in obigem Listing erkennt, startet die Erzeugung der Polynome aus 
$F[x]_{<\varphi(k)t}$ beim Polynom $2 \in F[x]$ (falls es die Charakteristik
zulässt), da $1\in F[x]$ ja wieder $(1)(\sigma)(u) = \id(u) = u$ liefert.
@maxLenPoly@ gibt dabei die maximale Länge der zu betrachtenden Polynome an (in
hiesiger Notation also @maxLenPoly@$= \varphi(k)t+1$). Die Polynome selbst
werden in zwei Schritten erzeugt: Sei $l := $@maxLenPoly@, so
durchläuft das @int@-Array @curPoly@ alle Elemente aus $\{0,\ldots,q-1\}^l$.
Das korrekte Polynom in $E[x]$ wird dann durch Einsetzen jeder Stelle 
dieses Tupels aus $\{0,\ldots,q-1\}^l$ in die @elementsF@ erzeugt und in 
@curFPoly@ gespeichert.

Die Anzahl der berechneten Erzeuger wird im @int@-Array @genCounts@ an der
Stelle @curGen@ gespeichert und da unsere Suche auf vollständig normale
Elemente abzielt, werden die konkreten Erzeuger durch @appendToEnd@ 
an die verkettete Liste @struct Node *root@ angehängt und 
damit für späteres Zusammensetzen gespeichert.


\subsubsection{Verkettete Listen zum Speichern berechneter vollständiger Erzeuger}

Die verkettete Liste ist dabei wie folgt aufgebaut.

\begin{ccode}[caption={[\texttt{struct Node} aus 
 \url{../Sage/enumeratePCNs.c}]Aus \url{../Sage/enumeratePCNs.c}},
  label=lst:structNode]
struct Node {
    struct FFElem *ff;
    struct Node *next;
};
\end{ccode}

Ebenfalls implementiert sich das Anheften eines Elements ans Ende der Liste 
erwartungsgemäß, wobei zu bemerken gilt, dass der neue Endknoten zurückgegeben 
wird.
Auf diese Weise muss nicht bei jedem Anheften die komplette Liste durchlaufen
werden. Das Element @struct FFElem *element@ wird dabei kopiert, so dass es
anschließend weiterverwendet werden kann und die Liste unverändert bleibt 
(vgl. Anwendung in \autoref{lst:calcSubmoduleElements}).

\begin{ccode}[caption={[\texttt{Node *appendToEnd} aus 
 \url{../Sage/enumeratePCNs.c}]Aus \url{../Sage/enumeratePCNs.c}},
  label=lst:appendToEnd]
/**
 * appends element to end of root, where element is copied to new FFElem.
 */
inline struct Node *appendToEnd(struct Node *root, struct FFElem *element,int m){
    struct Node *nextNode = root;
    if( nextNode != 0){
        while(nextNode->next != 0){
            nextNode = nextNode->next;
        }
        if( nextNode->ff != 0){
            nextNode->next = malloc( sizeof(struct Node) );
            nextNode = nextNode->next;
        }
        if( nextNode != 0){
            nextNode->next = 0;
            nextNode->ff = mallocFFElem(m);
            copyFFElem(element,nextNode->ff);
            return nextNode;
        }
    }
    return NULL;
}
\end{ccode}


Wie üblich in \Clang, ist es hilfreich das Freigeben von Speicher in eine 
eigene Funktion zu setzen.

\begin{ccode}[caption={[\texttt{void freeNode} aus 
 \url{../Sage/enumeratePCNs.c}]Aus \url{../Sage/enumeratePCNs.c}},
  label=lst:freeNode]
inline void freeNode(struct Node* head){
    struct Node *next_n = NULL;
    struct Node *tmp_n = NULL;
    for(tmp_n=head; tmp_n !=NULL; ){
        next_n = tmp_n->next;
        freeFFElem(tmp_n->ff);
        free(tmp_n);
        tmp_n = next_n;
    }
    head = 0;
}
\end{ccode}

\subsection{Dynamische Enumeration des größten Kreisteilungsmoduls}
\label{sub:dynamische_enumeration}

Da der Zerlegungssatz (\thref{satz:zerlegungssatz}) nicht immer eine echte
Zerlegung liefert (sich also alle vollständig normalen Elemente auf einen
einzigen Modul konzentrieren können) und in vielen Zerlegungen ein 
verallgemeinerter
Kreisteilungsmodul vorkommt, der verglichen mit den anderen Moduln dieser
Zerlegung besonders viele Elemente enthält, hat sich die Speicherung
\emph{aller} Erzeuger als schlecht erwiesen. Daher sind wir dazu übergegangen,
den größten Kreisteilungsmodul dynamisch zu enumerieren. Das bedeutet, dass
alle anderen verallgemeinerten Kreisteilungsmoduln vorab durch 
@calcSubmoduleElements@ (\autoref{lst:calcSubmoduleElements}) behandelt werden.
Bei der Enumeration des größten nutzen wir dann diese Informationen und setzen
die gefundenen Erzeuger zu einem vollständig normalen Element zusammen. Dies
können wir auf Primitivität durch @isPrimitive@
(\autoref{lst:isPrimitive}) testen und abschließend verwerfen, da es uns ja nur
auf eine Enumeration und nicht auf die konkrete Angabe der 
(primitiv) vollständig normalen Elemente ankommt.

Die bereits berechneten vollständigen Erzeuger werden durch 
das Array von Listen @struct Node **roots@ übergeben. @decompCount@ gibt dabei
die Anzahl aller (also inklusive des größten) verallgemeinerten
Kreisteilungsmoduln an. Alle anderen Variablen wurden bereits in den vorherigen
Funktionen erklärt, wobei noch bemerkt werden sollte, dass diesmal die
Erzeugung der Polynome bei $1 \in F[x]$ startet, da der bereits gefundene
Erzeuger des größten Kreisteilungsmoduls auch Teil einer gültigen Kombination
zu einem vollständig normalen Element ist. Dieser „Fehler” in der Berechnung
der Anzahl @genCounts@ wird in Zeile \texttt{120} am Ende der Funktion
korrigiert. Der Erzeuger selbst befindet sich wieder am aktuellen Knoten der
letzten Liste des Arrays @roots@, da die Datenstrukturen so
aufgebaut werden, dass dieser größte verallgemeinerte Kreisteilungsmodul der
letzte ist (siehe Zeilen \texttt{26} und \texttt{28}).


\begin{ccode}[caption={[\texttt{unsigned long long
  processLastSubmoduleAndTestPrimitivity} aus 
 \url{../Sage/enumeratePCNs.c}]Aus \url{../Sage/enumeratePCNs.c}},
  label=lst:processLastSubmodule]
unsigned long long processLastSubmoduleAndTestPrimitivity(struct Node **roots,
        struct FFElem *mipo, int decompCount,
        int maxLenPoly, 
        unsigned long long *genCounts, 
        struct FFPoly **polys, int polysCount, bool *evalToZero,
        struct FFElem **mats, int matLen, int *frobPowers,
        struct FFElem **elementsF,
        int m, int q, 
        int *barFactors, int *lenBarFactors, int countBarFactors,
        int *commonBarFactor, int lenCommonBarFactor,
        int *commonBiggestBarFactor, int lenCommonBiggestBarFactor,
        struct FFElem **matCharac,
        struct FFElem **matmulCache, bool *matmulCacheCalced,
        int *multTable, int *addTable){
    //generate temporary variables
    struct FFElem *fff = mallocFFElem(m);
    struct FFElem *ffff = mallocFFElem(m);
    struct FFElem *ffTmp = mallocFFElem(m);
    struct FFElem *ffTmp2 = mallocFFElem(m);
    struct FFElem *ffTmp3 = mallocFFElem(m);
    struct FFElem *ffTmp4 = mallocFFElem(m);
    struct FFElem *ffTmp5 = mallocFFElem(m);
    int *tmp = malloc(m*sizeof(int));

    int i,j;
    int curGen = decompCount-1;
    struct Node **curRoots = malloc( decompCount*sizeof(struct Node*) );
    struct FFElem *ff = roots[curGen]->ff;
    int *curPoly = malloc( maxLenPoly*sizeof(int) );
    struct FFPoly *curFPoly = malloc( sizeof(struct FFPoly) );
    curFPoly->poly = malloc( maxLenPoly*sizeof(struct FFElem*) );
    curFPoly->lenPoly = 0;
    
    initPoly(curPoly,maxLenPoly);
    curPoly[0] = 1;
    int curLenPoly = 1;

    unsigned long long pcn = 0;
    while(true){
        //setup curFPoly
        for(i=0;i<curLenPoly;i++)
            curFPoly->poly[i] = elementsF[curPoly[i]];
        curFPoly->lenPoly = curLenPoly;
        //apply Frobenius
        applyFrob_noCache(ff,mipo,
                curFPoly,
                mats,1, fff, //return value
                m,tmp,ffTmp,ffTmp2,
                multTable,addTable);
        //test generated element
        for(i=0;i<matLen;i++) matmulCacheCalced[i] = false;
        if(testSubmod(fff, mipo,
                polys,polysCount,evalToZero,
                mats,frobPowers,m,tmp,
                ffTmp,ffTmp2,ffTmp3,
                matmulCache,matmulCacheCalced, multTable,addTable)){
            genCounts[curGen]++;
            // build element as sum of already calced Nodes and ffTmp
            for(i=0;i<decompCount;i++) curRoots[i] = roots[i];
            // cycle through Nodes, build element and test primitivity
            while(true){
                copyFFElem(fff,ffff);
                //build element
                for(i=0;i<decompCount-1;i++){
                    addFFElem(ffff, curRoots[i]->ff, ffff, tmp,
                            multTable,addTable);
                }
                //test primitivity
                if(countBarFactors > 0){
                    if(isPrimitive(ffff, mipo,m,
                                barFactors,lenBarFactors,countBarFactors,
                                commonBarFactor,lenCommonBarFactor,
                                commonBiggestBarFactor,lenCommonBiggestBarFactor,
                                matCharac,
                                ffTmp,ffTmp2,ffTmp3,ffTmp4,ffTmp5,
                                tmp,multTable,addTable)){
                        pcn++;
                    }
                }

                //next element
                curRoots[0] = curRoots[0]->next;
                if( curRoots[0] == 0 ){
                   for(i=0;i<decompCount-1 && curRoots[i]==0;i++){
                       curRoots[i] = roots[i];
                       curRoots[i+1] = curRoots[i+1]->next;
                   }
                }
                if( curRoots[decompCount-1] == 0){
                   break;
                }
            }
        }
        //generate next element
        curPoly[0] += 1;
        if( curPoly[0] == q ){
            for(i=0;i<maxLenPoly-1 && curPoly[i]==q;i++){
                curPoly[i] = 0;
                curPoly[i+1] += 1;
            }
            if(i+1>curLenPoly)
                curLenPoly = i+1;
            if( curPoly[maxLenPoly-1]==q){
                break;
            }
        }
    }
    free(curPoly);
    free(curFPoly->poly);
    free(curFPoly);
    freeFFElem(fff);
    freeFFElem(ffff);
    freeFFElem(ffTmp);
    freeFFElem(ffTmp2);
    freeFFElem(ffTmp3);
    freeFFElem(ffTmp4);
    freeFFElem(ffTmp5);

    //we added first element twice
    genCounts[curGen]--;
    return pcn;
}
\end{ccode}


\begin{bemerkung}
  \label{bem:kein_test_auf_primitivitaet}
  Wie Zeile \texttt{69} zu erkennen gibt, kann man durch das Setzen von
  @countBarFactors = 0@ den Test auf Primitivität überspringen. Dies ist
  sinnvoll, wenn man nur an der Anzahl der vollständig normalen Elemente
  interessiert ist.
\end{bemerkung}


\subsubsection{Vorbereiten der Enumeration zum Auffinden vollständiger
  Erzeuger}

Alle bisher betrachteten Verfahren basierten immer auf der Annahme, dass
bereits ein vollständiger Erzeuger eines Kreisteilungsmoduls bereits gefunden
ist. Es ist klar, dass man diese irgendwann suchen muss, was die Funktion
@processFiniteField@ bewerkstelligt. Gleichzeitig bildet sie den Wrapper, der
von \sage aufgerufen wird und als Rückgabewert @unsigned long long@ die Anzahl
der primitiv vollständig normalen Elemente trägt. Alle zu übergebenden
Parameter werden in \sage erzeugt und wurden bereits erklärt.


\begin{ccode}[caption={[\texttt{unsigned long long processFiniteField} aus 
 \url{../Sage/enumeratePCNs.c}]Aus \url{../Sage/enumeratePCNs.c}},
  label=lst:processFiniteField]
unsigned long long processFiniteField(struct FFElem *mipo, int decompCount,
        struct FFPoly **polys, int *polysCountPerDecomp,
        bool *evalToZero, int *maxLenPolysPerDecomp,
        struct FFElem **mats, int matLen, int *frobPowers,
        unsigned long long *genCounts, int m, int charac, int q,
        int *barFactors, int *lenBarFactors, int countBarFactors,
        int *commonBarFactor, int lenCommonBarFactor,
        int *commonBiggestBarFactor, int lenCommonBiggestBarFactor,
        struct FFElem **matCharac, struct FFElem **elementsF,
        int *multTable, int *addTable){
    time_t TIME = time(NULL);
    int i,j;

    //setup temporary variables ----------------------------------------------
    int *tmp = malloc(m*sizeof(int));
    struct FFElem *ff = mallocFFElem(m);
    initPoly(ff->el,m);
    struct FFElem *ffRet = mallocFFElem(m);
    struct FFElem *ffTmp = mallocFFElem(m);
    struct FFElem *ffTmp2 = mallocFFElem(m);
    struct FFElem *ffTmp3 = mallocFFElem(m);
    struct FFElem *ffTmp4 = mallocFFElem(m);
    
    struct FFElem **matmulCache = malloc(matLen*sizeof(struct FFElem));
    for(i=0;i<matLen;i++) matmulCache[i] = mallocFFElem(m);
    bool *matmulCacheCalced = malloc(matLen*sizeof(bool));
    
    bool *toTestIndicator = malloc(decompCount*sizeof(bool));
    struct Node **roots = malloc( decompCount*sizeof(struct Node) );
    struct Node **curRoots = malloc(decompCount*sizeof(struct Node*));
    for(i=0;i<decompCount;i++){
        roots[i] = malloc( sizeof(struct Node) );
        roots[i]->ff = 0;
        roots[i]->next = 0;
        curRoots[i] = roots[i];
        toTestIndicator[i] = true;
    }
    //------------------------------------------------------------------------
    
    int foundCounter = 0;
    for(i=0;i<decompCount;i++) genCounts[i] = 0;

    // chase for elements ----------------------------------------------------
    while(true){
        for(i=0;i<matLen;i++) matmulCacheCalced[i] = 0;
        int curGen = testAllSubmods(ff,mipo,decompCount,
                polys,polysCountPerDecomp,evalToZero,
                mats,frobPowers,toTestIndicator,
                m,tmp,ffTmp,ffTmp2,ffTmp3,
                matmulCache,matmulCacheCalced,
                multTable,addTable);
        if( curGen != -1 ){
            if(toTestIndicator[curGen] == true){
                genCounts[curGen]++;
                appendToEnd(roots[curGen], ff, m);
                foundCounter++;
                toTestIndicator[curGen] = false;
            }
            if(foundCounter == decompCount) break;
        }
        //generate next element
        // (for sure there is a more efficient method)
        ff->el[0] += 1;
        if( ff->el[0] == charac ){
            for(i=0; i<m-1 && ff->el[i]==charac; i++){
                ff->el[i] = 0;
                ff->el[i+1] += 1;
            }
            if( ff->el[m-1] == charac )
                break;
        }
        updateFFElem(ff,m);
    }
    if( foundCounter != decompCount ){
        printf("BAAAD ERROR!!! foundCounter=%i < decompCount=%i\n",
                foundCounter,decompCount);
        exit(0);
    }
    printf("finding time: %.2f\n", (double)(time(NULL)-TIME));
    //------------------------------------------------------------------------
    


    // Process found elements ------------------------------------------------
    int curDecompPosition = 0;
    for(i=0;i<decompCount-1;i++){
        calcSubmoduleElements(roots[i], mipo,
                maxLenPolysPerDecomp[i],  // *** == maxLenPoly
                genCounts,i, // *** i == curGen
                polys+curDecompPosition, polysCountPerDecomp[i],
                evalToZero+curDecompPosition,
                mats, matLen, frobPowers+curDecompPosition,
                elementsF,
                m,q,tmp,
                ffTmp,ffTmp2,ffTmp3,ffTmp4,
                matmulCache,matmulCacheCalced,
                multTable,addTable);
        curDecompPosition += polysCountPerDecomp[i];
    }
    printf("all not last time: %.2f\n", (double)(time(NULL)-TIME));
    //------------------------------------------------------------------------
    
    // Process last Decomposition --------------------------------------------
    int curGen = decompCount-1;
    unsigned long long pcn = 
        processLastSubmoduleAndTestPrimitivity(roots,mipo,decompCount,
            maxLenPolysPerDecomp[curGen],  // *** == maxLenPoly
            genCounts,
            polys+curDecompPosition,polysCountPerDecomp[curGen],
            evalToZero+curDecompPosition,
            mats,matLen,frobPowers+curDecompPosition,
            elementsF,
            m,q,
            barFactors,lenBarFactors,countBarFactors,
            commonBarFactor,lenCommonBarFactor,
            commonBiggestBarFactor,lenCommonBiggestBarFactor,
            matCharac,
            matmulCache,matmulCacheCalced,
            multTable,addTable);
    //------------------------------------------------------------------------

    //free variables
    for(i=0;i<decompCount;i++)
        freeNode(roots[i]);
    free(roots); free(curRoots);

    //free temporary variables
    free(tmp);
    freeFFElem(ff);
    freeFFElem(ffRet);
    freeFFElem(ffTmp);
    freeFFElem(ffTmp2);
    freeFFElem(ffTmp3);
    freeFFElem(ffTmp4);
    for(i=0;i<matLen;i++) freeFFElem(matmulCache[i]);
    free(matmulCache);
    free(matmulCacheCalced);
    free(toTestIndicator);

    
    printf("total time: %.2f\n", (double)(time(NULL)-TIME));
    return pcn;
}
\end{ccode}

Wie zu erkennen ist, erfolgt die Suche nach vollständigen Erzeugern zunächst
durch iterative Enumeration aller Elemente. Wurde ein vollständiger Erzeuger
gefunden, so wird die jeweilige Stelle des @toTestIndicator@s umgeschaltet,
wodurch der zugehörige verallgemeinerte Kreisteilungsmodul in 
@testAllSubmods@ nicht mehr berücksichtigt wird. Ist für jeden
Kreisteilungsmodul ein vollständiger Erzeuger gefunden, werden wie oben
beschrieben durch @calcSubmoduleElements@ (\autoref{lst:calcSubmoduleElements})
alle, bis auf den letzten, verarbeitet. Dieser wird abschließend separat in
\autoref{lst:processLastSubmodule}
betrachtet und liefert die Anzahl der primitiven vollständig normalen Elemente.


\subsection{Top-Level-Implementierung in \sage}

Eingangs wurde zwar erwähnt, dass \sage nicht ausreichend performant ist, um
die hier angestrebten Ziele zu erreichen, doch wollen wir nicht gänzlich auf
die hochsprachlichen Funktionen dieses Computeralgebrasystems verzichten.
Insbesondere eignet sich \sage hier, die Daten für
@processFiniteField@ (\autoref{lst:processFiniteField}) bereitzustellen.

\subsubsection{Anwendung des Zerlegungssatzes}
Es ist klar, dass am Anfang der Berechnung von primitiv vollständig normalen
Elementen einer Erweiterung endlicher Körper stets die Anwendung des
Zerlegungssatzes (\thref{satz:zerlegungssatz}) steht.


\begin{sagecode}[caption={[\texttt{decompose} aus 
 \url{../Sage/enumeratePCNs.spyx}]Aus \url{../Sage/enumeratePCNs.spyx}},
  label=lst:decompose]
# Application of the Decomposition Theorem (Section 19)
# for x^n-1 over F_p^e
def decompose(p,e, n):
    pi = largestDiv(p,n)
    return decompose_cycl_module(p,e, 1, n/pi, pi)
\end{sagecode}


\begin{sagecode}[caption={[\texttt{decompose\_cycl\_module} aus 
 \url{../Sage/enumeratePCNs.spyx}]Aus \url{../Sage/enumeratePCNs.spyx}},
  label=lst:decompose_cycl_module]
# internal application of the Decomposition Theorem
# for Phi_k(x^(t*pi)) over F_p^e
def decompose_cycl_module(p,e, k,t,pi):
    if p.divides(k*t): print "ERROR p | kt"
    #test all prime divisors, start with largest one
    flag = False
    for r,l in reversed(factor(t)):
        if not (r**l).divides(ordn(squarefree(k*t),p**e)):
            R = largestDiv(r,t)
            return decompose_cycl_module(p,e, k, t/r, pi) \
                    + decompose_cycl_module(p,e, k*R, t/R, pi)
    return [(k,t,pi)]
\end{sagecode}

\begin{beispiel}
  Wollen wir einmal den Zerlegungssatz auf $E := \F_{3^{20}}$ über 
  $F := \F_3$ anwenden, so rufen wir "decompose(3,1,20)" auf und 
  erhalten
  \begin{center}
    "[(1, 1, 1), (2, 1, 1), (4, 1, 1), (5, 4, 1)]".
  \end{center}
  Umformuliert bedeutet das, dass
  \[ x^{20} - 1 \speq= \Phi_1(x)\ \Phi_2(x)\ \Phi_4(x)\ \Phi_5(x^4)\quad
    \in \F_3[x]\,,\]
  eine verträgliche Zerlegung ist. 
  Oder in Termen der erweiterten Kreisteilungsmoduln ist 
  \[ \C_{1,20} \speq= \C_{1,1} \oplus \C_{2,1} \oplus
    \C_{4,1} \oplus \C_{5,4}\]
  eine verträgliche Zerlegung über $\F_3$.
\end{beispiel}

Die benutzten Funktionen "largestDiv", "ordn" und "squarefree" sind dabei 
wie folgt gegeben.

\begin{sagecode}[caption={[\texttt{largestDiv} aus 
 \url{../Sage/enumeratePCNs.spyx}]Aus \url{../Sage/enumeratePCNs.spyx}}]
# returns the largest power of p dividing n
def largestDiv(p,n):
    l = 0
    while (p**l).divides(n):
        l = l+1
    return p**(l-1);
\end{sagecode}


\begin{sagecode}[caption={[\texttt{ordn} aus 
 \url{../Sage/enumeratePCNs.spyx}]Aus \url{../Sage/enumeratePCNs.spyx}}]
# computes ordn m(q) = min{ k: q ** k = 1 mod m }
def ordn(m,q):
    Zn = IntegerModRing(m)
    return Zn(q).multiplicative_order()
\end{sagecode}  

\begin{sagecode}[caption={[\texttt{squarefree} aus 
 \url{../Sage/enumeratePCNs.spyx}]Aus \url{../Sage/enumeratePCNs.spyx}}]
# computes the quadratic free part of an integer
def squarefree(n):
    return prod(map(lambda x: x[0], factor(Integer(n))))
\end{sagecode}  


\subsubsection{Ausnutzen einfacher Zerlegungen}

Zunächst müsste man für jeden erweiterten Kreisteilungsmodul
\emph{alle} Teiler des Modulcharakters testen, um vollständige Erzeuger zu
finden. Jedoch garantiert \thref{satz:einfache_erweiterungen}, dass dies in
manchen Fällen überflüssig ist, da beispielsweise bei 
einer einfachen Erweiterung ein Erzeuger
eines Kreisteilungsmoduls $\C_{k,t}$ über $\F_q$ bereits ein vollständiger
Erzeuger ist. Ist eine Erweiterung nicht einfach, so sollte man die Hoffnung
nach einer Vereinfachung der Suche nach vollständigen Erzeugern nicht aufgeben,
sondern sich überlegen, dass es einen Teiler $d \mid n$ geben kann, für den
die Erweiterung $\F_{q^n}$ über $\F_{q^d}$ einfach ist. Dann müssten keine
Teiler von Modulcharaktern getestet werden, die $d$ als echten Faktor
enthalten, da -- wie man sich sehr leicht überlegt -- 
falls $\F_{q^n}$ über $\F_{q^d}$ einfach
ist, auch $\F_{q^n}$ über $\F_{q^{d\,e}}$ für alle
Teiler $e \mid \tfrac n d$ einfach ist.

Dies wollen wir in nachstehender Funktion nutzen, die
gerade die zu betrachtenden Teiler einer Erweiterung liefert.
%(und deren Benennung vielleicht etwas kontraintuitiv gewählt wurde).

\begin{sagecode}[caption={[\texttt{get\_not\_completely\_basic\_divisors} aus 
 \url{../Sage/enumeratePCNs.spyx}]Aus \url{../Sage/enumeratePCNs.spyx}},
  label=lst:get_completely_basic_divisors]
# returns the NOT completely basic divisors of an 
# extension n over GF(p^e)
def get_not_completely_basic_divisors(p,e,n):
    n = Integer(n)
    q = Integer(p**e)
    divs = []
    divsN = divisors(n)
    while len(divsN) > 0:
        d = divsN.pop(0)
        isComplBasic = True
        for r in prime_divisors(n/d):
            if r.divides(ordn(p_free_part(n/d/r,p),q**d)):
                isComplBasic = False
                break
        divs += [d]
        if isComplBasic: 
            divsN = filter(lambda k: not d.divides(k), divsN)
    return divs
\end{sagecode}

"p_free_part" gibt, wie in \thref{satz:einfache_erweiterungen} (3) zu sehen ist,
gerade den größten Teiler des ersten Arguments an, der nicht mehr durch $p$,
dem zweiten Argument, teilbar ist. (Es wird dabei nicht überprüft, ob das zweite
Argument eine Primzahl ist.)


\begin{sagecode}[caption={[\texttt{p\_free\_part} aus 
 \url{../Sage/enumeratePCNs.spyx}]Aus \url{../Sage/enumeratePCNs.spyx}}]
# p-free part of t
def p_free_part(t,p):
    while p.divides(t):
        t /= p
    return t
\end{sagecode}




\subsubsection{Ausnutzen regulärer Kreisteilungsmoduln}

Sicherlich wollen wir auch Regularität (\thref{def:regulaer}) nicht unbeachtet
lassen, um uns die Suche nach vollständig normalen Elementen zu erleichtern.
Also haben wir auch einen Test auf Regularität nach \sage übersetzt.

\begin{sagecode}[caption={[\texttt{isRegular} aus 
 \url{../Sage/enumeratePCNs.spyx}]Aus \url{../Sage/enumeratePCNs.spyx}}]
# tests if cyclotomic module C_k,t is regular over F_p^e
def isRegular(p,e, k,t,pi):
    return gcd( ordn( squarefree(k*p_free_part(t,p)), p**e ),  k*t*pi) == 1
\end{sagecode}

Ist ein Kreisteilungsmodul regulär, so ist ein Test auf vollständige
Erzeuger-Eigenschaft durch \thref{satz:regulare_erweiterungen} gegeben.
Da Regularität lediglich die Anzahl der Teiler des Erweiterungsgrades, deren
zugehörige Kreisteilungsmoduln auf vollständige Erzeuger getestet werden
müssen, reduziert, wird \thref{satz:regulare_erweiterungen} durch Rückgabe der
Teiler $\tau$ bzw. $\tau$ und $2\tau$ (in Notation dieses Satzes)
im ausfallenden Fall realisiert, wie die Funktion 
"get_tau_divisors" zeigt. Die zu übergebenden Parameter bestehen wieder aus 
$q = p^e$ und den Daten $(k,t,\pi)$ des zu betrachtenden Kreisteilungsmoduls 
$\C_{k,\,t\pi}$ über $\F_q$.

\begin{sagecode}[caption={[\texttt{get\_tau\_divisors} aus 
 \url{../Sage/enumeratePCNs.spyx}]Aus \url{../Sage/enumeratePCNs.spyx}}]
# returns tau-divisors for complete generator test of 
# the cyclotomic module C_k,t*pi over F_p^e
def get_tau_divisors(p,e, k,t,pi):
    if t != 1:
        print "ERROR get_tau_divisors: t != 1 for p=",p," e=",e\
                ," k=",k," t=",t," pi=",pi
        raise Exception("Error t!=1")
    q = p**e
    tau = ordn(k,q) / ordn(squarefree(k),q)
    tau = prod(map(lambda ra: ra[0]**floor(ra[1]/2), factor(tau)))
    if isExceptional(p,e, k):
        return [ tau, 2*tau ]
    else:
        return [ tau ]
\end{sagecode}

Wie im Absatz vor der Definition von Regularität (\thref{def:regulaer})
erwähnt, ist die
kanonische Zerlegung im regulären Fall verträglich. Daher tritt ein Fehler auf,
wird obiger Funktion ein Kreisteilungsmodul $\C_{l,m}$ übergeben mit 
$m \neq p^b$ für ein $b\geq 0$.

Es bleibt natürlich noch ein Test anzugeben, der überprüft, ob 
die Parameter $(p,e,k)$ ausfallend sind (vgl. \thref{def:ausfallend}).

\begin{sagecode}[caption={[\texttt{isExceptional} aus 
 \url{../Sage/enumeratePCNs.spyx}]Aus \url{../Sage/enumeratePCNs.spyx}}]
# tests if n is exceptional over F_p^e
def isExceptional(p,e, n):
    q = p**e
    c = 0
    nbar = n
    while Integer(2).divides(nbar):
        c += 1
        nbar /= 2
    if (q).mod(4) == 3  and c >= 3  and ordn(q, 2**c) == 2:
        return True
    return False
\end{sagecode}  


\subsubsection{Die zentrale \sage-Funktion 
  \texttt{countCompleteSubmoduleGenerators}}

Als übergeordnete Funktion, die die Anzahl aller (primitiv) vollständig
normale Elemente und aller vollständigen Erzeuger im Sinne des Zerlegungssatzes
liefert, stellen wir \\
"countCompleteSubmoduleGenerators" bereit. Als Argumente
sind selbstredend ein endlicher Körper zu übergeben und der Grad der zu
betrachtenden Erweiterung. Ferner gibt es die Möglichkeit durch das optionale
Argument "binaryPowers=False" den Test auf Primitivität durch $p$-adische
Exponentiation durchführen zu lassen, wie in dem Absatz vor 
\autoref{lst:isPrimitive} erwähnt wurde (vgl. auch 
\autoref{subsub:primitivitaetstest}). Der Test auf Primitivität lässt sich
durch die Übergabe von "testPrimitivity=False"
(vgl. \thref{bem:kein_test_auf_primitivitaet}) vollständig deaktivieren.
Durch den Parameter "onlyNormal=True" wird lediglich die Anzahl der
(primitiv) normalen Elemente durch Auslassen aller echten Teiler 
von $n$ (vgl. Zeile \texttt{21} in
\autoref{lst:countcompletesubmodulegenerators}) bestimmt.

Der Rückgabewert der Funktion enthält die Anzahl aller vollständig normalen
Elemente der Erweiterung, die Anzahl aller primitiv vollständig normalen (oder
$0$, falls der Test auf Primitivität deaktiviert wurde), 
die Anzahl der jeweiligen vollständigen Erzeuger der Zerlegung in
verallgemeinerte Kreisteilungsmodule nach \thref{satz:zerlegungssatz} und
abschließend die Dauer der Berechnung.

Im Gegensatz zu den bisherigen Listings werden wir
"countCompleteSubmoduleGenerators" in mehrere Teile aufspalten, um ein besseres
Verständnis zu gewährleisten. Wir beginnen mit den ersten Zeilen, die in
offensichtlicher Weise die Datenstrukturen der Zerlegung bereitstellen, wie sie
in @testAllSubmods@ (\autoref{lst:testAllSubmods}) bzw. @testSubmod@ 
(\autoref{lst:testSubmod}) benötigt werden.


\begin{sagecode}[caption={[\texttt{countCompleteSubmoduleGenerators} aus 
 \url{../Sage/enumeratePCNs.spyx}]Aus \url{../Sage/enumeratePCNs.spyx}},
  label={lst:countcompletesubmodulegenerators}]
def countCompleteSubmoduleGenerators(F,n, binaryPowers=True, \
        testPrimitivity=True,\
        onlyNormal=False):
    TIME = time.time()
    p = F.characteristic()
    q = F.order();
    e = q.log(p)
    E = F.extension(Integer(n),'a');
    P = E.prime_subfield()
    #generate factors
    polys = []
    polysCount = []
    evalToZero = []
    frobPowers = []
    maxLenPolysPerDecomp = []
    notComplBasicDivisors = get_not_completely_basic_divisors(p,e,n)
    decomposition = decompose(p,e,n)
    for decomp in decomposition:
        k,t,pi = decomp
        if onlyNormal:
            divs = [1]
        else:
            divs = divisors(get_module_character(*decomp))
            divs = filter(lambda x: x in notComplBasicDivisors, divs)
            if isRegular(p,e,k,t,pi):
                if get_tau_divisors(p,e, k,t,pi) != divs:
                    divs = get_tau_divisors(p,e, k,t,pi)
        maxLenPolysPerDecomp += [ euler_phi(k)*t*pi ]
        countPolysForThisDecomp = 0
        for d in divs:
            G = F.extension(Integer(d), 'c');
            Gx = PolynomialRing(G,'x'); 
            h = Hom(G,E)[0]
            cycl = Gx.cyclotomic_polynomial(squarefree(k))\
                    (Gx.gen()**(k*t*pi/squarefree(k)/d))
            if countPolysForThisDecomp == 0:
                polys += [map(lambda x: x.polynomial().list(),
                    cycl.map_coefficients(h).list())]
                frobPowers += [d]
                evalToZero += [1]
                countPolysForThisDecomp += 1
            # add Co-Factors
            for f,mult in cycl.factor():
                g = cycl.quo_rem(f)[0]
                gE = g.map_coefficients(h)
                polys += [map(lambda x: x.polynomial().list(), gE.list())]
                frobPowers += [d]
                evalToZero += [0]
                countPolysForThisDecomp +=1
        polysCount += [countPolysForThisDecomp]
\end{sagecode}

Wie man gut erkennen kann, werden einfache Zerlegungen in den Zeilen
\texttt{16} und \texttt{24} ausgenutzt und für reguläre Kreisteilungsmoduln
wird die Funktion "get_tau_divisors" (Zeilen \texttt{25}f) aufgerufen.

Anschließend (\autoref{lst:countcompletesubmodulegenerators:i}) 
berechnen wir, falls "testPrimitivity=True",
die Kofaktoren, wie sie beim Test auf Primitivität
in @isPrimitive@ (\autoref{lst:isPrimitive}) verwendet werden. Dabei ist zu
unterscheiden, ob die Faktoren in binärer (ab Zeile \texttt{23})
oder $p$-adischer Form (ab Zeile \texttt{28}) genutzt werden
sollen. Bei $p$-adischer Darstellung muss, wie in dem Absatz vor 
@powerFFElem@ (\autoref{lst:powerffelem}) erwähnt, die Länge des maximal
auftretenden $0$-Intervalls berechnet werden (ab Zeile \texttt{36}).

\begin{sagecode}[caption={\texttt{countCompleteSubmoduleGenerators}
  Fortsetzung (I)}, label=lst:countcompletesubmodulegenerators:i]
    charac = int(E.characteristic())
        #mipo, idcsMipo
    mipo = E.modulus().list()
    m = len(mipo)-1

    #calc prime factors of order
    barFactors = []
    primitiveOrder = E.order()-1
    if testPrimitivity:
        factors = reversed(factor(primitiveOrder))
        for r,k in factors:
            barFactors += [primitiveOrder/r]
        countBarFactors = len(barFactors)
        commonBarFactor = gcd(barFactors)
        commonBiggestBarFactor = max(gcd(barFactors[1:]) / commonBarFactor,1)
        barFactors = map(lambda b: b/commonBarFactor, barFactors)
        curF = 0
        barFactors_tmp = [barFactors[0]]
        for b in barFactors[1:]:
            barFactors_tmp += [ b/commonBiggestBarFactor - curF]
            curF = b/commonBiggestBarFactor
        barFactors = barFactors_tmp
        if binaryPowers:
            barFactors = map(lambda b: get_padic_representation(b,2),barFactors)
            commonBarFactor = get_padic_representation(commonBarFactor,2)
            commonBiggestBarFactor = \
                    get_padic_representation(commonBiggestBarFactor,2)
        else:
            barFactors = map(lambda b: get_padic_representation(b,p),barFactors)
            commonBarFactor = get_padic_representation(commonBarFactor,p)
            commonBiggestBarFactor = \
                    get_padic_representation(commonBiggestBarFactor,p)
        lenCommonBarFactor = len(commonBarFactor)
        lenCommonBiggestBarFactor = len(commonBiggestBarFactor)

        lenBiggestZeroGap = 0
        if not binaryPowers:
            #find biggest gap (i.e. zero-interval)
            lenCurGap = 0
            for b in barFactors+[commonBarFactor]+[commonBiggestBarFactor]:
                i = 0
                while i < len(b):
                    lenCurGap = 0
                    while i<len(b) and b[i] == 0:
                        lenCurGap+= 1
                        i += 1
                    lenBiggestZeroGap = max(lenBiggestZeroGap, lenCurGap)
                    i += 1
    else:
        countBarFactors = 0
        barFactors = []
        commonBarFactor = []
        commonBiggestBarFactor = []
        lenBiggestZeroGap = 0
\end{sagecode}


Im letzten Teil der reinen \sage-Aufbereitung
(\autoref{lst:countcompletesubmodulegenerators:ii}), 
liften wir die Elemente des
Grundkörpers mittels eines injektiven Körperhomomorphismus in den
Erweiterungskörper, wie sie in @calcSubmoduleElements@
(\autoref{lst:calcSubmoduleElements}) bzw.
@processLastSubmoduleAndTestPrimitivity@
(\autoref{lst:processLastSubmodule}) benötigt werden. Ferner stellen wir die 
Additions- und Multiplikationstabellen nach 
\autoref{sub:arithmetik_in_endlichen_körpern} auf.

\begin{sagecode}[caption={\texttt{countCompleteSubmoduleGenerators}
  Fortsetzung (II)}, label=lst:countcompletesubmodulegenerators:ii]
        #generate F elements in E
    elementsF = []
    if e == 1:
        elementsF = map(lambda e: [e], list(F))
    else:
        h = Hom(F,E)[0]
        for e in itertools.product(xrange(p),repeat=e):
            elementsF += [h( F(list(reversed(e))) ).polynomial().list()]

        #calculate addition and multiplication tables
    ps = range(p)
    addTable = ps[P(-2*(p-1)):] + ps*2 + ps[:Integer(P(2*(p-1)))+1]
    multTable = ps[P(-(p-1)**2):] + ps*(2*(p-2)) + ps[:Integer(P((p-1)**2))+1]
\end{sagecode}  


Nun sind wir bereit, alle Daten nach \Clang zu transferieren. Dies ist ein
notwendiges Übel, da die interne Repräsentation von \sage-Objekten nicht mit
denen in \Clang vereinbar ist. Beispielsweise sind Listen von Ganzzahlen
in \sage keineswegs \Clang-kompatible Arrays. 
Es existiert jedoch gerade für diesen Zweck die Möglichkeit,
die komfortable Syntax von \texttt{numpy}-Arrays zu nutzen, die direkt auf 
\Clang-Arrays basieren.\footnote{Siehe z.B.
\url{http://www.sagemath.org/doc/numerical_sage/numpy.html} für die Benutzung
von \texttt{numpy}-Arrays in \sage.}

Andere Datenstrukturen, wie die selbst erstellten @struct FFElem@s, müssen
händisch übersetzt werden. Da \cython das (etwas merkwürdig wirkende) Mischen
von \python und \Clang erlaubt, schieben wir die hierfür erstellten Funktionen
der Übersetzung von \python-Listen in die jeweilige \Clang-Datenstruktur kurz
ein.

\begin{sagecode}[caption={[\texttt{FFElem *pyList2FFElem} aus 
 \url{../Sage/enumeratePCNs.spyx}]Aus \url{../Sage/enumeratePCNs.spyx}}]
cdef FFElem *pyList2FFElem(element,int m):
    cdef FFElem *ff = mallocFFElem(<int>m)
    initPoly(ff.el,m)
    for i,e in enumerate(element):
        ff.el[i] = e
    updateFFElem(ff,m)
    return ff
\end{sagecode}  

\begin{sagecode}[caption={[\texttt{FFElem **pyList2PointFFElem} aus 
 \url{../Sage/enumeratePCNs.spyx}]Aus \url{../Sage/enumeratePCNs.spyx}}]
cdef FFElem **pyList2PointFFElem(pyList, int m):
    lenList = len(pyList)
    cdef FFElem **ffs = <FFElem**>malloc(lenList*sizeof(FFElem*))
    for i,e in enumerate(pyList):
        ffs[i] = pyList2FFElem(e,m)
    return ffs
\end{sagecode}

\begin{sagecode}[caption={[\texttt{FFPoly *pyList2FFPoly} aus 
 \url{../Sage/enumeratePCNs.spyx}]Aus \url{../Sage/enumeratePCNs.spyx}}]
cdef FFPoly *pyList2FFPoly(listPoly, int m):
    lenPoly = len(listPoly)
    cdef FFPoly *poly = <FFPoly*>malloc(sizeof(FFPoly))
    poly.poly = <FFElem**>malloc(lenPoly*sizeof(FFElem*))
    poly.lenPoly = lenPoly
    for i,e in enumerate(listPoly):
        poly.poly[i] = pyList2FFElem(e,m)
    return poly
\end{sagecode}

\begin{sagecode}[caption={[\texttt{FFPoly **pyList2PointFFPoly} aus 
 \url{../Sage/enumeratePCNs.spyx}]Aus \url{../Sage/enumeratePCNs.spyx}}]
cdef FFPoly **pyList2PointFFPoly(listPolys, int m):
    countPolys = len(listPolys)
    cdef FFPoly **polys = <FFPoly**>malloc(countPolys*sizeof(FFPoly*))
    for i,e in enumerate(listPolys):
        polys[i] = pyList2FFPoly(e,m)
    return polys
\end{sagecode}


Nun können wir die Beschreibung von "countCompleteSubmoduleGenerators"
fortsetzen und erkennen sofort die gerade vorgestellten Funktionen der
Übersetzung sowie die Benutzung der \texttt{numpy}-Arrays.

\begin{sagecode}[caption={\texttt{countCompleteSubmoduleGenerators}
  Fortsetzung (III)}, label=lst:countcompletesubmodulegenerators:iii]
    # SETUP C DATA ===========================================================
    maxMatPower = max(map(lambda d: euler_phi(d[0])*d[1]*d[2], decomposition))
        # multiplication and addition table
    cdef np.ndarray[int,ndim=1,mode="c"] multTableRawC\
        = np.array(multTable, dtype=np.int32)
    cdef np.ndarray[int,ndim=1,mode="c"] addTableRawC\
        = np.array(addTable, dtype=np.int32)
    cdef int* multTableC = <int*>multTableRawC.data + <int>((p-1)**2)
    cdef int* addTableC = <int*>addTableRawC.data + <int>(2*(p-1))
        #setup mipo
    cdef FFElem *mipoC = pyList2FFElem(mipo,m+1)
       #setup matrices
    cdef FFElem **matsC  = genFrobMats(mipoC,m,maxMatPower,q,
            multTableC, addTableC)
        # mat charac
    cdef FFElem **matCharacC
    if binaryPowers:
        matCharacC = <FFElem**>0
    else:
        matCharacC = genFrobMats(mipoC,m,lenBiggestZeroGap+1,
                p, multTableC, addTableC)
    #setup polynomials, polyLength, frobPowers, evaltoZero
    decompCount = int(len(polysCount))
        #evalToZeroC
    cdef np.ndarray[char,ndim=1,mode="c",cast=True] evalToZeroC\
            = np.array(evalToZero, dtype=np.uint8)
        #frobPowersC
    cdef np.ndarray[int,ndim=1,mode="c"] frobPowersC\
            = np.array(frobPowers, dtype=np.int32)
        #polysCountC
    cdef np.ndarray[int,ndim=1,mode="c"] polysCountC\
            = np.array(polysCount, dtype=np.int32)
    cdef FFPoly **polysC = pyList2PointFFPoly(polys,m)
    cdef np.ndarray[int,ndim=1,mode="c"] maxLenPolysPerDecompC\
            = np.array(maxLenPolysPerDecomp, dtype=np.int32)
        # bar Factors
    cdef np.ndarray[int,ndim=1,mode="c"] barFactorsC \
        = np.array(list(itertools.chain(*barFactors)), dtype=np.int32)
    cdef np.ndarray[int,ndim=1,mode="c"] lenBarFactorsC \
        = np.array(map(len,barFactors), dtype=np.int32)
    cdef np.ndarray[int,ndim=1,mode="c"] commonBarFactorC \
        = np.array(commonBarFactor, dtype=np.int32)
    cdef np.ndarray[int,ndim=1,mode="c"] commonBiggestBarFactorC \
        = np.array(commonBiggestBarFactor, dtype=np.int32)
        # F elements in E
    cdef FFElem **elementsFC = pyList2PointFFElem(elementsF,m)
    #=========================================================================
\end{sagecode}

Es gilt anzumerken, dass die Erzeugung der Darstellungsmatrizen des Frobenius
in \Clang durch die Funktion @genFrobMats@ 
(die in \url{../Sage/enumeratePCNs.c} zu finden ist und hier nicht näher
erläutert wird, da sie weder vom mathematischen Standpunkt her besonders
spannend ist, noch programmiertechnisch besondere Aufmerksamkeit
verdient)
geschieht, wobei die maximal zu berechnende
Matrixpotenz gerade durch den Grad des größten auftretenden Polynoms der
Zerlegung gegeben ist. Wir wissen jedoch genau, wie der Grad eines
verallgemeinerten Kreisteilungspolynoms zu berechnen ist, wie Zeile
\texttt{2} in \autoref{lst:countcompletesubmodulegenerators:iii} erkennen lässt.

In einem letzten Schritt können wir (nun endlich) die bereitgestellte
\Clang-Funktion @processFiniteField@ (\autoref{lst:processFiniteField}) 
aufrufen und die Rückgabewerte verwalten. Hier gilt es anzumerken, dass die
Anzahl der vollständigen Erzeuger direkt in das Array @genCountsC@ geschrieben
wird und nicht als expliziter Rückgabewert erkennbar ist.


\begin{sagecode}[caption={\texttt{countCompleteSubmoduleGenerators}
  Fortsetzung (IV)}]
    #setup return values
    cdef np.ndarray[unsigned long long,ndim=1,mode="c"] genCountsC
    genCountsC = np.zeros(decompCount, dtype=np.ulonglong)

    cdef unsigned long long pcn = \
            processFiniteField(mipoC, decompCount,
                    polysC,<int*>polysCountC.data,
                    <char*>evalToZeroC.data,
                    <int*>maxLenPolysPerDecompC.data,
                    matsC,maxMatPower,<int*>frobPowersC.data,
                    <unsigned long long*>genCountsC.data, m, p, q,
                    <int*>barFactorsC.data, <int*>lenBarFactorsC.data,
                    countBarFactors,
                    <int*>commonBarFactorC.data,lenCommonBarFactor,
                    <int*>commonBiggestBarFactorC.data,lenCommonBiggestBarFactor,
                    matCharacC,elementsFC,
                    multTableC,addTableC)

    genCounts = dict()
    for i,d in enumerate(decomposition):
        genCounts[d] = Integer(genCountsC[i])

    # Free all malloced variables at the end =================================
    freeFFElem(mipoC)
    freeFFElemMatrix(matsC,m*maxMatPower)
    for i in range(len(polys)):
        freeFFPoly(polysC[i])
    free(polysC)
    freeFFElemMatrix(matCharacC,m*(lenBiggestZeroGap+1))
    freeFFElemMatrix(elementsFC,len(elementsF))
    #=========================================================================
    return prod(genCounts.values()), Integer(pcn), genCounts,\
            strfdelta(datetime.timedelta(seconds=(time.time()-TIME)))
\end{sagecode}


\subsection{Ein ausführliches Beispiel}

Wir wollen nun einmal das gesamte Verfahren zur Berechnung der Anzahl 
der primitiv vollständig normalen Elemente einer Erweiterung endlicher Körper
anhand eines Beispiels nachvollziehen. Dazu wählen wir $F := \F_2$ und 
$n := 6$, also $E := \F_{2^6}$. Die Wahl des Minimalpolynoms dieser Erweiterung
überlassen wir \sage und erhalten 
\[ E = \F_2[a] \big/ ( a^6 + a^4 + a^3 + a + 1 ) \,.\]

Gehen wir erneut den Code von "countCompleteSubmoduleGenerators"
Zeile für Zeile durch, so beginnen wir mit der Festlegung der grundlegenden
Parameter:
\[ p := 2,\qquad q := 2,\qquad e := 1,\qquad P := \F_2\,.\]


\paragraph{Berechnung der nicht einfachen Teiler} Im nächsten Schritt berechnen
wir die nicht einfachen Teiler mithilfe "get_not_completely_basic_divisors" 
(\autoref{lst:get_completely_basic_divisors}).
Dazu gehen wir alle Teiler von $n=6$ durch und überprüfen, ob die jeweiligen
Erweiterungen einfach sind, d.h. für jeden Teiler $d \mid n$ testen wir für
jeden Primteiler $r \mid \tfrac{n}{d}$, ob $r \nmid \ord_{(\frac{n}{dr})'}(q^d)$
(vgl. \thref{satz:einfache_erweiterungen}). Wir brechen jeweils ab, falls ein
$r$ die Teilbarkeitsbedingung nicht erfüllt.

\[ \begin{array}{l|l|l|l|l|ll} 
  d & \frac{n}{d} & r & (\frac{n}{dr})' & \ord_{(\frac{n}{dr})'}(q^d) & 
    \multicolumn{2}{l}{r \nmid \ord_{(\frac{n}{dr})'}(q^d)}\\\hline\hline
  1 & 6 & 2 & 3 & 2 & \lightning &\leadsto\text{\small nicht einfach}\\\hline
  2 & 3 & 3 & 1 & 1 & \checkmark & \leadsto\text{\small einfach}\\\hline
  3 & 2 & 2 & 1 & 1 & \checkmark & \leadsto\text{\small einfach}
  \end{array}\]

Damit sind alle zu betrachtenden Teiler von $n$ gegeben durch
\[ \text{"notComplBasicDivisors"} := [1,2,3]\,. \]
Wie man erkennt, wollen wir in diesem Beispiel alle auftretenden Listen in der
\python/\sage-üblichen Notation $[\ ,\ ,\ldots]$ angeben.


\paragraph{Anwendung des Zerlegungssatzes} Anschließend folgt die Berechnung
der Zerlegung in Kreisteilungsmoduln durch den Zerlegungssatz. Da wir in der
konkreten Implementierung stets drei Parameter für die Angabe von
Kreisteilungsmoduln verwenden, d.h. Potenzen der Charakteristik immer
„ausklammern”, wollen wir dies auch hier so notieren. Der zu
$x^n-1 = x^6-1$ über $\F_2$ gehörige Kreisteilungsmodul ist offenbar
\[ \C_{1,6} \speq= \C_{1,3\cdot 2}\]
und wir erhalten damit das Parametertripel $(k,t,\pi) := (1,3,2)$.
Hier startet der Zerlegungssatz rekursiv und wie in "decompose_cycl_module" 
(\autoref{lst:decompose_cycl_module}) erkennbar, durchlaufen wir die Primteiler
von $t$ in der Größe nach absteigend sortierter Reihenfolge.

\begin{center}
  \begin{tikzpicture}[level distance=2cm,sibling distance=3cm]
    \node (root) {$(1,3,2)$}
      child {node {$(1,1,2)$}
        edge from parent
          node[above, sloped, 
            font=\scriptsize,text=gray]
            {$(k,\tfrac t r, \pi)$}}
      child {node {$(3,1,2)$}
        edge from parent
          node[above,sloped,
          font=\scriptsize,text=gray]
          {$(kr,\tfrac t r, \pi)$}};
    \begin{scope}[overlay]
      \node[right=0.5cm of root, fill=gray!5,
        font=\small] 
        {$r=3$ $\leadsto$ 
          $3^1 \nmid \ord_{\nu(kt')}(q) = \ord_{3}(2) = 2$};
    \end{scope}
  \end{tikzpicture}
\end{center}
Da an den beiden Blättern $t=1$ gilt, endet hier die Möglichkeit einer weiteren
Rekursionsstufe und wir fassen zusammen, dass
\[ x^6-1 \speq= \Phi_1(x)^2 \ \Phi_3(x)^2 \]
die feinste verträgliche Zerlegung des Kreisteilungsmoduls $\C_{1,6}$
über $\F_2$ ist.


\paragraph{Polynome aufstellen} Nun sind wir in der Lage, die Polynome zu
berechnen, die wir für den Test von vermeindlichen vollständigen Erzeugen
benötigen werden. 
\begin{enumerate}
  \item Wir starten beim ersten erweiterten Kreisteilungspolynom
    \[ \Phi_{1,1}^2 \speq= x^2 + 1 \qquad \in \F_2[x]\,. \]
    Es ist nun $(k,t,\pi) := (1,1,2)$ und wir 
    müssten alle Teiler des Modulcharakters $\frac{k\,t\,\pi}{\nu(k)} = 2$,
    betrachten. Wie man an obig berechnetem "notComplBasicDivisors" erkennt, 
    lässt sich in diesem Fall einer der drei Teiler $\{1,2,3\}$
    streichen. Ein zweiter Kniff schafft eine weitere Reduktion der Teilerzahl, 
    da $(1,1,2)$ regulär über $\F_2$ ist:
    \[ \ord_{\nu(k\,t')}(q) \speq= \ord_{1}(2) \speq= 1 \,. \]
    Da $(1,1,2)$ nicht ausfallend über $\F_2$ ist, reicht es, den einzigen
    Teiler zu berechnen, den wir benötigen:
    \[ \tau(q,k) \speq= \tau(2,1) \speq= 1\,,\]
    da $\ord_k(q) = \ord_1(2) = \ord_{\nu(k)}(q)$.
    \begin{description}
      \item[$d=1.$] Nun sind alle Kofaktoren einer vollständigen Faktorisierung
        von $\Phi_{1,1}(x)^2$ über $\F_{2^d} = \F_2$ zu berechnen:
        \[ \Phi_{1,1}(x)^2 \speq= (x+1)^2\]
        und der einzige Kofaktor ist durch
        \[ g_{1,1,1}(x) \speq{:=} x+1 \]
        gegeben.
    \end{description}
  \item Nun zum zweiten Kreisteilungsmodul $(k,t,\pi) := (3,1,2)$. Der
    Modulcharakter ist wiederum $\frac{k\,t\,\pi}{\nu(k)} = 2$. Auch hier
    können wir von "notComplBasicDivisors" einen Teiler wegdiskutieren. Anders
    als in obigem Fall ist dieser Kreisteilungsmodul nicht regulär, da 
    \[ \ord_{\nu(kt')}(q) \speq= \ord_3(2) = 2\]
    nicht teilerfremd zu $kt=2$ ist.
    Also bleiben die beiden Teiler $\{1,2\}$ übrig.
    \begin{description}
      \item[$d=1.$] Wir faktorisieren 
        \[ \Phi_{3,1}(x)^2 \speq= (x^2+x+1)^2 \qquad \in \F_2[x] \]
        und erhalten als einzigen Kofaktor dieses Teilers
        \[ g_{2,1,1}(x) \speq{:=} x^2+x+1\,.\]
      \item[$d=2.$] Blicken wir noch einmal in die Definition eines
        vollständigen Erzeugers (\thref{def:vollst_erzeuger}), so sehen wir,
        dass $\C_{3,1\cdot 2}$ als $\F_{2^2}[x]$-Modul zu betrachten
        ist. Orientiert man sich an der Definition eines verallgemeinerten
        Kreisteilungsmoduls (\thref{def:verallgemeinerter_kreisteilungsmodul}),
        so sind wir gezwungen $\Phi_3(x^{\frac{2}{2}})$ über $\F_{2^2}$ zu
        faktorisieren. Dazu überlassen wir wiederum \sage die Repräsentation
        des endlichen Körpers
        \[ \F_{2^2} \speq= \F_2[b] \big/ (b^2+b+1)\]
        und faktorisieren
        \[ \Phi_{3,1}(x) \speq= (x+b) (x+b+1)\,. \]
        Ergo erhalten wir die beiden Kofaktoren in $\F_{2^2}[x]$:
        \begin{align*}
          g_{2,2,1}(x) &\speq{:=} x + b+1\,,\\
          g_{2,2,2}(x) &\speq{:=} x + b\,.
        \end{align*}
        Wie aber in der Beschreibung der Implementierung erwähnt, bietet es
        sich an, diese Polynome mittels eines injektiven Körperhomomorphismus
        in $E = \F_{2^6}$ zu lesen. Auch die Berechnung eines solchen
        überlassen wir \sage und wählen
        \[ h:\ \funcdef{\F_2[b]\big/(b^2+b+1) &\to& 
          \F_2[a] \big/ ( a^6 + a^4 + a^3 + a + 1 )\,,\\[10pt]
          b &\mapsto& a^2+a^2+a\,.}\]
        Damit schreiben wir obige Kofaktoren zu
        \begin{align*}
          g_{2,2,1}(x) &\speq{:=} x + a^3+a^2+a+1\,,\\
          g_{2,2,2}(x) &\speq{:=} x + a^3+a^2+a
        \end{align*}
        um, gelesen als Elemente von $\big(\F_2[a]\big/(a^6+a^4+a^3+a+1)\big)[x]$.
    \end{description}
\end{enumerate}
Als letzten Schritt des Aufstellens der Polynome fassen wir alle Ergebnisse
zusammen und erinnern uns an die Implementierung, wo neben den Polynomen auch
die Information, welche Polynome bei Vorliegen eines vollständigen Erzeugers in
der Frobenius-Auswertung zu Null ausgewertet werden müssen auch die Angabe der
Frobenius-Potenzen benötigt werden. Da alle Polynome in \emph{eine einzige}
Liste geschrieben werden, muss man selbstredend die Anzahl der Polynome des
jeweiligen Kreisteilungsmoduls abspeichern. Zusammengefasst erhalten wir
folgende Daten:
\[\small\begin{array}{r@{\ := [}llllll@{]}}
    \text{"polys"} &  x^2+1,& x+1, & x^4+x^2+1, &x^2+x+1, 
      & x+a^3+a^2+a+1, & x+a^3+a^2+a\\
    \text{"evalToZero"} & 1, & 0, & 1, & 0, & 0, & 0\\
    \text{"frobPowers"} & 1, & 1, & 1, & 1, & 2, & 2\\
    \text{"polysCount"} & 2, &    & 4  &    &    &\\
  \end{array}\]
Wie man sicherlich bemerkt, führen wir das den zweiten Kreisteilungsmodul
definierende Polynom $\Phi_{3,2}$ lediglich für den Teiler $d=1$ auf. Für den
Teiler $d=2$ hätten wir $\Phi_{3,1}$ jedoch mit Frobenius-Potenz $2$. Da ein
Element $u\in E$ jedoch genau dann $\Phi_{3,2}(\sigma)(u) = 0$ erfüllt, wenn 
$\Phi_{3,1}(\sigma^2)(u) = 0$, ist dieser Berechnungsschritt obsolet.

\paragraph{Daten für einen Primitivitätstest}
Für den in \autoref{lst:isPrimitive} beschriebenen Primitivitätstest, müssen
wir zunächst $q^n-1 = 2^6-1$ faktorisieren:
\[ 2^6-1 = 3^2\cdot 7\,.\]
Also sind die zu testenden Kofaktoren gerade $9$ und $21$. Wir erkennen sofort,
dass der größte gemeinsame Teiler beider Faktoren $3$ ist und setzen daher
in Benennung von "isPrimitive" (\autoref{lst:isPrimitive})
\[ \text{"commonBarFactor"}\ := 3\,.\]
Ergo reduzieren sich die Kofaktoren auf $3$ und $7$.
Im nächsten Schritt betrachten wir nur noch alle Kofaktoren, die den größten
Primfaktor obiger Faktorisierung enthalten. Hier ist dies nur einer:
$7$. Wieder berechnen wir den $\ggT$ all dieser: $7$. Damit haben wir alle
restlichen Daten:
\[ \begin{array}{r@{\ :=\ }l}
    \text{"commonBiggestBarFactor"} & 7\\
    \text{"barFactors"} & [3,1]
  \end{array} \]
Benutzen wir binäre Exponentiation so übersetzen wir die erhaltenen Zahlen ins
Binärsystem:
\[ \begin{array}{r@{\ :=\ }l}
    \text{"commonBarFactor"} & [1,1]\\
    \text{"commonBiggestBarFactor"} & [1,1,1]\\
    \text{"barFactors"} & \big[[1,1],\ [1]\big]
  \end{array} \]
In diesem Fall wären die Zahlen in $p$-adischer Schreibweise identisch, da ja
$p=2$.

\paragraph{Aufstellen der Frobenius-Matrizen}
Um den Frobenius von $F$, also 
$\bar F\to \bar F, x\mapsto x^2$,
effizient auf Elemente aus $E$ anwenden zu können,
müssen wir seine Darstellungsmatrix bezüglich der kanonischen Basis
\[\{ 1, a, a^2, a^3, a^4, a^5\} 
  \subseteq \F_2[a]\big/(a^6 + a^4 + a^3 + a + 1)\]
berechnen. Dazu fassen wir selbstredend die Elemente aus $E$ als Vektoren in
$\F_2^6$ auf:
\[\begin{array}{l@{\ =\ }l@{\quad\cong\quad}l}
  1^2 & 1 & \begin{bmatrix}1&0&0&0&0&0\end{bmatrix}^T\\
  a^2 & a^2 & \begin{bmatrix}0&0&1&0&0&0\end{bmatrix}^T\\
  (a^2)^2 & a^4 & \begin{bmatrix}0&0&0&0&1&0\end{bmatrix}^T\\
  (a^3)^2 & a^4+a^3+a+1 & \begin{bmatrix}1&1&0&1&1&0\end{bmatrix}^T\\
  (a^4)^2 & a^5 + a^4 + a^2 + a + 1 & \begin{bmatrix}1&1&1&0&1&1\end{bmatrix}^T\\
  (a^5)^2 & a^5+a^4+1 & \begin{bmatrix}1&0&0&0&1&1\end{bmatrix}^T
\end{array}\]

Damit erhalten wir eine Darstellungsmatrix des Frobenius:
\[\Gamma_\sigma \speq{:=} \begin{bmatrix}
  1 & 0 & 0 & 1 & 1 & 1 \\
  0 & 0 & 0 & 1 & 1 & 0 \\
  0 & 1 & 0 & 0 & 1 & 0 \\
  0 & 0 & 0 & 1 & 0 & 0 \\
  0 & 0 & 1 & 1 & 1 & 1 \\
  0 & 0 & 0 & 0 & 1 & 1 \\
  \end{bmatrix}\]

Wie man an den obigen Polynomen in "polys" erkennen kann, ist die maximale
Potenz des Frobenius gerade $4$. Daher bleibt noch $\Gamma_\sigma^2$,
$\Gamma_\sigma^3$ und $\Gamma_\sigma^4$ zu berechnen:

\[ \Gamma_\sigma^2 \speq{=}
  \begin{bmatrix}
    1 & 0 & 1 & 1 & 1 & 1 \\
    0 & 0 & 1 & 0 & 1 & 1 \\
    0 & 0 & 1 & 0 & 0 & 1 \\
    0 & 0 & 0 & 1 & 0 & 0 \\
    0 & 1 & 1 & 0 & 1 & 0 \\
    0 & 0 & 1 & 1 & 0 & 0 \\
    \end{bmatrix},
  \quad\Gamma_\sigma^3 \speq=
  \begin{bmatrix}
    1 & 1 & 1 & 1 & 0 & 1 \\
    0 & 1 & 1 & 1 & 1 & 0 \\
    0 & 1 & 0 & 0 & 0 & 1 \\
    0 & 0 & 0 & 1 & 0 & 0 \\
    0 & 1 & 1 & 0 & 1 & 1 \\
    0 & 1 & 0 & 1 & 1 & 0 \\
  \end{bmatrix},
  \quad\Gamma_\sigma^4 \speq=
  \begin{bmatrix}
    1 & 1 & 0 & 1 & 0 & 0 \\
    0 & 1 & 1 & 1 & 1 & 1 \\
    0 & 0 & 0 & 1 & 0 & 1 \\
    0 & 0 & 0 & 1 & 0 & 0 \\
    0 & 1 & 1 & 0 & 0 & 0 \\
    0 & 0 & 1 & 1 & 0 & 1 \\
  \end{bmatrix}.\]

Wie man in der Implementierung erkennen kann übergeben wir die
Frobenius-Matrizen stets als @FFElem **mats@, d.h. man sollte sich obige vier
Matrizen eher als eine $(24\times 6)$-Matrix vorstellen, deren Zeilen jeweils
aus einem @FFElem@ bestehen (vgl. \autoref{sec:impl_endl_körper}). 
Unter dieser Analogie ist dies gerade das Ergebnis der Funktion "genFrobMats".

\paragraph{Iteration von $E$ auf der Suche nach vollständigen Erzeugern}
Wie in der Beschreibung von @processFiniteField@
(\autoref{lst:processFiniteField}) angegeben, starten wir die Suche nach
vollständig normalen und primitiven Elementen bei einer Iteration des endlichen
Körpers $E$, bis wir für jeden Kreisteilungsmodul der Zerlegung einen
vollständigen Erzeuger gefunden haben. Die konkrete Iteration erfolgt dabei
lexikographisch in $\F_2^6$, wobei wir der besseren Lesbarkeit geschuldet
zwischen den verschiedenen Schreibweisen von Vektoren in $\F_2^6$ und
Polynomen in $\F_2[a]\big/(a^6 + a^4 + a^3 + a + 1)$ ohne besondere
Kennzeichnung wechseln werden.
\begin{description}
  \item[$u := \vec{0&0&0&0&0&0}^T.$] Hier gibt es nichts zu tun.
  \item[$u := \vec{1&0&0&0&0&0}^T.$] Wir blicken auf "polys" und berechnen
    \[ (x^2+1)(\sigma)(1) \speq= 0\ \checkmark,
    \qquad (x+1)(\sigma)(1) \speq= 0\ \lightning\,.\]
    Also ist $1$ kein Erzeuger von $\C_{1,1\cdot 2}$ über $\F_2$.
    Beim zweiten Kreisteilungsmodul scheitern wir bereits am ersten Polynom:
    \[ (x^4+x+1)(\sigma)(1) \speq= 1\ \lightning\,.\]
  \item[\ldots] Hier sind weitere Elemente zu denken, die ebenfalls keine 
    vollständigen Erzeuger liefern.
  \item[$u := \vec{0&1&1&0&0&0}^T.$] Dieses Element liefert einen vollständigen
    Erzeuger des zweiten Kreisteilungsmoduls:
    \[\begin{array}{r@{\ =\ }l@{\quad}r@{\ =\ }l} 
      (x^4+x^2+1)(\sigma)(u) & 0, &
      (x^2+x+1)(\sigma)(u) & a^5+a^4+a^2+1, \\[8pt]
      (x+a^3+a^2+a^1)(\sigma^2)(u) & a^5 + a^4 + a^3 + a, &
      (x+a^3+a^2+a)(\sigma^2)(u) & a^5+a^4\,.
      \end{array}\]
    Die Anwendung des Frobenius ist dabei jeweils durch obige Matrizen zu
    denken.
  \item[\ldots]
  \item[$u := \vec{0&1&1&1&0&0}^T.$] Hier haben wir einen vollständigen Erzeuger
    des ersten Kreisteilungsmoduls, wie nachstehende Rechnung zeigt.
    \[\begin{array}{r@{\ =\ }l@{\qquad}r@{\ =\ }l} 
      (x^2+1)(\sigma)(u) & 0, &
      (x+1)(\sigma)(u) & 1\,.
      \end{array}\]
\end{description}
Die berechneten vollständigen Erzeuger speichern wir in einem Array aus
verketteten Listen (vgl.
\autoref{sub:enumeration_eines_verallg_kreisteilungsmoduls}). Die verketteten
Listen sowie das Array wollen wir hier jedoch wieder in \python-üblicher 
Notation angeben. Bisher haben wir also für jeden Kreisteilungsmodul einen
Erzeuger gefunden:
\[ \text{"roots"}\ =\ \big[\, [a^3+a^2+a],\ [a^2+a] \,\big]\]
Die Benennung "roots" ist hier konsistent mit @processFiniteField@
(\autoref{lst:processFiniteField}) gewählt. Jedoch sind die Elemente der Listen
natürlich wieder als @FFElem@ zu denken.


\paragraph{Enumeration der einzelnen Kreisteilungsmoduln}
An diesem Punkt haben wir für jeden Kreisteilungsmodul einen Erzeuger gefunden
und können anhand diesem den jeweiligen Modul vollständig enumerieren.
\begin{description}
  \item[$\Phi_{1,1}^2.$] Sei $u := a^3+a^2+a$ unser gefundener Erzeuger, so können
    wir nach \thref{lemma:moduln_durch_polys} den Modul durch Polynome über
    $F$, deren Grad kleiner $2$ ist, enumerieren:
    \[\begin{array}{l|l|l}
      f(x) & f(\sigma)(u) & \text{vollst. Erz.}\\\hline
      1 & a^3+a^2+a & \checkmark\\
      x & a^3+a^2+a+1 & \checkmark\\
      x+1 & 1 & \lightning
      \end{array}\]
  \item [$\Phi_{3,1}^2.$] Sei in diesem Fall $u:= a^2+a$ der gefundene
    Erzeuger, so müssen wir Polynome bis zum Grad $3$ über $F$ betrachten:
    \[\begin{array}{l|l|l}
      f(x) & f(\sigma)(u) & \text{vollst. Erz.}\\\hline
      1 & a^2+a & \checkmark\\
      x & a^4+a^2 & \checkmark\\
      x+1 & a^4+a & \lightning\\
      x^2 & a^5+a^2+a+1 & \checkmark\\
      x^2+1 & a^5+1 & \checkmark\\
      x^2+x & a^5+a^4+a+1 & \lightning\\
      x^2+x+1 & a^5+a^4+a^2+1 & \lightning\\
      x^3 & a^5+a^2 & \checkmark\\
      x^3+1 & a^5+a & \lightning\\
      x^3+x & a^5+a^4 & \checkmark\\
      x^3+x+1 & a^5+a^4+a^2+a & \lightning\\
      x^3+x^2 & a+1 & \lightning\\
      x^3+x^2+1 & a^2+1 & \lightning\\
      x^3+x^2+x & a^4+a^2+a+1 & \lightning\\
      x^3+x^2+x+1 & a^4+1 & \lightning
      \end{array}\]
    In der konkreten Implementierung speichern wir diese Ergebnisse nicht ab,
    sondern erzeugen die weiteren Erzeuger des letzten Kreisteilungsmoduls
    dynamisch (vgl.  
    \autoref{sub:dynamische_enumeration}), was
    wir hier zur besseren Übersichtlichkeit nicht tun wollen.
\end{description}

Nun können wir die aktualisierte Liste "roots" angeben:
\[\text{"roots"}\ =\ \big[\, [a^3+a^2+a,\, a^3+a^2+a+1],\ 
  [a^2+a,\, a^4+a^2,\, a^5+a^2+a+1,\, a^5+1,\, a^5+a^2,\, a^5+a^4]\,\big]\]

An dieser Stelle können wir bereits festhalten, dass in der Erweiterung von
Grad $6$ über $\F_2$ genau $6\cdot 2=12$ vollständig normale Elemente
existieren.

\paragraph{Primitivitätstest}
Für einen Primitivitätstest müssen wir die $12$ vollständig normalen Elemente
natürlich erst einmal „zusammenbauen”. Dazu durchlaufen wir das kartesische
Produkt aus den Listen in "roots" und bilden jeweils die Summe der einzelnen
Elemente (vgl. \thref{def:vertraeglich}).
\begin{description}
  \item[$(a^3+a^2+a)\ +\ (a^2+a).$] 
    Wie in der Beschreibung zu @isPrimitive@ (\autoref{lst:isPrimitive})
    erläutert, berechnen wir zunächst $v^\texttt{\scriptsize commonBarFactor} = v^3$,
    wobei $v := a^3+a^2+a + a^2+a = a^3$ das zu testende Element ist.
    Dies führen wir mittels binärer Exponentiation durch, wie in 
    @powerFFElemSqM@ (\autoref{lst:powerffelemsqm}) beschrieben, geben hier
    jedoch nur das Ergebnis an. Es ist 
    \[ v^3 \speq=  a^5 + a^4 + a^2 + 1 \speq{=:} w\,.\]
    Da $w \neq 1$ müssen wir mit dem ersten Faktor aus 
    @barFactors@ fortfahren:
    \[ w^3 \speq= a^5 + a^4 + a^2\,. \]
    Auch dies ist ungleich $1$, also fahren wir fort mit 
    $w^\texttt{\scriptsize commonBiggestBarFactor}$ wie in 
    @isPrimitive@ (\autoref{lst:isPrimitive}) angegeben:
    \[ w^7 \speq= 1\,.\]
    An dieser Stelle können wir abbrechen und wissen, dass $v$ kein
    primitives Element ist.
  \item[$(a^3+a^2+a+1)\ +\ (a^2+a).$]
    Auch hier beginnen wir mit $v := a^3+a^2+a+1 + a^2+a = a^3+1$ und 
    berechnen
    \[ v^3 \speq= a^5+a^2+a+1 \speq{=:} w \,.\]
    Wieder ist $w\neq 1$ und wir fahren fort mit dem ersten @barFactor@.
    \[ w^3 \speq= a^4+a^2+a\,.\]
    Für den Exponenten $\text{"biggestCommonBarFactor"} = 7$ erhalten wir:
    \[ w^7 \speq= a^3+a^2+a \speq{=:} z\,.\]
    Diesen müssen wir nun mit allen verbleibenden @barFactor@s potenzieren. In
    unserem Fall lediglich einer:
    \[ z^1 \speq= a^3+a^2+a \]
    und somit ist $v$ ein primitives Element in $E$.
\end{description}

Für alle weiteren Elemente wollen wir nur das Ergebnis der Primitivitätstests
in tabellarischer Form angeben.

\[\begin{array}{l|l|l|l}
  \text{Erz. von } \C_{1,1\cdot 2} & \text{Erz. von } \C_{3,1\cdot 2} &
    \text{vollst. norm. Element} & 
    \text{primitiv} \\\hline
  a^3+a^2+a & a^4+a^2 & a^4+a^3 +a & \checkmark\\
  a^3+a^2+a+1 & a^4+a^2 & a^4+a^3+a+1 & \lightning\\
  a^3+a^2+a & a^5+a^2+a+1 & a^5+a^3+1 & \lightning\\
  a^3+a^2+a+1 &a^5+a^2+a+1 & a^5+a^3 & \checkmark\\
  a^3+a^2+a & a^5+1 & a^5 +a^3+a^2+a+1& \lightning\\
  a^3+a^2+a+1 &a^5+1 & a^5+a^3+a^2+a &\checkmark\\
  a^3+a^2+a & a^5+a^2 & a^5+a^3+a & \checkmark\\
  a^3+a^2+a+1 & a^5+a^2& a^5+a^3+a+1 &\lightning\\
  a^3+a^2+a & a^5+a^4 & a^5+a^4+a^3+a^2+a &\checkmark\\
  a^3+a^2+a+1 & a^5+a^4 & a^5+a^4+a^3+a^2+a+1 &\lightning\\
  \end{array}\]

Zusammenfassend existieren also $6$ primitiv vollständig normale Elemente in
der Erweiterung von Grad $6$ über $\F_2$.

\section{Auswertung der Ergebnisse}

Mit Hilfe obig vorgestellter Implementierung konnten wir die Werte
$\CN(q,n)$ und $\PCN(q,n)$, sowie die konkreten Anzahlen der Erzeuger der
Teilmoduln nach Anwendung des Zerlegungssatzes für die in 
\autoref{tab:bereiche_pcn} aufgeführten Paare $(q,n)$ bestimmen.
Falls ein Paar $(q,n)$ nicht einfach ist, so haben wir für die Primzahlen
$p=2,3,5$ auch einige Anzahlen normaler und primitiv normaler Elemente bestimmt
und diese Bereiche in \autoref{tab:bereiche_pn} gelistet.

\begin{table}[ht]
\caption{Bereiche der $\PCN$-Enumerationen}
\label{tab:bereiche_pcn}
\begin{center}
\begin{tabular}[t]{>{$}l<{$}|>{$}l<{$}}
  q & n \\\hline
  2 & 2,\ldots,31\\
  3 & 2,\ldots,20\\
  4 & 2,\ldots,14\\
  5 & 2,\ldots,12\\
  7 & 2,\ldots,11\\
  8,9 & 2,\ldots,9\\
  11, 13, 16, 17, 19, 23 & 2,\ldots,7\\
  25 & 2,\ldots,6\\
  27 & 2,\ldots,4\\
  29 & 2,\ldots,7\\
  31 & 2,\ldots,6\\
  32 & 2,\ldots,4\\
  37, 41, 43 & 2,\ldots,6\\
  121,169 & 2,\ldots,4\\
  361, 529, 841, 961 & 2,3\\
  1369, 1681, 1849 & 2\\
\end{tabular}\hspace{2cm}
\begin{tabular}[t]{>{$}l<{$}|>{$}l<{$}}
  q & n \\\hline
  2,\ldots,961 \text{ Primzahlpotenzen} & 3\\
  2,\ldots,243 \text{ Primzahlpotenzen} & 4\\
  2,\ldots,43 \text{ Primzahlpotenzen} & 6\\
\end{tabular}
\end{center}
\end{table}
  
\begin{table}[ht]
\caption{Bereiche der $\PN$-Enumerationen}
\label{tab:bereiche_pn}
\begin{center}
\begin{tabular}[t]{>{$}l<{$}|>{$}l<{$}}
  q & n \\\hline
  2 & 6,10,12,18,20,21,22,24,26,27,30\\
  3 & 8,10,14,16,20\\
  4 & 10\\
  5 & 6,12\\
\end{tabular}
\end{center}
\end{table}


Alle berechneten Ergebnisse werden in Tabellen im Anhang bereitgestellt.
Wie man erkennen kann, konnten wir alle Werte von 
\citeauthor{morgan1996} aus \autocite{morgan1996} reproduzieren
und viele weitere Werte liefern. Darüber
hinaus konnten wir die konkreten Anzahlen der Erzeuger der jeweiligen 
nicht weiter zerlegbaren Kreisteilungsmoduln bestimmen (vgl. Tabellen), 
was gerade aus theoretischer Sicht ein interessantes Resultat ist, da diese
Zahlen möglicherweise auf der Suche nach einer allgemeinen Formel für die
vollständigen Erzeuger eines Kreisteilungsmoduls helfen können (Wir erinnern
uns, dass \thref{satz:anzahl_vollst_erzeuger} lediglich für 
reguläre Kreisteilungsmodule gilt). 
Der kleinste Fall, für den wir nicht in der Lage waren ein Ergebnis zu
erlangen, ist $(q,n) = (2,32)$.

Zuletzt sei erwähnt, dass die obig vorgestellte Implementierung nur
eine unwesentliche Menge an Arbeitsspeicher erfordert, wodurch weitere
Ergebnisse problemlos produziert werden könnten, wenn man Rechenzeiten von über
einer Woche pro Körpererweiterung in Kauf nimmt. Die längste hier in Kauf
genommene Rechenzeit lag für $(q,n)=(243,4)$ bei $98$ Stunden und $4$ Minuten.



\section{Existenz von primitiv vollständig normalen Elementen}
\label{sec:existenz_pcn}

\subsection{Theoretische Aspekte}

Zu Beginn dieses Kapitels wurde die Bezeichnung $\G$ als die Menge der
$n\in \N^\ast$ eingeführt, für die die Erweiterungen von Grad $n$ eines 
\emph{jeden} endlichen Köpers ein $\PCN$-Element enthalten und in
\thref{prob:g} haben wir uns das Ziel gesetzt, möglichst viele $n$
anzugeben, die in $\G$ liegen. Zunächst können wir alle Erweiterungen
aufnehmen, für die $n$ über jedem $q$ einfach ist,
da hier die Existenz eines primitiv vollständig normalen Elementes
der eines primitiv normalen entspricht, die nach 
\thref{satz:primitive_normalbasis} gesichert ist. 
Des Weiteren sichert \thref{satz:pcn_in_regular}, dass wir auch 
a priori alle ungeraden Erweiterung aufnehmen können, die regulär über jedem
Grundkörper sind. Dazu geben wir einige Beispiele an 
(vgl. \autocite[Abschnitt vor Section 2]{hachenberger2001}):

\begin{lemma}
  \label{lemma:regular_speziell}
  Sei $n \in \N^\ast$. Dann gilt: $n$ ist regulär
  über jeder Primzahlpotenz $q>1$, falls eine der nachstehenden
  Bedingungen erfüllt ist:
  \begin{enumerate}
    \item $n$ ist Potenz einer beliebigen Primzahl.
    \item $n = N^s$ für $s\geq 1$ und $N$ ist eine 
      \emph{Carmichael Zahl}%
      \footnote{Eine \emph{Carmichael Zahl} ist eine ungerade
        natürliche Zahl $N$, sodass für jeden Primteiler $r$ von $N$ gilt: 
        $r-1$ teilt $N-1$}.
  \end{enumerate}
\end{lemma}
\begin{proof}
  \begin{enumerate}
    \item Sei $n = r^s$ für eine Primzahl $r$ und 
      $q = p^e$ für eine Primzahl $p$, so ist für  $r=p$ klar,
      dass $\ord_{\nu(n)}(q) = \ord_{1}(q)=1$ gilt. Für $r \neq p$ haben wir
      $\ord_{\nu(n)}(q) = \ord_r(q) \mid \varphi(r) = r-1$ nach
      \thref{lemma:rechenregeln_ordn} und damit ist $(n,q)$ regulär, da
      $\ggT(r,r-1) = 1$.
    \item Schreibe $\nu(N^s) = \prod_{i=1}^k r_i$ für Primzahlen 
      $r_1,\ldots,r_k$. Dann ist wie oben $\ord_{\nu(N^s)}(q)$ ein
      Teiler von $\varphi(\nu(N^s)) = \prod_{i=1}^k\varphi(r_i) = 
      \prod_{i=1}^k (r_i-1)$. Nun gilt jedoch $(r_i-1) \mid (N-1) \mid (N^s-1)$
      und damit $\ggT(r_i-1, N^s) = 1$ für alle $i=1,\ldots,k$.
  \end{enumerate}
\end{proof}


Damit können wir bereits eine große Teilmenge von $\G$ ausmachen:

\begin{proposition}
  \label{prop:G}
  Es gilt:
  \[ \{ N^s:\ N\text{ Carmichael Zahl},\ s\in \N^\ast\} \ \cup\ 
    \{ r^s:\ r \text{ Primzahl},\ s\in \N^\ast\} \quad\subset\quad \G\,. \]
\end{proposition}
\begin{proof}
  Klar nach \thref{kor:einfache_erweiterungen} und 
  \thref{lemma:regular_speziell}.
\end{proof}

Doch wie können wir $\G$ noch größer werden lassen? 
Betrachten wir einmal alle $n\in \{ 2,3,\ldots,33\}$, so sehen wir mit
obiger Proposition, dass für $n$ gleich
\[ 6,\ 12,\ 14,\ 15,\ 18,\ 20,\ 21,\ 22,\ 24,\ 26,\ 28,\ 30\]
  %34,\ 35,\ 36,\ 38,\ 40\]
diese Frage noch offen ist.
Mit Hilfe von \thref{satz:pcn_schranke} brauchen wir für jedes $n$ 
„nur” alle $q < n^4$ auf Existenz eines $\PCN$-Elements zu überprüfen. Daher
bietet es sich an, an dieser Stelle erneut \sage zu bemühen, um letztlich
nachstehenden Satz beweisen zu können:

\begin{satz}
  \label{satz:hauptsatz_g}
  %Es gilt
  %\[ \{ 2,3,4, \ldots, 40 \} \speq\subseteq \G \,.\]
  Für alle $n\in \N^\ast$ mit $2 \leq n \leq 33$ gilt
  $n \in \G$.
\end{satz}


\subsection{Implementierung einer $\PCN$-Suche I}

Im Gegensatz zur Enumeration einer einzigen Körpererweiterung geht es nun
darum, viele Körpererweiterungen mit immer gleichem Erweiterungsgrad zu
betrachten. Beispielsweise sind für $n = 30$ genau
$64902$ Körpererweiterungen zu prüfen, wobei das größte auftauchende $q$
gleich $809993$ ist.
Da in diesen Bereichen das Aufstellen von Frobenius-Matrizen, Additions- und
Multiplikationstabellen nicht mehr praktikabel ist, haben wir uns entschieden,
die gesamte Existenzsuche in \sage zu implementieren.


Den Test auf vollständige Normalität organisieren wir anhand nachstehendem
Lemma: 

\begin{lemma}
  Sei $E := \F_{q^n}$ über $F := \F_q$ eine Erweiterung endlicher Körper.
  Sei $x^n-1 = \prod_{i=1}^l \Phi_{k_i,t_i}^\pi$
  die feinst mögliche Zerlegung über $F$ nach 
  dem Zerlegunssatz (\thref{satz:zerlegungssatz}).
  Notiere
  \[ D_1 \speq{:=} \{ d\mid n:\ 
    \nexists d'\mid d,\ d'\neq d:\ \F_{q^n} \text{ über } \F_{q^{d'}}
    \text{ einfach} \}\]
  und 
  \[ D_2 \speq{:=} \{ d\in \N^\ast:\ d \mid \tfrac{k_it_i\pi}{\nu(k_i)}
    \text{ für ein } i=1,\ldots,l\}\,.\]
  Dann sind für $u\in E$ äquivalent:
  \begin{enumerate}
    \item $u$ ist vollständig normal über $F$.
    \item Für alle $d \in D := D_1 \cap D_2$ ist
      $\Ord_{q^d}(u) = x^{\frac n d}-1$.
    \item Für alle $d\in D$ gilt:
      Seien $x^{\frac n d}-1 = \prod_{i=1}^{l_d} f_i(x)^{\nu_i}$ 
      die vollständige Faktorisierung  von $x^{\frac n d}-1$ 
      über $\F_{q^d}$ und 
      $\bar f_i(x) = \tfrac{x^{\frac n d}-1}{f_i(x)}$ 
      für $i=1,\ldots,l_d$ die jeweiligen
      Kofaktoren, so gilt
      \[ \bar f_i(\sigma^d)(u) \speq\neq 0 \]
      für alle $i=1,\ldots,l_d$, wobei wie immer 
      $\sigma: \bar F \to \bar F, x\mapsto x^q$ den Frobenius-Endomorphismus
      von $F$ bezeichne.
  \end{enumerate}
\end{lemma}
\begin{proof}
  Die Äquivalenz von (2) und (3) ist klar mit der Definition der $q$-Ordnung,
  wobei es klar ist, dass $(x^{\frac n d}-1)(\sigma^d)(u) = 0$ nicht mehr
  überprüft werden muss, da es das Minimalpolynom von $\sigma^d$ über 
  $\F_{q^d}$ ist.
  Letztlich bleibt also nur ein Wort darüber zu verlieren, warum es ausreicht
  nur Teiler aus $D$ zu betrachten: Hier stellen wir fest, dass es klar ist,
  lediglich Teiler aus $D_1$ zu betrachten, da alle weiteren Teiler von $n$
  bereits Zwischenkörper einer einfachen Erweiterung liefern.
  Bleibt $D_2$ zu klären. Dazu zerlege $u = u_1+\ldots+u_l$ mit
  $u_i \in \C_{k_i,t_i\pi}$ für $i=1,\ldots,l$. Nun gilt
  nach der Definition von vollständigen Erzeugern
  (\thref{def:vollst_erzeuger}), dass
  $u_i$ genau dann ein vollständiger Erzeuger von $\C_{k_i,t_i\pi}$ ist,
  wenn $\Ord_{q^d}(u_i) = \Phi_{\nu(k_i),\frac{k_it_i\pi}{\nu(k_i)d}}$ für alle
  Teiler $d$ des Modulcharakters $\frac{k_it_i\pi}{\nu(k_i)}$.
  Ist damit (3) erfüllt, so sind alle $u_i$ vollständige Erzeuger und nach
  Definition verträglicher Zerlegungen (\thref{def:vertraeglich}) folgt
  (1).
\end{proof}


Konkret ist der Test auf vollständige Normalität wie folgt gegeben.


\begin{sagecode}[caption={[\texttt{isCompletelyNormal} aus 
 \url{../Sage/findAnyPCN_trinom.spyx}]Aus \url{../Sage/findAnyPCN_trinom.spyx}}]
# Tests x as Element of E on complete normality, i.e. tests for each 
# d in divs, if the corresponding polynomials in prodsAll over the corresponding
# field in fieldsAll vanishes on frobenius evaluation of x.
# fieldsAll and facsAll are dicts indexed by the divisors of divs, where
# fieldsAll[d] is the corresponding intermediate field of order q^d
# and facsAll[d] is the factorization of x^(n/d)-1 over GF(q^d).
# prodsAll[d] is the list of all possible cofactors of above factorization.
def isCompletelyNormal(x,E, q, divs, fieldsAll, facsAll, prodsAll):
    if x == E.zero(): return False
    #test isNormal for each divisor
    pows = dict()
    for d in divs:
        h = Hom(fieldsAll[d],E)[0];
        for idx,(f,mult) in enumerate(facsAll[d]):
            g = prodsAll[d][idx];
            ret = E.zero();
            iold = 0
            xiold = x
            for i,gi in enumerate(list(g)):
                if pows.has_key(i*d):
                    xi = pows[i*d];
                    iold = i*d
                    xiold = xi
                else:
                    xi = xiold**(q**(d*i-iold));
                    pows[i*d] = xi;
                    xiold = xi
                    iold = i*d
                ret += h(gi)*xi
            if ret == 0: return False;
    return True
\end{sagecode}

Hier wurde auf eine Anwendung der Frobenius-Auswertung durch
Matrixmultiplikation verzichtet, da dies via purem \sage-Code wesentlich
langsamer ist als Potenzieren. Wie man jedoch erkennen kann, werden bereits
berechnete Potenzen wiederverwendet, um hier unnötigen Rechenaufwand
einzusparen.
Die Parameter "divs", "fieldsAll", "facsAll" und "prodsAll" werden jeweils in
den übergeordneten Funktionen "findAnyPCN_polynom" 
(\autoref{lst:findAnyPCN_polynom}) und
"findAnyPCN_polynom_prime" (\autoref{lst:findAnyPCN_polynom_prime})
wie folgt generiert.

\begin{sagecode}[caption={Aus \url{../Sage/findAnyPCN_trinom.spyx}}]
    #setup factors of x^n-1
    divs = get_proper_subfield_divisors(p,r,n)
    facsAll = dict();
    prodsAll = dict();
    fieldsAll = dict();
    for d in divs:
        G = F.extension(Integer(d), 'c');
        Gx = PolynomialRing(G,'x');
        fieldsAll[d] = G;
        facsAll[d] = list((Gx.gen()**(n/d)-1).factor());
        prodsAll[d] = dict();
        for idx,(f,mult) in enumerate(facsAll[d]):
            prodsAll[d][idx] = (Gx.gen()**(n/d)-1).quo_rem(f)[0]
\end{sagecode}




Bereits \citeauthor{morgan1996} haben in \autocite{morgan1996} für alle 
$p^n < 10^{50}$ mit $p\leq 97$ ein primitives, vollständig normales Polynom 
von Grad $n$ über $\F_p$ angegeben. Betrachtet man diese Tabellen, so ist
auffällig, dass sehr viele dieser Polynome Trinome, also Polynome, bei denen
lediglich drei Koeffizienten nicht verschwinden, sind. Man kann sich leicht
überlegen, dass ein Binom nicht Minimalpolynom eines vollständig normalen
Elements einer Körpererweiterung sein kann und bemerke, dass sich 
zwei Koeffizienten des Minimalpolynoms eines primitiv vollständig normalen
Elements etwas eingrenzen lassen:

\begin{lemma}
  \label{lemma:pcn_koeff}
  Sei $u \in \F_{q^n}$ über $\F_q$ ein primitiv vollständig normales Element
  und $f(x) \speq= x^n + a_{n-1}x^{n-1}+\ldots+a_0 \in \F_q[x]$ 
  sein Minimalpolynom. Dann gilt
  \begin{enumerate}
    \item $a_{n-1} \speq= -\Tr_{\F_{q^n}\mid \F_q}(u) \speq\neq 0$ und 
    \item $(-1)^na_0 \speq= \Nm_{\F_{q^n}\mid \F_q}(u)$ ist primitiv in $\F_q$.
  \end{enumerate}
\end{lemma}
\begin{proof}
  Die beiden Identitäten sind klar durch Koeffizientenvergleich von
  $f(x) = (x-u)(x-u^q)(x-u^{q^2})\cdot\ldots\cdot(x-u^{q^{n-1}})$.%
  \footnote{Man kennt diesen Zusammenhang der 
    Koeffizienten eines Polynoms mit seinen Nullstellen auch unter dem Namen
    \emph{elementarsymmetrische Funktionen}.}
  Die Spur eines (vollständig) normalen Elements einer Körpererweiterung ist
  stets ungleich Null, da sie ja gerade die Summe aller Basiselemente der 
  von jenem Element erzeugten Normalbasis ist und diese ja über dem Grundkörper
  linear unabhängig sind. 
  Die Primitivität der Norm erhalten wir sofort durch
  \[ \Nm_{\F_{q^n}\mid \F_q}(u) \speq= u\cdot u^q\cdot \ldots\cdot u^{q^n-1}
    = u^{\frac{q^n-1}{q-1}}\,.\]
  Aus der Primitivität von $u$ folgt nun $\ord(u^{\frac{q^n-1}{q-1}}) = q-1$
  und damit $\ord((-1)^na_0) = q-1$.
\end{proof}

In der Hoffnung möglichst viele Trinome vorzufinden, haben wir uns das Ziel
gesetzt für jede Primzahlpotenz $q$ mit $q < n^4$ für einen gegebenen
Erweiterungsgrad $n$ das „kleinste” primitiv vollständig normale Polynom
über $\F_q$ von Grad $n$ zu bestimmen. Das „kleinste” beziehe sich dabei auf
folgende Ordnungsrelation:

\begin{definition}
  \label{def:polynomordnung}
  Seien $f(x) = x^n+a_{n-1}x^{n-1}+\ldots+a_0$ und
  $g(x) = x^n + b_{n-1}x^{n-1}+\ldots+b_0$ zwei Polynome gleichen Grades über 
  $\F_q$. Ferner bezeichne $i(f)$ dasjenige Wort über dem Alphabet
  $\{ 0,\ldots,n\}$, das die Indizes der nicht verschwindenden 
  Koeffizienten von $f$ in
  aufsteigender Reihenfolge repräsentiert, d.h. ist $i(f) = i_1i_2\ldots i_k$,
  so gilt $i_1<i_2<\ldots<i_k$ und $a_{j} \neq 0$ genau für 
  $j \in \{i_1,\ldots,i_k\}$. Analog sei $a(f)$ das Wort über dem Alphabet 
  $\F_q$, das die nicht verschwindenden Koeffizienten von $f$ 
  ihrem Index absteigend nach repräsentiert, d.h. ist 
  $a(f) = a_{i_1}a_{i_2}\ldots a_{i_k}$, so ist 
  $i_1>i_2>\ldots>i_k$ und für alle $j \in \{i_1,\ldots,i_k\}$ gilt:
  $a_{j}$ ist der nicht verschwindende Koeffizient
  von $x^{j}$ in $f$.

  Dann heißt $f$ \emph{kleiner oder gleich} $g$, 
  geschrieben $f\preceq g$, falls gilt:
  \begin{enumerate}
    \item $i(f)$ ist kürzer oder gleich lang $i(g)$
      und bei Gleichheit gilt zusätzlich:
    \item $i(f)$ lexikographisch kleiner oder gleich $i(g)$ und bei
      Gleichheit gilt zusätzlich:
    \item $a(f)$ ist lexikographisch kleiner oder gleich $a(g)$, 
      wobei eine Ordnung auf $\F_q$ wie folgt gegeben ist:
      \begin{itemize}
        \item Ist $q = p$ für eine Primzahl $p$, so wähle die natürliche Ordnung von
          $\F_p \cong \{ 0,1,\ldots,p-1\}$.
        \item Sonst wähle einmalig eine Repräsentation von 
          $\F_q$ durch $\F_p[x]\big/(h(x))$ und definiere 
          $u$ kleiner gleich $v$ für $u,v\in \F_q$ als
          $a(x) \preceq b(x)$ via dieser Definition für
          Repräsentanten $a(x)$ von $u$ und $b(x)$ von $v$ mit
          $a(x),b(x) \in \F_p[x]$ und $\deg(a),\deg(b) < \deg(h)$.
      \end{itemize}
  \end{enumerate}
\end{definition}


Wir haben uns für diese Definition einer Ordnung auf $\F_q[x]$ entschieden, da
hier stets Polynome 
\begin{itemize}
  \item mit kleinerem Hamming-Gewicht, d.h. mit kleinerer Anzahl an 
    nicht verschwindenden Koeffizienten, 
  \item mit kleineren Exponenten,
  \item mit „kleineren“ Koeffizienten (vgl. obige Definition)
\end{itemize}
bevorzugt werden.

\begin{bemerkung}
  Die oben vorgestellte Ordnung entspricht nicht ganz derer, die von 
  \citeauthor{morgan1996} in \autocite{morgan1996} verwendet wurde. Die dortige
  basierte auf einem Vergleich der Hamming-Gewichte und bei Gleichheit auf 
  der Zahl $N_f \in \N^\ast$, die für
  $f(x) = x^n+\sum_{i=0}^{n-1} a_ix^i$ über $\F_p$ durch
  $N_f := p^n + \sum_{i=0}^{n-1} a_ip^i$ (für $a_i \in \{0,\ldots,p-1\}$)
  gegeben war. Wir empfanden jedoch beispielsweise das Polynom
  $f(x) = x^4+2x^3+x+2$ kleiner als das Polynom
  $g(x) = x^4+x^3+x^2+2$. In der Ordnung von \citeauthor{morgan1996} ist jedoch
  $N_f = 140$ größer als $N_g = 119$.
\end{bemerkung}


Die konkrete Suche nach einem $\PCN$-Polynom für ein Paar $(q,n)$ 
teilen wir in zwei verschiedene Funktionen auf, wobei wir
"findAnyPCN_polynom_prime" nutzen wollen, falls $q$ eine Primzahl ist und
"findAnyPCN_polynom", falls $q$ eine echte Primzahlpotenz ist. Der wesentliche
Unterschied liegt dabei in der Enumeration des Grundkörpers $\F_q$, einmal
simplerweise durch $\{0,1,\ldots,p-1\}$ darstellbar, falls $q = p$ eine
Primzahl ist. Andernfalls müssen wir uns wieder des Polynomrings
über dem Primkörper $\F_p$ bedienen, um die Elemente in
$\F_q \cong \F_p[x]\big/(f(x))$ für $q=p^r$ und $r>1$ darstellen zu können.
Konkret:

\begin{sagecode}[caption={[\texttt{findAnyPCN\_polynom\_prime} aus 
 \url{../Sage/findAnyPCN_trinom.spyx}]Aus \url{../Sage/findAnyPCN_trinom.spyx}},
  label=lst:findAnyPCN_polynom_prime]
# special function for testing extensions of PrimeFields
def findAnyPCN_polynom_prime(p,n,primitivity_with_bar_factors=False):
    p = Integer(p)
    n = Integer(n)
    F = GF(p)
    
    Fx = PolynomialRing(F,'x')
    
    orderE = p**n
    primOrder = orderE-1

    primitives = []

    #setup factors of x^n-1
    divs = get_proper_subfield_divisors(p,1,n)
    facsAll = dict();
    prodsAll = dict();
    fieldsAll = dict();
    for d in divs:
        G = F.extension(Integer(d), 'c');
        Gx = PolynomialRing(G,'x');
        fieldsAll[d] = G;
        facsAll[d] = list((Gx.gen()**(n/d)-1).factor());
        prodsAll[d] = dict();
        for idx,(f,mult) in enumerate(facsAll[d]):
            prodsAll[d][idx] = (Gx.gen()**(n/d)-1).quo_rem(f)[0]

    firstRun = True
    # first test trinoms!
    for coeffT in xrange(1,p):
        if firstRun:
            if is_even(n):
                prange = xrange(1,p)
            else:
                prange = xrange(p-1,0,-1)
            for coeffN in prange:
                if F(coeffN).multiplicative_order() != p-1: continue
                if is_even(n):
                    primitives += [coeffN]
                else:
                    coeffN *= (-1)
                    primitives = [coeffN] + primitives
                f = Fx.gen()**n + coeffT*Fx.gen()**(n-1) + coeffN
                if not f.is_irreducible(): continue
                
                E = GF(orderE, name='a', modulus=f)

                if E.gen().multiplicative_order() == primOrder \
                        and isCompletelyNormal(E.gen(),E,p,\
                            divs,fieldsAll,facsAll,prodsAll):
                    return E.gen(),f
            firstRun = False
        else:
            for coeffN in primitives:
                f = Fx.gen()**n + coeffT*Fx.gen()**(n-1) + coeffN
                if not f.is_irreducible(): continue
                E = GF(orderE, name='a', modulus=f)

                if E.gen().multiplicative_order() == primOrder \
                        and isCompletelyNormal(E.gen(),E,p,\
                            divs,fieldsAll,facsAll,prodsAll):
                    return E.gen(),f
    # test rest
    for length in xrange(1,n-1):
        for idcs in itertools.combinations(xrange(1,n-1),length):
            for xs in itertools.product(xrange(1,p),repeat=length+1):
                for x in primitives:
                    f = Fx.gen()**n + xs[0]*Fx.gen()**(n-1) + x
                    for j,j2 in enumerate(idcs):
                        f += xs[length-j] * Fx.gen() ** j2
                    if not f.is_irreducible(): continue
                    E = GF(orderE, name='a', modulus=f)

                    if isPrimitive(E.gen(),primOrder,barFactors) \
                            and isCompletelyNormal(E.gen(),E,p,\
                                divs,fieldsAll,facsAll,prodsAll):
                        return E.gen(),f
\end{sagecode}

Wie man erkennen kann stützen wir uns auf die Hoffnung Trinome vorzufinden und
beginnen unsere Suche daher bei deren Enumeration. Ferner machen wir uns
\thref{lemma:pcn_koeff} zu Nutze und bestimmen anfangs dynamisch die primitiven
Elemente des Grundkörpers (vgl. Zeilen \texttt{37}-\texttt{43}).
Um die Ordnung nach \thref{def:polynomordnung} beizubehalten, sind wir hier
gezwungen eine Fallunterscheidung in $n$ gerade oder ungerade zu vollziehen
(vgl. Zeile \texttt{38}). Ebenfalls ein Vorteil gegenüber 
"findAnyPCN_polynom" ist die Tatsache, dass der Erweiterungskörper "E"
durch das zu testende Polynom gegeben werden kann und so dieses nicht erst über
"E" faktorisiert werden muss (vgl. Zeilen \texttt{46}, \texttt{48}).

Es sei ferner angemerkt, dass \sage eine Funktion "is_primitive" für Polynome
besitzt, die gerade testet, ob ein Polynom primitiv ist oder nicht. Jedoch hat
sich herausgestellt, dass dies um ein Vielfaches länger dauert, als die hier
beschriebene Methode in Zeile \texttt{48}.


Falls der Grundkörper kein Primkörper ist, organisieren wir die Suche analog,
wobei wir gezwungen sind, einige „Umwege“ zu gehen.


\begin{sagecode}[caption={[\texttt{findAnyPCN\_polynom} aus 
 \url{../Sage/findAnyPCN_trinom.spyx}]Aus \url{../Sage/findAnyPCN_trinom.spyx}},
  label=lst:findAnyPCN_polynom]
def findAnyPCN_polynom(p,r,n):
    if r == 1:
        return findAnyPCN_polynom_prime(p,n)

    q = p**r
    F = GF(q,'a')

    E = F.extension(n,'a')
    P = E.prime_subfield()

    Px = PolynomialRing(P,'x')
    Fx = PolynomialRing(F,'x')
    Ex = PolynomialRing(E,'x')
    h = Hom(F,E)[0]
    primOrder = E.order()-1
   
    primitives = []
    
    #setup factors of x^n-1
    divs = get_proper_subfield_divisors(p,r,n)
    facsAll = dict();
    prodsAll = dict();
    fieldsAll = dict();
    for d in divs:
        G = F.extension(Integer(d), 'c');
        Gx = PolynomialRing(G,'x');
        fieldsAll[d] = G;
        facsAll[d] = list((Gx.gen()**(n/d)-1).factor());
        prodsAll[d] = dict();
        for idx,(f,mult) in enumerate(facsAll[d]):
            prodsAll[d][idx] = (Gx.gen()**(n/d)-1).quo_rem(f)[0]

    # list elements of F
    Flist = [F.zero()]
    for i in xrange(1,r+1):
        for idcs in itertools.combinations(xrange(0,r),i):
            for koeffs in itertools.product(xrange(1,p),repeat=i):
                Flist += [F(list(sum([e * Px.gen()**idcs[j] for \
                        j,e in enumerate(reversed(koeffs))])))]

    if not is_even(n):
        FprimList = [F.zero()]
        for i in xrange(1,r+1):
            for idcs in itertools.combinations(xrange(0,r),i):
                for koeffs in itertools.product(xrange(p-1,0,-1),repeat=i):
                    FprimList += [F(list(sum([e * Px.gen()**idcs[j] for \
                            j,e in enumerate(reversed(koeffs))])))]

    firstRun = True
    # first test trinoms!
    for coeffT in xrange(1,q):
        coeffTF = Flist[coeffT]
        if firstRun:
            for coeffN in xrange(p,q):
                if is_even(n):
                    coeffNF = Flist[coeffN]
                else:
                    coeffNF = FprimList[coeffN]
                if coeffNF.multiplicative_order() != F.order()-1: continue
                if not is_even(n): coeffNF *= (-1)
                primitives += [coeffNF]
                f = Fx.gen()**n + coeffTF*Fx.gen()**(n-1) + coeffNF
                if not f.is_irreducible(): continue
                for fac,mul in Ex(f.map_coefficients(h)).factor():
                    if fac.degree() == 1:
                        x = -fac[0]
                        if x.multiplicative_order() == primOrder \
                                and isCompletelyNormal(x,E,q,divs,\
                                fieldsAll,facsAll,prodsAll):
                            return x,f
                        else: break
                    else: break
            firstRun = False
        else:
            for coeffNF in primitives:
                f = Fx.gen()**n + coeffTF*Fx.gen()**(n-1) + coeffNF
                if not f.is_irreducible(): continue
                for fac,mul in Ex(f.map_coefficients(h)).factor():
                    if fac.degree() == 1:
                        x = -fac[0]
                        if x.multiplicative_order() == primOrder \
                                and isCompletelyNormal(x,E,q,divs,\
                                fieldsAll,facsAll,prodsAll):
                            return x,f
                        else: break
                    else: break
    # test rest
    for length in xrange(1,n-1):
        for idcs in itertools.combinations(xrange(1,n-1),length):
            for xs in itertools.product(xrange(1,q),repeat=length+1):
                for x in primitives:
                    f = Fx.gen()**n + Flist[xs[0]]*Fx.gen()**(n-1) + x
                    for j,j2 in enumerate(idcs):
                        f += Flist[xs[length-j]] * Fx.gen() ** j2
                    if not f.is_irreducible(): continue
                    for fac,mul in Ex(f.map_coefficients(h)).factor():
                        if fac.degree() == 1:
                            y = -fac[0]
                            if y.multiplicative_order() == primOrder \
                                    and isCompletelyNormal(y,E,q,divs,\
                                    fieldsAll,facsAll,prodsAll):
                                return y,f
                            else: break
                        else: break
\end{sagecode}

Man erkennt, dass wir hier die Elemente aus "F" eigens generieren müssen, um
den Bedingungen aus \thref{def:polynomordnung} gerecht zu werden.

Sicherlich ist klar, dass man auch eine Funktion braucht, die gleich alle
$q$ für gegebenes $n$ testet:


\begin{sagecode}[caption={[\texttt{findAnyPCN\_polynom\_wrapper} aus 
 \url{../Sage/findAnyPCN_trinom.spyx}]Aus \url{../Sage/findAnyPCN_trinom.spyx}}]
def findAnyPCN_polynom_wrapper(n, border=lambda n:n**4, \
        fileoutput=False, filepath="pcns_trinom_", \
        startPrime=1, stopPrime=0, onlyR=None, \
        cpuNum=1):
    if fileoutput:
        st = datetime.datetime.\
                fromtimestamp(time.time()).strftime('%Y-%m-%d_%H:%M:%S')
        filepath += str(n)+"_"
        if onlyR != None: filepath += str(onlyR)+"_"
        filepath += st
    border = border(n)
    p = startPrime
    if onlyR != None and p**onlyR > border: return

    gen = runGenerator(border,[startPrime,stopPrime],[onlyR,onlyR])
    pool = Pool(cpuNum)
    for p,r,n,(x,pol) in pool.imap( findAnyPCN_polynom__star, \
            ((p,r,n) for p,r in gen) ):
        print "(",p,", ",r,") = ", pol
        if fileoutput:
            with open(filepath,'a') as f:
                f.write(str(p)+"\t"+str(r)
                        +"\t"+str(pol)+"\n")
            f.close();
    pool.close()
    pool.join()
\end{sagecode}

Wie man erkennen kann, hat es sich als vorteilhaft erwiesen die Argumente
"startPrime", "stopPrime" und "onlyR" einzuführen, wobei letzteres lediglich
Paare $(q,n)$ testet, bei denen $q = p^r$ mit $r=$ "onlyR". (Insbesondere eine
Trennung zwischen "onlyR" $=1$ und dem Rest war hilfreich, da es stets sehr
viele Primzahlen $p$ mit $p<n^4$ gibt, jedoch nur wenige, für die eine echte
Potenz immer noch kleiner $n^4$ ist.)


Wir schließen mit den beiden Hilfsfunktionen, die in obiger Funktion
benutzt werden, um der Syntax von "Pool.imap" gerecht zu werden.

\begin{sagecode}[caption={[\texttt{findAnyPCN\_polynom\_\_star} aus 
 \url{../Sage/findAnyPCN_trinom.spyx}]Aus \url{../Sage/findAnyPCN_trinom.spyx}}]
def findAnyPCN_polynom__star(prn):
    return prn[0],prn[1],prn[2],findAnyPCN_polynom(*prn)
\end{sagecode}


\begin{sagecode}[caption={[\texttt{runGenerator} aus 
 \url{../Sage/findAnyPCN_trinom.spyx}]Aus \url{../Sage/findAnyPCN_trinom.spyx}}]
def runGenerator(border,pRange=None,rRange=None):
    if pRange == None:
        p = 1
    else:
        p = pRange[0]
    while p < border :
        p = next_prime(p)
        if pRange != None and p > pRange[1]: return
        # consider only rs in rRange
        if rRange != None:
            for r in xrange(rRange[0],rRange[1]+1):
                if p**r > border: break
                yield p,r
        # consider all rs
        else:
            r = 1
            q = p**r
            while q < border:
                yield p,r
                r += 1
                q = p**r
\end{sagecode}

In den Tabellen der angehängten CD findet man die Ergebnisse unsere
computergestützten Suche. (Die genaue Syntax der \texttt{csv}-Dateien wird
später beschrieben.)


\subsection{Implementierung einer $\PCN$-Suche II}
\label{subsec:impl_pcn_ii}

Leider waren wir nicht in der Lage für alle Paare $(p^r,n)$ obige Implementierung
zu nutzen, da gerade für große $r$ in sinnvoller Rechenzeit kein $\PCN$-Polynom
gefunden werden konnte. Um für diese Ausnahmen dennoch ein $\PCN$-Polynom
präsentieren zu können, geben wir für diese Erweiterungen ein beliebiges
$\PCN$-Polynom an (also nicht wie oben das beste im Sinne der Ordnung aus
\thref{def:polynomordnung}). Dieses finden wir wie folgt: Für gegebenes $(q,n)$
bestimmen wir ein primitives Element $u \in \F_{q^n}$ via der \sage-Funktion
"primitive_element()" oder übergeben bereits ein bekanntes
primitives Element durch das Argument "primitive_element".
Anschließend iterieren wir aufsteigend über alle 
$i\in\N^\ast$ mit $\ggT(i,q^n-1) = 1$ und beenden die Suche mit
Ausgabe des Minimalpolynoms von $u^i$ über $\F_q$, falls
$u^i$ vollständig normal ist. Nach \thref{satz:zykl_gruppen} (5) ist klar, dass
wir $u^i$ nicht mehr auf Primitivität zu testen brauchen.
In \sage sieht dieses Vorgehen konkret wie folgt aus.

\begin{sagecode}[caption={[\texttt{findAnyPCN} aus 
 \url{../Sage/findAnyPCN_additional.spyx}]Aus \url{../Sage/findAnyPCN_additional.spyx}}]
def findAnyPCN(p,r,n, primitive_element=None):
    q = p**r
    F = GF(q,'a')

    E = F.extension(n,'a')
    P = E.prime_subfield()

    Px = PolynomialRing(P,'x')
    Fx = PolynomialRing(F,'x')
    Ex = PolynomialRing(E,'x')
    h = Hom(F,E)[0]
    primOrder = E.order()-1
    
    primitives = []
    
    #setup factors of x^n-1
    divs = get_proper_subfield_divisors(p,r,n)
    facsAll = dict();
    prodsAll = dict();
    fieldsAll = dict();
    for d in divs:
        G = F.extension(Integer(d), 'c');
        Gx = PolynomialRing(G,'x');
        fieldsAll[d] = G;
        facsAll[d] = list((Gx.gen()**(n/d)-1).factor());
        prodsAll[d] = dict();
        for idx,(f,mult) in enumerate(facsAll[d]):
            prodsAll[d][idx] = (Gx.gen()**(n/d)-1).quo_rem(f)[0]
    
    # get one primitive element
    if primitive_element == None:
        x = E.primitive_element()
    else:
        x = primitive_element
        E = primitive_element.parent()
    
    lasti = 0
    y = x
    for i in itertools.count(1):
        if gcd(i,E.order()-1) != 1: continue
        y = y*x**(i-lasti)
        lasti = i
        if isCompletelyNormal(y,E,q,divs,fieldsAll,facsAll,prodsAll):
            mipo = y.minpoly()
            for f,i in Fx(mipo).factor():
                if f.map_coefficients(h)(y) == E.zero():
                    return f
\end{sagecode}


Die „unschönen“ Ergebnisse dieser Funktion wurden in den Tabellen speziell
gekennzeichnet.


\subsection{Auswertung der Ergebnisse}

Mit Hilfe der hier vorgestellten Funktionen zur Findung von $\PCN$-Polynomen
und dem asymptotischen Resultat \thref{satz:pcn_schranke} von 
\citeauthor{hachenberger2014}
konnte nun bewiesen werden, dass für alle $n \in \{ 2,3,\ldots,33\}$ und für
alle Primzahlpotenzen $q$ ein primitiv vollständig normales Polynom von Grad
$n$ über $\F_q$ existiert (\thref{satz:hauptsatz_g}). Dies stellt in der Tat
eine Neuerung in der Forschung über die Existenz von primitiv vollständig
normalen Elementen in Erweiterungen endlicher Körper dar und trägt letztendlich
dazu bei, einen Teil der Antwort auf die große Frage der generellen Existenz
von $\PCN$-Elementen zu liefern.

In der angehängten CD befinden sich im Ordner \texttt{Tables/PCNs} die 
Ergebnisse der $\PCN$-Suche. Die Dateinamen entsprechen dem Schema 
\texttt{pcns\_$n$\_$r$.csv}
und ein Eintrag \texttt{$p$, poly, modulus} bedeutet, dass
\texttt{poly} ein primitiv vollständig normales Polynom von Grad $n$ über 
\[ \F_{p^r} \cong \F_p[a]\big/(\texttt{modulus}) \]
ist. \texttt{modulus} entfällt selbstredend, falls $r = 1$ gilt. Die Primzahlen
$p$ sind innerhalb einer Datei in aufsteigender Reihenfolge sortiert.
Für die Ergebnisse der Funktion "findAnyPCN" aus \autoref{subsec:impl_pcn_ii}
wurde an die Primzahl $p$ ein \texttt{!} angehängt, um herauszustellen, dass
diese Polynome nicht die kleinsten im Sinne der gewählten Ordnung 
(\thref{def:polynomordnung}) sind.

Da \citeauthor{morgan1996} ihre Suche nach $\PCN$-Polynomen keinem
konkreten Ziel widmeten (wir wollten \thref{satz:hauptsatz_g} beweisen),
präsentiert ihre Arbeit 
für jede Primzahl $p\leq 97$ und jede ganze Zahl $n \geq 2$ mit
$p^n < 10^{50}$ ein $\PCN$-Polynom. Auch diesen Bereich konnten wir erweitern
und geben auf beiliegender CD (vgl. Anhang \ref{anh:sec:cd}) 
für $r = 1$ und alle Primzahlen $p < 1000$ mit $p^n < 10^{70}$
primitiv vollständig normale Polynome von Grad $n$ über
$\F_{p^r}$ an.



Wir schließen die vorliegende Arbeit mit einer interessanten Entdeckung, die
bei Betrachtung der berechneten $\PCN$-Polynome besonders auffällig
erscheint:
Für sehr viele Erweiterungen existieren primitiv vollständig normale Trinome. 
Inbesondere kann man erkennen, dass 
für konstantes $n$ und $r$, wenn $p$ nur groß genug wird,
offenbar stets ein $\PCN$-Trinom vorhanden ist. Dieses faszinierende Resultat 
legt die folgende Vermutung nahe, die bisher gänzlich unklar bleibt und für die 
zumindest im Rahmen dieser Arbeit keine Möglichkeit eines 
Beweises gefunden werden konnte.

\begin{vermutung}
  Seien $n\in \N^\ast$ und $r\in \N^\ast$ beliebig.
  Dann existiert ein $P_{n,r}\in \N^\ast$, so dass für alle
  Primzahlen $p \geq P_{n,r}$ 
  ein primitiv vollständig normales Trinom von Grad
  $n$ über $\F_{p^r}$ existiert.
\end{vermutung}

\input{extro}

\clearpage
\addcontentsline{toc}{chapter}{Listings}
\lstlistoflistings

\nocite{*}
\printbibliography


\appendix
\chapter{Tabellen}

Im Folgenden stellen wir die mit Hilfe der vorgestellten Algorithmen
berechneten Werte vor. Dabei ist folgende Legende zu beachten:
\begin{description}
  \item[$q, p,r$] sind die Daten des betrachteten Grundkörpers $\F_q$, wobei
    $q = p^r$ gilt.
  \item[$\CN(q,n)$] gibt die Anzahl der vollständig normalen Elemente
    der Erweiterung von Grad $n$ über $\F_q$ an.
  \item[$\PCN(q,n)$] gibt die Anzahl der primitiv vollständig normalen Elemente 
    der Erweiterung von Grad $n$ über $\F_q$ an.
  \item[\normalfont Erzeuger.] Hier ist die Anzahl der vollständigen Erzeuger
    der Zerlegung nach \thref{satz:zerlegungssatz} gegeben, wobei ein Datum
    $(k,t,\pi):\, N$ bedeutet, dass für den Kreisteilungsmodul 
    $\C_{k,t\pi}$ gerade $N$ vollständige Erzeuger in $\F_q$ existieren.
  \item[$(\speq.)^\ast$] gibt bei Vorhandensein in der Spalte $\CN(q,n)$ an, 
    ob die aktuelle Körpererweiterung einfach (\thref{def:einfach}) ist.
    Falls ja, so gilt per definitionem 
    $\CN(q,n) = \cal N(q,n)$ und $\PCN(q,n) = \PN(q,n)$.
  \item[$(\speq.)^\dagger$] gibt bei Vorhandensein hinter einem Erzeuger-Datum
    an, ob dieser regulär ist (\thref{def:regulaer}).
\end{description}

\begin{longtable}{llllllp{7cm}}
  \caption{Enumerationen $p=2$}\\
  $q$ & $p$ & $r$ & $n$ & $\CN(q,n)$ & $\PCN(q,n)$ & Erzeuger \\\hline
  \endhead
  2 & 2 & 1 & 2 & 2$^\ast$ & 2 & $(1,1,2)^\dagger$: 2\\
2 & 2 & 1 & 3 & 3$^\ast$ & 3 & $(1,1,1)^\dagger$: 1,\ $(3,1,1)^\dagger$: 3\\
2 & 2 & 1 & 4 & 8$^\ast$ & 4 & $(1,1,4)^\dagger$: 8\\
2 & 2 & 1 & 5 & 15$^\ast$ & 15 & $(1,1,1)^\dagger$: 1,\ $(5,1,1)^\dagger$: 15\\
2 & 2 & 1 & 6 & 12 & 6 & $(1,1,2)^\dagger$: 2,\ $(3,1,2)$: 6\\
2 & 2 & 1 & 7 & 49$^\ast$ & 49 & $(1,1,1)^\dagger$: 1,\ $(7,1,1)^\dagger$: 49\\
2 & 2 & 1 & 8 & 128$^\ast$ & 56 & $(1,1,8)^\dagger$: 128\\
2 & 2 & 1 & 9 & 189$^\ast$ & 171 & $(1,1,1)^\dagger$: 1,\ $(3,1,1)^\dagger$: 3,\ $(9,1,1)^\dagger$: 63\\
2 & 2 & 1 & 10 & 420 & 250 & $(1,1,2)^\dagger$: 2,\ $(5,1,2)$: 210\\
2 & 2 & 1 & 11 & 1023$^\ast$ & 957 & $(1,1,1)^\dagger$: 1,\ $(11,1,1)^\dagger$: 1023\\
2 & 2 & 1 & 12 & 768 & 360 & $(1,1,4)^\dagger$: 8,\ $(3,1,4)$: 96\\
2 & 2 & 1 & 13 & 4095$^\ast$ & 4095 & $(1,1,1)^\dagger$: 1,\ $(13,1,1)^\dagger$: 4095\\
2 & 2 & 1 & 14 & 6272$^\ast$ & 4074 & $(1,1,2)^\dagger$: 2,\ $(7,1,2)^\dagger$: 3136\\
2 & 2 & 1 & 15 & 10125$^\ast$ & 8430 & $(1,1,1)^\dagger$: 1,\ $(3,1,1)^\dagger$: 3,\ $(5,1,1)^\dagger$: 15,\ $(15,1,1)^\dagger$: 225\\
2 & 2 & 1 & 16 & 32768$^\ast$ & 16272 & $(1,1,16)^\dagger$: 32768\\
2 & 2 & 1 & 17 & 65025$^\ast$ & 65025 & $(1,1,1)^\dagger$: 1,\ $(17,1,1)^\dagger$: 65025\\
2 & 2 & 1 & 18 & 46872 & 24948 & $(1,1,2)^\dagger$: 2,\ $(3,1,2)$: 6,\ $(9,1,2)$: 3906\\
2 & 2 & 1 & 19 & 262143$^\ast$ & 262143 & $(1,1,1)^\dagger$: 1,\ $(19,1,1)^\dagger$: 262143\\
2 & 2 & 1 & 20 & 329280 & 150320 & $(1,1,4)^\dagger$: 8,\ $(5,1,4)$: 61440\\
2 & 2 & 1 & 21 & 259308 & 220374 & $(1,1,1)^\dagger$: 1,\ $(3,1,1)^\dagger$: 3,\ $(7,3,1)$: 194481\\
2 & 2 & 1 & 22 & 2091012 & 1317250 & $(1,1,2)^\dagger$: 2,\ $(11,1,2)$: 1047552\\
2 & 2 & 1 & 23 & 4190209$^\ast$ & 4099957 & $(1,1,1)^\dagger$: 1,\ $(23,1,1)^\dagger$: 4190209\\
2 & 2 & 1 & 24 & 3145728 & 1246752 & $(1,1,8)^\dagger$: 128,\ $(3,1,8)$: 49152\\
2 & 2 & 1 & 25 & 15728625$^\ast$ & 15188050 & $(1,1,1)^\dagger$: 1,\ $(5,1,1)^\dagger$: 15,\ $(25,1,1)^\dagger$: 1048575\\
2 & 2 & 1 & 26 & 33529860 & 22345232 & $(1,1,2)^\dagger$: 2,\ $(13,1,2)$: 16764930\\
2 & 2 & 1 & 27 & 47258883 & 39950874 & $(1,1,1)^\dagger$: 1,\ $(3,1,1)^\dagger$: 3,\ $(9,1,1)^\dagger$: 63,\ $(27,1,1)^\dagger$: 250047\\
2 & 2 & 1 & 28 & 102760448$^\ast$ & 50821260 & $(1,1,4)^\dagger$: 8,\ $(7,1,4)^\dagger$: 12845056\\
2 & 2 & 1 & 29 & 268435455$^\ast$ & 266908663 & $(1,1,1)^\dagger$: 1,\ $(29,1,1)^\dagger$: 268435455\\
2 & 2 & 1 & 30 & 111132000 & 55308540 & $(1,1,2)^\dagger$: 2,\ $(3,1,2)$: 6,\ $(5,1,2)$: 210,\ $(15,1,2)$: 44100\\
2 & 2 & 1 & 31 & 887503681$^\ast$ & 887503681 & $(1,1,1)^\dagger$: 1,\ $(31,1,1)^\dagger$: 887503681\\
4 & 2 & 2 & 2 & 12$^\ast$ & 8 & $(1,1,2)^\dagger$: 12\\
4 & 2 & 2 & 3 & 27$^\ast$ & 18 & $(1,1,1)^\dagger$: 3,\ $(3,1,1)^\dagger$: 9\\
4 & 2 & 2 & 4 & 192$^\ast$ & 96 & $(1,1,4)^\dagger$: 192\\
4 & 2 & 2 & 5 & 675$^\ast$ & 400 & $(1,1,1)^\dagger$: 3,\ $(5,1,1)^\dagger$: 225\\
4 & 2 & 2 & 6 & 1728$^\ast$ & 792 & $(1,1,2)^\dagger$: 12,\ $(3,1,2)^\dagger$: 144\\
4 & 2 & 2 & 7 & 11907$^\ast$ & 7784 & $(1,1,1)^\dagger$: 3,\ $(7,1,1)^\dagger$: 3969\\
4 & 2 & 2 & 8 & 49152$^\ast$ & 24448 & $(1,1,8)^\dagger$: 49152\\
4 & 2 & 2 & 9 & 107163$^\ast$ & 57186 & $(1,1,1)^\dagger$: 3,\ $(3,1,1)^\dagger$: 9,\ $(9,1,1)^\dagger$: 3969\\
4 & 2 & 2 & 10 & 529200 & 241400 & $(1,1,2)^\dagger$: 12,\ $(5,1,2)$: 44100\\
4 & 2 & 2 & 11 & 3139587$^\ast$ & 1978020 & $(1,1,1)^\dagger$: 3,\ $(11,1,1)^\dagger$: 1046529\\
4 & 2 & 2 & 12 & 7077888$^\ast$ & 2803392 & $(1,1,4)^\dagger$: 192,\ $(3,1,4)^\dagger$: 36864\\
4 & 2 & 2 & 13 & 50307075$^\ast$ & 33525908 & $(1,1,1)^\dagger$: 3,\ $(13,1,1)^\dagger$: 16769025\\
4 & 2 & 2 & 14 & 195084288$^\ast$ & 96481224 & $(1,1,2)^\dagger$: 12,\ $(7,1,2)^\dagger$: 16257024\\
8 & 2 & 3 & 2 & 56$^\ast$ & 36 & $(1,1,2)^\dagger$: 56\\
8 & 2 & 3 & 3 & 441$^\ast$ & 378 & $(1,1,1)^\dagger$: 7,\ $(3,1,1)^\dagger$: 63\\
8 & 2 & 3 & 4 & 3584$^\ast$ & 1512 & $(1,1,4)^\dagger$: 3584\\
8 & 2 & 3 & 5 & 28665$^\ast$ & 23760 & $(1,1,1)^\dagger$: 7,\ $(5,1,1)^\dagger$: 4095\\
8 & 2 & 3 & 6 & 218736 & 117288 & $(1,1,2)^\dagger$: 56,\ $(3,1,2)$: 3906\\
8 & 2 & 3 & 7 & 823543$^\ast$ & 698544 & $(1,1,1)^\dagger$: 7,\ $(7,1,1)^\dagger$: 117649\\
8 & 2 & 3 & 8 & 14680064$^\ast$ & 5804640 & $(1,1,8)^\dagger$: 14680064\\
8 & 2 & 3 & 9 & 110270727$^\ast$ & 93223872 & $(1,1,1)^\dagger$: 7,\ $(3,1,1)^\dagger$: 63,\ $(9,1,1)^\dagger$: 250047\\
16 & 2 & 4 & 3 & 3375$^\ast$ & 1440 & $(1,1,1)^\dagger$: 15,\ $(3,1,1)^\dagger$: 225\\
16 & 2 & 4 & 4 & 61440$^\ast$ & 30720 & $(1,1,4)^\dagger$: 61440\\
16 & 2 & 4 & 6 & 13824000$^\ast$ & 5469696 & $(1,1,2)^\dagger$: 240,\ $(3,1,2)^\dagger$: 57600\\
32 & 2 & 5 & 3 & 31713$^\ast$ & 26100 & $(1,1,1)^\dagger$: 31,\ $(3,1,1)^\dagger$: 1023\\
32 & 2 & 5 & 4 & 1015808$^\ast$ & 465000 & $(1,1,4)^\dagger$: 1015808\\
32 & 2 & 5 & 6 & 1037141952 & 516358800 & $(1,1,2)^\dagger$: 992,\ $(3,1,2)$: 1045506\\
64 & 2 & 6 & 3 & 250047$^\ast$ & 134136 & $(1,1,1)^\dagger$: 63,\ $(3,1,1)^\dagger$: 3969\\
64 & 2 & 6 & 4 & 16515072$^\ast$ & 6531840 & $(1,1,4)^\dagger$: 16515072\\
128 & 2 & 7 & 3 & 2080641$^\ast$ & 1764882 & $(1,1,1)^\dagger$: 127,\ $(3,1,1)^\dagger$: 16383\\
128 & 2 & 7 & 4 & 266338304$^\ast$ & 131721408 & $(1,1,4)^\dagger$: 266338304\\
256 & 2 & 8 & 3 & 16581375$^\ast$ & 6561792 & $(1,1,1)^\dagger$: 255,\ $(3,1,1)^\dagger$: 65025\\
512 & 2 & 9 & 3 & 133955073$^\ast$ & 113245776 & $(1,1,1)^\dagger$: 511,\ $(3,1,1)^\dagger$: 262143\\

\end{longtable}

\begin{longtable}{llllllp{7cm}}
  \caption{Enumerationen $p=3$}\\
  $q$ & $p$ & $r$ & $n$ & $\CN(q,n)$ & $\PCN(q,n)$ & Erzeuger \\\hline
  3 & 3 & 1 & 2 & 4$^\ast$ & 4 & $(1,1,1)^\dagger$: 2,\ $(2,1,1)^\dagger$: 2\\
3 & 3 & 1 & 3 & 18$^\ast$ & 9 & $(1,1,3)^\dagger$: 18\\
3 & 3 & 1 & 4 & 32$^\ast$ & 16 & $(1,1,1)^\dagger$: 2,\ $(2,1,1)^\dagger$: 2,\ $(4,1,1)^\dagger$: 8\\
3 & 3 & 1 & 5 & 160$^\ast$ & 75 & $(1,1,1)^\dagger$: 2,\ $(5,1,1)^\dagger$: 80\\
3 & 3 & 1 & 6 & 324$^\ast$ & 144 & $(1,1,3)^\dagger$: 18,\ $(2,1,3)^\dagger$: 18\\
3 & 3 & 1 & 7 & 1456$^\ast$ & 728 & $(1,1,1)^\dagger$: 2,\ $(7,1,1)^\dagger$: 728\\
3 & 3 & 1 & 8 & 1536 & 576 & $(1,1,1)^\dagger$: 2,\ $(2,1,1)^\dagger$: 2,\ $(4,1,1)^\dagger$: 8,\ $(8,1,1)^\dagger$: 48\\
3 & 3 & 1 & 9 & 13122$^\ast$ & 6075 & $(1,1,9)^\dagger$: 13122\\
3 & 3 & 1 & 10 & 24960 & 11160 & $(1,1,1)^\dagger$: 2,\ $(2,1,1)^\dagger$: 2,\ $(5,2,1)$: 6240\\
3 & 3 & 1 & 11 & 117128$^\ast$ & 55979 & $(1,1,1)^\dagger$: 2,\ $(11,1,1)^\dagger$: 58564\\
3 & 3 & 1 & 12 & 209952$^\ast$ & 65424 & $(1,1,3)^\dagger$: 18,\ $(2,1,3)^\dagger$: 18,\ $(4,1,3)^\dagger$: 648\\
3 & 3 & 1 & 13 & 913952$^\ast$ & 456976 & $(1,1,1)^\dagger$: 2,\ $(13,1,1)^\dagger$: 456976\\
3 & 3 & 1 & 14 & 2114112 & 1054368 & $(1,1,1)^\dagger$: 2,\ $(2,1,1)^\dagger$: 2,\ $(7,2,1)$: 529984\\
3 & 3 & 1 & 15 & 9447840$^\ast$ & 3962700 & $(1,1,3)^\dagger$: 18,\ $(5,1,3)^\dagger$: 524880\\
3 & 3 & 1 & 16 & 6291456 & 2289984 & $(1,1,1)^\dagger$: 2,\ $(2,1,1)^\dagger$: 2,\ $(4,1,1)^\dagger$: 8,\ $(8,1,1)^\dagger$: 64,\ $(16,1,1)^\dagger$: 6400\\
3 & 3 & 1 & 17 & 86093440$^\ast$ & 43022053 & $(1,1,1)^\dagger$: 2,\ $(17,1,1)^\dagger$: 43046720\\
3 & 3 & 1 & 18 & 172186884$^\ast$ & 62696736 & $(1,1,9)^\dagger$: 13122,\ $(2,1,9)^\dagger$: 13122\\
3 & 3 & 1 & 19 & 774840976$^\ast$ & 387177364 & $(1,1,1)^\dagger$: 2,\ $(19,1,1)^\dagger$: 387420488\\
3 & 3 & 1 & 20 & 1184481280 & 423266160 & $(1,1,1)^\dagger$: 2,\ $(2,1,1)^\dagger$: 2,\ $(4,1,1)^\dagger$: 8,\ $(5,4,1)$: 37015040\\
3 & 3 & 1 & 21 & 6935383728 & -- & $(1,1,3)^\dagger$: 18,\ $(7,1,3)$: 385299096\\
3 & 3 & 1 & 22 & 13718968384$^\ast$ & -- & $(1,1,1)^\dagger$: 2,\ $(2,1,1)^\dagger$: 2,\ $(11,1,1)^\dagger$: 58564,\ $(22,1,1)^\dagger$: 58564\\
9 & 3 & 2 & 2 & 64$^\ast$ & 32 & $(1,1,1)^\dagger$: 8,\ $(2,1,1)^\dagger$: 8\\
9 & 3 & 2 & 3 & 648$^\ast$ & 264 & $(1,1,3)^\dagger$: 648\\
9 & 3 & 2 & 4 & 4096$^\ast$ & 1536 & $(1,1,1)^\dagger$: 8,\ $(2,1,1)^\dagger$: 8,\ $(4,1,1)^\dagger$: 64\\
9 & 3 & 2 & 5 & 51200$^\ast$ & 23000 & $(1,1,1)^\dagger$: 8,\ $(5,1,1)^\dagger$: 6400\\
9 & 3 & 2 & 6 & 419904$^\ast$ & 130848 & $(1,1,3)^\dagger$: 648,\ $(2,1,3)^\dagger$: 648\\
9 & 3 & 2 & 7 & 4239872$^\ast$ & 2115008 & $(1,1,1)^\dagger$: 8,\ $(7,1,1)^\dagger$: 529984\\
9 & 3 & 2 & 8 & 16777216$^\ast$ & 6117376 & $(1,1,1)^\dagger$: 8,\ $(2,1,1)^\dagger$: 8,\ $(4,1,1)^\dagger$: 64,\ $(8,1,1)^\dagger$: 4096\\
9 & 3 & 2 & 9 & 344373768$^\ast$ & 125421768 & $(1,1,9)^\dagger$: 344373768\\
27 & 3 & 3 & 3 & 18954$^\ast$ & 8748 & $(1,1,3)^\dagger$: 18954\\
27 & 3 & 3 & 4 & 492128$^\ast$ & 154368 & $(1,1,1)^\dagger$: 26,\ $(2,1,1)^\dagger$: 26,\ $(4,1,1)^\dagger$: 728\\
27 & 3 & 3 & 6 & 359254116$^\ast$ & 130838112 & $(1,1,3)^\dagger$: 18954,\ $(2,1,3)^\dagger$: 18954\\
81 & 3 & 4 & 3 & 524880$^\ast$ & 163584 & $(1,1,3)^\dagger$: 524880\\
81 & 3 & 4 & 4 & 40960000$^\ast$ & 14962688 & $(1,1,1)^\dagger$: 80,\ $(2,1,1)^\dagger$: 80,\ $(4,1,1)^\dagger$: 6400\\
243 & 3 & 5 & 3 & 14289858$^\ast$ & 5994450 & $(1,1,3)^\dagger$: 14289858\\
243 & 3 & 5 & 4 & 3458087072$^\ast$ & 1235872000 & $(1,1,1)^\dagger$: 242,\ $(2,1,1)^\dagger$: 242,\ $(4,1,1)^\dagger$: 59048\\
729 & 3 & 6 & 3 & 386889048$^\ast$ & 140901120 & $(1,1,3)^\dagger$: 386889048\\

\end{longtable}

\begin{longtable}{llllllp{7cm}}
  \caption{Enumerationen $p=5$}\\
  $q$ & $p$ & $r$ & $n$ & $\CN(q,n)$ & $\PCN(q,n)$ & Erzeuger \\\hline
  \endhead
  5 & 5 & 1 & 2 & 16$^\ast$ & 8 & $(1,1,1)^\dagger$: 4,\ $(2,1,1)^\dagger$: 4\\
5 & 5 & 1 & 3 & 96$^\ast$ & 48 & $(1,1,1)^\dagger$: 4,\ $(3,1,1)^\dagger$: 24\\
5 & 5 & 1 & 4 & 256$^\ast$ & 64 & $(1,1,1)^\dagger$: 4,\ $(2,1,1)^\dagger$: 4,\ $(4,1,1)^\dagger$: 16\\
5 & 5 & 1 & 5 & 2500$^\ast$ & 1130 & $(1,1,5)^\dagger$: 2500\\
5 & 5 & 1 & 6 & 8448 & 2376 & $(1,1,1)^\dagger$: 4,\ $(2,1,1)^\dagger$: 4,\ $(3,2,1)$: 528\\
5 & 5 & 1 & 7 & 62496$^\ast$ & 31248 & $(1,1,1)^\dagger$: 4,\ $(7,1,1)^\dagger$: 15624\\
5 & 5 & 1 & 8 & 147456$^\ast$ & 44928 & $(1,1,1)^\dagger$: 4,\ $(2,1,1)^\dagger$: 4,\ $(4,1,1)^\dagger$: 16,\ $(8,1,1)^\dagger$: 576\\
5 & 5 & 1 & 9 & 1499904$^\ast$ & 687132 & $(1,1,1)^\dagger$: 4,\ $(3,1,1)^\dagger$: 24,\ $(9,1,1)^\dagger$: 15624\\
5 & 5 & 1 & 10 & 6250000$^\ast$ & 1862760 & $(1,1,5)^\dagger$: 2500,\ $(2,1,5)^\dagger$: 2500\\
5 & 5 & 1 & 11 & 39037504$^\ast$ & 19518752 & $(1,1,1)^\dagger$: 4,\ $(11,1,1)^\dagger$: 9759376\\
5 & 5 & 1 & 12 & 71368704 & 18178944 & $(1,1,1)^\dagger$: 4,\ $(2,1,1)^\dagger$: 4,\ $(3,2,1)$: 528,\ $(4,1,1)^\dagger$: 16,\ $(12,1,1)$: 528\\
25 & 5 & 2 & 3 & 13824$^\ast$ & 3888 & $(1,1,1)^\dagger$: 24,\ $(3,1,1)^\dagger$: 576\\
25 & 5 & 2 & 4 & 331776$^\ast$ & 101376 & $(1,1,1)^\dagger$: 24,\ $(2,1,1)^\dagger$: 24,\ $(4,1,1)^\dagger$: 576\\
25 & 5 & 2 & 6 & 191102976$^\ast$ & 48691008 & $(1,1,1)^\dagger$: 24,\ $(2,1,1)^\dagger$: 24,\ $(3,1,1)^\dagger$: 576,\ $(6,1,1)^\dagger$: 576\\
125 & 5 & 3 & 3 & 1937376$^\ast$ & 887220 & $(1,1,1)^\dagger$: 124,\ $(3,1,1)^\dagger$: 15624\\
125 & 5 & 3 & 4 & 236421376$^\ast$ & 60235200 & $(1,1,1)^\dagger$: 124,\ $(2,1,1)^\dagger$: 124,\ $(4,1,1)^\dagger$: 15376\\
625 & 5 & 4 & 3 & 242970624$^\ast$ & 61910784 & $(1,1,1)^\dagger$: 624,\ $(3,1,1)^\dagger$: 389376\\

\end{longtable}

\begin{longtable}{llllllp{7cm}}
  \caption{Enumerationen $p=7$}\\
  $q$ & $p$ & $r$ & $n$ & $\CN(q,n)$ & $\PCN(q,n)$ & Erzeuger \\\hline
  \endhead
  7 & 7 & 1 & 2 & 36$^\ast$ & 16 & $(1,1,1)^\dagger$: 6,\ $(2,1,1)^\dagger$: 6\\7 & 7 & 1 & 3 & 216$^\ast$ & 72 & $(1,1,1)^\dagger$: 6,\ $(3,1,1)^\dagger$: 36\\7 & 7 & 1 & 4 & 1728$^\ast$ & 480 & $(1,1,1)^\dagger$: 6,\ $(2,1,1)^\dagger$: 6,\ $(4,1,1)^\dagger$: 48\\7 & 7 & 1 & 5 & 14400$^\ast$ & 4800 & $(1,1,1)^\dagger$: 6,\ $(5,1,1)^\dagger$: 2400\\7 & 7 & 1 & 6 & 46656$^\ast$ & 14832 & $(1,1,1)^\dagger$: 6,\ $(2,1,1)^\dagger$: 6,\ $(3,1,1)^\dagger$: 36,\ $(6,1,1)^\dagger$: 36\\7 & 7 & 1 & 7 & 705894$^\ast$ & 227010 & $(1,1,7)^\dagger$: 705894\\7 & 7 & 1 & 8 & 3815424 & 1016320 & $(1,1,1)^\dagger$: 6,\ $(2,1,1)^\dagger$: 6,\ $(4,1,1)^\dagger$: 48,\ $(8,1,1)^\dagger$: 2208\\7 & 7 & 1 & 9 & 25264224$^\ast$ & 7753806 & $(1,1,1)^\dagger$: 6,\ $(3,1,1)^\dagger$: 36,\ $(9,1,1)^\dagger$: 116964\\7 & 7 & 1 & 10 & 207187200 & 62435920 & $(1,1,1)^\dagger$: 6,\ $(2,1,1)^\dagger$: 6,\ $(5,2,1)$: 5755200\\7 & 7 & 1 & 11 & 1694851488$^\ast$ & 564443264 & $(1,1,1)^\dagger$: 6,\ $(11,1,1)^\dagger$: 282475248\\
\end{longtable}

\begin{longtable}{llllllp{7cm}}
  \caption{Enumerationen $n=3$}\\
  $q$ & $p$ & $r$ & $n$ & $\CN(q,n)$ & $\PCN(q,n)$ & Erzeuger \\\hline
  \endhead
  2 & 2 & 1 & 3 & 3$^\ast$ & 3 & $(1,1,1)^\dagger$: 1,\ $(3,1,1)^\dagger$: 3\\3 & 3 & 1 & 3 & 18$^\ast$ & 9 & $(1,1,3)^\dagger$: 18\\4 & 2 & 2 & 3 & 27$^\ast$ & 18 & $(1,1,1)^\dagger$: 3,\ $(3,1,1)^\dagger$: 9\\5 & 5 & 1 & 3 & 96$^\ast$ & 48 & $(1,1,1)^\dagger$: 4,\ $(3,1,1)^\dagger$: 24\\7 & 7 & 1 & 3 & 216$^\ast$ & 72 & $(1,1,1)^\dagger$: 6,\ $(3,1,1)^\dagger$: 36\\8 & 2 & 3 & 3 & 441$^\ast$ & 378 & $(1,1,1)^\dagger$: 7,\ $(3,1,1)^\dagger$: 63\\9 & 3 & 2 & 3 & 648$^\ast$ & 264 & $(1,1,3)^\dagger$: 648\\11 & 11 & 1 & 3 & 1200$^\ast$ & 384 & $(1,1,1)^\dagger$: 10,\ $(3,1,1)^\dagger$: 120\\13 & 13 & 1 & 3 & 1728$^\ast$ & 576 & $(1,1,1)^\dagger$: 12,\ $(3,1,1)^\dagger$: 144\\16 & 2 & 4 & 3 & 3375$^\ast$ & 1440 & $(1,1,1)^\dagger$: 15,\ $(3,1,1)^\dagger$: 225\\17 & 17 & 1 & 3 & 4608$^\ast$ & 2304 & $(1,1,1)^\dagger$: 16,\ $(3,1,1)^\dagger$: 288\\19 & 19 & 1 & 3 & 5832$^\ast$ & 1944 & $(1,1,1)^\dagger$: 18,\ $(3,1,1)^\dagger$: 324\\23 & 23 & 1 & 3 & 11616$^\ast$ & 4440 & $(1,1,1)^\dagger$: 22,\ $(3,1,1)^\dagger$: 528\\25 & 5 & 2 & 3 & 13824$^\ast$ & 3888 & $(1,1,1)^\dagger$: 24,\ $(3,1,1)^\dagger$: 576\\27 & 3 & 3 & 3 & 18954$^\ast$ & 8748 & $(1,1,3)^\dagger$: 18954\\29 & 29 & 1 & 3 & 23520$^\ast$ & 9180 & $(1,1,1)^\dagger$: 28,\ $(3,1,1)^\dagger$: 840\\31 & 31 & 1 & 3 & 27000$^\ast$ & 7200 & $(1,1,1)^\dagger$: 30,\ $(3,1,1)^\dagger$: 900\\32 & 2 & 5 & 3 & 31713$^\ast$ & 26100 & $(1,1,1)^\dagger$: 31,\ $(3,1,1)^\dagger$: 1023\\37 & 37 & 1 & 3 & 46656$^\ast$ & 13176 & $(1,1,1)^\dagger$: 36,\ $(3,1,1)^\dagger$: 1296\\41 & 41 & 1 & 3 & 67200$^\ast$ & 26880 & $(1,1,1)^\dagger$: 40,\ $(3,1,1)^\dagger$: 1680\\43 & 43 & 1 & 3 & 74088$^\ast$ & 21168 & $(1,1,1)^\dagger$: 42,\ $(3,1,1)^\dagger$: 1764\\47 & 47 & 1 & 3 & 101568$^\ast$ & 46596 & $(1,1,1)^\dagger$: 46,\ $(3,1,1)^\dagger$: 2208\\49 & 7 & 2 & 3 & 110592$^\ast$ & 34272 & $(1,1,1)^\dagger$: 48,\ $(3,1,1)^\dagger$: 2304\\53 & 53 & 1 & 3 & 146016$^\ast$ & 57744 & $(1,1,1)^\dagger$: 52,\ $(3,1,1)^\dagger$: 2808\\59 & 59 & 1 & 3 & 201840$^\ast$ & 97440 & $(1,1,1)^\dagger$: 58,\ $(3,1,1)^\dagger$: 3480\\61 & 61 & 1 & 3 & 216000$^\ast$ & 52848 & $(1,1,1)^\dagger$: 60,\ $(3,1,1)^\dagger$: 3600\\64 & 2 & 6 & 3 & 250047$^\ast$ & 134136 & $(1,1,1)^\dagger$: 63,\ $(3,1,1)^\dagger$: 3969\\67 & 67 & 1 & 3 & 287496$^\ast$ & 72000 & $(1,1,1)^\dagger$: 66,\ $(3,1,1)^\dagger$: 4356\\71 & 71 & 1 & 3 & 352800$^\ast$ & 120960 & $(1,1,1)^\dagger$: 70,\ $(3,1,1)^\dagger$: 5040\\73 & 73 & 1 & 3 & 373248$^\ast$ & 124416 & $(1,1,1)^\dagger$: 72,\ $(3,1,1)^\dagger$: 5184\\79 & 79 & 1 & 3 & 474552$^\ast$ & 122040 & $(1,1,1)^\dagger$: 78,\ $(3,1,1)^\dagger$: 6084\\81 & 3 & 4 & 3 & 524880$^\ast$ & 163584 & $(1,1,3)^\dagger$: 524880\\83 & 83 & 1 & 3 & 564816$^\ast$ & 260280 & $(1,1,1)^\dagger$: 82,\ $(3,1,1)^\dagger$: 6888\\89 & 89 & 1 & 3 & 696960$^\ast$ & 316800 & $(1,1,1)^\dagger$: 88,\ $(3,1,1)^\dagger$: 7920\\97 & 97 & 1 & 3 & 884736$^\ast$ & 294912 & $(1,1,1)^\dagger$: 96,\ $(3,1,1)^\dagger$: 9216\\121 & 11 & 2 & 3 & 1728000$^\ast$ & 364608 & $(1,1,1)^\dagger$: 120,\ $(3,1,1)^\dagger$: 14400\\125 & 5 & 3 & 3 & 1937376$^\ast$ & 887220 & $(1,1,1)^\dagger$: 124,\ $(3,1,1)^\dagger$: 15624\\128 & 2 & 7 & 3 & 2080641$^\ast$ & 1764882 & $(1,1,1)^\dagger$: 127,\ $(3,1,1)^\dagger$: 16383\\169 & 13 & 2 & 3 & 4741632$^\ast$ & 1325376 & $(1,1,1)^\dagger$: 168,\ $(3,1,1)^\dagger$: 28224\\243 & 3 & 5 & 3 & 14289858$^\ast$ & 5994450 & $(1,1,3)^\dagger$: 14289858\\256 & 2 & 8 & 3 & 16581375$^\ast$ & 6561792 & $(1,1,1)^\dagger$: 255,\ $(3,1,1)^\dagger$: 65025\\289 & 17 & 2 & 3 & 23887872$^\ast$ & 6283008 & $(1,1,1)^\dagger$: 288,\ $(3,1,1)^\dagger$: 82944\\343 & 7 & 3 & 3 & 40001688$^\ast$ & 12279276 & $(1,1,1)^\dagger$: 342,\ $(3,1,1)^\dagger$: 116964\\361 & 19 & 2 & 3 & 46656000$^\ast$ & 10584000 & $(1,1,1)^\dagger$: 360,\ $(3,1,1)^\dagger$: 129600\\512 & 2 & 9 & 3 & 133955073$^\ast$ & 113245776 & $(1,1,1)^\dagger$: 511,\ $(3,1,1)^\dagger$: 262143\\529 & 23 & 2 & 3 & 147197952$^\ast$ & 34848000 & $(1,1,1)^\dagger$: 528,\ $(3,1,1)^\dagger$: 278784\\625 & 5 & 4 & 3 & 242970624$^\ast$ & 61910784 & $(1,1,1)^\dagger$: 624,\ $(3,1,1)^\dagger$: 389376\\729 & 3 & 6 & 3 & 386889048$^\ast$ & 140901120 & $(1,1,3)^\dagger$: 386889048\\841 & 29 & 2 & 3 & 592704000$^\ast$ & 122760576 & $(1,1,1)^\dagger$: 840,\ $(3,1,1)^\dagger$: 705600\\961 & 31 & 2 & 3 & 884736000$^\ast$ & 191020032 & $(1,1,1)^\dagger$: 960,\ $(3,1,1)^\dagger$: 921600\\
\end{longtable}

\begin{longtable}{llllllp{7cm}}
  \caption{Enumerationen $n=4$}\\
  $q$ & $p$ & $r$ & $n$ & $\CN(q,n)$ & $\PCN(q,n)$ & Erzeuger \\\hline
  \endhead
  2 & 2 & 1 & 4 & 8$^\ast$ & 4 & $(1,1,4)^\dagger$: 8\\
3 & 3 & 1 & 4 & 32$^\ast$ & 16 & $(1,1,1)^\dagger$: 2,\ $(2,1,1)^\dagger$: 2,\ $(4,1,1)^\dagger$: 8\\
4 & 2 & 2 & 4 & 192$^\ast$ & 96 & $(1,1,4)^\dagger$: 192\\
5 & 5 & 1 & 4 & 256$^\ast$ & 64 & $(1,1,1)^\dagger$: 4,\ $(2,1,1)^\dagger$: 4,\ $(4,1,1)^\dagger$: 16\\
7 & 7 & 1 & 4 & 1728$^\ast$ & 480 & $(1,1,1)^\dagger$: 6,\ $(2,1,1)^\dagger$: 6,\ $(4,1,1)^\dagger$: 48\\
8 & 2 & 3 & 4 & 3584$^\ast$ & 1512 & $(1,1,4)^\dagger$: 3584\\
9 & 3 & 2 & 4 & 4096$^\ast$ & 1536 & $(1,1,1)^\dagger$: 8,\ $(2,1,1)^\dagger$: 8,\ $(4,1,1)^\dagger$: 64\\
11 & 11 & 1 & 4 & 12000$^\ast$ & 3200 & $(1,1,1)^\dagger$: 10,\ $(2,1,1)^\dagger$: 10,\ $(4,1,1)^\dagger$: 120\\
13 & 13 & 1 & 4 & 20736$^\ast$ & 4352 & $(1,1,1)^\dagger$: 12,\ $(2,1,1)^\dagger$: 12,\ $(4,1,1)^\dagger$: 144\\
16 & 2 & 4 & 4 & 61440$^\ast$ & 30720 & $(1,1,4)^\dagger$: 61440\\
17 & 17 & 1 & 4 & 65536$^\ast$ & 16896 & $(1,1,1)^\dagger$: 16,\ $(2,1,1)^\dagger$: 16,\ $(4,1,1)^\dagger$: 256\\
19 & 19 & 1 & 4 & 116640$^\ast$ & 31104 & $(1,1,1)^\dagger$: 18,\ $(2,1,1)^\dagger$: 18,\ $(4,1,1)^\dagger$: 360\\
23 & 23 & 1 & 4 & 255552$^\ast$ & 60640 & $(1,1,1)^\dagger$: 22,\ $(2,1,1)^\dagger$: 22,\ $(4,1,1)^\dagger$: 528\\
25 & 5 & 2 & 4 & 331776$^\ast$ & 101376 & $(1,1,1)^\dagger$: 24,\ $(2,1,1)^\dagger$: 24,\ $(4,1,1)^\dagger$: 576\\
27 & 3 & 3 & 4 & 492128$^\ast$ & 154368 & $(1,1,1)^\dagger$: 26,\ $(2,1,1)^\dagger$: 26,\ $(4,1,1)^\dagger$: 728\\
29 & 29 & 1 & 4 & 614656$^\ast$ & 139776 & $(1,1,1)^\dagger$: 28,\ $(2,1,1)^\dagger$: 28,\ $(4,1,1)^\dagger$: 784\\
31 & 31 & 1 & 4 & 864000$^\ast$ & 207360 & $(1,1,1)^\dagger$: 30,\ $(2,1,1)^\dagger$: 30,\ $(4,1,1)^\dagger$: 960\\
32 & 2 & 5 & 4 & 1015808$^\ast$ & 465000 & $(1,1,4)^\dagger$: 1015808\\
37 & 37 & 1 & 4 & 1679616$^\ast$ & 420864 & $(1,1,1)^\dagger$: 36,\ $(2,1,1)^\dagger$: 36,\ $(4,1,1)^\dagger$: 1296\\
41 & 41 & 1 & 4 & 2560000$^\ast$ & 564224 & $(1,1,1)^\dagger$: 40,\ $(2,1,1)^\dagger$: 40,\ $(4,1,1)^\dagger$: 1600\\
43 & 43 & 1 & 4 & 3259872$^\ast$ & 659712 & $(1,1,1)^\dagger$: 42,\ $(2,1,1)^\dagger$: 42,\ $(4,1,1)^\dagger$: 1848\\
47 & 47 & 1 & 4 & 4672128$^\ast$ & 1036288 & $(1,1,1)^\dagger$: 46,\ $(2,1,1)^\dagger$: 46,\ $(4,1,1)^\dagger$: 2208\\
49 & 7 & 2 & 4 & 5308416$^\ast$ & 1413120 & $(1,1,1)^\dagger$: 48,\ $(2,1,1)^\dagger$: 48,\ $(4,1,1)^\dagger$: 2304\\
53 & 53 & 1 & 4 & 7311616$^\ast$ & 1794816 & $(1,1,1)^\dagger$: 52,\ $(2,1,1)^\dagger$: 52,\ $(4,1,1)^\dagger$: 2704\\
59 & 59 & 1 & 4 & 11706720$^\ast$ & 3014144 & $(1,1,1)^\dagger$: 58,\ $(2,1,1)^\dagger$: 58,\ $(4,1,1)^\dagger$: 3480\\
61 & 61 & 1 & 4 & 12960000$^\ast$ & 3340800 & $(1,1,1)^\dagger$: 60,\ $(2,1,1)^\dagger$: 60,\ $(4,1,1)^\dagger$: 3600\\
64 & 2 & 6 & 4 & 16515072$^\ast$ & 6531840 & $(1,1,4)^\dagger$: 16515072\\
67 & 67 & 1 & 4 & 19549728$^\ast$ & 4453760 & $(1,1,1)^\dagger$: 66,\ $(2,1,1)^\dagger$: 66,\ $(4,1,1)^\dagger$: 4488\\
71 & 71 & 1 & 4 & 24696000$^\ast$ & 5644800 & $(1,1,1)^\dagger$: 70,\ $(2,1,1)^\dagger$: 70,\ $(4,1,1)^\dagger$: 5040\\
73 & 73 & 1 & 4 & 26873856$^\ast$ & 6279168 & $(1,1,1)^\dagger$: 72,\ $(2,1,1)^\dagger$: 72,\ $(4,1,1)^\dagger$: 5184\\
79 & 79 & 1 & 4 & 37964160$^\ast$ & 9345024 & $(1,1,1)^\dagger$: 78,\ $(2,1,1)^\dagger$: 78,\ $(4,1,1)^\dagger$: 6240\\
81 & 3 & 4 & 4 & 40960000$^\ast$ & 14962688 & $(1,1,1)^\dagger$: 80,\ $(2,1,1)^\dagger$: 80,\ $(4,1,1)^\dagger$: 6400\\
83 & 83 & 1 & 4 & 46314912$^\ast$ & 9351040 & $(1,1,1)^\dagger$: 82,\ $(2,1,1)^\dagger$: 82,\ $(4,1,1)^\dagger$: 6888\\
89 & 89 & 1 & 4 & 59969536$^\ast$ & 13620480 & $(1,1,1)^\dagger$: 88,\ $(2,1,1)^\dagger$: 88,\ $(4,1,1)^\dagger$: 7744\\
97 & 97 & 1 & 4 & 84934656$^\ast$ & 19390976 & $(1,1,1)^\dagger$: 96,\ $(2,1,1)^\dagger$: 96,\ $(4,1,1)^\dagger$: 9216\\
121 & 11 & 2 & 4 & 207360000$^\ast$ & 54374400 & $(1,1,1)^\dagger$: 120,\ $(2,1,1)^\dagger$: 120,\ $(4,1,1)^\dagger$: 14400\\
125 & 5 & 3 & 4 & 236421376$^\ast$ & 60235200 & $(1,1,1)^\dagger$: 124,\ $(2,1,1)^\dagger$: 124,\ $(4,1,1)^\dagger$: 15376\\
128 & 2 & 7 & 4 & 266338304$^\ast$ & 131721408 & $(1,1,4)^\dagger$: 266338304\\
169 & 13 & 2 & 4 & 796594176$^\ast$ & 171343872 & $(1,1,1)^\dagger$: 168,\ $(2,1,1)^\dagger$: 168,\ $(4,1,1)^\dagger$: 28224\\
243 & 3 & 5 & 4 & 3458087072$^\ast$ & 1235872000 & $(1,1,1)^\dagger$: 242,\ $(2,1,1)^\dagger$: 242,\ $(4,1,1)^\dagger$: 59048\\

\end{longtable}

\begin{longtable}{llllllp{7cm}}
  \caption{Enumerationen $n=6$}\\
  $q$ & $p$ & $r$ & $n$ & $\CN(q,n)$ & $\PCN(q,n)$ & Erzeuger \\\hline
  \endhead
  2 & 2 & 1 & 6 & 12 & 6 & $(1,1,2)^\dagger$: 2,\ $(3,1,2)$: 6\\
3 & 3 & 1 & 6 & 324$^\ast$ & 144 & $(1,1,3)^\dagger$: 18,\ $(2,1,3)^\dagger$: 18\\
4 & 2 & 2 & 6 & 1728$^\ast$ & 792 & $(1,1,2)^\dagger$: 12,\ $(3,1,2)^\dagger$: 144\\
5 & 5 & 1 & 6 & 8448 & 2376 & $(1,1,1)^\dagger$: 4,\ $(2,1,1)^\dagger$: 4,\ $(3,2,1)$: 528\\
7 & 7 & 1 & 6 & 46656$^\ast$ & 14832 & $(1,1,1)^\dagger$: 6,\ $(2,1,1)^\dagger$: 6,\ $(3,1,1)^\dagger$: 36,\ $(6,1,1)^\dagger$: 36\\
8 & 2 & 3 & 6 & 218736 & 117288 & $(1,1,2)^\dagger$: 56,\ $(3,1,2)$: 3906\\
9 & 3 & 2 & 6 & 419904$^\ast$ & 130848 & $(1,1,3)^\dagger$: 648,\ $(2,1,3)^\dagger$: 648\\
11 & 11 & 1 & 6 & 1416000 & 298848 & $(1,1,1)^\dagger$: 10,\ $(2,1,1)^\dagger$: 10,\ $(3,2,1)$: 14160\\
13 & 13 & 1 & 6 & 2985984$^\ast$ & 834048 & $(1,1,1)^\dagger$: 12,\ $(2,1,1)^\dagger$: 12,\ $(3,1,1)^\dagger$: 144,\ $(6,1,1)^\dagger$: 144\\
16 & 2 & 4 & 6 & 13824000$^\ast$ & 5469696 & $(1,1,2)^\dagger$: 240,\ $(3,1,2)^\dagger$: 57600\\
17 & 17 & 1 & 6 & 21086208 & 5546304 & $(1,1,1)^\dagger$: 16,\ $(2,1,1)^\dagger$: 16,\ $(3,2,1)$: 82368\\
19 & 19 & 1 & 6 & 34012224$^\ast$ & 7711200 & $(1,1,1)^\dagger$: 18,\ $(2,1,1)^\dagger$: 18,\ $(3,1,1)^\dagger$: 324,\ $(6,1,1)^\dagger$: 324\\
23 & 23 & 1 & 6 & 134420352 & 31821840 & $(1,1,1)^\dagger$: 22,\ $(2,1,1)^\dagger$: 22,\ $(3,2,1)$: 277728\\
25 & 5 & 2 & 6 & 191102976$^\ast$ & 48691008 & $(1,1,1)^\dagger$: 24,\ $(2,1,1)^\dagger$: 24,\ $(3,1,1)^\dagger$: 576,\ $(6,1,1)^\dagger$: 576\\
27 & 3 & 3 & 6 & 359254116$^\ast$ & 130838112 & $(1,1,3)^\dagger$: 18954,\ $(2,1,3)^\dagger$: 18954\\
29 & 29 & 1 & 6 & 551873280 & 114307056 & $(1,1,1)^\dagger$: 28,\ $(2,1,1)^\dagger$: 28,\ $(3,2,1)$: 703920\\
31 & 31 & 1 & 6 & 729000000$^\ast$ & 157394880 & $(1,1,1)^\dagger$: 30,\ $(2,1,1)^\dagger$: 30,\ $(3,1,1)^\dagger$: 900,\ $(6,1,1)^\dagger$: 900\\
32 & 2 & 5 & 6 & 1037141952 & 516358800 & $(1,1,2)^\dagger$: 992,\ $(3,1,2)$: 1045506\\
37 & 37 & 1 & 6 & 2176782336$^\ast$ & 548654688 & $(1,1,1)^\dagger$: 36,\ $(2,1,1)^\dagger$: 36,\ $(3,1,1)^\dagger$: 1296,\ $(6,1,1)^\dagger$: 1296\\
41 & 41 & 1 & 6 & 215496704 & 1028522880 & $(1,1,1)^\dagger$: 40,\ $(2,1,1)^\dagger$: 40,\ $(3,2,1)$: 2819040\\
43 & 43 & 1 & 6 & 1194064448$^\ast$ & 1304511264 & $(1,1,1)^\dagger$: 42,\ $(2,1,1)^\dagger$: 42,\ $(3,1,1)^\dagger$: 1764,\ $(6,1,1)^\dagger$: 1764\\

\end{longtable}

\chapter{CD- und Online-Resourcen}
\newcommand{\normalcomma}{{\normalfont ,} }

\section{CD}
\label{anh:sec:cd}
Auf beiliegender CD findet man die folgenden Ordner:

\begin{tabular}{>{\ttfamily}l}
  ./Tables/Enumerations/ \\
  ./Tables/PCNs/ \\
  ./Sage
\end{tabular}

\subsection{\texttt{./Tables/Enumerations}}

In analoger Syntax zu Anhang \ref{anh:sec:enumerationen} finden sich in diesem
Ordner die Tabellen der Enumerationen als digitale Version wieder.
Wie in Anhang \ref{anh:sec:enumerationen} sind die Tabellen dort nach konstanten
$p$ (\texttt{enumerationsPCN\_P\_$p$.csv}) und konstanten $n$ 
(\texttt{enumerationsPCN\_N\_$n$.csv}) separiert. Die Enumerationen, bei denen
lediglich die Anzahl normaler und primitiv normaler Elemente bestimmt wurde,
liegen in den Dateien \texttt{enumerationsPN\_P\_$p$.csv}.

Die Syntax der Einträge entspricht genau der der entsprechenden 
Tabellen. Lediglich das Symbol $^\dagger$ zur Kennzeichnung von regulären
Kreisteilungsmoduln wurde durch \texttt{*} ersetzt.

Es sind folgende Dateien in diesem Ordner enthalten:

\begin{tabular}{>{\small\ttfamily}p{\textwidth}}
enumerationsPCN\_P\_2.csv\normalcomma
enumerationsPCN\_P\_3.csv\normalcomma
enumerationsPCN\_P\_5.csv\normalcomma
enumerationsPCN\_P\_7.csv\normalcomma
enumerationsPCN\_P\_11.csv\normalcomma
enumerationsPCN\_P\_13.csv\normalcomma
enumerationsPCN\_P\_17.csv\normalcomma
enumerationsPCN\_P\_19.csv\normalcomma
enumerationsPCN\_P\_23.csv\normalcomma
enumerationsPCN\_P\_29.csv\normalcomma
enumerationsPCN\_P\_31.csv\normalcomma
enumerationsPCN\_P\_37.csv\normalcomma
enumerationsPCN\_P\_41.csv\normalcomma
enumerationsPCN\_P\_43.csv
\end{tabular}

\begin{tabular}{>{\small\ttfamily}p{\textwidth}}
enumerationsPCN\_N\_3.csv\normalcomma
enumerationsPCN\_N\_4.csv\normalcomma
enumerationsPCN\_N\_6.csv
\end{tabular}

\begin{tabular}{>{\small\ttfamily}p{\textwidth}}
enumerationsPN\_P\_2.csv\normalcomma
enumerationsPN\_P\_3.csv\normalcomma
enumerationsPN\_P\_5.csv
\end{tabular}

\subsection{\texttt{./Tables/PCNs}}

In diesem Ordner sind die Ergebnisse der Suche nach primitiv vollständig
normalen Polynome (\autoref{sec:existenz_pcn}) gespeichert. 
In der Datei \texttt{pcns\_$n$\_$r$.csv} ist für jede Primzahlpotenz 
$p^r < n^4$ ein primitiv vollständig normales Polynom hinterlegt. 
Ein Eintrag \texttt{$p$, poly, modulus} bedeutet, dass
\texttt{poly} ein primitiv vollständig normales Polynom von Grad $n$ über 
\[ \F_{p^r} = \F_p[a]\big/(\texttt{modulus}) \]
ist. \texttt{modulus} entfällt selbstredend, falls $r = 1$ gilt. Die Primzahlen
$p$ sind innerhalb einer Datei in aufsteigender Reihenfolge sortiert.
Das Polynom ist in den meisten Fällen
kleinste bezüglich der Ordnung aus \thref{def:polynomordnung}. 
Ist dies bei einem Eintrag nicht der Fall (vgl. \autoref{subsec:impl_pcn_ii}),
so wurde ein \texttt{!} der Primzahl $p$ angehängt, um dies kenntlich zu
machen.

Es sind folgende Dateien in diesem Ordner enthalten:


\begin{tabular}{>{\small\ttfamily}p{\textwidth}}
pcns\_6\_1.csv\normalcomma
pcns\_6\_2.csv\normalcomma
pcns\_6\_3.csv\normalcomma
pcns\_6\_4.csv\normalcomma
pcns\_6\_5.csv\normalcomma
pcns\_6\_6.csv\normalcomma
pcns\_6\_7.csv\normalcomma
pcns\_6\_8.csv\normalcomma
pcns\_6\_9.csv\normalcomma
pcns\_6\_10.csv
\end{tabular}

\begin{tabular}{>{\small\ttfamily}p{\textwidth}}
pcns\_10\_1.csv\normalcomma
pcns\_10\_2.csv\normalcomma
pcns\_10\_3.csv\normalcomma
pcns\_10\_4.csv\normalcomma
pcns\_10\_5.csv\normalcomma
pcns\_10\_6.csv\normalcomma
pcns\_10\_7.csv\normalcomma
pcns\_10\_8.csv\normalcomma
pcns\_10\_9.csv\normalcomma
pcns\_10\_10.csv\normalcomma
pcns\_10\_11.csv\normalcomma
pcns\_10\_12.csv\normalcomma
pcns\_10\_13.csv
\end{tabular}

\begin{tabular}{>{\small\ttfamily}p{\textwidth}}
pcns\_12\_1.csv\normalcomma
pcns\_12\_2.csv\normalcomma
pcns\_12\_3.csv\normalcomma
pcns\_12\_4.csv\normalcomma
pcns\_12\_5.csv\normalcomma
pcns\_12\_6.csv\normalcomma
pcns\_12\_7.csv\normalcomma
pcns\_12\_8.csv\normalcomma
pcns\_12\_9.csv\normalcomma
pcns\_12\_10.csv\normalcomma
pcns\_12\_11.csv\normalcomma
pcns\_12\_12.csv\normalcomma
pcns\_12\_13.csv\normalcomma
pcns\_12\_14.csv
\end{tabular}

\begin{tabular}{>{\small\ttfamily}p{\textwidth}}
pcns\_14\_1.csv\normalcomma
pcns\_14\_2.csv\normalcomma
pcns\_14\_3.csv\normalcomma
pcns\_14\_4.csv\normalcomma
pcns\_14\_5.csv\normalcomma
pcns\_14\_6.csv\normalcomma
pcns\_14\_7.csv\normalcomma
pcns\_14\_8.csv\normalcomma
pcns\_14\_9.csv\normalcomma
pcns\_14\_10.csv\normalcomma
pcns\_14\_11.csv\normalcomma
pcns\_14\_12.csv\normalcomma
pcns\_14\_13.csv\normalcomma
pcns\_14\_14.csv\normalcomma
pcns\_14\_15.csv
\end{tabular}

\begin{tabular}{>{\small\ttfamily}p{\textwidth}}
pcns\_15\_1.csv\normalcomma
pcns\_15\_2.csv\normalcomma
pcns\_15\_3.csv\normalcomma
pcns\_15\_4.csv\normalcomma
pcns\_15\_5.csv\normalcomma
pcns\_15\_6.csv\normalcomma
pcns\_15\_7.csv\normalcomma
pcns\_15\_8.csv\normalcomma
pcns\_15\_9.csv\normalcomma
pcns\_15\_10.csv\normalcomma
pcns\_15\_11.csv\normalcomma
pcns\_15\_12.csv\normalcomma
pcns\_15\_13.csv\normalcomma
pcns\_15\_14.csv\normalcomma
pcns\_15\_15.csv
\end{tabular}

\begin{tabular}{>{\small\ttfamily}p{\textwidth}}
pcns\_18\_1.csv\normalcomma
pcns\_18\_2.csv\normalcomma
pcns\_18\_3.csv\normalcomma
pcns\_18\_4.csv\normalcomma
pcns\_18\_5.csv\normalcomma
pcns\_18\_6.csv\normalcomma
pcns\_18\_7.csv\normalcomma
pcns\_18\_8.csv\normalcomma
pcns\_18\_9.csv\normalcomma
pcns\_18\_10.csv\normalcomma
pcns\_18\_11.csv\normalcomma
pcns\_18\_12.csv\normalcomma
pcns\_18\_13.csv\normalcomma
pcns\_18\_14.csv\normalcomma
pcns\_18\_15.csv\normalcomma
pcns\_18\_16.csv
\end{tabular}

\begin{tabular}{>{\small\ttfamily}p{\textwidth}}
pcns\_20\_1.csv\normalcomma
pcns\_20\_2.csv\normalcomma
pcns\_20\_3.csv\normalcomma
pcns\_20\_4.csv\normalcomma
pcns\_20\_5.csv\normalcomma
pcns\_20\_6.csv\normalcomma
pcns\_20\_7.csv\normalcomma
pcns\_20\_8.csv\normalcomma
pcns\_20\_9.csv\normalcomma
pcns\_20\_10.csv\normalcomma
pcns\_20\_11.csv\normalcomma
pcns\_20\_12.csv\normalcomma
pcns\_20\_13.csv\normalcomma
pcns\_20\_14.csv\normalcomma
pcns\_20\_15.csv\normalcomma
pcns\_20\_16.csv\normalcomma
pcns\_20\_17.csv
\end{tabular}

\begin{tabular}{>{\small\ttfamily}p{\textwidth}}
pcns\_21\_1.csv\normalcomma
pcns\_21\_2.csv\normalcomma
pcns\_21\_3.csv\normalcomma
pcns\_21\_4.csv\normalcomma
pcns\_21\_5.csv\normalcomma
pcns\_21\_6.csv\normalcomma
pcns\_21\_7.csv\normalcomma
pcns\_21\_8.csv\normalcomma
pcns\_21\_9.csv\normalcomma
pcns\_21\_10.csv\normalcomma
pcns\_21\_11.csv\normalcomma
pcns\_21\_12.csv\normalcomma
pcns\_21\_13.csv\normalcomma
pcns\_21\_14.csv\normalcomma
pcns\_21\_15.csv\normalcomma
pcns\_21\_16.csv\normalcomma
pcns\_21\_17.csv
\end{tabular}

\begin{tabular}{>{\small\ttfamily}p{\textwidth}}
pcns\_22\_1.csv\normalcomma
pcns\_22\_2.csv\normalcomma
pcns\_22\_3.csv\normalcomma
pcns\_22\_4.csv\normalcomma
pcns\_22\_5.csv\normalcomma
pcns\_22\_6.csv\normalcomma
pcns\_22\_7.csv\normalcomma
pcns\_22\_8.csv\normalcomma
pcns\_22\_9.csv\normalcomma
pcns\_22\_10.csv\normalcomma
pcns\_22\_11.csv\normalcomma
pcns\_22\_12.csv\normalcomma
pcns\_22\_13.csv\normalcomma
pcns\_22\_14.csv\normalcomma
pcns\_22\_15.csv\normalcomma
pcns\_22\_16.csv\normalcomma
pcns\_22\_17.csv
\end{tabular}

\begin{tabular}{>{\small\ttfamily}p{\textwidth}}
pcns\_24\_1.csv\normalcomma
pcns\_24\_2.csv\normalcomma
pcns\_24\_3.csv\normalcomma
pcns\_24\_4.csv\normalcomma
pcns\_24\_5.csv\normalcomma
pcns\_24\_6.csv\normalcomma
pcns\_24\_7.csv\normalcomma
pcns\_24\_8.csv\normalcomma
pcns\_24\_9.csv\normalcomma
pcns\_24\_10.csv\normalcomma
pcns\_24\_11.csv\normalcomma
pcns\_24\_12.csv\normalcomma
pcns\_24\_13.csv\normalcomma
pcns\_24\_14.csv\normalcomma
pcns\_24\_15.csv\normalcomma
pcns\_24\_16.csv\normalcomma
pcns\_24\_17.csv\normalcomma
pcns\_24\_18.csv
\end{tabular}

\begin{tabular}{>{\small\ttfamily}p{\textwidth}}
pcns\_26\_1.csv\normalcomma
pcns\_26\_2.csv\normalcomma
pcns\_26\_3.csv\normalcomma
pcns\_26\_4.csv\normalcomma
pcns\_26\_5.csv\normalcomma
pcns\_26\_6.csv\normalcomma
pcns\_26\_7.csv\normalcomma
pcns\_26\_8.csv\normalcomma
pcns\_26\_9.csv\normalcomma
pcns\_26\_10.csv\normalcomma
pcns\_26\_11.csv\normalcomma
pcns\_26\_12.csv\normalcomma
pcns\_26\_13.csv\normalcomma
pcns\_26\_14.csv\normalcomma
pcns\_26\_15.csv\normalcomma
pcns\_26\_16.csv\normalcomma
pcns\_26\_17.csv\normalcomma
pcns\_26\_18.csv
\end{tabular}

\begin{tabular}{>{\small\ttfamily}p{\textwidth}}
pcns\_28\_1.csv\normalcomma
pcns\_28\_2.csv\normalcomma
pcns\_28\_3.csv\normalcomma
pcns\_28\_4.csv\normalcomma
pcns\_28\_5.csv\normalcomma
pcns\_28\_6.csv\normalcomma
pcns\_28\_7.csv\normalcomma
pcns\_28\_8.csv\normalcomma
pcns\_28\_9.csv\normalcomma
pcns\_28\_10.csv\normalcomma
pcns\_28\_11.csv\normalcomma
pcns\_28\_12.csv\normalcomma
pcns\_28\_13.csv\normalcomma
pcns\_28\_14.csv\normalcomma
pcns\_28\_15.csv\normalcomma
pcns\_28\_16.csv\normalcomma
pcns\_28\_17.csv\normalcomma
pcns\_28\_18.csv\normalcomma
pcns\_28\_19.csv
\end{tabular}

\begin{tabular}{>{\small\ttfamily}p{\textwidth}}
pcns\_30\_1.csv\normalcomma
pcns\_30\_2.csv\normalcomma
pcns\_30\_3.csv\normalcomma
pcns\_30\_4.csv\normalcomma
pcns\_30\_5.csv\normalcomma
pcns\_30\_6.csv\normalcomma
pcns\_30\_7.csv\normalcomma
pcns\_30\_8.csv\normalcomma
pcns\_30\_9.csv\normalcomma
pcns\_30\_10.csv\normalcomma
pcns\_30\_11.csv\normalcomma
pcns\_30\_12.csv\normalcomma
pcns\_30\_13.csv\normalcomma
pcns\_30\_14.csv\normalcomma
pcns\_30\_15.csv\normalcomma
pcns\_30\_16.csv\normalcomma
pcns\_30\_17.csv\normalcomma
pcns\_30\_18.csv\normalcomma
pcns\_30\_19.csv
\end{tabular}

Ferner ist die Datei

\begin{tabular}{>{\small\ttfamily}p{\textwidth}}
pcns\_range.csv
\end{tabular}

enthalten, in der die jeweils kleinsten (bzgl. \thref{def:polynomordnung})
$\PCN$-Polynome von Grad $n$ über $\F_{p^r}$ 
für $r = 1$ und alle Primzahlen $p<1000$ mit $p^n < 10^{70}$
aufgeführt werden.
Die Syntax der Einträge ist dabei analog zu oben, wobei $n$ ergänzt
wurde, d.h. der Eintrag
\texttt{$p$, $n$, poly} bedeutet, dass
\texttt{poly} das kleinste $\PCN$-Polynom von Grad $n$ über 
$\F_p$ ist.


\subsection{\texttt{./Sage}}

Dieser Ordner beinhaltet die Quellcodes der in 
\autoref{chap:existenz_und_enumeration} vorgestellten Funktionen:

\begin{tabular}{>{\small\ttfamily}p{\textwidth}}
enumeratePCNs.c\\
enumeratePCNs.spyx\\
findAnyPCN\_additional.spyx\\
findAnyPCN\_trinom.spyx
\end{tabular}


\section{Online}
Das gesamte Arbeitsverzeichnis dieser Arbeit ist unter

\url{https://github.com/hackenbergstefan/masterarbeit/}

zugänglich. Die Datenstruktur ist dabei wie folgt gegeben:

\begin{tabular}{>{\ttfamily}lp{10cm}}
  ./CD & Vollständiger Inhalt der angehängten CD wie in 
    Anhang \ref{anh:sec:cd} beschrieben\\
  ./Latex & Ordner mit \LaTeX-Quellcodes\\
  ./Sage & Ordner mit \sage-Quellcodes und Ausgabedateien\\
  ./Tables & Ergebnisse der \sage-Funktionen, identisch mit
    dem Ordner \texttt{Tables} auf angehängter CD (vgl. Anhang
    \ref{anh:sec:cd})\\
  ./Masterarbeit.pdf & Kompilierte Version der Arbeit
\end{tabular}

\vfill
\begin{center}
  \begin{tikzpicture}
    \path[draw=gray] (0,0) rectangle (12.4cm,12.4cm);
  \end{tikzpicture}
\end{center}




%\section{$\PCN$-Polynome}

\newcommand{\insertPCNS}[3]{%
  \subsection{$\PCN$-Polynome für $n=#1$}
  \begin{description}[leftmargin=0pt,labelindent=20pt,font=\normalsize]
    \foreach \r/\max in {#2} {%
      \item[$r=\r$:]
        \foreach \num in {0,...,\max} {%
          \input{./tables/pcns_#1_\r__\num.tex}
        }
    }
    \foreach \r in {#3} {%
      \item[$r=\r$:] \input{./tables/pcns_#1_\r__0.tex}
    }
  \end{description}}



%\tiny
\fontsize{4}{5}\selectfont


\insertPCNS{6}{}{1,...,10}
\insertPCNS{10}{1/1}{2,...,8}
\insertPCNS{12}{1/2}{2,...,7}
\insertPCNS{14}{1/4}{2,...,11}
\insertPCNS{15}{1/5}{2,3}
\insertPCNS{18}{1/10}{2,...,8}
\insertPCNS{20}{1/14}{2,...,5}
\insertPCNS{21}{1/17}{2,...,5}
\insertPCNS{22}{1/20}{2,...,4}
\insertPCNS{24}{1/28}{2,...,4}
\insertPCNS{26}{1/36}{2,...,3}
\insertPCNS{28}{1/50}{2,...,9}





%\chapter{\texttt{Sage}-Quellcodes}
%\lstset{language=python,
  %basicstyle = \footnotesize\normalfont\ttfamily,
  %commentstyle = \itshape\color{gray},
  %caption ={\lstname},
  %frame = tb,
  %framexleftmargin = 0pt,
  %numbers = left,
  %numberstyle = \tiny,
%% numbersep = 5pt,
  %breaklines = true,
  %xleftmargin = 0.1\linewidth,
  %xrightmargin = 0.1\linewidth,
  %showstringspaces=false,
  %columns=fullflexible,
  %tabsize=3}

%\lstinputlisting{../Sage/algorithmen.spyx}
%\lstinputlisting{../Sage/examples/scheerhorn1.sage}
%\lstinputlisting{../Sage/examples/satz1.sage}
%\lstinputlisting{../Sage/examples/satz1_1.sage}
%\lstinputlisting{../Sage/examples/satz2.sage}


%\addchap{Fragen 1}
%\input{fragen1}

%\addchap*{Fragen 2014-12-04}
%\input{fragen2}

\end{document}
