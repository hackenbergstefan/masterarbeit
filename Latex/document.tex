%% Basierend auf einer TeXnicCenter-Vorlage von Mark Müller
%%%%%%%%%%%%%%%%%%%%%%%%%%%%%%%%%%%%%%%%%%%%%%%%%%%%%%%%%%%%%%%%%%%%%%%

% Wählen Sie die Optionen aus, indem Sie % vor der Option entfernen  
% Dokumentation des KOMA-Script-Packets: scrguide

%%%%%%%%%%%%%%%%%%%%%%%%%%%%%%%%%%%%%%%%%%%%%%%%%%%%%%%%%%%%%%%%%%%%%%%
%% Optionen zum Layout des Artikels                                  %%
%%%%%%%%%%%%%%%%%%%%%%%%%%%%%%%%%%%%%%%%%%%%%%%%%%%%%%%%%%%%%%%%%%%%%%%
\documentclass[%
%a5paper,             % alle weiteren Papierformat einstellbar
%landscape,           % Querformat
%10pt,                % Schriftgröße (12pt, 11pt (Standard))
%BCOR1cm,             % Bindekorrektur, bspw. 1 cm
%DIVcalc,             % führt die Satzspiegelberechnung neu aus
%                       s. scrguide 2.4
%twoside,             % Doppelseiten
%twocolumn,           % zweispaltiger Satz
halfparskip*,       % Absatzformatierung s. scrguide 3.1
%headsepline,         % Trennline zum Seitenkopf  
%footsepline,         % Trennline zum Seitenfuß
titlepage,            % Titelei auf eigener Seite
%normalheadings,      % Überschriften etwas kleiner (smallheadings)
%idxtotoc,            % Index im Inhaltsverzeichnis
%liststotoc,          % Abb.- und Tab.verzeichnis im Inhalt
bibtotoc,           % Literaturverzeichnis im Inhalt
%abstracton,          % Überschrift über der Zusammenfassung an 
%leqno,               % Nummerierung von Gleichungen links
%fleqn,               % Ausgabe von Gleichungen linksbündig
%draft                % überlangen Zeilen in Ausgabe gekennzeichnet
DIV = 15,
headsepline,
openany,
BCOR=0.5cm,
pointlessnumbers,        %keine Punkte nach Überschriften
chapterprefix=true
]
{scrbook}
\synctex=1


%% Deutsche Anpassungen %%%%%%%%%%%%%%%%%%%%%%%%%%%%%%%%%%%%%

\usepackage[ngerman]{babel}
\usepackage[T1]{fontenc}
\usepackage[utf8]{inputenc}


\usepackage{lmodern} %Type1-Schriftart für nicht-englische Texte

\usepackage{amsmath,amssymb,MnSymbol}
\usepackage{xcolor}


\usepackage{enumitem}
\setlist[enumerate]{label=(\arabic*)}

\usepackage{array}


%% Packages für Grafiken & Abbildungen %%%%%%%%%%%%%%%%%%%%%%
\usepackage{graphicx} %%Zum Laden von Grafiken
%\usepackage{subfig} %%Teilabbildungen in einer Abbildung
\usepackage{calc}

\usepackage{tikz}
\usetikzlibrary{calc,positioning,backgrounds}
\pgfdeclarelayer{background}
\pgfdeclarelayer{foreground}
\pgfsetlayers{background,main,foreground}
\usepackage{tikzpagenodes}
\usepackage{tikz-cd} %%PSTricks - nicht verwendbar mit pdfLaTeX
\usepackage[colorlinks=false, pdfborder={0 0 0}]{hyperref}
\usepackage[nameinlink,german]{cleveref}

\usepackage{listings}
\usepackage[automark]{scrpage2} % Headline styles 


%% Listings setup %%%%%%%%%%%%%%%%
\lstset{
  mathescape = true,
  basicstyle = \small\normalfont\sffamily,
  frame = tb,
  framexleftmargin = 15pt,
% numbers = left,
  numberstyle = \tiny,
% numbersep = 5pt,
  breaklines = true,
  xleftmargin = 0.1\linewidth,
  xrightmargin = 0.1\linewidth,
  escapeinside = {(*}{*)},
  tabsize=3,
  morekeywords={if, and, or, is, then, else, endif, while, endwhile, for, from,
  to, do, endfor, Input, Output, Algorithmus, return},
  morecomment=[l]{//},
  columns=flexible
}

%%%%%%%%%%%%%%%%%%%%%%%%%%%%%%%%%%%%%%%%%%%%%%%%%%%%%%%%%%%%%%%%%%%%%%%%%%%%%%%
%% Style Anpassungen
%%%%%%%%%%%%%%%%%%%%%%%%%%%%%%%%%%%%%%%%%%%%%%%%%%%%%%%%%%%%%%%%%%%%%%%%%%%%%%%
%\usepackage[sc]{mathpazo}
%\renewcommand{\sfdefault}{fav}
%\setkomafont{disposition}{\sffamily}
%%%%%%%%%%%%%%%%%%%%%%%%%%%%%%%%%%%%%%%%%%%%%%%%%%%%%%%%%%%%%%%%%%%%%%%%%%%%%%%
\colorlet{mycol}{red!50!black}

%% Theorems %%%%%%%%%%%%%%%%%%%%%%
\usepackage{tikztheorems}
\newtikztheorem[
  style=elegantbreak,
  color=mycol,
  font header=\normalfont\sffamily\bfseries,
  counter zero=chapter,
  postskip=20pt
  ]{satz}{Satz}
  
\newtikztheorem[
  style=elegantbreak,
  color=mycol,
  font header=\normalfont\sffamily\bfseries,
  font body=\normalfont,
  counter parent=satz,
  postskip=20pt
  ]{definition}{Definition}

\theoremstyle{plain}
\theorembodyfont{\itshape}
\newtheorem{lemma}[satz]{Lemma}
\newtheorem{kor}[satz]{Korollar}
\newtheorem{prop}[satz]{Proposition}
\newtheorem{algorithmus}[satz]{Algorithmus}

%Custom theorems
\makeatletter
\newenvironment{plainthm}[1]{\let\plthm\@undefined
\newtheorem{plthm}[satz]{#1} \begin{plthm}}{\end{plthm}}
\makeatother

\theoremstyle{plain}
\theoremheaderfont{\normalfont\sffamily\itshape}
\theorembodyfont{\normalfont}
\newtheorem{bemerkung}[satz]{Bemerkung}
\newtheorem{beispiel}[satz]{Beispiel}
                                     
\theoremstyle{plain}
\theoremheaderfont{\normalfont\sffamily\bfseries}
\theorembodyfont{\normalfont}                

\newtheorem{notation}[satz]{Notation}



%% Autoref Names %%%%%%%%%%%%%%%%%
\crefname{lemma}{Lemma}{Lemmas}
\crefname{equation}{Gleichung}{Gleichungen}
\crefname{definition}{Definition}{Definitionen}
\crefname{algorithmus}{Algorithmus}{Algorithmen}
\crefname{kor}{Korollar}{Korollare}
\crefname{satz}{Satz}{Sätze}

%% Amsmath options %%%%%%%%%%%%%%%%%
\numberwithin{equation}{chapter}
\allowdisplaybreaks

%% Pagestyle %%%%%%%%%%%%%%%%%%%%%%%%%%
\usepackage{scrpage2}
\addtokomafont{pagenumber}{\sffamily\color{gray!50!black}}
\addtokomafont{pagehead}{\sffamily\upshape\color{gray!50!black}}
%\setheadwidth[0pt]{textwithmarginpar}}
\pagestyle{scrheadings}
\clearscrheadfoot
\setheadsepline{0pt}
\chead{\headmark}
\ohead[\myheadcmd]{\myheadcmd}
\ofoot{}

\def\myheadcmd{\tikz[remember picture]{
  \node[outer xsep=1pt, outer ysep=3pt, inner ysep=0pt,
    inner xsep=5pt, font=\sffamily] (a) {\pagemark};}%
    \tikz[remember picture, overlay]{
    \ifthenelse{\isodd{\thepage}}{
      \draw[gray,line width=1pt]
        ($(a.north west)+(0pt,0)$) |- ($(a.south east)+(-3pt,-3pt)$);
      \draw[gray] ($(a.south west)+(0pt,0pt)$)  -- 
        (a.south west -| current page text area.north west);
    }{%
      \draw[gray,line width=1pt]
        ($(a.north east)+(0pt,0)$) |- ($(a.south west)+(3pt,-3pt)$);
      \draw[gray] ($(a.south east)+(0pt,0pt)$)-- 
        (a.south east -| current page text area.north east);
    }%
  }%
}

\def\myheadcmdempty{\tikz[remember picture]{\node[outer sep=4pt,inner sep=0pt] (a) {\sffamily\pagemark};}
    \tikz[remember picture, overlay]{
    \ifthenelse{\isodd{\thepage}}{
      \draw[gray,line width=1pt]
        ($(a.north west)+(0pt,0)$) |- ($(a.south east)+(0,-3pt)$);
    }{
      \draw[gray,line width=1pt]
        ($(a.north east)+(0pt,0)$) |- ($(a.south west)+(0pt,-3pt)$);
    }
  }
}

%% titlesec %%%%%%%%%%%%%%%%%%%%%%%%%%%%%%%%%%%%%%%%%%%%%%%%%
\usepackage{titlesec}
  
\titleformat{\chapter}[display]%
  {\normalfont\sffamily\huge\bfseries\color{mycol!95}}%
  {\color{gray}\chaptertitlename~\thechapter}{0ex}{}

\titleformat{\section}[hang]%
  {\normalfont\sffamily\huge\bfseries\color{mycol!95}}%
  {\color{gray}\thesection\hspace{1ex}\raisebox{-0.1\baselineskip}{\rule{1pt}{.8\baselineskip}}}{1ex}{}

\titleformat{\subsection}[hang]%
  {\normalfont\sffamily\Large\bfseries\color{mycol!95}}%
  {\color{gray}\thesubsection\hspace{1ex}\raisebox{-0.1\baselineskip}{\rule{0.5pt}{.8\baselineskip}}}{1ex}{}

  
\titleformat{\subsubsection}[hang]%
  {\normalfont\sffamily\large\bfseries\color{mycol!95}}%
  {\color{gray}\thesubsubsection\hspace{1ex}\raisebox{-0.1\baselineskip}{\rule{0.5pt}{.8\baselineskip}}}{1ex}{}


%% Makros %%%%%%%%%%%%%%%%%%%%%%
\newcommand{\A}{\ensuremath \mathbb{A}}
\newcommand{\R}{\ensuremath \mathcal{R}}
\newcommand{\N}{\ensuremath \mathbb{N}}
\newcommand{\Q}{\ensuremath \mathbb{Q}}
\newcommand{\Z}{\ensuremath \mathbb{Z}}
\newcommand{\C}{\ensuremath \mathbb{C}}
\newcommand{\F}{\ensuremath \mathbb{F}}
\newcommand{\K}{\ensuremath \mathbb{K}}
\renewcommand{\P}{\ensuremath \mathbb{P}}
\newcommand{\Kb}{\ensuremath \overline K}
\renewcommand{\O}{\ensuremath \mathcal{O}}
\renewcommand{\L}{\ensuremath \mathcal{L}}
\renewcommand{\l}{\ensuremath \ell}
\renewcommand{\S}{\ensuremath\mathcal{S}}
\newcommand{\m}{\ensuremath \mathfrak{m}}
\newcommand{\speq}[1]{\ #1\ }
\newcommand{\const}{\ensuremath \mathrm{const}}
\newcommand{\divp}[1]{\ensuremath [#1]} %Divisorpunkt
\newcommand{\probn}[1]{{\sffamily #1}} %Problem Name
\newcommand{\obda}{\small oBdA\space}
\newcommand{\inv}{^{-1}}
\newcommand{\kl}[1]{^{(#1)}}

\let\div\undefined
\DeclareMathOperator{\charak}{char}
\DeclareMathOperator{\div}{div}
\DeclareMathOperator{\ord}{ord}
\DeclareMathOperator{\summ}{sum}
\DeclareMathOperator{\comp}{\circ}
\DeclareMathOperator{\ggT}{ggT}
\DeclareMathOperator{\kgV}{kgV}
\DeclareMathOperator{\supp}{supp}
\DeclareMathOperator{\Div0}{Div^0}
\DeclareMathOperator{\Divv}{Div}
\DeclareMathOperator{\Pic0}{Pic^0}
\DeclareMathOperator{\Gal}{Gal}
\DeclareMathOperator{\End}{End}
\DeclareMathOperator{\Ord}{Ord}
\DeclareMathOperator{\im}{im}
\DeclareMathOperator{\id}{id}
\DeclareMathOperator{\Tr}{Tr}
\DeclareMathOperator{\spann}{span}


\newcommand{\funcdef}[1]{%
  \begin{array}[t]{>{\displaystyle}r>{\displaystyle}c>{\displaystyle}l}%
  #1\end{array}}


\let\grqqnospace\grqq
\renewcommand{\grqq}{\grqqnospace\space}

%% Others  %%%%%%%%%%%%%%%%%%%%%%%
\newcommand{\?}{{\huge \color{red} ?}}
\newcommand{\TODO}{{\sffamily\bfseries\large \color{red} TODO}}

\newcommand{\overbox}[2]{\ensuremath\begin{array}[b]{c}%
\makebox[0pt]{\fbox{\scriptsize#2}}\\[-2pt]\text{\small$\downarrow$}\\[-3pt]%
{\displaystyle#1}\end{array}}%

\let\marginparold\marginpar
\renewcommand{\marginpar}[1]{%
  \marginparold[\raggedleft\scriptsize\sffamily #1]%
  {\raggedright\scriptsize\sffamily #1}}


%% Bibliographiestil %%%%%%%%%%%%%%%%%%%%%%%%%%%%%%%%%%%%%%%%%%%%%%%%%%
\usepackage[german=guillemets]{csquotes}
\usepackage[style=numeric,backend=biber,
  isbn=false,
  firstinits=true
  ]{biblatex}
\addbibresource{bib.bib}




\begin{document}

%% Trennungen %%%%%%%%%%%%%%%%%%%%%%%%%%%%%%%%%%%%%%%%%%%%%%%%%%%%%%%%%
%%%%%%%%%%%%%%%%%%%%%%%%%%%%%%%%%%%%%%%%%%%%%%%%%%%%%%%%%%%%%%%%%%%%%%%

\frontmatter


%%%%%%%%%%%%%%%%%%%%%%%%%%%%%%%%%%%%%%%%%%%%%%%%%%%%%%%%%%%%%%%%%%%%%%%
%% Ihr Artikel                                                       %%
%%%%%%%%%%%%%%%%%%%%%%%%%%%%%%%%%%%%%%%%%%%%%%%%%%%%%%%%%%%%%%%%%%%%%%%

%% eigene Titelseitengestaltung %%%%%%%%%%%%%%%%%%%%%%%%%%%%%%%%%%%%%%%    
\begin{titlepage}
\thispagestyle{empty}
\newcommand{\Rule}{\rule{\textwidth}{1mm}}
\begin{center}\sffamily\bfseries
\LARGE\textcolor{gray}{Masterarbeit}
\par\vspace*{2cm}
\tikz[baseline]{ \node[anchor=base, minimum width=\textwidth,
  text=mycol,
  inner xsep=5pt,
  inner ysep=10pt,
  align=center,
  font=\Huge]
  (main title) {Über explizite Konstruktionen\\ von Normalbasen\\
  über endlichen Körpern};
  \draw[overlay, line width=1mm, gray,
    line cap=round]
    (main title.south west)
    ++(0,-10pt) -- +(\textwidth,0)
    (main title.north west)
    ++(0,10pt) -- +(\textwidth,0);
}
\vfill
\normalfont\sffamily\large vorgelegt von\par
\bfseries\LARGE Stefan Hackenberg
\vfill
\normalfont\sffamily\large am\\
\bfseries\Large Institut für Mathematik\\
\normalfont\sffamily\large der\\
\bfseries\Large Universität Augsburg
\vfill
\normalfont\sffamily\large betreut durch \\
\bfseries\Large Prof. Dr. Dirk Hachenberger\par
\vfill
\normalfont\sffamily\large Stand\\
\bfseries\Large \today
\end{center}
\end{titlepage}


%% Angaben zur Standardformatierung des Titels %%%%%%%%%%%%%%%%%%%%%%%%
%\titlehead{Titelkopf }
\subject{\large Masterarbeit}
\title{\Huge }
%\subtitle{Grundlegende Resultate zu elliptischen Kurven, Konstruktionen und
%Eigenschaften der Weil-Paarung, ein
%algorithmischer Überblick zu elliptischen Kurven in der Kryptographie und die
%kryptographische Anwendung der Weil-Paarung mit Hilfe des MOV-Algorithmus}
\author{\vspace*{2cm}\\\normalsize von\\\Large Stefan Hackenberg}
%\and{Der Name des Co-Autoren}
%\thanks{Fußnote}     % entspr. \footnote im Fließtext
%\date{}              % falls anderes, als das aktuelle gewünscht
\publishers{{\small geschrieben an der} \\ Universität Augsburg}

%% Widmungsseite %%%%%%%%%%%%%%%%%%%%%%%%%%%%%%%%%%%%%%%%%%%%%%%%%%%%%%
%\dedication{Für Sandra}

%\maketitle             % Titelei wird erzeugt

%% Zusammenfassung nach Titel, vor Inhaltsverzeichnis %%%%%%%%%%%%%%%%%
%\begin{abstract}
% Für eine kurze Zusammenfassung des folgenden Artikels.
% Für die Überschrift s. \documentclass[abstracton].
%\end{abstract}


\cleardoubleemptypage
\tableofcontents


%% Der Text %%%%%%%%%%%%%%%%%%%%%%%%%%%%%%%%%%%%%%%%%%%%%%%%%%%%%%%%%%%

%\include{intro.tex}
\mainmatter
\chapter{Grundbegriffe}
\label{chap:grundbegriffe}

Wir eröffnen den Hauptteil der Arbeit mit dem Zusammentragen
einiger grundlegender Resultate, die dem Leser sicherlich bekannt sind. 
Daher werden wir die meisten Aussagen 
lediglich ohne Beweis zitieren. Wir beginnen dabei bei der Gruppentheorie und
insbesondere zyklische Gruppen. Diese werden uns später
helfen, die Untergruppe der (primitiven) Einheitswurzeln
in \autoref{chap:kreisteilungspolynome} zu verstehen. Im anschließenden Abschnitt
rekapitulieren wir ein wenig die Galoistheorie von endlichen Körpern.
Insbesondere wollen wir wiederholen, dass die 
Galoisgruppe endlicher Körper zyklisch ist und von einem
speziellen Automorphismus erzeugt wird.

\section{Ein wenig Gruppentheorie}

%Um später den Zerfall der Kreisteilungspolynome über endlichen Körpern zu
%verstehen, wiederholen wir zunächst ein paar Aussagen über zyklische Gruppen.

\autocite[Theorem 1.15]{lidl1997finite} fasst alle notwendigen Resultate
zusammen.

\begin{satz}
  \label{satz:zykl_gruppen}
  \begin{enumerate}
    \item Jede Untergruppe einer zyklischen Gruppe ist wieder zyklisch.
    \item Sei $\langle a \rangle$ eine zyklische Gruppe der Ordnung $m$,
      so erzeugt $a^k$ eine Untergruppe der Ordnung $\frac{m}{\ggT(m,k)}$.
    \item Sei $\langle a\rangle$ eine zyklische Gruppe der Ordnung $m$ und
      $d \mid m$, so enthält $\langle a \rangle$ genau eine Untergruppe der
      Ordnung $d$.
     \item Sei $f$ ein positiver Teiler der Gruppenordnung einer endlichen
        zyklischen Gruppe $\langle a \rangle$. Dann enthält $\langle a \rangle$
        genau $\varphi(f)$ Elemente der Ordnung $f$.
        ($\varphi$ bezeichne die Eulersche Phifunktion)
     \item Eine zyklische Gruppe der Ordnung $m$ enthält genau $\varphi(m)$
        Erzeuger. Ist $a$ ein Erzeuger, so sind alle Erzeuger der Form
        $a^r$ mit $\ggT(r,m) = 1$.
  \end{enumerate}
\end{satz}

Da wir später ein paar Eigenschaften benötigen
werden, wiederholen wir die wohlbekannte Definition der Eulerschen
Phifunktion und geben wir dann die wichtigsten Rechenregeln an.

\begin{definition}[Eulersche Phifunktion]
  %\nomenclature{$\varphi$}{Eulersche Phifunktion}
  Die Funktion
  \[ \varphi: \funcdef{\N^\ast &\to& \N^\ast,\\
    n &\mapsto& |\{ a \in \N:\ 1\leq a\leq n,\ \ggT(a,n)=1 \}|}\]
  heißt \emph{Eulersche $\varphi$-Funktion}.
\end{definition}

\begin{definition}[quadratfreier Teil]
  Sei $n \in \N$ und $n = p_1^{r_1}\cdot\ldots\cdot p_l^{r_l}$ seine
  Primfaktorzerlegung. Dann heißt
  \[ \nu(n) \speq{:=} p_1\cdot \ldots\cdot p_l\]
  \emph{quadratfreier Teil von $n$}.
\end{definition}


\begin{lemma}[Rechenregeln der Eulerschen Phifunktion]
  \label{lemma:rechenregeln_phifunktion}
  Seien $a,b\in\N^\ast$, so gilt
  \begin{enumerate}
    \item $\varphi(ab) = \varphi(a)\varphi(b)$, falls $\ggT(a,b) = 1$,
    \item $a = \sum_{d\mid a} \varphi(d)$ und
    \item $\varphi(a) = \tfrac{a}{\nu(a)}\varphi(\nu(a))$.
  \end{enumerate}
\end{lemma}


Zyklische Gruppen und endliche Körper hängen eng zusammen, da bekanntlich die
multiplikative Gruppe eines endlichen Körpers immer zyklisch ist. Dies können
wir nutzen, um Erzeugern (im Sinne der Gruppentheorie) der multiplikativen
Gruppe eines endlichen Körpers einen Namen zu geben.

\begin{satz}
  \label{satz:mult_gruppe_endl_korper_zyklisch}
  Die multiplikative Gruppe eines endlichen Körpers ist zyklisch.
\end{satz}
\begin{proof}
  \autocite[Theorem 2.8]{lidl1997finite}.
\end{proof}


\begin{definition}[primitiv]
  \label{def:primitiv}
  Sei $\F_q$ ein endlicher Körper. $u\in \F_q$ heißt \emph{primitiv} 
  (oder \emph{primitives Element}), falls $\langle u \rangle = \F_q^\ast$, 
  also $u$ ein Erzeuger der multiplikativen Gruppe $\F_q^\ast$ ist.
\end{definition}


\begin{bemerkung}
  Es ist klar, dass $u\in \F_q$ genau dann primitiv ist, wenn 
  $\ord(u) = q-1$, also seine gruppentheoretische Ordnung in $\F_q^\ast$ genau
  der Gruppenordnung entspricht.
\end{bemerkung}


\section{Automorphismen über endlichen Körpern}

\begin{satz}
  \label{satz:frob_auto}
  Seien $F = \F_q$ ein endlicher Körper der Charakteristik $p\neq 0$ und 
  $n\in \N^\ast$. Dann ist
  \[ \sigma_n: \funcdef{F &\to& F\\
    a &\mapsto& a^{p^n}}\]
  ein Automorphismus auf $F$.
\end{satz}
\begin{proof}
  \autocite[Corollary 3.18]{wan2003lectures}.
\end{proof}

\begin{bemerkung}
  Insbesondere gilt also für alle $a,b\in F$, $F$ wie oben:
  \[ (a\pm b)^{p^n} = a^{p^n} \pm b^{p^n}\,.\]
\end{bemerkung}

\begin{satz}
  \label{satz:frob_fix}
  Sei $q$ eine Primzahlpotenz und $n\in \N^\ast$. Der Automorphismus
  \[ \sigma: \funcdef{\F_{q^n} &\to& \F_{q^n}\\
    a &\mapsto& a^q}\]
  hält die Elemente von $\F_q$ fest, also 
  \[ \sigma |_{\F_q} = \id_{\F_q}\,.\]
  Ferner ist $\sigma^k \neq \id_{\F_{q^n}}$ für $k=1,\ldots,n-1$, alle
  $\sigma^k$s sind paarweise verschiedene Automorphismen und 
  $\sigma^n = \id_{\F_{q^n}}$.
  $\sigma$ heißt auch \emph{Frobenius-Endomorphismus} oder
  \emph{Frobenius-Automorphismus}.
\end{satz}


Bezeichne $\Gal(E \mid F)$ die Galoisgruppe einer Galoiserweiterung $E$ über
$F$, so können wir das folgende zentrale Resultat zitieren:

\begin{satz}
  \label{satz:frob_sind_alle_autos}
  Es gilt
  \[ \Gal(\F_{q^n}\mid \F_q) \speq= \langle \sigma\rangle\,.\]
  Das bedeutet, dass es neben $\sigma^0,\sigma,\ldots,\sigma^{n-1}$ keine weiteren
  Automorphismen von $\F_{q^n}$ gibt, die $\F_q$ fixieren.
\end{satz}
\begin{proof}
  \autocite[Theorem 7.3]{wan2003lectures}.
\end{proof}

Neben der Tatsache, dass der Frobenius-Automorphismus 
alle Elemente der Galoisgruppe erzeugt, können wir
auch zeigen, dass alle Potenzen des Frobenius von $\F_q^n$ über $\F_q$ linear
unabhängig sind. Dies gilt sogar in einem größeren Kontext:

\begin{satz}[Dedekindsches Lemma]
  \label{satz:dedekindsches_lemma}
  Seinen $K,L$ zwei Körper, $n \in \N$ und $\tau_1,\ldots,\tau_n: K\to L$
  verschiedene injektive Körperhomorphismen. Dann ist für jedes $x \in K$
  \[ \{\tau_1(x),\ldots,\tau_n(x) \}\]
  linear unabhängig über $L$.
\end{satz}
\begin{proof}
  \autocite[Satz 27.2]{karpfinger2010algebra}.
\end{proof}


Mit \thref{satz:frob_sind_alle_autos} wird klar, dass für ein
irreduzibles Polynom $f(x) \in \F_q[x]$, das in $\F_{q^n}$ eine Nullstelle 
$\alpha$ besitzt, auch $\sigma^i(\alpha)$ für alle $i=1,\ldots,n-1$
Nullstellen sind. Ferner kann man sich auch relativ leicht überlegen, dass auch
jedes Polynom $f(x) \in \F_q[x]$ vom Grad $n$ eine Nullstelle in 
$\F_{q^n}$ besitzt. Beides fasst nachstehender Satz zusammen.

\begin{satz}
  \label{satz:nst_irred_polys}
  Ist $f(x) \in \F_q[x]$ ein irreduzibles Polynom vom Grad $n$. Dann 
  existiert eine Nullstelle $\alpha$ von $f(x)$ in $\F_{q^n}$, alle 
  Nullstellen von $f(x)$ sind einfach und gegeben durch
  \[ \alpha, \alpha^q, \alpha^{q^2}, \ldots, \alpha^{q^{n-1}}\ \in \F_{q^n}\,.\]
\end{satz}
\begin{proof}
  \autocite[Theorem 2.14]{lidl1997finite}.
\end{proof}


\chapter{Der Zerfall von $x^n-1$ und die Kreisteilungspolynome}
\label{chap:kreisteilungspolynome}

Sei $K$ ein beliebiger Körper der Charakteristik $p$ 
und $\bar K$ ein fest gewählter algebraischer
Abschluss. Wir wollen nun untersuchen, wie das Polynom
$x^n-1 \in K[x]$ über $K$ zerfällt. Dazu orientieren wir uns 
an \autocite{lidl1997finite} und \autocite{wan2003lectures}.

\begin{definition}[Kreisteilungskörper, Einheitswurzeln]
  \label{def:kreisteilungskorper}
  Sei $n\in\N^\ast$. Der Zerfällungskörper von $x^n-1 \in K[x]$ heißt
  der \emph{$n$-te Kreisteilungskörper} und wird mit $K\kl n$ notiert.
  Die Nullstellen von $x^n-1$ in $K\kl n$ heißen \emph{$n$-te
  Einheitswurzeln} und die Menge derer wird mit $U\kl n$ bezeichnet.
\end{definition}


\begin{satz}
  Sei $n\in \N^\ast$.
  \begin{enumerate}
    \item Sei $p\nmid n$. Dann ist $U\kl n$ eine zyklische Gruppe (bzgl. der
      Multiplikation in $K\kl n$) der Ordnung $n$.
    \item Ist $p \mid n$ und schreibt man $n = p^e m$ 
      für positive ganze Zahlen $m$ und $e$ mit $p\nmid m$, so
      ist $K\kl n = K\kl m$ und $U\kl n= U\kl m$ und die Nullstellen von
      $x^n-1 \in K[x]$ sind gerade die Elemente in $U\kl m$ jedoch jeweils mit
      Multiplizität $p^e$.
  \end{enumerate}
\end{satz}
\begin{proof}
  \autocite[Theorem 2.42]{lidl1997finite}.
\end{proof}


\begin{definition}[primitive Einheitswurzeln]
  \label{def:primitive_einheitswurzeln}
  Sei $n\in \N^\ast$ und $p\nmid n$. Dann heißen die Erzeuger von 
  $U\kl n$ \emph{primitive $n$-te Einheitswurzeln}. 
  Die Menge der primitiven $n$-ten Einheitswurzeln wird mit
  $C\kl n$ bezeichnet.
\end{definition}


\begin{definition}[Kreisteilungspolynom]
  \label{def:kreisteilungspolynom}
  Seien $n \in \N^\ast$, $p\nmid n$. Das Polynom 
  \[ \Phi_n(x) \speq{:=} \prod_{\zeta \in C\kl n} (x - \zeta)\ \in 
    K\kl n[x]\]
  heißt \emph{$n$-tes Kreisteilungspolynom}.
\end{definition}

\begin{satz}
  \label{satz:zerfall_xn_1}
  Seien $K$ ein Körper der Charakteristik $p$ und $n \in \N^\ast$ mit $p\nmid
  n$. Dann gilt:
  \begin{enumerate}
    \item $x^n-1 = \prod_{d\mid n} \Phi_d(x)$.
    \item $\Phi_n(x) \in P[x]$, wobei $P$ den Primkörper von $K$ notiere.
  \end{enumerate}
\end{satz}
\begin{proof}
  \begin{enumerate}
    \item Dies ist eine einfache Folgerung aus \thref{satz:zykl_gruppen}.
    \item Lässt sich per Induktion recht einfach beweisen 
      (vgl. \autocite[Theorem 2.45 (ii)]{lidl1997finite}).
  \end{enumerate}
\end{proof}


\begin{definition}
  \label{def:multiplikative_ordnung_mod}
  Für zwei teilerfremde natürliche Zahlen $q,n$ größer Null sei
  \[ \ord_n(q) \speq{:=} \ord([q]_n)\]
  die \emph{multiplikative Ordnung von $q$ modulo $n$},
  wobei $[q]_n$ die Restklasse von $q$ in $\Z_n$ bezeichnet und 
  die Ordnung in der Einheitengruppe von $\Z_n$, notiert 
  durch $\Z_n^\times$, zu lesen ist.
\end{definition}

\begin{lemma}[Rechenregeln der multiplikation Ordnung modulo $n$]
  \label{lemma:rechenregeln_ordn}
  Seien $m,n,q \in \N^\ast$ mit \newline $\ggT(n,q)=1$,
  $\ggT(m,q) = 1$ und $\ggT(m,n) = 1$, so gilt
  \begin{enumerate}
    \item $\ord_n(q) \mid \varphi(n)$,
    \item $\ord_{mn}(q) = \kgV\{ \ord_m(q), \ord_n(q)\}$.
  \end{enumerate}
\end{lemma}
\begin{proof}
  \begin{enumerate}
    \item Klar, da $[q]_n$ in $\Z_n^\times$ eine Untergruppe der Ordnung
      $\ord_n(q)$ erzeugt. 
      Nach dem Satz von Lagrange teilt deren Ordnung die 
      Gruppenordnung $|\Z_n^\times| = \varphi(n)$.
    \item Nach dem Chinesischen Restsatz 
      (z.B. \autocite[Kapitel 2 Satz 12]{bosch2009algebra}) ist
      \[ f: \funcdef{ \Z_{nm} &\xrightarrow{\cong}& \Z_n \times \Z_m\,,\\{} 
          [x]_{nm} &\mapsto& ([x]_n, [x]_m)}\]
      ein Isomorphismus von Ringen, 
      da algebraisch $\Z_n$ ja nichts anderes ist, als
      $\Z\big/(n)$, wobei $(n)$ das von $n$ im Ring $\Z$ erzeugte Ideal meint.
      Dieser liefert einen Gruppenhomomorphismus auf
      den Einheiten:
      \[ f: \Z_{nm}^\times \to \Z_n^\times \times \Z_m^\times\,.\]
      Nun ist per definitionem von $\ord_{\bullet}(q)$ die Behauptung
      klar.
  \end{enumerate}
\end{proof}

Damit können wir nun zu einem zentralen Resultat dieses Abschnittes kommen, das
uns über die gesamte Arbeit hinweg begleiten wird.

\begin{satz}
  \label{satz:zerfall_kreisteilungspolys}
  Seien $q$ eine Primzahlpotenz und $n\in \N^\ast$ mit $\ggT(q,n)=1$. Dann
  zerfällt das $n$-te Kreisteilungspolynom $\Phi_n(x)$ über $\F_q$ in
  \[ \frac{\varphi(n)}{\ord_n(q)}\]
  irreduzible paarweise teilerfremde Polynome von jeweils Grad $\ord_n(q)$.
\end{satz}
\begin{proof}
  Sei $f(x) \mid \Phi_n(x)$ ein irreduzibler Teiler über $\F_q$. Ist dann
  $\zeta \in C\kl n$ eine Nullstelle von $f(x)$, so sind 
  nach \thref{satz:nst_irred_polys} auch
  \[ \zeta^q, \zeta^{q^2}, \ldots, \zeta^{q^{n-1}} \]
  Nullstellen von $f(x)$. Jedoch sind offenbar nur $\ord_n(q)$ dieser 
  verschieden und da $f$ als irreduzibles Polynom wieder nach 
  \thref{satz:nst_irred_polys} nur einfache Nullstellen besitzt,
  können wir folgern, dass $\deg f = \ord_n(q)$.
  Da $f(x)$ als beliebiger irreduzibler Teiler von $\Phi_n(x)$ gewählt wurde,
  folgt sofort die Behauptung, wenn man sich überlegt, dass der Grad des
  $n$-ten Kreisteilungspolynoms per Definition gerade $\varphi(n)$ ist.
\end{proof}


Im Beweis obigen Satzes haben wir gesehen, dass die Wirkung der Galoisgruppe
$\Gal(\F_{q^n}\mid \F_q)$ auf der Menge der primitiven $n$-ten Einheitswurzeln
$C\kl n$ (die Wirkung ist selbstredend durch Einsetzen gegeben) diese in 
Teilmengen der Mächtigkeit $\ord_n(q)$ zerlegt.
Dies lässt sich natürlich auf $U\kl n$ übertragen, da ja gerade 
nach \thref{satz:zykl_gruppen}
$U\kl n = \bigcupdot_{d\mid n} C\kl d$. 
Dies motiviert nachstehende Definition.


%\begin{lemma}
  %\label{lemma:uber-pi-m-1}
  
%\end{lemma}

%\begin{lemma}
  %\label{lemma:uber-pi-m-2}
  %Für natürliche Zahlen $m$, $k$ gilt
  %\[ k \speq= \cl_m(k)\cdot l \]
  %mit $\ggT(m \cl_m(k), l) = 1$.
%\end{lemma}
%\begin{proof}
  %Dies sieht man sehr leicht, wenn man sich die Primfaktorzerlegungen der
  %gegebenen Zahlen zu Gemüte führt: Sei also 
  %\begin{align*}
    %m \quad&=\quad p_1^{\nu_1} \cdot\ldots\cdot p_s^{\nu_s}\ \cdot\
      %r_1^{\eta_1}\cdot\ldots\cdot r_t^{\eta_t}\,,\\
    %k \quad&=\quad p_1^{\nu'_1} \cdot\ldots\cdot p_s^{\nu'_s}
      %\ \cdot\ {r'}_1^{{\eta'}_1} \cdot\ldots\cdot {r'}_{t'}^{{\eta'}_{t'}}\,,
  %\end{align*}
  %für geeignete $\nu_\cdot,\nu'_\cdot,\eta_\cdot,\eta'_\cdot > 0$, wobei
  %natürlich $\ggT(r_1\cdot\ldots\cdot r_t, r'_1\cdot\ldots\cdot r'_{t'}) = 1$.
  %Dann ist $\cl_m(k)$ der größte Teiler von $k$, dessen quadratfreier Teil, den
  %quadratfreien Teil von $m$ teilt, also
  %\[ \cl_m(k) \speq= p_1^{\nu'_1}\cdot\ldots\cdot p_s^{\nu'_s}\,.\]
  %Damit ist $l = {r'}_1^{{\eta'}_1}\cdot\ldots\cdot {r'}_{t'}^{{\eta'}_{t'}}$ 
  %und die Behauptung folgt sofort.
%\end{proof} 

\begin{definition}
  \label{def:nebenklassen_mod_m}
  Für $m,q\in\N$ mit $\ggT(m,q) = 1$ und $j \in \{0,\ldots,m-1\}$ definieren wir
  \[ M_q(j\bmod m) \speq{:=} \{ j\,q^i \bmod m:\ i\in\N\} \speq= 
    \{j,\ jq,\ jq^2,\ jq^3,\ldots\ \bmod m\}\,.\]
  Ein vollständiges Repräsentantensystem von Nebenklassen 
  der Untergruppe $M_q(1\bmod m)$
  in $\Z_m$ sei mit $R_q(m)$ bezeichnet. Für $l=1,\ldots,m-1$
  bezeichne ferner
  $r_q(l \bmod m) := |\{lq^i:\ i\in \N \}|$ die Länge der zugehörigen Bahn.
\end{definition}

\begin{bemerkung}
  Per Definition von $\ord_m(q)$ ist für $l\neq 0$
  \[ r_q(l\bmod m) = \ord_{\frac{m}{\ggT(m,l)}}(q)\,. \]
\end{bemerkung}


\begin{beispiel}
  \label{beispiel:zerfall_x21_1_1}
  Wollen wir den Zerfall von $x^{21}-1$ über $\F_2$ untersuchen, so berechnen
  wir erst ein Vertretersystem von Restklassen modulo 21:
  \[\begin{array}[t]{r|l}
    l \in R_2(21) & M_2(l \bmod{21}) \\\hline
    0 & 0 \\
    1 & 1, 2, 4, 8, 11, 16 \\
    3 & 3, 6, 12 \\
    5 & 5, 10, 13, 17, 19, 20 \\
    7 & 7, 14 \\
    9 & 9, 15, 18
  \end{array}\]
  Nun wissen wir aus \thref{satz:zerfall_xn_1}, dass 
  \[ x^{21} -1 = \Phi_1(x) \cdot \Phi_3(x) \cdot \Phi_7(x) \cdot
  \Phi_{21}(x)\,.\]
  Die Nullstellen von $\Phi_{21}(x)$ partitionieren sich gerade in
  diejenigen $M_2(l\bmod{21})$ für die $l=1,5$.
  Also haben wir 
  \[\footnotesize\setlength{\arraycolsep}{2pt}\everymath{\displaystyle}
    \begin{array}{rcccc} 
      \Phi_{21}(x) &=& (x^{6} + x^{4} + x^{2} + x + 1) 
        &\cdot& (x^{6} + x^{5} + x^{4} + x^{2} + 1) \\
      &=& (x-\zeta)(x-\zeta^2)(x-\zeta^4)(x-\zeta^8)(x-\zeta^{11})
        (x-\zeta^{16}) &\cdot&
        (x-\zeta^5)(x-\zeta^{10})(x-\zeta^{13})(x-\zeta^{17})
        (x-\zeta^{19})(x-\zeta^{20})\,
    \end{array}\]
  falls wir $\zeta \in C\kl{21}$ als Nullstelle von
  $x^6+x^4+x^2+x+1$ setzen.
  Analog erhalten wir den Zerfall von $\Phi_7(x)$ durch Betrachtung der
  $M_2(l\bmod{21})$ für $l=3,9$. 
  \[\footnotesize\setlength{\arraycolsep}{2pt}\everymath{\displaystyle}
    \begin{array}{rcccc} 
      \Phi_7(x)&=& (x^{3} + x + 1) &\cdot& (x^{3} + x^{2} + 1) \\
      &=& (x-\zeta^3)(x-\zeta^{3\cdot 2})(x-\zeta^{3\cdot 4}) &\cdot&
        (x-\zeta^{3\cdot 3})(x-\zeta^{3\cdot 5})(x-\zeta^{3\cdot 6}) 
    \end{array}\]
  Sammeln wir den Rest auf, erhalten wir die Partitionierung für
  $\Phi_3(x)$ und den trivialen Fall $\Phi_1(x)$.
  \[ \begin{array}{rcc} 
    \Phi_3(x) &=& x^2 + x + 1\\
              &=& (x-\zeta^7)(x-\zeta^{14})\,,\\[10pt]
    \Phi_1(x) &=& x-1\\
             &=& x-\zeta^0\,.
    \end{array}\]
\end{beispiel}


Nun können wir uns überlegen, ob und wie unterschiedliche Kreisteilungspolynome
zusammenhängen und kommen dabei auf die bekannten Resultate, die z.B. in 
\autocite[Proposition 10.6, 10.7]{hachenberger1997finite} zu finden sind. Um
diese anzugeben, benötigen wir jedoch noch eine Definition und zitieren
einige Eigenschaften.

\begin{definition}
  \label{def:closure}
  Seien $r,n\in \N$, so definiere
  \[ \cl_r(n) \speq{:=} \max\{k \in \N^\ast:\ k \mid n,\ \nu(k) \mid \nu(r)
  \}\,.\]
\end{definition}


\begin{lemma}
  \label{lemma:cl_1}
  Seien $q>1$ eine ganze Zahl, $n\in \N^\ast$ und $r$ ein Primteiler von $q-1$.
  Dann gilt:
  \begin{enumerate}
    \item Ist $r\neq 2$ oder $q\equiv 1 \bmod 4$, so gilt
      \[ \cl_r(q^{r^n}-1) \speq= r^n\, \cl_r(q-1) \,.\]
    \item Ist $q \equiv 3 \bmod 4$, so gilt
      \[ \cl_2(q^{2^n}-1) \speq= 2^{n-1}\, \cl_2(q^2-1)\,.\]
  \end{enumerate}
\end{lemma}
\begin{proof}
  \autocite[Lemma 19.4]{hachenberger1997finite}.
\end{proof}


\begin{lemma}
  \label{lemma:cl_2}
  Seien $q,m,k > 1$ ganze Zahlen mit $\nu(k) \mid \nu(m)\mid q-1$. Dann gilt
  \begin{enumerate}
    \item Ist $m$ ungerade oder $q \equiv 1 \bmod 4$ oder $k$ ungerade, so
      gilt
      \[ \cl_m(q^k-1) \speq= k\,\cl_m(q-1)\,. \]
    \item Ist $m$ gerade, $q \equiv 3 \bmod 4$ und $k$ gerade, so ist
      \[ \cl_m(q^k-1) \speq= \tfrac k 2\, \cl_m(q^2-1)\,.\]
  \end{enumerate}
\end{lemma}
\begin{proof}
  \autocite[Lemma 19.5]{hachenberger1997finite}.
\end{proof}


\begin{satz}
  \label{satz:zusammenhang_unterschiedlicher_kreisteilungspolys}
  Seien $t,k \in \N^\ast$ und $K$ ein Körper der Charakteristik $p$.
  \begin{enumerate}
    \item Ist $\nu(t) \mid k$, so gilt
      \[ \Phi_k(x^t) \speq= \Phi_{kt}(x) \ \in K[x]\,.\]
    \item Sind $t$ und $k$ teilerfremd, so gilt
      \[ \Phi_k(x^t) \speq= \prod_{d\mid t} \Phi_{kd}(x)\ \in K[x]\,.\]
    \item Insbesondere gilt: Seien $q = p^r$ eine Primzahlpotenz,
      $t,k\in\N^\ast$ mit $p\nmid t,k$ und $\pi$ eine Potenz von $p$. Sei ferner
      $t = \cl_k(t)\cdot \bar t$, so gilt
      \[ \Phi_k(x^{t\pi}) \speq= 
        \left(\prod_{d\mid \bar t} \Phi_{k\,d\,\cl_k(t)} (x)\right)^\pi
        \ \in \F_q[x]\,. \]
  \end{enumerate}
\end{satz}
\begin{proof}
  Dass sich Potenzen von $p$ aus dem Argument herausziehen lassen, ist klar,
  da $\id_P = (.)^\pi: P \to P$ für den Primkörper $P \subset K$ nach 
  \thref{satz:frob_auto} eine lineare Abbildung ist. 
  Ferner haben nach \thref{satz:zerfall_xn_1} die
  Kreisteilungspolynome nur Koeffizienten in $P$.

  Der Kern des Beweises des Rests liegt in der Betrachtung des 
  Gruppenhomomorphismus
  \[ \psi_n:\ \bar K^\ast \to \bar K^\ast,\ x \mapsto x^n\]
  für $p\nmid n$. Denn nun ist offensichtlich, dass die Nullstellen von 
  $\Phi_k(x^t)$ gerade alle Elemente in $\bar K^\ast$, deren $t$-te Potenz
  eine primitive $k$-te Einheitswurzel ist, sind, also $\psi_t\inv(C\kl k)$.
  Ergo formulieren sich die Aussagen wie folgt um:
  \begin{enumerate}[label=(\arabic*')]
    \item Ist $\nu(t) \mid k$, so gilt
      $\psi_t\inv(C\kl k) \speq= C\kl{kt}$.
    \item Ist $\ggT(t,k)=1$, so gilt
      $\psi_t\inv(C\kl k) \speq= \bigcupdot_{d\mid t} C\kl{kd}$.
    \item Ist $k,t\in \N^\ast$ mit $p\nmid t,k$ und $k = \cl_k(t) \bar t$,
      so gilt
      \[ \psi_t\inv(C\kl k) \speq= 
        \bigcupdot_{d\mid \bar t} C\kl{k\,d\,\cl_k(t)}\]
  \end{enumerate}
  Nun ist offensichtlich, dass es reicht (3') zu zeigen. 
  Dazu notiere $t_0 := \cl_k(t)$ und seien $d\mid \bar t$ 
  und $\zeta \in C\kl{kd t_0}$ beliebig. Dann ist 
  \[ \ord(\zeta^t) = \ord((\zeta^{t_0d})^{\frac{\bar t}{d}})
    = k\,,\]
  da per Definition von $\cl_k(t)$ gerade $\ggT(\bar t, kt_0) = 1$.
  Also gilt $\psi_t(C\kl{kdt_0}) \subseteq C\kl k$
  und damit
  \[ \bigcupdot_{d\mid \bar t} C\kl{kdt_0} \speq\subseteq 
    \psi_t\inv \psi_t ( \cupdot_{d\mid \bar t} C\kl{kdt_0}) 
    \speq\subseteq \psi_t\inv(C\kl k)\]
  Die Gleichheit folgt mit einem Zählargument:
  Auf der einen Seite ist
  \[ \big|\bigcupdot_{d\mid \bar t} C\kl{kdt_0}\big| = 
    \sum_{d\mid \bar t} \varphi(kdt_0) = 
    \varphi(kt_0)\sum_{d\mid \bar t}\varphi(d) = \varphi(kt_0) \cdot \bar t
     = \varphi(k)t\,,\]
  wobei an \thref{lemma:rechenregeln_phifunktion} erinnert sei.
  Auf der anderen Seite haben wir
  \[ \big| \psi_t\inv(C \kl k)\big| \speq= t |C\kl k| \speq= t \varphi(k)\,,\]
  was den Beweis abschließt.
\end{proof}

Bevor wir den Zerfall der Kreisteilungspolynome noch genauer untersuchen,
kann man als einfache Folgerung angeben, wann genau ein
Binom $x^n-\beta \in \F_q[x]$ irreduzibel ist.

\begin{satz}
  \label{satz:binom_irreduzibel}
  Seien $\beta \in \F_q^\ast$ und $n\in \N$. Es gilt:
  $x^n -\beta \in \F_q[x]$ ist genau
  dann irreduzibel, wenn 
  \begin{enumerate}
    \item $p := \charak(\F_q) \nmid n$,
    \item $\nu(n) \mid e := \ord(\beta)$ und
    \item $\ord_{ne}(q) = n$.
  \end{enumerate}
\end{satz}
\begin{proof}
  Zunächst ist klar, dass $p \nmid n$ erfüllt sein muss, da ansonsten 
  $\beta' \in \F_q^\ast$ existiert mit $\beta'^p = \beta$ 
  (vgl. \thref{satz:frob_auto}), also wäre
  $x^n-\beta = (x^\frac n p - \beta')^p$ eine Faktorisierung.
  Nun sei $u$ eine Nullstelle von $x^n-\beta$, so lässt sich beobachten, dass
  (in Notation des Beweises von
  \thref{satz:zusammenhang_unterschiedlicher_kreisteilungspolys})
  $u \in \psi_n\inv(C\kl{e})$. Damit gilt nach
  \thref{satz:zusammenhang_unterschiedlicher_kreisteilungspolys} (3)
  \[ x^n-\beta \speq= \prod_{d\mid \bar n} \ggT(x^n-\beta, \Phi_{en_0d}(x))\]
  für $n = \cl_{e}(n) \bar n$ und diese Zerlegung ist, wie man sich analog zum
  Beweis von \thref{satz:zerfall_kreisteilungspolys} überlegen kann, 
  nicht trivial (vgl. \autocite[Proposition 5.3.5]{hachenberger2015}). 
  Damit ist (2) der Behauptung
  klar, so dass $x^n-\beta \mid \Phi_{ne}(x)$. Ferner zerfällt $\Phi_{ne}(x)$
  nach \thref{satz:zerfall_kreisteilungspolys} in 
  $\frac{\varphi(ne)}{\ord_{ne}(q)}$ irreduzible Faktoren von jeweils Grad
  $\ord_{ne}(q)$. Damit wird auch (3) der Behauptung augenblicklich klar.
\end{proof}

Für die letzte Bedingung in obigem Satz existieren noch verschiedene weitere
äquivalente Charakterisierungen, die nachstehend zu finden sind.

\begin{satz}
  \label{satz:binom_irreduzibel_aquiv}
  Seien $p$ eine Primzahl, $q$ eine Potenz von $p$ und $n,e\in \N^\ast$ 
  mit $p\nmid n$ und $\nu(n)\mid e\mid q-1$. Dann sind äquivalent:
  \begin{enumerate}
    \item $\ord_{ne}(q)  = n$,
    \item $\ggT(\tfrac{q-1}{e},n)=1$ und $q\equiv 1 \bmod 4$, falls $4\mid n$,
      und
    \item $\cl_n(q-1) \mid e$ und $q\equiv 1 \bmod 4$, falls $4\mid n$.
  \end{enumerate}
\end{satz}
\begin{proof}
  \autocite[Theorem 5.3.7]{hachenberger2015} und 
  \autocite[Corollary 5.3.8]{hachenberger2015}.
\end{proof}


Wir haben nun erkannt, wann genau Binome über einem endlichen Körper
irreduzibel sind. Doch wenn man dem Titel dieses Kapitels Glauben schenken mag,
interessieren wir uns hier vorangig für den Zerfall der Kreisteilungspolynome
über endlichen Körpern. Diese sind im Allgemeinen keine Binome, 
aber genau das Wissen über die Irreduzibiltät von
Binomen lässt uns Bedingungen formulieren, die dazu führen, dass ein
Kreisteilungspolynom über einem gegebenen endlichen Körper in irreduzible
Binome zerfällt.
Später (\autoref{sec:stark_regulare_erweiterungen}) werden wir diese
Bedingungen \emph{stark regulär} (\thref{def:stark_regular}) nennen und
einsehen, dass sie eine wesentliche Rolle bei der expliziten Konstruktion 
von Normalbasen spielen. Nachstehender Satz hat seinen Ursprung in 
\autocite[Lemma 22.2]{hachenberger1997finite}, jedoch mit anderem
Beweis.

\begin{satz}
  \label{satz:kreisteilungspolynome_binome}
  Seien $\F_q$ ein endlicher Körper von Charakteristik $p$ und $m \in \N^\ast$.
  Es gelte $p\nmid m$, $\nu(m)\mid q-1$ und $4\mid q-1$, falls $2\mid m$.
  Setze $l := \cl_m(q-1)$, $a := \ggT(l,m)$ und $I_a := \{ j\in \N^\ast:\ 
  j\leq a,\ \ggT(j,a)=1\}$. Ist $\zeta \in \F_q^\ast$ eine primitive 
  $a$-te Einheitswurzel, so ist
  \[ \Phi_m(x) \speq= \prod_{j\in I_a} \big( x^\frac m a - \zeta^j\big) \]
  die vollständige Faktorisierung des $m$-ten Kreisteilungspolynoms über
  $\F_q$.
\end{satz}
\begin{proof}
  Wir stellen fest, dass $\F_q$ in der Tat $a$-te Einheitswurzeln enthält, da 
  $\ord_a(q) = 1$. Dies ist klar, da $l$ per definitionem $q-1$ teilt und 
  $a = \ggT(l,m)$.
  Nun wollen wir uns klar werden, dass beide Seiten obiger Gleichung auch
  identisch sind: Für $j \in I_a$ durchläuft $\zeta^j$ alle primitiven $a$-ten
  Einheitswurzeln und damit sind die Nullstellen der rechten Seite der
  Gleichung gerade alle primitiven $m$-ten Einheitswurzeln.
  Bleibt die Irreduzbilität von $x^\frac m a - \zeta^j$ zu zeigen, wobei wir
  ohne Einschränkung $j=1$ wählen können: Klar ist, dass $p\nmid \frac m a$, da
  $p \nmid m$ nach Voraussetzungen. Ferner ist $\nu(l) = \nu(m)$, da wegen
  $\nu(m) \mid q-1$ gilt:
  \[ \cl_m(q-1) = \max\{k\in \N^\ast:\ k\mid q-1,\ \nu(k)= \nu(m) \}\,. \]
  Also ist auch $\nu(a) = \nu(\ggT(l,m)) = \nu(m)$ und damit folgt
  $\nu(\tfrac m a) \mid \nu(m) = \nu(a) \mid a$, womit auch (2) in 
  \thref{satz:binom_irreduzibel} erfüllt wäre.
  Da $\nu(m)\mid q-1$ ist $\cl_{\frac m a}(q-1) \mid m$, also auch
  $\cl_{\frac m a}(q-1)\mid a$. Damit wäre durch die Bedingung $q\equiv 1 \bmod
  4$, falls $2\mid m$, auch (3) in \thref{satz:binom_irreduzibel_aquiv}
  erfüllt.
\end{proof}

\begin{bemerkung}
  Man hätte obigen Beweis auch ohne das Wissen über irreduzible Binome führen
  können, in dem man sich \thref{satz:zerfall_kreisteilungspolys} bedient. So
  findet man dies auch in \autocite[Lemma 22.2]{hachenberger1997finite}.
\end{bemerkung}



Erinnert man sich nun erneut an \thref{satz:zerfall_kreisteilungspolys}, so
kann man sich die Frage stellen, ob man den
Zusammenhang unterschiedlicher Kreisteilungspolynome aus 
\thref{satz:zusammenhang_unterschiedlicher_kreisteilungspolys}
in dem Sinne verfeinern kann, dass man sich nicht für das gesamte 
Kreisteilungspolynom interessiert, sondern lediglich für einen irreduziblen
Teiler. Diese Frage beantwortet nachstehender Satz.


\begin{satz}
  \label{satz:zerfall_f_x_s}
  Seien $q=p^r$ eine Primzahlpotenz und
  $m,t\in \N$ mit $p\nmid m$, $p\nmid t$ und $\ggT(m,t) = 1$.
  Definieren wir für $d\mid t$ ferner
  \[ \Delta_q(m,d) \speq{:=} \frac{\varphi(d) \ord_m(q)}{\ord_{md}(q)}\,,\]
  so gilt:
  \begin{enumerate}
    \item Ist $f(x) \mid \Phi_m(x)$ ein über $\F_q$ irreduzibler monischer Teiler des
      $m$-ten Kreisteilungspolynoms, so gilt
      \[ f(x^t) \speq= \prod_{d\mid t} 
        \prod_{i=1}^{\Delta_q(m,d)} f_{d,i}(x)\,,\]
      wobei für alle $i=1,\ldots,\Delta_q(m,d)$ 
      \[ f_{d,i}\in \F_q[x] 
        \text{ monisch, irreduzibel und } f_{d,i}(x) \mid \Phi_{md}(x)\,.\]
      Ferner sind alle $f_{d,i}(x)$ paarweise teilerfremd.
    \item Sind $f(x) \mid \Phi_m(x)$ und $g(x) \mid \Phi_m(x)$ zwei
      teilerfremde, monische, über $\F_q$ irreduzible Teiler des $m$-ten
      Kreisteilungspolynoms, so sind auch $f(x^t)$ und $g(x^t)$ teilerfremd.
  \end{enumerate}
\end{satz}
\begin{proof}
  Wie schon im Beweis von
  \thref{satz:zusammenhang_unterschiedlicher_kreisteilungspolys} betrachten wir
  den Gruppenhomomorphismus $\psi_t$, diesmal eingeschränkt auf $U\kl{mt}$:
  \[ \psi_t:\ \funcdef{U\kl{mt} & \to& U\kl m\,,\\
    x &\mapsto& x^t\,,}\]
  was offenbar ein wohldefinierter Gruppenhomomorphismus bleibt.
  Offensichtlich ist $\ker\psi_t = U\kl t$. Da $\ggT(m,t) = 1$, also 
  $U\kl{mt} = U\kl m \odot U\kl t$ als leichte Folgerung aus
  \thref{satz:zykl_gruppen}, ist $\psi_t$ auch surjektiv.

  Ist nun $\alpha \in C\kl m$ eine Nullstelle von $f(x)$, 
  so existiert -- wiederum
  weil $m$ und $t$ teilerfremd sind -- genau ein $\beta \in C\kl m$ mit
  $\beta^t = \alpha$. Damit ist also
  \[ \psi_t\inv(\{\alpha\}) \speq= \beta\, U\kl t \speq=
    \bigcupdot_{d\mid t}\ \beta\, C\kl d \,.\]
  Notiert wieder $\sigma$ der Frobenius von
  $\F_q$, so sind nach \thref{satz:nst_irred_polys}
  $\sigma^j(\alpha)$, $j=0,\ldots,\delta-1$ für 
  $\delta=\ord_q(m)$ die Nullstellen von $f(x)$. Da $p\nmid t$ bleibt
  die Menge der $t$-ten Einheitswurzeln invariant unter $\sigma $ und 
  damit ist die Menge der Nullstellen von $f(x^t)$ gerade
  \begin{equation}
    \bigcupdot_{j=0}^{\delta-1} \sigma^j(\beta)\, U\kl t \speq=
    \bigcupdot_{j=0}^{\delta-1}\ \bigcupdot_{d\mid t}\ \beta^{q^j}\, C\kl d
    \speq=
    \bigcupdot_{d\mid t}\ \bigcupdot_{j=0}^{\delta-1}\ \beta^{q^j}\, C\kl d
    \speq{=:} \bigcupdot_{d\mid t} N_d\,. 
  \end{equation}
  %Wir bemerken kurz, dass $\alpha$ nicht in obiger Nullstellenmenge liegt, da
  %$q^i \neq t$ für alle $i=1,\ldots,\delta-1$ und damit 
  %$\beta^{q^i} \neq \alpha$ für alle $i=1,\ldots,\delta-1$, was
  %$f(x)\nmid f(x^t)$ zeigt.
  Wollen wir nun einsehen, wie $f(x^t)$ über $\F_q$ zerfällt, so müssen wir
  überlegen, wie obige Nullstellenmenge in $\sigma$-invariante Teilmengen
  zerfällt. Für jedes $d\mid t$ und jedes $j\in\{0,\ldots,\delta -1\}$ ist
  $\zeta \in \beta^{q^j}\,C\kl d$ ein Element mit $\ord(\zeta) = md$, also
  Nullstelle von $\Phi_{md}(x)$. Ferner gilt
  offenbar $\forall d\mid t:\ |N_d| = \delta\varphi(d)$
  und wir können folgern, dass $N_d$ in genau
  \[ \frac{\delta\varphi(d)}{\ord_{md}(q)} \speq= \frac{\ord_m(q)\,
    \varphi(d)}{\ord_{md}(q)} \speq=
    \Delta_q(m,d)\]
  $\sigma$-invariante Teilmengen zerfällt. $\Delta_q(m,d)$ ist in der Tat eine
  natürliche Zahl größer 0, da nach \thref{lemma:rechenregeln_ordn} (2)
  \[ \frac{\ord_m(q)\,\varphi(d)}{\ord_{md}(q)} \speq=
    \frac{\varphi(d)\, \ggT(\ord_m(q),\ord_d(q))}{\ord_d(q)}\]
  und $\ord_d(q) \mid \varphi(d)$ nach \thref{lemma:rechenregeln_ordn} (1).
  Damit ist alles in (1) gezeigt.
  Der Zusatz (2) folgt sofort, denn ist $\alpha_f \in C\kl m$ 
  bzw. $\alpha_g \in C\kl m$ Nullstelle von $f$ bzw. $g$, so gehören diese zu 
  verschiedenen $\sigma$-invarianten Teilmengen von $C\kl m$ (vgl. 
  auch \thref{beispiel:zerfall_x21_1_1}) und folglich gehören auch
  $\beta_f\in C\kl m$ bzw. $\beta_g\in C\kl m$ mit $\beta_f^t = \alpha_f$ bzw.
  $\beta_g^t = \alpha_g$ zu verschiedenen und damit disjunkten
  $\sigma$-invarianten Teilmengen von $C\kl m$.
\end{proof}

%\begin{satz}
  %\label{satz:f_x_s_ist_teiler_von_phimd}
  %Seien $q=p^r$ eine Primzahlpotenz, $F = \F_q$ ein endlicher Körper,
  %$m$ eine natürliche Zahl und $s = \ord_{\nu(m)}(q)$ mit 
  %$\ggT(s,m) = 1$. Weiter sei $s = \bar s p^\beta$ mit 
  %$\ggT(\bar s, p) = 1$. Ist ferner $f(x) \mid \Phi_m(x)$ ein über $\F_q$
  %irreduzibler monischer Teiler des $m$-ten Kreisteilungspolynoms, so gilt:
  %\[ f(x^s) \speq= \prod_{i=1}^? f_{i}(x)^{p^\beta}\,,\]
  %so dass $f_i(x)$ ein irreduzibler monischer Teiler von $\Phi_{md(i)}(x)$ ist
  %für $d:\ \{1,\ldots,?\} \to \{\text{Teiler von } \bar s\},\ 
  %i\mapsto d(i)$ surjektiv und monoton. Ferner gilt sogar 
  %$d(1) < d(2)$.
%\end{satz}
%\marginpar{Leider habe ich keine Ahnung, wie $i\mapsto d(i)$ aussieht und
  %was $?$ ist. Insbesondere gelingt es mir nicht, die Surjektivität zu
  %beweisen.}
%\begin{proof}
  %Zunächst ist klar, dass wir \obda annehmen können, dass $p \nmid s$. Ist
  %nämlich $s = \bar s p^\beta$ wie oben, so gilt
  %\[ f(x^s) = f(x^{\bar s})^{p^\beta}\,,\]
  %da $\F_q \to \F_q, x \mapsto x^p$ bekanntlich ein Ringhomomorphismus ist.
  %Ist dann $\zeta \in E$ eine primitive $(ms)$-te Einheitswurzel für
  %$E = F_e$, $e = \ord_{ms}(q)$, so ist $\ord(\zeta^s) = m$ und es gilt nach
  %\cref{} \marginpar{Reference!}
  %\[ \Phi_m(x) \speq= \prod_{j \in R_q(m)\atop \ggT(j,m) = 1}
    %\prod_{i\in M_q(j\bmod m)} \ (x - \zeta^{si} ) \]
  %Also haben wir
  %\[ f(x) \speq= \prod_{i\in M_q(j_0 \bmod m)} \ (x - \zeta^{si})\]
  %für ein $j_0\in R_q(m)$.
  %Betrachten wir nun einen Linearfaktor von $f(x^s)$, so wollen wir 
  %zeigen, dass
  %\[g(x) := (x^s - \zeta^s) \speq= \prod_{i=0}^{s-1} ( x - \zeta^{im +1} )\,.\]
  %Dies ist aber nicht schwer zu sehen, da für $i \in \{0,\ldots,s-1\}$
  %\[ g(\zeta^{in+1}) \speq= \zeta^{ism + s} - \zeta^s \speq= 0 \,.\]
  %Da $\zeta$ primitive $(ms)$-te Einheitswurzel ist, haben wir $s$ paarweise
  %verschiedene Nullstellen von $g$ gefunden und obige Behauptung gezeigt.
  %Nun betrachten wir die Aufteilung der Kreisteilungspolynome:
  %Nach \cref{} haben wir
  %\[ \Phi_m(x^s) \speq= \prod_{d \mid s} \Phi_{md}(x)\,. \]
  %Nun wollen wir einsehen, dass jedes $\zeta^{in+1}$ ein 
  %$\Phi_{md}(x)$ für $d\mid s$ \glqq trifft\grqq: \TODO
  %Sei nun $I \subseteq \{0,\ldots,s-1\}$, so dass $\forall d\mid s$ genau ein
  %$i \in I$ existiert mit $(x-\zeta^{im+1}) \mid \Phi_{md}(x)$. Dann sind wir
  %jedoch fertig, da
  %\[ f(x^s) \speq= \prod_{i \in I} \prod_{j\in M_q(i \bmod md_i)}
    %\ (x - \zeta^{(im+1)j}) \, \]
  %wobei $d_i$ gerade der zu $i \in I$ korrespondierende Teiler $d_i\mid s$ sei,
  %das kleinste Produkt ist, das $f(x^s)$ zu einem Polynom über $F$ macht.
%\end{proof}


\begin{beispiel}
  Greifen wir noch einmal \thref{beispiel:zerfall_x21_1_1} auf und betrachten
  einen irreduziblen Teiler $f(x)$ von $\Phi_7(x)$ über $\F_2$, sagen wir
  \[ f(x) \speq{:=} x^3+ x +1\,.\]
  Sei $t := 3$.
  Nun wissen wir nach \thref{satz:zerfall_f_x_s}, dass $f(x^3)$ wie folgt über
  $\F_2$ zerfällt:
  \[ f(x^3) \speq= 
    \tikz[baseline]{\node[anchor=base,rounded corners,fill=gray!5]
      (n)
      {$\displaystyle\prod_{i=1}^{\Delta_2(7,1)} f_{1,i}(x)$};
      \node[above=0pt of n, font=\scriptsize, text=gray]{$d=1 \mid 3$};}
    \cdot
    \tikz[baseline]{\node[anchor=base,rounded corners,fill=gray!5]
      (n)
      {$\displaystyle\prod_{i=1}^{\Delta_2(7,3)} f_{3,i}(x)$};
      \node[above=0pt of n, font=\scriptsize, text=gray]{$d=3 \mid 3$};}
    \speq=
    \tikz[baseline]{\node[anchor=base,rounded corners,fill=gray!5]
      (n)
      {$\displaystyle f_{1,1}(x)$};}
    \cdot
    \tikz[baseline]{\node[anchor=base,rounded corners,fill=gray!5]
      (n)
      {$\displaystyle f_{3,1}(x)$};}
    \]
    da 
    \begin{alignat*}{6}
      \Delta_2(7,1) &\speq=& \frac{\varphi(1)\ord_7(2)}{\ord_{7}(2)} &\speq=&
        \frac{1\cdot 3}{3} &\speq=& 1\,, \\
      \Delta_2(7,3) &\speq=& \frac{\varphi(3)\ord_7(2)}{\ord_{21}(2)} &\speq=&
        \frac{2\cdot 3}{6} &\speq=& 1\,. 
    \end{alignat*}
    Wir wollen nun herausfinden, welche Teiler $f_{1,1}(x)$ und $f_{3,1}(x)$ 
    von $\Phi_7(x)$ und $\Phi_{21}(x)$ sind. Wir übernehmen den Zerfall
    der Kreisteilungspolynome aus \thref{beispiel:zerfall_x21_1_1}
    und können einsehen, dass
    \[ f_{1,1}(x) \speq= x^3+x^2+1\,,\qquad
      f_{3,1}(x) \speq= x^6+x^5+x^4+x^2+1\,.\]
\end{beispiel}

%\begin{beispiel}
  %Sei $q = p = 2$ und $m = 7$, so haben wir über $\F_2$
  %\begin{align*}
    %\Phi_m(x) = \Phi_7(x) &= 
      %x^{6} + x^{5} + x^{4} + x^{3} + x^{2} + x + 1 \\
    %&= (x^{3} + x + 1) \cdot (x^{3} + x^{2} + 1)
  %\end{align*}
  %Hieraus wählen wir einen irreduziblen Teiler, sagen wir
  %\[ f(x) = x^3 + x +1\,.\]
  %Nun wollen wir den Beweis von \cref{satz:f_x_s_ist_teiler_von_phimd}
  %nachvollziehen. Dazu berechnen wir erst $\ord_7(2) = 3  =: s$. 
  %Sei dann $\zeta$ eine 
  %$(ms) = 21$-te Einheitswurzel. Um $\Phi_7$ in Termen von $\zeta$ darstellen
  %zu können, brauchen wir ein Vertretersystem von Restklassen ${}\bmod 21$
  %\[\begin{array}[t]{r|l}
    %l \in R_2(21) & M_2(l \bmod 21) \\\hline
    %0 & 0 \\
    %1 & 1, 2, 4, 8, 11, 16 \\
    %3 & 3, 6, 12 \\
    %5 & 5, 10, 13, 17, 19, 20 \\
    %7 & 7, 14 \\
    %9 & 9, 15, 18
  %\end{array}\]
  %Damit haben wir 
  %\[ \setlength{\arraycolsep}{2pt}\begin{array}{rcccc}
    %\Phi_7(x) &=& (x^{3} + x + 1) &\cdot& (x^{3} + x^{2} + 1) \\
      %&=& (x-\zeta^3)(x-\zeta^{3\cdot 2})(x-\zeta^{3\cdot 4}) &\cdot&
        %(x-\zeta^{3\cdot 3})(x-\zeta^{3\cdot 5})(x-\zeta^{3\cdot 6})
  %\end{array}\]
  %Nun folgt 
  %\[\Phi_7(x^3) 
        %= \prod_{d \mid \bar s} \Phi_{md}(x) = \Phi_7(x) \cdot \Phi_{21}(x)\]
  %für 
  %\[\footnotesize\setlength{\arraycolsep}{2pt}\everymath{\displaystyle}
    %\begin{array}{rcccc} 
      %\Phi_7(x)&=& (x^{3} + x + 1) &\cdot& (x^{3} + x^{2} + 1) \\
      %&=& (x-\zeta^3)(x-\zeta^{3\cdot 2})(x-\zeta^{3\cdot 4}) &\cdot&
        %(x-\zeta^{3\cdot 3})(x-\zeta^{3\cdot 5})(x-\zeta^{3\cdot 6}) \\[10pt]
      %\Phi_{21}(x) &=& (x^{6} + x^{4} + x^{2} + x + 1) 
        %&\cdot& (x^{6} + x^{5} + x^{4} + x^{2} + 1) \\
      %&=& (x-\zeta)(x-\zeta^2)(x-\zeta^4)(x-\zeta^8)(x-\zeta^{11})
        %(x-\zeta^{16}) &\cdot&
        %(x-\zeta^5)(x-\zeta^{10})(x-\zeta^{13})(x-\zeta^{17})
        %(x-\zeta^{19})(x-\zeta^{20}) 
    %\end{array}\]
  %Ferner ist 
  %\begin{align*}
    %f(x^s) = f(x^3) &= x^{9} + x^{3} + 1\\
    %&= (x^{3} + x^{2} + 1) \cdot (x^{6} + x^{5} + x^{4} + x^{2} + 1)
  %\end{align*}
  %und wir erkennen, dass bereits alles durch
  %\[ (x-\zeta^3)(x^3) \speq= (x-\zeta) (x-\zeta^8) (x-\zeta^{15})\]
  %festgelegt ist.
  %Hier wäre also $f(x^s) = f_1(x)f_2(x)$ mit 
  %\[ d:\ \{1,2\} \to \{1,3\},\ 1\mapsto 1,\ 2\mapsto 3\,,\]
  %was in diesem Fall sogar bijektiv ist.
%\end{beispiel}

\begin{beispiel}
  Als zweites Beispiel wollen wir uns einen Fall betrachten, 
  in dem $\Delta_q(d,m)$ nicht immer $1$ ist.
  Sei $p=q=3$, $m=5$ und $t= 4$.
  Also müssen wir ein Vertretersystem von Restklassen modulo $20$ betrachten:
  \[\begin{array}[t]{r|l}
    l \in R_3(20) & M_2(l \bmod 20) \\\hline
    0 & 0 \\
    1 & 1, 3, 7, 9 \\
    2 & 2, 6, 14, 18 \\
    4 & 4, 8, 12, 16 \\
    5 & 5, 15 \\
    10& 10 \\
    11& 11, 13, 17, 19 
    \end{array}\]
  Wir sehen, dass $\Phi_{20}(x)$ für $l=1,11$ in 
  2 Polynome von jeweils Grad $4$ zerfällt:
  \[\setlength{\arraycolsep}{2pt}\everymath{\displaystyle}
    \begin{array}{rcccc} 
      \Phi_{20}(x) &=& 
        (x^{4} + x^{3} + 2 x + 1) &\cdot& (x^{4} + 2 x^{3} + x + 1) \\
      &=& (x-\zeta^{11})(x-\zeta^{13})(x-\zeta^{17})(x-\zeta^{19}) &\cdot&
        (x-\zeta)(x-\zeta^{3})(x-\zeta^{7})(x-\zeta^{9})\,,
    \end{array}\]
  wobei wir $\zeta \in C\kl{20}$ mit Minimalpolynom $x^4+2x^3+x+1$ gewählt
  haben.
  Nun können wir den Zerfall von $\Phi_5(x)$ und $\Phi_{10}(x)$ in Termen von
  $\zeta$ anhand der Restklassen modulo $20$ beschreiben:
  \[\setlength{\arraycolsep}{2pt}\everymath{\displaystyle}
    \begin{array}{rcccc} 
      \Phi_5(x)&=& x^4 + x^3 + x^2 + x + 1\\
      &=& (x-\zeta^4)(x-\zeta^{8})(x-\zeta^{12})(x-\zeta^{16})\,, \\[10pt]
      \Phi_{10}(x) &=& x^{4} + 2 x^{3} + x^{2} + 2 x + 1 \\
      &=& (x-\zeta^2)(x-\zeta^6)(x-\zeta^{14})(x-\zeta^{18})\,.
    \end{array}\]
  Die Restklassen für $l=0,5,10$ gehören zu den Kreisteilungspolynomen 
  $\Phi_1(x), \Phi_4(x)$ und $\Phi_2(x)$, die wir für ein Beispiel zu
  \thref{satz:zerfall_f_x_s} nicht benötigen.
  Nun brauchen wir wieder einen irreduziblen monischen Teiler von
  $\Phi_m(x)$ und setzen daher $f(x) = \Phi_m(x)$.
  Wir berechnen wie oben
  \begin{alignat*}{6}
    \Delta_3(5,1) &\speq=& \frac{\varphi(1)\ord_5(3)}{\ord_{5}(3)} &\speq=&
      \frac{1\cdot 4}{4} &\speq=& 1\,, \\
    \Delta_3(5,2) &\speq=& \frac{\varphi(1)\ord_5(3)}{\ord_{10}(3)} &\speq=&
      \frac{1\cdot 4}{4} &\speq=& 1\,, \\
    \Delta_3(5,4) &\speq=& \frac{\varphi(4)\ord_5(3)}{\ord_{20}(3)} &\speq=&
      \frac{2\cdot 4}{4} &\speq=& 2\,. 
  \end{alignat*}
  Nun ist klar, wie $f(x^t)$ über $\F_3$ zerfällt:
  \[\small f(x^4) \speq= \big(
    \tikz[baseline]{\node[anchor=base,rounded corners,fill=gray!5]
      (n)
      {$\displaystyle x^4+x^3+x^2+x+1$};
      \node[above=0pt of n, font=\scriptsize, text=gray]{$d=1 \mid 4$};}
    \big)\cdot\big(
    \tikz[baseline]{\node[anchor=base,rounded corners,fill=gray!5]
      (n)
      {$\displaystyle x^4+2x^3+x^2+2x+1$};
      \node[above=0pt of n, font=\scriptsize, text=gray]{$d=2 \mid 4$};}
    \big)\cdot\big(
    \tikz[baseline]{\node[anchor=base,rounded corners,fill=gray!5]
      (n)
      {$\displaystyle (x^4+x^3+2x+1)(x^4+2x^3+x+1)$};
      \node[above=0pt of n, font=\scriptsize, text=gray]{$d=4 \mid 4$};}
    \big) \]
\end{beispiel}

%\begin{beispiel}
  %Sei $q = p = 3$ und $m = 22$, so haben wir über $\F_q$
  %\begin{align*} 
    %\Phi_m(x) = \Phi_{22}(x) \speq{&=} 
      %x^{10} + 2 x^{9} + x^{8} + 2 x^{7} + x^{6} + 2 x^{5} + x^{4} + 
      %2 x^{3} + x^{2} + 2 x + 1 \\
    %\speq{&=}
      %(x^{5} + 2 x^{3} + 2 x^{2} + 2 x + 1) \cdot 
      %(x^{5} + 2 x^{4} + 2 x^{3} + 2 x^{2} + 1) \,.
  %\end{align*}
  %Es ist ferner $s = \ord_{\nu(m)}(q) = \ord_{22}(3) = 5 = \bar s$. Sei
  %\[ f(x) \speq= x^{5} + 2 x^{3} + 2 x^{2} + 2 x + 1 \quad\in\F_q[x]\]
  %ein irreduzibler Teiler von $\Phi_m(x)$ in $\F_q[x]$.
  
  %Wählen wir nun wie im Beweis von \cref{satz:f_x_s_ist_teiler_von_phimd} eine
  %$(ms) = 100$-te Einheitswurzel $\zeta$, so müssen wir ein Vertretersystem von
  %Restklassen ${}\bmod m$ berechnen, um die passenden Zerlegungen in
  %Linearfaktoren angeben zu können:
  %\begin{center}
    %\[\begin{array}{r|l}
      %l \in R_q & M_q(l \bmod m) \\\hline
      %0 & 0 \\
      %1 & 1, 3, 5, 9, 15    \\
      %2 & 2, 6, 8, 10, 18   \\
      %4 & 4, 12, 14, 16, 20 \\
      %7 & 7, 13, 17, 19, 21 \\
      %11 & 11
    %\end{array}\]
  %\end{center}
  %Da nur $1$ und $7$ teilerfremd zu $22$ sind, ist also
  %\[\setlength{\arraycolsep}{3pt}\begin{array}{rcccc}
    %\Phi_{22}(x) &=& (x^{5} + 2 x^{3} + 2 x^{2} + 2 x + 1) &\cdot& 
      %(x^{5} + 2 x^{4} + 2 x^{3} + 2 x^{2} + 1) \\
    %&=& (x - \zeta)(x-\zeta^3)(x-\zeta^5)(x-\zeta^9)(x-\zeta^{15}) &\cdot&
      %(x-\zeta^7)(x-\zeta^{13})(x-\zeta^{17})(x-\zeta^{19})(x-\zeta^{21})
  %\end{array}\]
  %Gehen wir nun über $f(x^s)$ zu betrachten, so erhalten wir über $\F_q$
  %\begin{align*}
    %f(x^s) &= x^{25} + 2 x^{15} + 2 x^{10} + 2 x^{5} + 1\\
    %&= (x^{5} + 2 x^{3} + 2 x^{2} + 2 x + 1) \\
    %&\ \  \cdot (x^{20} + x^{18} + x^{17} + 2 x^{16} + x^{15} + 
      %x^{14} + 2 x^{10} + 2 x^{9} + 2 x^{8} + x^{7} + 2 x^{5} + x^{4} + x^{3} + 
      %2 x^{2} + x + 1)\,.
  %\end{align*}

  %Dies teilt $\Phi_{m}(x^s) = \Phi_{22}(x^5)$, was über $\F_q$ wie folgt
  %zerfällt:
  %\[\begin{array}{rcc}
    %\Phi_{22}(x^5) &=& \prod_{d\mid \bar s} \Phi_{md}(x)
      %=  \Phi_{22}(x) \cdot \Phi_{110}(x) \\
    %&=&
  %\end{array}\]
  %Wie $\Phi_{22}$ in Termen von $\zeta$ zerfällt haben wir oben bereits
  %gesehen,

%\end{beispiel}

\chapter{Moduln}

Nähern wir uns der Situation von Normalbasen in möglichst allgemeiner Form, so
beginnt die Reise bei der Betrachtung folgender Situation:
\begin{definition}[$(V,\tau)$]
  Sei $\K$ ein Körper und $V$ ein $\K$-Vektorraum und 
  $\tau \in \End_\K(V)$, so können wir $V$ als $\K[x]$-Modul auffassen:
  \[ f(x) \cdot v \speq{:=} f(\tau)(v)\]
  für alle $f(x) \in \K[x]$ und $v\in V$.
  Nenne das Paar $(V,\tau)$ \emph{$\K[x]$-Modul bzgl. $\tau$}.
\end{definition}

\begin{notation}
  Sei $\tau\in \End_\K(V)$.
  \begin{itemize}
  \item Es bezeichne $\mu_\tau$ das Minimalpolynom von 
    $\tau$, also das normierte Polynom kleinsten Grades $f(x)\in \K[x]$ mit 
    $f(\tau) = 0$.
  \item Ferner schreibe $\chi_\tau$ für das charakteristische Polynom von 
    $\tau$, also $\chi_\tau(x) := \det(x \id_V - \tau) \in \K[x]$.
  \end{itemize}
\end{notation}


\begin{bemerkung}
  Ist $\K  = F :=\F_q$ ein endlicher Körper, 
  $V = E := \F_{q^n}$ eine Körpererweiterung
  von Grad $n$ und 
  \[\tau = \sigma: \funcdef{E & \to & E\\
    v &\mapsto & v^q}\]
  der Frobenius von $E$, so ist
  \[ \mu_\tau(x) \ =\ \chi_\tau(x) \ =\ x^n - 1\,,\]
  denn: Es ist klar, dass $n = \deg \chi_\tau$ und da nach dem Satz von
  Cayley-Hamilton ist $\sigma$ Nullstelle von $\chi_\tau$. Daher teilt
  $\mu_\tau$ das charakteristische Polynom. Jedoch kennen wir das
  Minimalpolynom von $\tau$: Nach Dedekinds-Unabhängigkeitslemma ist 
  $\id_E,\sigma,\ldots,\sigma^{n-1}$ linear unabhänig über $E$, also insbesondere
  über $F$, und $\sigma^n = \id_E$.\marginpar{References!}
\end{bemerkung}


\begin{definition}[$\tau$-Ordnung, Teilmodul]
  Sei $(V,\tau)$ ein $\K[x]$-Modul. Zu jedem $v \in V$ betrachte den
  $\K[x]$-Modulhomomorphismus
  \[ \psi_w: \funcdef{\K[x] & \to & V \\
    f(x) & \mapsto & f(x)\cdot v }  \]
  Sei ferner $\dim V < \infty$.
  \begin{enumerate}
    \item Ist $\ker\psi_v = (g(x))$ für $g(x)\in \K[x]$ normiert, so heißt
      $g(x)$ \emph{$\tau$-Ordnung von $v$}\@. Ferner ist $g(x)$ eindeutig.
      Schreibe $\Ord_\tau(v) := g(x)$.
    \item $\K[\tau]\cdot v := \im{\psi_v}$ heißt der von \emph{$v$ erzeugte
      $\K[x]$-Teilmodul von $V$}.
  \end{enumerate}
\end{definition}
\marginpar{Eindeutigkeit!}


\begin{notation}
  Für $\K = \F_q$ einen endlichen Körper, $V = E \mid \F_q$ eine 
  Körpererweiterung und $\tau = \sigma$ den Frobenius-Endomorphismus schreibe
  \[ \Ord_q := \Ord_\tau \]
  und bezeichne $\Ord_q$ mit \emph{$q$-Ordnung}.
\end{notation}

\begin{lemma}
  \label{lemma:eigenschaften-tau-ordnung}
  Sei $(V,\tau)$ ein $\K[x]$-Modul. Ferner seien
  $u,v\in V$ mit $g(x) := \Ord_\tau(u)$, $h(x) := \Ord_\tau(v)$ und 
  $f(x) \in \K[x]$. Dann gilt
  \begin{enumerate}
    \item $\Ord_\tau(f(x)\cdot u) = \frac{g(x)}{\ggT(f(x),g(x))}$.
    \item $\Ord_\tau(u+v) = g(x)h(x)$, falls $\ggT(g,h) = 1$.
  \end{enumerate}
\end{lemma}
\begin{proof}
  \begin{enumerate}
    \item \TODO
    \item \TODO
  \end{enumerate}
\end{proof}

\begin{lemma}
  Sei $(V,\tau)$ ein $\K[x]$-Modul. Sei $v\in V$. Dann gilt:
  \[ \dim_\K( \K[x]\cdot v ) \speq= \deg( \Ord_\tau(v) )\,.\]
\end{lemma}
\begin{proof}
  Nach dem Homomorphiesatz gilt: \marginpar{References!}
  $ \im\psi_v \cong \K[x] \big/ \ker \psi_v$.
\end{proof}


\begin{definition}[zyklischer Modul]
  $(V,\tau)$ heißt \emph{zyklischer $\K[x]$-Modul bzgl. $w$}, falls es ein 
  $w\in \K$ gibt, sodass $K[\tau]\cdot w = V$.
\end{definition}


\begin{satz}
  Es gilt:
  \[ (V,\tau) \text{ ist ein zyklischer Modul} \quad\Leftrightarrow\quad
    \mu_\tau = \chi_\tau\]
\end{satz}
\begin{proof}
  Fassen wir zunächst ein paar einfache Tatsachen zusammen:
  Ist $u \in V$, so haben wir 
  \[ \dim(\K[x]\cdot v) = \deg( \Ord_\tau(v) ) \speq\leq 
    \deg\mu_\tau \speq\leq \deg \chi_\tau \]
  und 
  \[ \Ord_\tau (v) \speq\mid \mu_\tau \speq\mid \chi_\tau \,,\]
  wobei die erste Teilbarkeitsrelation per definitionem erfüllt ist und die
  zweite gerade der Satz von Cayley-Hamilton ist.
  Damit kommen wir zum direkten Beweis:
  \begin{description}
    \item["`$\Rightarrow$"'] Sei $V$ also zyklisch bzgl. $w$, so ist dies nach
      obigem äquivalent zu $\deg(\Ord_\tau(w)) = n$. Daraus folgt aber sofort
      $\mu_t = \chi_\tau$, da beide normiert sind.
    \item["`$\Leftarrow$"'] Zunächst sei behauptet, dass es stets ein 
      $w \in V$ gibt mit $\Ord_\tau(w) = \mu_\tau$. Sei dazu 
      $\mu_\tau(x) = \prod_{i=1}^r p_i(x)^{a_i}$ die Zerlegung in irreduzible
      Faktoren über $\K[x]$, so existieren $w_i \in V$ mit
      $\Ord_\tau(w_i) = p_i^{a_i}$. Andernfalls hätten wir einen Widerspruch 
      zum Minimalpolynom von $\tau$!
      Nach \autoref{lemma:eigenschaften-tau-ordnung} ist dann aber 
      $w := \sum_{i=1}^r w_i$ ein Element in $V$ mit $\tau$-Ordnung $\mu_\tau$.

      Ist dann also $\mu_\tau = \chi_\tau$, so hat obiges $w$ genau
      $\tau$-Ordnung $\chi_\tau$; erzeugt also $V$ als $\K[x]$-Modul.
  \end{description}
\end{proof}

Nun wollen wir spezielle Untermoduln von $V$ betrachten, welche uns guten
Aufschluss über die Struktur von $V$ geben können:

\begin{notation}
  Seien $(V,\tau)$ ein $\K[x]$-Modul und $g(x) \in \K[x]$.
  Definiere
  \[ V_g \speq{:=} \{ v \in V \mid g(x)\cdot v = 0 \}\,.\]
\end{notation}

Zunächst ist klar, dass $V_g \neq 0$ nur für $g$ Teiler von $\mu_\tau$ gelten
kann. Damit können wir folgende "`Rechenregeln"' formulieren:

\begin{lemma}
  Seien $g(x), h(x) \in \K[x]$ mit $g,h \mid \mu_\tau$. Dann gilt:
  \begin{enumerate}
    \item $V_g \cap V_h \speq= V_{\ggT(g,h)}$
    \item $V_g + V_h \speq= V_{\kgV(g,h)}$
  \end{enumerate}
\end{lemma}
\begin{proof}
  Per definitionem ist klar, dass für $v \in V_g$ gerade
  $\Ord_\tau(v) \mid g$. Also können wir $V_g$ auch wie folgt auffassen:
  \[ V_g \speq= \{ v\in V:\ \Ord_\tau(v) \mid g\} \,,\]
  Damit sind die Behautungen nach \cref{lemma:eigenschaften-tau-ordnung} klar,
  denn für $v \in V$ gilt:
  \[ v\in V_g \cap V_h \speq\Leftrightarrow 
    \Ord_\tau(v) \mid g \land \Ord_\tau(v) \mid h \speq\Leftrightarrow
    \Ord_\tau(v) \mid \ggT(g,h) \speq\Leftrightarrow v \in V_{\ggT(g,h)}\]
  und ebenso
  \[ v \in V_g + V_h \speq\Leftrightarrow 
    \Ord_\tau(v) \mid g \lor \Ord_\tau(v) \mid h \speq\Leftrightarrow
    \Ord_\tau(v) \mid \kgV(g,h) \speq\Leftrightarrow v \in V_{\kgV(g,h)}\,.\]
\end{proof}


\begin{satz}
  Sei $(V,\tau)$ ein zyklischer Modul mit $\dim(V) = n$. Sei ferner 
  $g(x)\in \K[x]$ normiert mit $g\mid \mu_\tau$. Dann gilt:
  \begin{enumerate}
    \item $V_g$ ist ein $\K[x]$-Teilmodul von $V$.
    \item Alle $\K[x]$-Teilmoduln von $V$ sind von dieser Form.
    \item $V_g$ ist zyklisch bzgl. $\tau$ mit Minimalpolynom $g(x)$.
      Ferner ist $\dim(V_g) = \deg(g)$.
    \item Die Erzeuger von $V_g$ sind genau die Elemente $v\in V$ mit 
      $\Ord_\tau(v) = g$.
  \end{enumerate}
\end{satz}
\begin{proof}
  \begin{enumerate}
    \item 
    \item
    \item
    \item
  \end{enumerate}
\end{proof}

\chapter{Normalbasen -- Ein Überblick}

Seien wieder $F := \F_q$ ein endlicher Körper von Charakteristik $p$ und 
$E := \F_{q^n} \mid F$ eine Körpererweiterung.
Wir wiederholen kurz die Definition einer \emph{Normalbasis}

\begin{definition}[normales Element, normales Polynom, Normalbasis]
  Sei $F$ ein Körper und $E \mid F$ eine endliche Galoiserweiterung von Grad
  $n$. Sei ferner $w\in E$ mit $F(w) = E$. $w$ heißt \emph{normal über $F$},
  falls
  \[ \{ \gamma(w) \mid \gamma \in G\}\]
  eine $F$-Basis von $E$ ist. 
  $\{ \gamma(w) \mid \gamma \in G\}$ heißt entsprechend \emph{Normalbasis} und
  $g(x) \in F[x]$ mit 
  \[ g(x) = \prod_{\gamma \in G}(x - \gamma(w))\]
  heißt \emph{normales Polynom}.
\end{definition}

Um effizient normale Elemente in $E\mid F$ zu finden, betrachten wir 
$(E,\sigma)$ als $F[x]$-Modul und nutzen die Aussagen aus
\autoref{chap:moduln}.

\begin{satz}
  \begin{enumerate}
    \item Die Erzeuger von $(E,\sigma)$ als $F[x]$-Modul sind genau die 
      normalen Elemente in $E\mid F$.
    \item Man hat eine Bijektion von Mengen
      \[ \{V_g :\ g(x) \in F[x] \text{ monisch mit } g(x) \mid x^n-1\}
        \overset{1-1}{\longleftrightarrow}
        \{F[x] \text{-Teilmoduln von }E\}\]
        wobei $V_g := \{v \in E : g(x)\cdot v = 0\} = \ker(g(\sigma))$.
    \item Jedes $V_g$ ist ein zyklischer Modul und es gilt
      \[u \text{ erzeugt } V_g \quad\Leftrightarrow \quad
        \Ord_q(u) = g(x).\]
        Insbesondere sind die Erzeuger von $V$ genau die Elemente $v \in V$ mit 
        $\Ord_\tau(v) = x^n-1$.
  \end{enumerate}
\end{satz}
\begin{proof}
  Alles schon in \TODO~gezeigt.
\end{proof}

Dies liefert uns die grundlegende Idee für das Auffinden von normalen
Elementen:
\begin{lemma}
  Sei $x^n-1 = \prod_{i=1}^s r_i(x)$ eine Zerlegung in paarweise teilerfremde
  Polynome, so gilt:
  \[ E \speq= \bigoplus_{i=1}^s V_{r_i} \,.\]
\end{lemma}
\begin{proof}
  \TODO.
\end{proof}

\begin{kor}
  Sei $x^n-1 = \prod_{i=1}^s r_i(x)$ eine Zerlegung in paarweise teilerfremde
  Polynome. Seien ferner $u_i \in V_{r_i}$ Elemente mit 
  $\Ord_q(u_i) = r_i(x)$ $\forall i=1,\ldots,s$. Dann ist
  \[ u \speq= u_1 + u_2 + \ldots + u_s\]
  normal in $E \mid F$.
\end{kor}
\begin{proof}
  Da obige Zerlegung von $x^n-1$ paarweise teilerfremd ist, folgt nach
  \cref{lemma:eigenschaften-tau-ordnung}
  $\Ord_q(u) = \prod_{i=1}^s \Ord_q(u_i) = \prod_{i=1}^s r_i(x) = x^n -1$.
\end{proof}

\begin{beispiel}
  
\end{beispiel}


An diesem Punkt stellt sich natürlich die Frage, wie wir dies nutzbar machen
können. Ist nämlich $p\nmid n$, so kennen wir eine Faktorisierung von $x^n-1$:
\[ x^n-1 \speq= \prod_{d\mid n} \Phi_d(x)\,.\]
Des Weiteren können wir eine Zerlegung von $\Phi_d(x)$ sogar genauer angeben:



\nocite{*}
\printbibliography


\appendix
%\chapter{\texttt{Sage}-Quellcodes}
%\lstset{language=python,
  %basicstyle = \footnotesize\normalfont\ttfamily,
  %commentstyle = \itshape\color{gray},
  %caption ={\lstname},
  %frame = tb,
  %framexleftmargin = 0pt,
  %numbers = left,
  %numberstyle = \tiny,
%% numbersep = 5pt,
  %breaklines = true,
  %xleftmargin = 0.1\linewidth,
  %xrightmargin = 0.1\linewidth,
  %showstringspaces=false,
  %columns=fullflexible,
  %tabsize=3}

%\lstinputlisting{../Sage/algorithmen.spyx}
%\lstinputlisting{../Sage/examples/scheerhorn1.sage}
%\lstinputlisting{../Sage/examples/satz1.sage}
%\lstinputlisting{../Sage/examples/satz1_1.sage}
%\lstinputlisting{../Sage/examples/satz2.sage}


%\addchap{Fragen 1}
%\input{fragen1}

\end{document}
