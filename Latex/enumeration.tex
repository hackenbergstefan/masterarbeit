\chapter{Existenz und Enumeration primitiv vollständig normaler Elemente}
\label{chap:existenz_und_enumeration}

In den vorangegangenen Kapiteln haben wir lediglich normale und vollständig
normale Elemente in Erweiterungen endlicher Körper betrachtet. Es existiert
jedoch eine weitere besondere Eigenschaft, die gerade in der Anwendung von
großem Interesse ist: Primitivität (\thref{def:primitiv}). Zusammen haben wir
nun drei Eigenschaften eines Elements $u \in E$ einer Erweiterung endlicher
Körper $E\mid F$ kennengelernt, die von Interesse sind. Daher ist es nur
sinnvoll sich der Frage zu widmen, \emph{wie viele} Elemente mit den jeweiligen
Eigenschaften existieren. Mit \thref{satz:zykl_gruppen} und 
\thref{satz:mult_gruppe_endl_korper_zyklisch} ist sofort klar, dass es in
$\F_q$ genau $\varphi(q-1)$ primitive Elemente gibt! Daher ist die
Fragestellung nach der Anzahl primitiver Elemente schnell gelöst. Darüber
hinaus wollen wir die folgenden Notationen treffen.

\begin{definition}
  \label{def:anzahlen}
  Seien $\F_q$ ein endlicher Körper und $n \in \N^\ast$, so bezeichne
  $\cal N(q,n)$, $\CN(q,n)$, $\PN(q,n)$ bzw. $\PCN(q,n)$ die Anzahl der 
  normalen, vollständig normalen, primitiv normalen bzw. primitiv vollständig
  normalen Elemente in $\F_{q^n}$ über $\F_q$, d.h.
  \begin{align*}
    \cal N(q,n) &\speq{:=} 
      |\{ u \in \F_{q^n}:\ u\text{ ist normal über }\F_q\}|\,, \\
    \CN(q,n) &\speq{:=} 
      |\{ u \in \F_{q^n}:\ u\text{ ist vollständig normal über }\F_q\}|\,, \\
    \PN(q,n) &\speq{:=} 
      |\{ u \in \F_{q^n}:\ u\text{ ist primitiv und normal über }\F_q\}|\,, \\
    \PCN(q,n) &\speq{:=} 
      |\{ u \in \F_{q^n}:\ u\text{ ist primitiv und vollständig 
      normal über }\F_q\}|\,, \\
    \G &\speq{:=} 
      \{ n\in \N^\ast, n\geq 2:\ 
      \forall q\text{ Primzahlpotenz gilt } \PCN(q,n) > 0 \}\,.
  \end{align*}
\end{definition}

Vielleicht erscheint die Definition von $\G$ etwas überraschend, da für jedes
Element in $\G$ schließlich \emph{unendlich viele} Körpererweiterungen auf die
Existenz eines $\PCN$-Elements getestet werden müssen. Doch es sei an dieser
Stelle vorweg genommen, dass wir in der Lage sind mit Hilfe eines
asymptotischen Resultats und der konkreten Angabe von endlich vielen
$\PCN$-Elementen zu zeigen, dass $\G$ nicht leer ist!

Nun können wir folgende Probleme definieren:

\begin{problem}[$\cal N(q,n)=?$]
  \label{prob:n=}
  Seien $q$ eine Primzahlpotenz und $n\in \N^\ast$. Was ist
  $\cal N(q,n)$?
\end{problem}
\begin{problem}[$\CN(q,n)=?$]
  \label{prob:cn=}
  Seien $q$ eine Primzahlpotenz und $n\in \N^\ast$. Was ist
  $\CN(q,n)$?
\end{problem}
\begin{problem}[$\PN(q,n)=?$]
  \label{prob:pn=}
  Seien $q$ eine Primzahlpotenz und $n\in \N^\ast$. Was ist
  $\PN(q,n)$?
\end{problem}
\begin{problem}[$\PCN(q,n)=?$]
  \label{prob:pcn=}
  Seien $q$ eine Primzahlpotenz und $n\in \N^\ast$. Was ist
  $\PCN(q,n)$?
\end{problem}

Offensichtlich können wir die obigen Problemstellungen leicht abschwächen und
uns zunächst fragen, ob überhaupt Elemente mit den geforderten Eigenschaften
existieren. Auch dazu wollen wir passende Probleme formulieren.

\begin{problem}[$\cal N(q,n)>0?$]
  \label{prob:n>0}
  Seien $q$ eine Primzahlpotenz und $n\in \N^\ast$. Ist
  $\cal N(q,n)>0$?
\end{problem}
\begin{problem}[$\CN(q,n)>0?$]
  \label{prob:cn>0}
  Seien $q$ eine Primzahlpotenz und $n\in \N^\ast$. Ist
  $\CN(q,n)>0$?
\end{problem}
\begin{problem}[$\PN(q,n)>0?$]
  \label{prob:pn>0}
  Seien $q$ eine Primzahlpotenz und $n\in \N^\ast$. Ist
  $\PN(q,n)>0$?
\end{problem}
\begin{problem}[$\PCN(q,n)>0?$]
  \label{prob:pcn>0}
  Seien $q$ eine Primzahlpotenz und $n\in \N^\ast$. Ist
  $\PCN(q,n)>0$?
\end{problem}


Zuletzt wollen wir natürlich auch für $\G$ eine Problemstellung 
formulieren:

\begin{problem}[$n\in \G ?$]
  \label{prob:g}
  Finde möglichst viele $n\in \N^\ast$, $n\geq 2$ mit $n\in \G$.
\end{problem}

Bisher haben wir all diese Probleme nicht ausreichend geklärt. Doch im
folgenden Abschnitt wollen wir uns jenen Fragestellungen zunächst theoretisch
nähern, um in den darauffolgenden gezielte experimentelle Untersuchungen 
auf Grundlage der
theoretischen Resultate, die über (vollständig) normale Elemente im bisherigen
Verlauf erarbeitet wurden, durchzuführen.


\section{Theoretische Enumerationen und Exis\-tenz\-aus\-sa\-gen}


Wir starten mit einem wohlbekannten Resultat, das eine Antwort auf die Frage
nach der Existenz von normalen Elementen (\thref{prob:n>0}) gibt:

\begin{satz}[Satz von der Normalbasis]
  Zu jedem endlichen Körper $F$ und jeder endlichen Erweiterung $E$ von $F$
  existiert eine Normalbasis von $E$ über $F$.
\end{satz}
\begin{proof}
  \autocite[Theorem 2.35]{lidl1997finite}.
\end{proof}

Selbige Aussage können wir auch für vollständig normale Elemente treffen, was
zuerst \citeyear{blessenohl1986} von \citeauthor{blessenohl1986}
\autocite{blessenohl1986} bewiesen wurde.

\begin{satz}[Verschärfung des Satzes von der Normalbasis]
  Zu jedem endlichen Körper $F$ und jeder endlichen Erweiterung $E$ von $F$
  existiert eine vollständige Normalbasis von $E$ über $F$.
\end{satz}
\begin{proof}
  \autocite[Satz 1.2]{blessenohl1986}.
\end{proof}


Damit wäre auch \thref{prob:cn>0} beantwortet! Von den Existenzfragen bleibt
damit noch die Existenz von primitiv normalen und primitiv vollständig normalen
Elementen in beliebigen Erweiterungen offen. Erstere beantwortete
\citeauthor{lenstra1987} \citeyear{lenstra1987} \autocite{lenstra1987}
nach den Vorarbeiten von Carlitz \autocite{carlitz1952} und 
Davenport \autocite{davenport1968}.

\begin{satz}[Satz von der primitiven Normalbasis]
  \label{satz:primitive_normalbasis}
  Zu jedem endlichen Körper $F$ und jeder endlichen Erweiterung $E$ von $F$
  existiert eine primitive Normalbasis von $E$ über $F$.
\end{satz}
\begin{proof}
  \autocite{lenstra1987}.
\end{proof}


Bleibt also nur noch die Frage nach der Existenz primitiver vollständig
normaler Elemente. Auch wenn Hachenberger 
\citeyear{hachenberger2001} \autocite{hachenberger2001} und 
2014 \autocite{hachenberger2014} die beiden
nachstehenden bedeutsamen Resultate beweisen konnte, bleibt die Suche nach 
$\PCN$-Elementen weiterhin ein offenes Problem, dem wir uns im weiteren Verlauf
experimentell widmen wollen.

\begin{satz}
  \label{satz:pcn_in_regular}
  Seien $q$ eine Primzahlpotenz und $n \in \N^\ast$, so dass
  $\F_{q^n}$ über $\F_q$ eine reguläre Erweiterung ist. 
  %Sei ferner
  %$4\mid (q-1)$, falls $q$ ungerade und $n$ gerade ist. 
  Dann existiert ein
  primitives Element in $\F_{q^n}$, das vollständig normal über $\F_q$ ist.
\end{satz}
\begin{proof}
  \autocite[Theorem 1.4]{hachenberger2001} und 
  \autocite{hachenberger2014regexc}.
\end{proof}


\begin{satz}
  \label{satz:pcn_schranke}
  Sei $n\in \N^\ast$ mit $n\geq 2$. Dann gilt:
  Für Primzahlpotenzen $q$ mit $q \geq n^4$ existiert ein primitives Element in 
  $\F_{q^n}$, das vollständig normal über $\F_q$ ist.
\end{satz}
\begin{proof}
  \autocite[Theorem 2]{hachenberger2014}.
\end{proof}

Wir können nun zusammenfassen, dass von obigen Existenzproblemen lediglich
\thref{prob:pcn>0} überlebt hat und alle anderen durch theoretische Resultate
abgedeckt werden konnten. Nun können wir versuchen die Zählprobleme anzugehen
und starten mit der Wiederholung der Formel für normale Elemente aus
\thref{satz:anzahl_normal}.

\begin{satz}
  Seien $q$ eine Primzahlpotenz und $n\in \N^\ast$ mit $n\geq 2$, so existieren
  in $\F_{q^n}$ genau 
  \[ \phi_q(x^n-1) \speq= q^{n'(\pi-1)}\,\prod_{d\mid n}
    \left( q^{\ord_d(q)} -1 \right)^{\frac{\varphi(d)}{\ord_d(q)}}\]
  Elemente, die normal über $\F_q$ sind, wobei $n = n'\pi$ mit $\ggT(n',q) = 1$.
\end{satz}

Obiger Satz beantwortet also \thref{prob:n=} vollständig. Für die Anzahlen von 
primitiv normalen und primitiv vollständig normalen Elementen einer gegebenen
Körpererweiterung war es jedoch bisher nicht möglich Aussagen zu formulieren,
die alle Paare $(q,n)$ erfassen.

Ein kleiner Schritt in Richtung der Bestimmung der 
Anzahl von vollständig normalen Elementen 
besteht sicherlich in der Erkenntnis, dass für einfache Erweiterungen
(\thref{def:einfach}) diese Frage von \thref{satz:anzahl_normal} beantwortet
wird. Ferner war Hachenberger in \autocite[Section 21]{hachenberger1997finite}
in der Lage diese Frage für reguläre Kreisteilungsmoduln (und damit für
reguläre Erweiterungen) zu beantworten:

\begin{satz}
  \label{satz:anzahl_vollst_erzeuger}
  Seien $\F_q$ ein endlicher Körper von Charakteristik $p$ und $m \in \N^\ast$
  mit $\ggT(m,q)=1$. Ist dann $\C_{m,p^b}$ ein regulärer Kreisteilungsmodul
  über $\F_q$, so ist die Anzahl der vollständigen Erzeuger von 
  $\C_{m,p^b}$ gleich
  \begin{enumerate}
    \item $\displaystyle \left( q^{\frac{\ord_m(q)}{\tau}}-1 \right)%
      ^{\frac{\tau\,\varphi(m)}{\ord_m(q)}} \cdot 
      q^{(p^b-1)\varphi(m)}$,\quad falls $\C_{m,p^b}$ nicht ausfallend ist und
    \item $\displaystyle \left( q^{\frac{2\ord_m(q)}{\tau}}
      -4q^{\frac{\ord_m(q)}{\tau}}+3\right)%
      ^{\frac{\tau\,\varphi(m)}{2\ord_m(q)}} \cdot 
      q^{(p^b-1)\varphi(m)}$,\quad falls $\C_{m,p^b}$ ausfallend ist,
  \end{enumerate}
  wobei $\tau = \tau(q,n)$ aus \thref{def:tau}.
\end{satz}
\begin{proof}
  \autocite[Proposition 21.1, Proposition 21.2]{hachenberger1997finite}.
\end{proof}

Darüber hinaus lässt sich so eine Abschätzung für die Anzahl von vollständig
normalen Elementen einer regulären Körpererweiterung ableiten.

\begin{satz}
  Sei $\F_{q^m}$ über $\F_q$ eine reguläre Erweiterung endlicher Körper, so ist
  die Anzahl der vollständig normalen Elemente in $\F_{q^m}\mid \F_q$
  mindestens
  \[ (q-1)^{m'}\cdot q^{m'(p^b-1)} \,,\]
  wobei $m = m'\,p^b$ mit $\ggT(p,m')=1$ und $p$ der Charakteristik von $\F_q$.
  Gleichheit gilt genau dann, wenn $m' \mid (q-1)$ und die Erweiterung damit 
  einfach ist.
\end{satz}
\begin{proof}
  \autocite[Theorem 21.3, Theorem 6.1]{hachenberger1997finite}.
\end{proof}

Bei primitiv normalen und primitiv vollständig normalen Elementen
ist das Wissen über deren Anzahlen leider noch spärlicher gesät. 
Für $\PCN$-Elemente existieren die folgenden Abschätzungen.

\begin{satz}
  Sei $p$ eine Primzahl und $q$ eine Potenz von $p$. Dann gilt:
  \begin{enumerate}
    \item $\PCN(q,2^l) \geq 4(q-1)^{2^{l-2}}$, falls $q\equiv 3 \bmod 4$
      und $l\geq e+3$ für $e$ maximal, so dass $2^e\mid (q^2-1)$, oder
      $q\equiv 1\bmod 4$ und $l\geq 5$.
    \item $\PCN(q,r^l) \geq r^2(q-1)^{r^{l-2}}$, falls
      $r\neq p$ eine ungerade Primzahl und $l\geq 2$.
    \item $\PCN(q,r^l) \geq r(q-1)^{r^{l-1}}\varphi(q^{r^{l-1}}-1)$, falls
      $r\geq 7$, $\geq p$ eine Primzahl und $l\geq 2$.
    \item $\PCN(q,p^l) \geq p\,q^{p^{l-1}-1}\,(q-1)$, falls $l\geq 2$.
    \item $\PCN(q,p^l)\geq p\,q^{p^{l-1}-1}\,(q-1)\,\varphi(q^{p^{l-1}}-1)$,
      falls $p\geq 7$ und $l\geq 2$.
  \end{enumerate}
\end{satz}
\begin{proof}
  \autocite{hachenberger2010}.
\end{proof}

Wie man nun deutlich erkennen kann, besteht immer noch große Unklarheit über
die Anzahl von primitiv (vollständig) normalen Elementen und selbst die Zahl
vollständig normaler Elemente ist nicht hinreichend geklärt. Daher bietet es
sich freilich an, eine computergestützte Enumeration durchzuführen, um den
Nebel ein wenig mehr lichten zu können. 
\citeauthor{morgan1996} \autocite{morgan1996} haben bereits 
\citeyear{morgan1996} für $(q,n)$ aus 
$\{ (2,2),\ldots,(2,18),\allowbreak\ 
  (3,2),\ldots,(3,12),\allowbreak\ 
  (4,2),\ldots,(4,9),\allowbreak\ 
  (5,2),\ldots,(5,8),\allowbreak\ 
  (7,2),\ldots,(7,6),\allowbreak\ 
  (8,2),\ldots,(8,5),\allowbreak\ 
  (9,2),\ldots,(9,5)\}$ die Werte $\CN(q,n)$ und $\PCN(q,n)$ bestimmen können.
Betrachten wir einmal diese Zahlenkonstellationen und fragen uns, ob wir diese
nicht durch obige theoretische Resultate abdecken können, so müssen wir
feststellen, dass bei den Anzahlen der vollständig normalen Elemente lediglich
die Paare $(2,6), (2,10),(2,12),(2,18), (3,8), (3,10), (5,6)$ nicht einfach und
nicht regulär sind, also $\CN(q,n)$ nicht bereits aus obigen Sätzen
folgt.

Daher setzen wir uns im Folgenden das Ziel, die Tabelle von 
\citeauthor{morgan1996} einerseits zu verifizieren und andererseits zu
erweitern, um ein weitaus breiteres Spektrum an Zahlwerten präsentieren zu
können. Dabei sei angemerkt, dass in \autocite{morgan1996} die theoretischen
Überlegungen aus \autoref{chap:vollst_normalbasen} gänzlich unbeachtet blieben.
Wir wollen diese in nachfolgender Implementierung intensiv nutzen, um
bestmögliche Ausbeute vorhanderer Rechenleistung zu erhalten.
