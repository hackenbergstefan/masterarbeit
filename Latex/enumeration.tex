\chapter{Enumeration primitiv vollständig normaler Elemente}

In den vorangegangenen Kapiteln haben wir lediglich normale und vollständig
normale Elemente in Erweiterungen endlicher Körper betrachtet. Es existiert
jedoch eine weitere besondere Eigenschaft, die gerade in der Anwendung von
großem Interesse ist: Primitivität (\thref{def:primitiv}). Zusammen haben wir
nun drei Eigenschaften eines Elements $u \in E$ einer Erweiterung endlicher
Körper $E\mid F$ kennengelernt, die von Interesse sind. Daher ist es nur
sinnvoll sich der Frage zu widmen, \emph{wie viele} Elemente mit den jeweiligen
Eigenschaften es gibt. Mit \thref{satz:zykl_gruppen} und 
\thref{satz:mult_gruppe_endl_korper_zyklisch} ist sofort klar, dass es in
$\F_q$ genau $\varphi(q-1)$ primitive Elemente gibt! Daher ist die
Fragestellung nach der Anzahl primitiver Elemente schnell gelöst. Darüber
hinaus wollen wir die folgenden Notationen treffen.

\begin{definition}
  Seien $\F_q$ ein endlicher Körper und $n \in \N^\ast$, so bezeichne
  $\cal N(q,n)$, $\CN(q,n)$, $\PN(q,n)$ bzw. $\PCN(q,n)$ die Anzahl der 
  normalen, vollständig normalen, primitiv normalen bzw. primitiv vollständig
  normalen Elemente in $\F_{q^n}$ über $\F_q$, d.h.
  \begin{align*}
    \cal N(q,n) &\speq{:=} 
      |\{ u \in \F_{q^n}:\ u\text{ ist normal über }\F_q\}| \\
    \CN(q,n) &\speq{:=} 
      |\{ u \in \F_{q^n}:\ u\text{ ist vollständig normal über }\F_q\}| \\
    \PN(q,n) &\speq{:=} 
      |\{ u \in \F_{q^n}:\ u\text{ ist primitiv und normal über }\F_q\}| \\
    \PCN(q,n) &\speq{:=} 
      |\{ u \in \F_{q^n}:\ u\text{ ist primitiv und vollständig 
      normal über }\F_q\}| \\
    \G &\speq{:=} 
      \{ n\in \N^\ast, n\geq 2:\ 
      \forall q\text{ Primzahlpotenz gilt } \PCN(q,n) > 0 \}
  \end{align*}
\end{definition}

Vielleicht erscheint die Definition von $\G$ etwas überraschend, da für jedes
Element in $\G$ schließlich \emph{unendlich viele} Körpererweiterungen auf die
Existenz eines $\PCN$-Elements getestet werden müssen. Doch es sei an dieser
Stelle vorweg genommen, dass wir in der Lage sind mit Hilfe eines
asymptoptischen Resultats und der konkreten Angabe von endlich vielen
$\PCN$-Elementen zu zeigen, dass $\G$ nicht leer ist!

Nun können wir folgende Probleme definieren:

\begin{problem}[$\cal N(q,n)=?$]
  \label{prob:n=}
  Seien $q$ eine Primzahlpotenz und $n\in \N^\ast$. Was ist
  $\cal N(q,n)$?
\end{problem}
\begin{problem}[$\CN(q,n)=?$]
  \label{prob:cn=}
  Seien $q$ eine Primzahlpotenz und $n\in \N^\ast$. Was ist
  $\CN(q,n)$?
\end{problem}
\begin{problem}[$\PN(q,n)=?$]
  \label{prob:pn=}
  Seien $q$ eine Primzahlpotenz und $n\in \N^\ast$. Was ist
  $\PN(q,n)$?
\end{problem}
\begin{problem}[$\PCN(q,n)=?$]
  \label{prob:pcn=}
  Seien $q$ eine Primzahlpotenz und $n\in \N^\ast$. Was ist
  $\PCN(q,n)$?
\end{problem}

Offensichtlich können wir die obigen Problemstellungen leicht abschwächen und
uns zunächst fragen, ob überhaupt Elemente mit den geforderten Eigenschaften
existieren. Auch dazu wollen wir passende Probleme formulieren.

\begin{problem}[$\cal N(q,n)>0?$]
  \label{prob:n>0}
  Seien $q$ eine Primzahlpotenz und $n\in \N^\ast$. Ist
  $\cal N(q,n)>0$?
\end{problem}
\begin{problem}[$\CN(q,n)>0?$]
  \label{prob:cn>0}
  Seien $q$ eine Primzahlpotenz und $n\in \N^\ast$. Ist
  $\CN(q,n)>0$?
\end{problem}
\begin{problem}[$\PN(q,n)>0?$]
  \label{prob:pn>0}
  Seien $q$ eine Primzahlpotenz und $n\in \N^\ast$. Ist
  $\PN(q,n)>0$?
\end{problem}
\begin{problem}[$\PCN(q,n)>0?$]
  \label{prob:pcn>0}
  Seien $q$ eine Primzahlpotenz und $n\in \N^\ast$. Ist
  $\PCN(q,n)>0$?
\end{problem}


Zuletzt wollen wir natürlich auch für $\G$ eine Problemstellung 
zu formulieren:

\begin{problem}[$n\in \G ?$]
  Finde möglichst viele $n\in \N^\ast$, $n\geq 2$ mit $n\in \G$.
\end{problem}

Bisher haben wir all diese Probleme nicht ausreichend geklärt. Doch im
folgenden Abschnitt wollen wir uns jenen Fragestellungen zunächst theoretisch
widmen, um in den darauffolgenden gezielte Enumerationen auf Basis der
theoretischen Resultate, die über (vollständig) normale Elemente im bisherigen
Verlauf erarbeitet wurden, durchzuführen, um für die offen bleibenden Fragen 


\section{Theoretische Enumerationen}


Wir starten mit einem wohlbekannten Resultat, das eine Antwort auf die Frage
nach der Existenz von normalen Elementen (\thref{prob:n>0}) gibt:

\begin{satz}[Satz von der Normalbasis]
  Zu jedem endlichen Körper $F$ und jeder endlichen Erweiterung $E$ von $F$
  existiert eine Normalbasis von $E$ über $F$.
\end{satz}
\begin{proof}
  \autocite[Theorem 2.35]{lidl1997finite}.
\end{proof}

Selbige Aussage können wir auch für vollständig normale Elemente treffen, was
zuerst \citeyear{blessenohl1986} von \citeauthor{blessenohl1986}
\autocite{blessenohl1986} bewiesen wurde.

\begin{satz}[Verschärfung des Satzes von der Normalbasis]
  Zu jedem endlichen Körper $F$ und jeder endlichen Erweiterung $E$ von $F$
  existiert eine vollständige Normalbasis von $E$ über $F$.
\end{satz}
\begin{proof}
  \autocite[Satz 1.2]{blessenohl1986}.
\end{proof}


Damit wäre auch \thref{prob:cn>0} beantwortet! Von den Existenzfragen bleibt
damit noch die Existenz von primitiv normalen und primitiv vollständig normalen
Elementen in beliebigen Erweiterungen offen. Erstere beantwortete
\citeauthor{lenstra1987} \citeyear{lenstra1987} \autocite{lenstra1987}
nach den Vorarbeiten von Carlitz und Davenport.

\begin{satz}[Satz von der primitiven Normalbasis]
  Zu jedem endlichen Körper $F$ und jeder endlichen Erweiterung $E$ von $F$
  existiert eine primitive Normalbasis von $E$ über $F$.
\end{satz}
\begin{proof}
  \autocite{lenstra1987}.
\end{proof}


Bleibt also nur noch die Frage nach der Existenz primitiver vollständig
normaler Elemente. Auch wenn Hachenberger 
\citeyear{hachenberger2001} \autocite{hachenberger2001} und 
\citeyear{hachenberger2014} \autocite{hachenberger2014} die beiden
nachstehenden bedeutsamen Resultate beweisen konnte, bleibt die Suche nach 
$\PCN$-Elementen weiterhin ein offenes Problem, dem wir uns im weiteren Verlauf
experimentell widmen wollen.

\begin{satz}
  Seien $q$ eine Primzahlpotenz und $n \in \N^\ast$, so dass
  $\F_{q^n}$ über $\F_q$ eine reguläre Erweiterung ist. Sei ferner
  $4\mid (q-1)$, falls $q$ ungerade und $n$ gerade ist. Dann existiert ein
  primitives Element in $\F_{q^n}$, das vollständig normal über $\F_q$ ist.
\end{satz}
\begin{proof}
  \autocite[Theorem 1.4]{hachenberger2001}.
\end{proof}


\begin{satz}
  Sei $n\in \N^\ast$ mit $n\geq 2$. Dann gilt:
  Für Primzahlpotenzen $q$ mit $q \geq n^4$ existiert ein primitives Element in 
  $\F_{q^n}$, das vollständig normal über $\F_q$ ist.
\end{satz}
\begin{proof}
  \autocite[Theorem 2]{hachenberger2014}.
\end{proof}

Wir können nun zusammenfassen, dass von obigen Existenzproblemen lediglich
\thref{prob:pcn>0} überlebt hat und alle anderen durch theoretische Resultate
abgedeckt werden konnten. Nun können wir versuchen die Zählprobleme anzugehen
und starten mit einem allgemein bekannten Resultat.

\begin{definition}
  Sei $f(x) \in \F_q[x]$ ein Polynom über einem endlichen Körper. Definiere
  \[ \phi_q(f) \speq{:=} |\{ g(x) \in \F_q[x]:\ 
    \deg g < \deg f,\ \ggT(f,g) = 1\}|\,.\]
\end{definition}

\begin{bemerkung}
  $\phi_q(f)$ ist das Analogon zur Eulerschen Phifunktion für Polynome, da
   $\phi_q(f)$ gerade die Anzahl der Einheiten im Ring $\F_q[x]\big/(f(x))$
   angibt.
\end{bemerkung}

\begin{satz}
  Seien $q$ eine Primzahlpotenz und $n\in \N^\ast$ mit $n\geq 2$, so existieren
  in $\F_{q^n}$ genau 
  \[ \phi_q(x^n-1) \speq= q^{n'(\pi-1)}\,\prod_{d\mid n}
    \left( q^{\ord_d(q)} -1 \right)^{\frac{\varphi(d)}{\ord_d(q)}}\]
  Elemente, die normal über $\F_q$ sind, wobei $n = n'\pi$ mit $\ggT(n',q) = 1$.
\end{satz}
\begin{proof}
  \autocite[Theorem 3.73]{lidl1997finite} oder 
  \autocite[Theorem 10.5]{hachenberger1997finite}.
\end{proof}

Obiger Satz beantwortet also \thref{prob:n=} vollständig.
