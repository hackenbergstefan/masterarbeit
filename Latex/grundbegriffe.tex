\chapter{Grundbegriffe}
\label{chap:grundbegriffe}

%Wir eröffnen den Hauptteil der Arbeit mit dem Zusammentragen
Tragen wir zunächst 
einige grundlegende Resultate zusammen, die dem Leser sicherlich bekannt sind. 
Daher zitieren wir die meisten Aussagen 
lediglich ohne Beweis. Wir beginnen dabei bei der Gruppentheorie und
insbesondere mit zyklischen Gruppen. Diese werden uns später
helfen, die Untergruppe der Einheitswurzeln
in \autoref{chap:kreisteilungspolynome} zu verstehen. Im anschließenden Abschnitt
rekapitulieren wir ein wenig die Galoistheorie von endlichen Körpern.
Insbesondere wollen wir wiederholen, dass die 
Galoisgruppe endlicher Körper zyklisch ist und von einem
speziellen Automorphismus erzeugt wird.

\section{Ein wenig Gruppentheorie}

%Um später den Zerfall der Kreisteilungspolynome über endlichen Körpern zu
%verstehen, wiederholen wir zunächst ein paar Aussagen über zyklische Gruppen.

\autocite[Theorem 1.15]{lidl1997finite} fasst alle notwendigen Resultate
zusammen.

\begin{satz}
  \label{satz:zykl_gruppen}
  \begin{enumerate}
    \item Jede Untergruppe einer zyklischen Gruppe ist wieder zyklisch.
    \item Sei $\langle a \rangle$ eine zyklische Gruppe der Ordnung $m$,
      so erzeugt $a^k$ eine Untergruppe der Ordnung $\frac{m}{\ggT(m,k)}$.
    \item Sei $\langle a\rangle$ eine zyklische Gruppe der Ordnung $m$ und
      $d \mid m$, so enthält $\langle a \rangle$ genau eine Untergruppe der
      Ordnung $d$.
     \item Sei $f$ ein positiver Teiler der Gruppenordnung einer endlichen
        zyklischen Gruppe $\langle a \rangle$. Dann enthält $\langle a \rangle$
        genau $\varphi(f)$ Elemente der Ordnung $f$.
        ($\varphi$ bezeichne die Eulersche Phifunktion)
     \item Eine zyklische Gruppe der Ordnung $m$ enthält genau $\varphi(m)$
        Erzeuger. Ist $a$ ein Erzeuger, so sind alle Erzeuger der Form
        $a^r$ mit $\ggT(r,m) = 1$.
  \end{enumerate}
\end{satz}

Da wir später ein paar Eigenschaften benötigen
werden, wiederholen wir die wohlbekannte Definition der Eulerschen
Phifunktion und geben dann die wichtigsten Rechenregeln an.

\begin{definition}[Eulersche Phifunktion]
  \label{def:euler_phi}
  %\nomenclature{$\varphi$}{Eulersche Phifunktion}
  Die Funktion
  \[ \varphi: \funcdef{\N^\ast &\to& \N^\ast,\\
    n &\mapsto& |\{ a \in \N:\ 1\leq a\leq n,\ \ggT(a,n)=1 \}|}\]
  heißt \emph{Eulersche Phifunktion}.
\end{definition}

\begin{definition}[quadratfreier Teil]
  \label{def:quadratfreier_teil}
  Sei $n \in \N$ und $n = p_1^{r_1}\cdot\ldots\cdot p_l^{r_l}$ seine
  Primfaktorzerlegung. Dann heißt
  \[ \nu(n) \speq{:=} p_1\cdot \ldots\cdot p_l\]
  \emph{quadratfreier Teil von $n$}.
\end{definition}


\begin{lemma}[Rechenregeln der Eulerschen Phifunktion]
  \label{lemma:rechenregeln_phifunktion}
  Seien $a,b\in\N^\ast$, so gilt
  \begin{enumerate}
    \item $\varphi(ab) = \varphi(a)\varphi(b)$, falls $\ggT(a,b) = 1$,
    \item $a = \sum_{d\mid a} \varphi(d)$ und
    \item $\varphi(a) = \tfrac{a}{\nu(a)}\varphi(\nu(a))$.
  \end{enumerate}
\end{lemma}


Zyklische Gruppen und endliche Körper hängen eng zusammen, da bekanntlich die
multiplikative Gruppe eines endlichen Körpers immer zyklisch ist. Dies können
wir nutzen, um Erzeugern (im Sinne der Gruppentheorie) der multiplikativen
Gruppe eines endlichen Körpers einen Namen zu geben.

\begin{satz}
  \label{satz:mult_gruppe_endl_korper_zyklisch}
  Die multiplikative Gruppe eines endlichen Körpers ist zyklisch.
\end{satz}
\begin{proof}
  \autocite[Theorem 2.8]{lidl1997finite}.
\end{proof}


\begin{definition}[primitiv]
  \label{def:primitiv}
  Sei $\F_q$ ein endlicher Körper. $u\in \F_q$ heißt \emph{primitiv} 
  (oder \emph{primitives Element}), falls $\langle u \rangle = \F_q^\ast$, 
  also $u$ ein Erzeuger der multiplikativen Gruppe $\F_q^\ast$ ist.
\end{definition}


\begin{bemerkung}
  Es ist klar, dass $u\in \F_q$ genau dann primitiv ist, wenn 
  $\ord(u) = q-1$, also seine gruppentheoretische Ordnung in $\F_q^\ast$ genau
  der Gruppenordnung entspricht.
\end{bemerkung}


\section{Automorphismen über endlichen Körpern}

\begin{satz}
  \label{satz:frob_auto}
  Seien $F := \F_q$ ein endlicher Körper der Charakteristik $p\neq 0$ und 
  $n\in \N^\ast$. Dann ist
  \[ \sigma_n: \funcdef{F &\to& F\,,\\
    a &\mapsto& a^{p^n}}\]
  ein Automorphismus auf $F$.
\end{satz}
\begin{proof}
  \autocite[Corollary 3.18]{wan2003lectures}.
\end{proof}

\begin{bemerkung}
  Insbesondere gilt also für alle $a,b\in F$, $F$ wie oben:
  \[ (a\pm b)^{p^n} = a^{p^n} \pm b^{p^n}\,.\]
\end{bemerkung}

\begin{satz}
  \label{satz:frob_fix}
  Sei $q$ eine Primzahlpotenz und $n\in \N^\ast$. Der Automorphismus
  \[ \sigma: \funcdef{\F_{q^n} &\to& \F_{q^n}\,,\\
    a &\mapsto& a^q}\]
  hält die Elemente von $\F_q$ fest, also 
  \[ \sigma |_{\F_q} = \id_{\F_q}\,.\]
  Ferner ist $\sigma^k \neq \id_{\F_{q^n}}$ für $k=1,\ldots,n-1$, alle
  $\sigma^k$s sind paarweise verschiedene Automorphismen und 
  $\sigma^n = \id_{\F_{q^n}}$.
  $\sigma$ heißt auch \emph{Frobenius-Endomorphismus},
  \emph{Frobenius-Automorphismus} oder kurz \emph{Frobenius}.
\end{satz}
\begin{proof}
  \autocite[Theorem 7.1]{wan2003lectures}.
\end{proof}


Bezeichne $\Gal(E \mid F)$ die Galoisgruppe einer Galoiserweiterung $E$ über
$F$, so können wir das folgende zentrale Resultat zitieren:

\begin{satz}
  \label{satz:frob_sind_alle_autos}
  Es gilt
  \[ \Gal(\F_{q^n}\mid \F_q) \speq= \langle \sigma\rangle\,.\]
  Das bedeutet, dass es neben $\sigma^0,\sigma,\ldots,\sigma^{n-1}$ keine weiteren
  Automorphismen von $\F_{q^n}$ gibt, die $\F_q$ fixieren.
\end{satz}
\begin{proof}
  \autocite[Theorem 7.3]{wan2003lectures}.
\end{proof}

Neben der Tatsache, dass der Frobenius-Automorphismus 
alle Elemente der Galoisgruppe erzeugt, können wir
auch zeigen, dass alle Potenzen des Frobenius von $\F_{q^n}$ über $\F_q$ linear
unabhängig sind. Dies gilt sogar in einem größeren Kontext:

\begin{satz}[Dedekindsches Lemma]
  \label{satz:dedekindsches_lemma}
  Seien $K,L$ zwei Körper, $n \in \N$ und $\tau_1,\ldots,\tau_n: K\to L$
  verschiedene injektive Körperhomorphismen. 
  Sind $c_1,\ldots,c_n \in L$ mit
  \[ c_1\tau_1(x)+ \ldots+ c_n\tau_n(x)
    \quad\text{für jedes $x\in K$}\,,\]
  so gilt $c_1=\ldots=c_n=0$.
\end{satz}
\begin{proof}
  \autocite[Satz 27.2]{karpfinger2010algebra}.
\end{proof}


Mit \thref{satz:frob_sind_alle_autos} wird klar, dass für ein
irreduzibles Polynom $f(x) \in \F_q[x]$, das in $\F_{q^n}$ eine Nullstelle 
$\alpha$ besitzt, auch $\sigma^i(\alpha)$ für alle $i=1,\ldots,n-1$
Nullstellen sind. Ferner kann man sich auch relativ leicht überlegen, dass auch
jedes Polynom $f(x) \in \F_q[x]$ vom Grad $n$ eine Nullstelle in 
$\F_{q^n}$ besitzt. Beides fasst nachstehender Satz zusammen.

\begin{satz}
  \label{satz:nst_irred_polys}
  Sei $f(x) \in \F_q[x]$ ein irreduzibles Polynom vom Grad $n$. Dann 
  existiert eine Nullstelle $\alpha$ von $f(x)$ in $\F_{q^n}$, alle 
  Nullstellen von $f(x)$ sind einfach und gegeben durch
  \[ \alpha, \alpha^q, \alpha^{q^2}, \ldots, \alpha^{q^{n-1}}\ \in \F_{q^n}\,.\]
\end{satz}
\begin{proof}
  \autocite[Theorem 2.14]{lidl1997finite}.
\end{proof}

