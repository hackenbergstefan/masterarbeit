\chapter{Der Zerfall von $x^n-1$ und die Kreisteilungspolynome $\Phi_D(x)$}
\label{chap:kreisteilungspolynome}

Sei $K$ ein beliebiger Körper der Charakteristik $p$ 
und $\bar K$ ein fest gewählter algebraischer
Abschluss.. Wir wollen nun untersuchen, wie das Polynom
$x^n-1 \in K[x]$ über $K$ zerfällt. Dazu orientieren wir uns 
an \autocite{lidl1997finite} und \autocite{wan2003lectures}.

\begin{definition}[Kreisteilungskörper, Einheitswurzeln]
  Sei $n\in\N^\ast$. Der Zerfällungskörper von $X^n-1 \in K[X]$ heißt
  der \emph{$n$-te Kreisteilungskörper} und wird mit $K\kl n$ notiert.
  Die Nullstellen von $X^n-1$ in $K\kl n$ heißen \emph{$n$-ten
  Einheitswurzeln} und die Menge derer wird mit $E\kl n$ bezeichnet.
\end{definition}


\begin{satz}
  Sei $n\in \N^\ast$.
  \begin{enumerate}
    \item Sei $p\nmid n$. Dann ist $E\kl n$ eine zyklische Gruppe (bzgl. der
      Multiplikation in $K\kl n$) der Ordnung $n$.
    \item Ist $p \mid n$ und schreibt man $n = p^e m$ 
      für positive ganze Zahlen $m$ und $e$ mit $p\nmid m$, so
      ist $K\kl n = K\kl m$ und $E\kl n= E\kl m$ und die Nullstellen von
      $X^n-1 \in K[X]$ sind gerade die Elemente in $E\kl m$ jedoch jeweils mit
      Multiplizität $p^e$.
  \end{enumerate}
\end{satz}
\begin{proof}
  \autocite[Theorem 2.42]{lidl1997finite}.
\end{proof}


\begin{definition}[primitive Einheitswurzeln]
  Sei $n\in \N^\ast$ und $p\nmid n$. Dann heißen die Erzeuger von 
  $E\kl n$ \emph{primitive $n$-te Einheitswurzeln}. 
  Die Untergruppe der primitiven Einheitswurzeln wird mit
  $C\kl n$ bezeichnet.
\end{definition}


\begin{definition}[Kreisteilungspolynom]
  Seien $n \in \N^\ast$, $p\nmid n$ und $\zeta \in C\kl n$. Das Polynom 
  \[ \Phi_d(X) \speq{:=} \prod_{i=1\atop \ggT(i,n)=1}^n (X - \zeta^i)\]
  heißt \emph{$d$-tes Kreisteilungspolynom}.
\end{definition}

\begin{satz}
  \label{satz:zerfall_xn-1}
  Seien $K$ ein Körper der Charakteristik $p$ und $n \in \N^\ast$ mit $p\nmid
  n$. Dann gilt:
  \begin{enumerate}
    \item $X^n-1 = \prod_{d\mid n} \Phi_d(X)$.
    \item $\Phi_n(X) \in P[X]$, wobei $P\subset K$ den Unterkörper mit $p$
      Elementen notiert.
  \end{enumerate}
\end{satz}
\begin{proof}
  \begin{enumerate}
    \item Dies ist eine einfache Folgerung aus \thref{satz:zykl_gruppen}.
    \item Lässt sich per Induktion recht einfach beweisen 
      (vgl. \autocite[Theorem 2.45 (ii)]{lidl1997finite}).
  \end{enumerate}
\end{proof}


\begin{definition}
  Für zwei teilerfremde natürliche Zahlen $q,n$ größer Null sei
  \[ \ord_n(q) \speq{:=} \min\{k \mid q^k \equiv 1 \bmod n\}\,.\]
\end{definition}

Damit können wir nun zu einem zentralen Resultat dieses Abschnittes kommen, das
uns über die gesamte Arbeit hinweg begleiten wird.

\begin{satz}
  Seien $q$ eine Primzahlpotenz und $n\in \N^\ast$ mit $\ggT(q,n)=1$. Dann
  zerfällt das $n$-te Kreisteilungspolynom $\Phi_n(X)$ über $\F_q$ in
  \[ \frac{\varphi(n)}{\ord_n(q)}\]
  irreduzible paarweise teilerfremde Polynome von jeweils Grad $\ord_n(q)$.
\end{satz}
\begin{proof}
  Sei $f(X) \mid \Phi_n(X)$ ein irreduzibler Teiler über $\F_q$. Ist dann
  $\zeta \in C\kl n$ eine Nullstelle von $f(X)$, so sind 
  nach \thref{satz:nst_irred_polys} auch
  \[ \zeta^q, \zeta^{q^2}, \ldots, \zeta^{q^n} \]
  Nullstellen von $f(X)$. Jedoch sind offenbar nur $\ord_n(q)$ dieser 
  verschieden und da $f$ als irreduzibles Polynom wieder nach 
  \thref{satz:nst_irred_polys} nur einfache Nullstellen besitzt,
  können wir folgern, dass $\deg f = \ord_n(q)$.
  Da $f(X)$ als beliebiger irreduzibler Teiler von $\Phi_n(X)$ gewählt wurde,
  folgt sofort die Behauptung, wenn man sich überlegt, dass der Grad des
  $n$-ten Kreisteilungspolynoms per definitionem gerade $\varphi(n)$ ist.
\end{proof}


Im Beweis obigen Satzes haben wir gesehen, dass die Wirkung der Galoisgruppe
$\Gal(\F_{q^n}\mid \F_q)$ auf der Menge der primitiven $n$-ten Einheitswurzeln
$C\kl n$ (die Wirkung ist selbstredend durch Einsetzen gegeben) diese in 
Teilmengen der Mächtigkeit $\ord_n(q)$ zerlegt.
Dies lässt sich natürlich auf $E\kl n$ übertragen, da ja gerade 
nach \thref{satz:zykl_gruppen}
$E\kl n = \bigcupdot_{d\mid n} C\kl d$. 
Dies motiviert nachstehende Definition.


%\begin{lemma}
  %\label{lemma:uber-pi-m-1}
  
%\end{lemma}

%\begin{lemma}
  %\label{lemma:uber-pi-m-2}
  %Für natürliche Zahlen $m$, $k$ gilt
  %\[ k \speq= \pi_m(k)\cdot l \]
  %mit $\ggT(m \pi_m(k), l) = 1$.
%\end{lemma}
%\begin{proof}
  %Dies sieht man sehr leicht, wenn man sich die Primfaktorzerlegungen der
  %gegebenen Zahlen zu Gemüte führt: Sei also 
  %\begin{align*}
    %m \quad&=\quad p_1^{\nu_1} \cdot\ldots\cdot p_s^{\nu_s}\ \cdot\
      %r_1^{\eta_1}\cdot\ldots\cdot r_t^{\eta_t}\,,\\
    %k \quad&=\quad p_1^{\nu'_1} \cdot\ldots\cdot p_s^{\nu'_s}
      %\ \cdot\ {r'}_1^{{\eta'}_1} \cdot\ldots\cdot {r'}_{t'}^{{\eta'}_{t'}}\,,
  %\end{align*}
  %für geeignete $\nu_\cdot,\nu'_\cdot,\eta_\cdot,\eta'_\cdot > 0$, wobei
  %natürlich $\ggT(r_1\cdot\ldots\cdot r_t, r'_1\cdot\ldots\cdot r'_{t'}) = 1$.
  %Dann ist $\pi_m(k)$ der größte Teiler von $k$, dessen quadratfreier Teil, den
  %quadratfreien Teil von $m$ teilt, also
  %\[ \pi_m(k) \speq= p_1^{\nu'_1}\cdot\ldots\cdot p_s^{\nu'_s}\,.\]
  %Damit ist $l = {r'}_1^{{\eta'}_1}\cdot\ldots\cdot {r'}_{t'}^{{\eta'}_{t'}}$ 
  %und die Behauptung folgt sofort.
%\end{proof} 

\begin{definition}
  Für $m,q\in\N$ mit $\ggT(m,q) = 1$ und $j \in \Z_m$ definieren wir
  \[ M_q(j\bmod m) \speq{:=} \{ j\,q^i \bmod m\mid i\in\N\} \speq= 
    \{j,\ jq,\ jq^2,\ jq^3,\ldots\ \bmod m\}\]
  Ein vollständiges Repräsentantensystem von Nebenklassen von $M_q(1\bmod m)$
  in $\Z_m$ sei mit $R_q(m)$ bezeichnet.
\end{definition}


\begin{beispiel}
  Wollen wir den Zerfall von $X^{21}-1$ über $\F_2$ untersuchen, so berechnen
  wir erst ein Vertretersystem von Restklassen $\bmod{21}$:
  \[\begin{array}[t]{r|l}
    l \in R_2(21) & M_2(l \bmod{21}) \\\hline
    0 & 0 \\
    1 & 1, 2, 4, 8, 11, 16 \\
    3 & 3, 6, 12 \\
    5 & 5, 10, 13, 17, 19, 20 \\
    7 & 7, 14 \\
    9 & 9, 15, 18
  \end{array}\]
  Nun wissen wir aus \thref{satz:zerfall_xn-1}, dass 
  \[ X^{21} -1 = \Phi_1(X) \cdot \Phi_3(X) \cdot \Phi_7(X) \cdot
  \Phi_{21}(X)\,.\]
  Ist $\zeta \in C\kl{21}$ eine primitive $21$-te Einheitswurzel, so 
  partitionieren sich die Nullstellen von $\Phi_{21}(X)$ gerade in
  diejenigen $M_2(l\bmod{21})$ für die $l=1,5$, da wir ja gerade diejenigen
  $i\in \{1,\ldots,20\}$ suchen für die $\zeta^i$ Erzeuger von $C\kl{21}$ ist.
  Also haben wir 
  \[\footnotesize\setlength{\arraycolsep}{2pt}\everymath{\displaystyle}
    \begin{array}{rcccc} 
      \Phi_{21}(X) &=& (X^{6} + X^{4} + X^{2} + X + 1) 
        &\cdot& (X^{6} + X^{5} + X^{4} + X^{2} + 1) \\
      &=& (X-\zeta)(X-\zeta^2)(X-\zeta^4)(X-\zeta^8)(X-\zeta^{11})
        (X-\zeta^{16}) &\cdot&
        (X-\zeta^5)(X-\zeta^{10})(X-\zeta^{13})(X-\zeta^{17})
        (X-\zeta^{19})(X-\zeta^{20})\,.
    \end{array}\]
  Analog erhalten wir den Zerfall von $\Phi_7(X)$ durch Betrachtung der
  $M_2(l\bmod{21})$ für $l=3,9$. 
  \[\footnotesize\setlength{\arraycolsep}{2pt}\everymath{\displaystyle}
    \begin{array}{rcccc} 
      \Phi_7(X)&=& (X^{3} + X + 1) &\cdot& (X^{3} + X^{2} + 1) \\
      &=& (X-\zeta^3)(X-\zeta^{3\cdot 2})(X-\zeta^{3\cdot 4}) &\cdot&
        (X-\zeta^{3\cdot 3})(X-\zeta^{3\cdot 5})(X-\zeta^{3\cdot 6}) 
    \end{array}\]
  Sammeln wir den Rest auf, erhalten wir die Partitionierung für
  $\Phi_3(X)$ und den trivialen Fall $\Phi_1(X)$.
  \[ \begin{array}{rcc} 
    \Phi_3(X) &=& X^2 + X + 1\\
              &=& (X-\zeta^7)(X-\zeta^{14})\,,\\[10pt]
    \Phi_1(X) &=& X-1\,.
    \end{array}\]
\end{beispiel}


\begin{satz}
  \label{satz:f_x_s_ist_teiler_von_phimd}
  Seien $q=p^r$ eine Primzahlpotenz und
  $m,t\in \N$ mit $p\nmid m$, $p\nmid t$ und $\ggT(m,t) = 1$. Ist dann 
  $f(x) \mid \Phi_m(x)$ ein über $\F_q$ irreduzibler monischer Teiler des
  $m$-ten Kreisteilungspolynoms. Definieren wir 
  ferner 
  \[ \Delta_q(m,d) \speq{:=} \frac{\varphi(d) \ord_m(q)}{\ord_{md}(q)}\,,\]
  so gilt:
  \[ f(x^t) \speq= \prod_{d\mid t} \prod_{i=1}^{\Delta_q(m,d)} f_{d,i}(x)\,,\]
  wobei für alle $i=1,\ldots,\Delta_q(m,d)$ 
  \[ f_{d,i}\in \F_q[x] 
    \text{ monisch, irreduzibel und } f_{d,i}(x) \mid \Phi_{md}(x)\,.\]
  Ferner sind alle $f_{d,i}(x)$ paarweise teilerfremd.
\end{satz}
\begin{proof}
  Sei für $n\in \N$
  \[ U_n := \{ \lambda \in \bar\F_q:\ \ord(\lambda)\mid n\}\]
  die Menge der $n$-ten Einheitswurzeln und
  \[ C_n := \{\lambda \in \bar \F_q:\ \ord(\lambda) = n\}\]
  die Menge der primitiven $n$-ten Einheitswurzeln, so können wir folgenden
  Gruppenhomomorphismus betrachten:
  \[ \psi: \funcdef{U_{mt} &\to& U_t\,,\\ \lambda&\mapsto& \lambda^t\,.}\]
  Offensichtlich ist $\ker\psi = U_t$. Da $\ggT(m,t) = 1$, also 
  $U_{mt} = U_m \odot U_t$ nach \thref{}, ist $\psi$ auch surjektiv.

  Ist nun $\alpha \in C_m$ eine Nullstelle von $f(x)$, so existiert -- wiederum
  weil $m$ und $t$ teilerfremd sind -- genau ein $\beta \in C_m$ mit
  $\beta^t = \alpha$. Damit ist also
  \[ \psi\inv(\{\alpha\}) \speq= \beta\, U_t \speq=
    \bigcupdot_{d\mid t}\ \beta\, C_d \,.\]
  Sei nun $\sigma_F: \bar\F_q \to \bar\F_q, x\mapsto x^q$ der Frobenius von
  $\F_q$, so sind $\sigma^j(\alpha)$, $j=0,\ldots,\delta-1$ für 
  $\delta=\ord_q(m)$ die Nullstellen von $f(x)$. Damit ist die Menge der
  Nullstellen von $f(x^t)$ gerade
  \[ \bigcupdot_{j=0}^{\delta-1} \sigma^j(\beta)\, U_t \speq=
    \bigcupdot_{j=0}^{\delta-1}\ \bigcupdot_{d\mid t}\ \beta^{q^j}\, C_d
    \speq=
    \bigcupdot_{d\mid t}\ \bigcupdot_{j=0}^{\delta-1}\ \beta^{q^j}\, C_d
    \speq{=:} \bigcupdot_{d\mid t} N_d\,. \]
  Wollen wir nun einsehen, wie $f(x^t)$ über $\F_q$ zerfällt, so müssen wir
  überlegen, wie obige Nullstellenmenge in $\sigma$-invariante Teilmengen
  zerfällt. Für jedes $d\mit t$ und jedes $j\in\{0,\ldots,\delta -1\}$ ist
  $\zeta \in \beta^{q^j}\,C_d$ ein Element mit $\ord(\zeta) = md$, also
  Nullstelle von $\Phi_{md}(x)$. Ferner gilt
  offenbar $\forall d\mid t:\ |N_d| = \delta\varphi(d)$
  und wir können folgern, dass $N_d$ in genau
  \[ \frac{\delta\varphi(d)}{\ord_{md}(q)} \speq= \frac{\ord_m(q)\,
    \varphi(d)}{\ord_{md}(q)} \speq=
    \Delta_q(m,d)\]
  $\sigma$-invariante Teilmengen zerfällt. $\Delta_q(m,d)$ ist in der Tat eine
  natürliche Zahl größer 0, da nach \thref{}
  \[ \frac{\ord_m(q)\,\varphi(d)}{\ord_{md}(q)} \speq=
    \frac{\varphi(d)\, \ggT(\ord_m(q),\ord_d(q))}{\ord_d(q)}\]
  und $\ord_d(q)$, also die multiplikative Ordnung von $q$ in $\Z_d$, ein
  Teiler der Gruppenordnung der Einheiten von $\Z_d$, also $\varphi(d)$, ist.
  Damit ist alles gezeigt.
\end{proof}

%\begin{satz}
  %\label{satz:f_x_s_ist_teiler_von_phimd}
  %Seien $q=p^r$ eine Primzahlpotenz, $F = \F_q$ ein endlicher Körper,
  %$m$ eine natürliche Zahl und $s = \ord_{\nu(m)}(q)$ mit 
  %$\ggT(s,m) = 1$. Weiter sei $s = \bar s p^\beta$ mit 
  %$\ggT(\bar s, p) = 1$. Ist ferner $f(x) \mid \Phi_m(x)$ ein über $\F_q$
  %irreduzibler monischer Teiler des $m$-ten Kreisteilungspolynoms, so gilt:
  %\[ f(x^s) \speq= \prod_{i=1}^? f_{i}(x)^{p^\beta}\,,\]
  %so dass $f_i(x)$ ein irreduzibler monischer Teiler von $\Phi_{md(i)}(x)$ ist
  %für $d:\ \{1,\ldots,?\} \to \{\text{Teiler von } \bar s\},\ 
  %i\mapsto d(i)$ surjektiv und monoton. Ferner gilt sogar 
  %$d(1) < d(2)$.
%\end{satz}
%\marginpar{Leider habe ich keine Ahnung, wie $i\mapsto d(i)$ aussieht und
  %was $?$ ist. Insbesondere gelingt es mir nicht, die Surjektivität zu
  %beweisen.}
%\begin{proof}
  %Zunächst ist klar, dass wir \obda annehmen können, dass $p \nmid s$. Ist
  %nämlich $s = \bar s p^\beta$ wie oben, so gilt
  %\[ f(x^s) = f(x^{\bar s})^{p^\beta}\,,\]
  %da $\F_q \to \F_q, x \mapsto x^p$ bekanntlich ein Ringhomomorphismus ist.
  %Ist dann $\zeta \in E$ eine primitive $(ms)$-te Einheitswurzel für
  %$E = F_e$, $e = \ord_{ms}(q)$, so ist $\ord(\zeta^s) = m$ und es gilt nach
  %\cref{} \marginpar{Reference!}
  %\[ \Phi_m(x) \speq= \prod_{j \in R_q(m)\atop \ggT(j,m) = 1}
    %\prod_{i\in M_q(j\bmod m)} \ (x - \zeta^{si} ) \]
  %Also haben wir
  %\[ f(x) \speq= \prod_{i\in M_q(j_0 \bmod m)} \ (x - \zeta^{si})\]
  %für ein $j_0\in R_q(m)$.
  %Betrachten wir nun einen Linearfaktor von $f(x^s)$, so wollen wir 
  %zeigen, dass
  %\[g(x) := (x^s - \zeta^s) \speq= \prod_{i=0}^{s-1} ( x - \zeta^{im +1} )\,.\]
  %Dies ist aber nicht schwer zu sehen, da für $i \in \{0,\ldots,s-1\}$
  %\[ g(\zeta^{in+1}) \speq= \zeta^{ism + s} - \zeta^s \speq= 0 \,.\]
  %Da $\zeta$ primitive $(ms)$-te Einheitswurzel ist, haben wir $s$ paarweise
  %verschiedene Nullstellen von $g$ gefunden und obige Behauptung gezeigt.
  %Nun betrachten wir die Aufteilung der Kreisteilungspolynome:
  %Nach \cref{} haben wir
  %\[ \Phi_m(x^s) \speq= \prod_{d \mid s} \Phi_{md}(x)\,. \]
  %Nun wollen wir einsehen, dass jedes $\zeta^{in+1}$ ein 
  %$\Phi_{md}(x)$ für $d\mid s$ \glqq trifft\grqq: \TODO
  %Sei nun $I \subseteq \{0,\ldots,s-1\}$, so dass $\forall d\mid s$ genau ein
  %$i \in I$ existiert mit $(x-\zeta^{im+1}) \mid \Phi_{md}(x)$. Dann sind wir
  %jedoch fertig, da
  %\[ f(x^s) \speq= \prod_{i \in I} \prod_{j\in M_q(i \bmod md_i)}
    %\ (x - \zeta^{(im+1)j}) \, \]
  %wobei $d_i$ gerade der zu $i \in I$ korrespondierende Teiler $d_i\mid s$ sei,
  %das kleinste Produkt ist, das $f(x^s)$ zu einem Polynom über $F$ macht.
%\end{proof}

\begin{beispiel}
  Sei $q = p = 2$ und $m = 7$, so haben wir über $\F_2$
  \begin{align*}
    \Phi_m(x) = \Phi_7(x) &= 
      x^{6} + x^{5} + x^{4} + x^{3} + x^{2} + x + 1 \\
    &= (x^{3} + x + 1) \cdot (x^{3} + x^{2} + 1)
  \end{align*}
  Hieraus wählen wir einen irreduziblen Teiler, sagen wir
  \[ f(x) = x^3 + x +1\,.\]
  Nun wollen wir den Beweis von \cref{satz:f_x_s_ist_teiler_von_phimd}
  nachvollziehen. Dazu berechnen wir erst $\ord_7(2) = 3  =: s$. 
  Sei dann $\zeta$ eine 
  $(ms) = 21$-te Einheitswurzel. Um $\Phi_7$ in Termen von $\zeta$ darstellen
  zu können, brauchen wir ein Vertretersystem von Restklassen ${}\bmod 21$
  \[\begin{array}[t]{r|l}
    l \in R_2(21) & M_2(l \bmod 21) \\\hline
    0 & 0 \\
    1 & 1, 2, 4, 8, 11, 16 \\
    3 & 3, 6, 12 \\
    5 & 5, 10, 13, 17, 19, 20 \\
    7 & 7, 14 \\
    9 & 9, 15, 18
  \end{array}\]
  Damit haben wir 
  \[ \setlength{\arraycolsep}{2pt}\begin{array}{rcccc}
    \Phi_7(x) &=& (x^{3} + x + 1) &\cdot& (x^{3} + x^{2} + 1) \\
      &=& (x-\zeta^3)(x-\zeta^{3\cdot 2})(x-\zeta^{3\cdot 4}) &\cdot&
        (x-\zeta^{3\cdot 3})(x-\zeta^{3\cdot 5})(x-\zeta^{3\cdot 6})
  \end{array}\]
  Nun folgt 
  \[\Phi_7(x^3) 
        = \prod_{d \mid \bar s} \Phi_{md}(x) = \Phi_7(x) \cdot \Phi_{21}(x)\]
  für 
  \[\footnotesize\setlength{\arraycolsep}{2pt}\everymath{\displaystyle}
    \begin{array}{rcccc} 
      \Phi_7(x)&=& (x^{3} + x + 1) &\cdot& (x^{3} + x^{2} + 1) \\
      &=& (x-\zeta^3)(x-\zeta^{3\cdot 2})(x-\zeta^{3\cdot 4}) &\cdot&
        (x-\zeta^{3\cdot 3})(x-\zeta^{3\cdot 5})(x-\zeta^{3\cdot 6}) \\[10pt]
      \Phi_{21}(x) &=& (x^{6} + x^{4} + x^{2} + x + 1) 
        &\cdot& (x^{6} + x^{5} + x^{4} + x^{2} + 1) \\
      &=& (x-\zeta)(x-\zeta^2)(x-\zeta^4)(x-\zeta^8)(x-\zeta^{11})
        (x-\zeta^{16}) &\cdot&
        (x-\zeta^5)(x-\zeta^{10})(x-\zeta^{13})(x-\zeta^{17})
        (x-\zeta^{19})(x-\zeta^{20}) 
    \end{array}\]
  Ferner ist 
  \begin{align*}
    f(x^s) = f(x^3) &= x^{9} + x^{3} + 1\\
    &= (x^{3} + x^{2} + 1) \cdot (x^{6} + x^{5} + x^{4} + x^{2} + 1)
  \end{align*}
  und wir erkennen, dass bereits alles durch
  \[ (x-\zeta^3)(x^3) \speq= (x-\zeta) (x-\zeta^8) (x-\zeta^{15})\]
  festgelegt ist.
  Hier wäre also $f(x^s) = f_1(x)f_2(x)$ mit 
  \[ d:\ \{1,2\} \to \{1,3\},\ 1\mapsto 1,\ 2\mapsto 3\,,\]
  was in diesem Fall sogar bijektiv ist.
\end{beispiel}

\begin{beispiel}
  Als zweites Beispiel wollen wir uns einen Fall betrachten, in dem $d$ nicht
  bijektiv ist:
  Sei $p=q=3$ und $m=5$. Damit ist $s = \ord_{\nu(n)}(q) = 4 = \bar s$.
  Betrachten wir wieder ein Vertretersystem von Restklassen ${}\bmod 20$
  \[\begin{array}[t]{r|l}
    l \in R_3(20) & M_2(l \bmod 20) \\\hline
    0 & 0 \\
    1 & 1, 3, 7, 9 \\
    2 & 2, 6, 14, 18 \\
    4 & 4, 8, 12, 16 \\
    5 & 5, 15 \\
    10& 10 \\
    11& 11, 13, 17, 19 
    \end{array}\]
  und eine $ml = 20$-ste Einheitswurzel $\zeta$, so haben wir
  \[\setlength{\arraycolsep}{2pt}\everymath{\displaystyle}
    \begin{array}{rcccc} 
      \Phi_5(x)&=& x^4 + x^3 + x^2 + x + 1\\
      &=& (x-\zeta^4)(x-\zeta^{8})(x-\zeta^{12})(x-\zeta^{16}) \\[10pt]
      \Phi_{10}(x) &=& x^{4} + 2 x^{3} + x^{2} + 2 x + 1 \\
      &=& (x-\zeta^2)(x-\zeta^6)(x-\zeta^{14})(x-\zeta^{16})\\[10pt]
      \Phi_{20}(x) &=& 
        (x^{4} + x^{3} + 2 x + 1) &\cdot& (x^{4} + 2 x^{3} + x + 1) \\
      &=& (x-\zeta^{11})(x-\zeta^{13})(x-\zeta^{17})(x-\zeta^{19}) &\cdot&
        (x-\zeta)(x-\zeta^{3})(x-\zeta^{7})(x-\zeta^{9})
    \end{array}\]
  Hier haben wir nur eine Wahl für einen irreduziblen monischen Teiler von
  $\Phi_m(x)$ und setzen daher $f(x) = \Phi_m(x)$.
  Betrachten wir nun 
  \[ (x -\zeta^4)(x^4) \speq= (x-\zeta)(x-\zeta^6)(x-\zeta^{11})
    (x-\zeta^{16})\]
  so erkennen wir, dass in diesem Fall beide irreduziblen Faktoren von
  $\Phi_{20}(x)$ getroffen werden und können folgern:
  \[ f(x^s) \speq= [x^4 + x^3 + x^2 + x + 1]
    [x^{4} + 2 x^{3} + x^{2} + 2 x + 1]
    [(x^{4} + x^{3} + 2 x + 1) (x^{4} + 2 x^{3} + x + 1)]\]
  Dies ist auch nicht weiter überraschend, da in diesem Fall
  $f(x^s) = \Phi_m(x^s) = \prod_{d\mid \bar s} \Phi_{md}(x) = 
  \Phi_5(x)\Phi_{10}(x)\Phi_{20}(x)$ ist.
  Die Abbildung $d$ erschließt sich zu
  \[ d: \{1,2,3,4\} \to \{1,2,4\},\ 1\mapsto 1,\ 2\mapsto 2,\ 3,4\mapsto 4\]
\end{beispiel}

%\begin{beispiel}
  %Sei $q = p = 3$ und $m = 22$, so haben wir über $\F_q$
  %\begin{align*} 
    %\Phi_m(x) = \Phi_{22}(x) \speq{&=} 
      %x^{10} + 2 x^{9} + x^{8} + 2 x^{7} + x^{6} + 2 x^{5} + x^{4} + 
      %2 x^{3} + x^{2} + 2 x + 1 \\
    %\speq{&=}
      %(x^{5} + 2 x^{3} + 2 x^{2} + 2 x + 1) \cdot 
      %(x^{5} + 2 x^{4} + 2 x^{3} + 2 x^{2} + 1) \,.
  %\end{align*}
  %Es ist ferner $s = \ord_{\nu(m)}(q) = \ord_{22}(3) = 5 = \bar s$. Sei
  %\[ f(x) \speq= x^{5} + 2 x^{3} + 2 x^{2} + 2 x + 1 \quad\in\F_q[x]\]
  %ein irreduzibler Teiler von $\Phi_m(x)$ in $\F_q[x]$.
  
  %Wählen wir nun wie im Beweis von \cref{satz:f_x_s_ist_teiler_von_phimd} eine
  %$(ms) = 100$-te Einheitswurzel $\zeta$, so müssen wir ein Vertretersystem von
  %Restklassen ${}\bmod m$ berechnen, um die passenden Zerlegungen in
  %Linearfaktoren angeben zu können:
  %\begin{center}
    %\[\begin{array}{r|l}
      %l \in R_q & M_q(l \bmod m) \\\hline
      %0 & 0 \\
      %1 & 1, 3, 5, 9, 15    \\
      %2 & 2, 6, 8, 10, 18   \\
      %4 & 4, 12, 14, 16, 20 \\
      %7 & 7, 13, 17, 19, 21 \\
      %11 & 11
    %\end{array}\]
  %\end{center}
  %Da nur $1$ und $7$ teilerfremd zu $22$ sind, ist also
  %\[\setlength{\arraycolsep}{3pt}\begin{array}{rcccc}
    %\Phi_{22}(x) &=& (x^{5} + 2 x^{3} + 2 x^{2} + 2 x + 1) &\cdot& 
      %(x^{5} + 2 x^{4} + 2 x^{3} + 2 x^{2} + 1) \\
    %&=& (x - \zeta)(x-\zeta^3)(x-\zeta^5)(x-\zeta^9)(x-\zeta^{15}) &\cdot&
      %(x-\zeta^7)(x-\zeta^{13})(x-\zeta^{17})(x-\zeta^{19})(x-\zeta^{21})
  %\end{array}\]
  %Gehen wir nun über $f(x^s)$ zu betrachten, so erhalten wir über $\F_q$
  %\begin{align*}
    %f(x^s) &= x^{25} + 2 x^{15} + 2 x^{10} + 2 x^{5} + 1\\
    %&= (x^{5} + 2 x^{3} + 2 x^{2} + 2 x + 1) \\
    %&\ \  \cdot (x^{20} + x^{18} + x^{17} + 2 x^{16} + x^{15} + 
      %x^{14} + 2 x^{10} + 2 x^{9} + 2 x^{8} + x^{7} + 2 x^{5} + x^{4} + x^{3} + 
      %2 x^{2} + x + 1)\,.
  %\end{align*}

  %Dies teilt $\Phi_{m}(x^s) = \Phi_{22}(x^5)$, was über $\F_q$ wie folgt
  %zerfällt:
  %\[\begin{array}{rcc}
    %\Phi_{22}(x^5) &=& \prod_{d\mid \bar s} \Phi_{md}(x)
      %=  \Phi_{22}(x) \cdot \Phi_{110}(x) \\
    %&=&
  %\end{array}\]
  %Wie $\Phi_{22}$ in Termen von $\zeta$ zerfällt haben wir oben bereits
  %gesehen,

%\end{beispiel}
