\chapter{Der Zerfall von $x^n-1$ und die Kreisteilungspolynome $\Phi_D(x)$}
\label{chap:kreisteilungspolynome}

Sei $K$ ein beliebiger Körper der Charakteristik $p$ 
und $\bar K$ ein fest gewählter algebraischer
Abschluss. Wir wollen nun untersuchen, wie das Polynom
$x^n-1 \in K[x]$ über $K$ zerfällt. Dazu orientieren wir uns 
an \autocite{lidl1997finite} und \autocite{wan2003lectures}.

\begin{definition}[Kreisteilungskörper, Einheitswurzeln]
  Sei $n\in\N^\ast$. Der Zerfällungskörper von $X^n-1 \in K[X]$ heißt
  der \emph{$n$-te Kreisteilungskörper} und wird mit $K\kl n$ notiert.
  Die Nullstellen von $X^n-1$ in $K\kl n$ heißen \emph{$n$-ten
  Einheitswurzeln} und die Menge derer wird mit $E\kl n$ bezeichnet.
\end{definition}


\begin{satz}
  Sei $n\in \N^\ast$.
  \begin{enumerate}
    \item Sei $p\nmid n$. Dann ist $E\kl n$ eine zyklische Gruppe (bzgl. der
      Multiplikation in $K\kl n$) der Ordnung $n$.
    \item Ist $p \mid n$ und schreibt man $n = p^e m$ 
      für positive ganze Zahlen $m$ und $e$ mit $p\nmid m$, so
      ist $K\kl n = K\kl m$ und $E\kl n= E\kl m$ und die Nullstellen von
      $X^n-1 \in K[X]$ sind gerade die Elemente in $E\kl m$ jedoch jeweils mit
      Multiplizität $p^e$.
  \end{enumerate}
\end{satz}
\begin{proof}
  \autocite[Theorem 2.42]{lidl1997finite}.
\end{proof}


\begin{definition}[primitive Einheitswurzeln]
  Sei $n\in \N^\ast$ und $p\nmid n$. Dann heißen die Erzeuger von 
  $E\kl n$ \emph{primitive $n$-te Einheitswurzeln}. 
  Die Untergruppe der primitiven Einheitswurzeln wird mit
  $C\kl n$ bezeichnet.
\end{definition}


\begin{definition}[Kreisteilungspolynom]
  Seien $n \in \N^\ast$, $p\nmid n$. Das Polynom 
  \[ \Phi_n(X) \speq{:=} \prod_{\zeta \in C\kl n} (X - \zeta)\]
  heißt \emph{$d$-tes Kreisteilungspolynom}.
\end{definition}

\begin{satz}
  \label{satz:zerfall_xn_1}
  Seien $K$ ein Körper der Charakteristik $p$ und $n \in \N^\ast$ mit $p\nmid
  n$. Dann gilt:
  \begin{enumerate}
    \item $X^n-1 = \prod_{d\mid n} \Phi_d(X)$.
    \item $\Phi_n(X) \in P[X]$, wobei $P\subset K$ den Unterkörper mit $p$
      Elementen notiert.
  \end{enumerate}
\end{satz}
\begin{proof}
  \begin{enumerate}
    \item Dies ist eine einfache Folgerung aus \thref{satz:zykl_gruppen}.
    \item Lässt sich per Induktion recht einfach beweisen 
      (vgl. \autocite[Theorem 2.45 (ii)]{lidl1997finite}).
  \end{enumerate}
\end{proof}


\begin{definition}
  Für zwei teilerfremde natürliche Zahlen $q,n$ größer Null sei
  \[ \ord_n(q) \speq{:=} \ord([q]_n)\]
  die \emph{multiplikative Ordnung von $q$ modulo $n$},
  wobei $[q]_n$ die Restklasse von $q$ in $\Z_n$ bezeichnet und 
  die Ordnung als Gruppenordnung in den Einheiten von $\Z_n$, notiert 
  durch $\Z_n^\times$, zu lesen ist.
\end{definition}

\begin{lemma}[Rechenregeln der multiplikation Ordnung modulo $n$]
  \label{lemma:rechenregeln_ordn}
  Seien $m,n,q \in \N^\ast$ mit \newline $\ggT(n,q)=1$,
  $\ggT(m,q) = 1$ und $\ggT(m,n) = 1$, so gilt
  \begin{enumerate}
    \item $\ord_n(q) \mid \varphi(n)$,
    \item $\ord_{mn}(q) = \kgV\{ \ord_m(q), \ord_n(q)\}$.
  \end{enumerate}
\end{lemma}
\begin{proof}
  \begin{enumerate}
    \item Klar, da $[q]_n$ in $\Z_n^\times$ eine Untergruppe der Ordnung
      $\ord_n(q)$ erzeugt. 
      Nach dem Satz von Lagrange teilt deren Ordnung die 
      Gruppenordnung $|\Z_n^\times| = \varphi(n)$.
    \item Nach dem Chinesischen Restsatz 
      (z.B. \autocite[Kapitel 2 Satz 12]{bosch2009algebra}) ist
      \[ f: \funcdef{ \Z_{nm} &\xrightarrow{\cong}& \Z_n \times \Z_m\,,\\{} 
          [x]_{nm} &\mapsto& ([x]_n, [x]_m)}\]
      ein Isomorphismus von Ringen, 
      da algebraisch $\Z_n$ ja nichts anderes ist, als
      $\Z\big/(n)$, wobei $(n)$ das von $n$ im Ring $\Z$ erzeugte Ideal meint.
      Dieser liefert einen Gruppenhomomorphismus auf
      den Einheiten:
      \[ f: \Z_{nm}^\times \to \Z_n^\times \times \Z_m^\times\,.\]
      Nun ist per definitionem von $\ord_{\bullet}(q)$ die Behauptung
      klar.
  \end{enumerate}
\end{proof}

Damit können wir nun zu einem zentralen Resultat dieses Abschnittes kommen, das
uns über die gesamte Arbeit hinweg begleiten wird.

\begin{satz}
  \label{satz:zerfall_kreisteilungspolys}
  Seien $q$ eine Primzahlpotenz und $n\in \N^\ast$ mit $\ggT(q,n)=1$. Dann
  zerfällt das $n$-te Kreisteilungspolynom $\Phi_n(X)$ über $\F_q$ in
  \[ \frac{\varphi(n)}{\ord_n(q)}\]
  irreduzible paarweise teilerfremde Polynome von jeweils Grad $\ord_n(q)$.
\end{satz}
\begin{proof}
  Sei $f(X) \mid \Phi_n(X)$ ein irreduzibler Teiler über $\F_q$. Ist dann
  $\zeta \in C\kl n$ eine Nullstelle von $f(X)$, so sind 
  nach \thref{satz:nst_irred_polys} auch
  \[ \zeta^q, \zeta^{q^2}, \ldots, \zeta^{q^n} \]
  Nullstellen von $f(X)$. Jedoch sind offenbar nur $\ord_n(q)$ dieser 
  verschieden und da $f$ als irreduzibles Polynom wieder nach 
  \thref{satz:nst_irred_polys} nur einfache Nullstellen besitzt,
  können wir folgern, dass $\deg f = \ord_n(q)$.
  Da $f(X)$ als beliebiger irreduzibler Teiler von $\Phi_n(X)$ gewählt wurde,
  folgt sofort die Behauptung, wenn man sich überlegt, dass der Grad des
  $n$-ten Kreisteilungspolynoms per definitionem gerade $\varphi(n)$ ist.
\end{proof}


\begin{lemma}
  \label{lemma:kreisteilungspolynome_in_binome}

\end{lemma}


Im Beweis obigen Satzes haben wir gesehen, dass die Wirkung der Galoisgruppe
$\Gal(\F_{q^n}\mid \F_q)$ auf der Menge der primitiven $n$-ten Einheitswurzeln
$C\kl n$ (die Wirkung ist selbstredend durch Einsetzen gegeben) diese in 
Teilmengen der Mächtigkeit $\ord_n(q)$ zerlegt.
Dies lässt sich natürlich auf $E\kl n$ übertragen, da ja gerade 
nach \thref{satz:zykl_gruppen}
$E\kl n = \bigcupdot_{d\mid n} C\kl d$. 
Dies motiviert nachstehende Definition.


%\begin{lemma}
  %\label{lemma:uber-pi-m-1}
  
%\end{lemma}

%\begin{lemma}
  %\label{lemma:uber-pi-m-2}
  %Für natürliche Zahlen $m$, $k$ gilt
  %\[ k \speq= \pi_m(k)\cdot l \]
  %mit $\ggT(m \pi_m(k), l) = 1$.
%\end{lemma}
%\begin{proof}
  %Dies sieht man sehr leicht, wenn man sich die Primfaktorzerlegungen der
  %gegebenen Zahlen zu Gemüte führt: Sei also 
  %\begin{align*}
    %m \quad&=\quad p_1^{\nu_1} \cdot\ldots\cdot p_s^{\nu_s}\ \cdot\
      %r_1^{\eta_1}\cdot\ldots\cdot r_t^{\eta_t}\,,\\
    %k \quad&=\quad p_1^{\nu'_1} \cdot\ldots\cdot p_s^{\nu'_s}
      %\ \cdot\ {r'}_1^{{\eta'}_1} \cdot\ldots\cdot {r'}_{t'}^{{\eta'}_{t'}}\,,
  %\end{align*}
  %für geeignete $\nu_\cdot,\nu'_\cdot,\eta_\cdot,\eta'_\cdot > 0$, wobei
  %natürlich $\ggT(r_1\cdot\ldots\cdot r_t, r'_1\cdot\ldots\cdot r'_{t'}) = 1$.
  %Dann ist $\pi_m(k)$ der größte Teiler von $k$, dessen quadratfreier Teil, den
  %quadratfreien Teil von $m$ teilt, also
  %\[ \pi_m(k) \speq= p_1^{\nu'_1}\cdot\ldots\cdot p_s^{\nu'_s}\,.\]
  %Damit ist $l = {r'}_1^{{\eta'}_1}\cdot\ldots\cdot {r'}_{t'}^{{\eta'}_{t'}}$ 
  %und die Behauptung folgt sofort.
%\end{proof} 

\begin{definition}
  Für $m,q\in\N$ mit $\ggT(m,q) = 1$ und $j \in \Z_m$ definieren wir
  \[ M_q(j\bmod m) \speq{:=} \{ j\,q^i \bmod m\mid i\in\N\} \speq= 
    \{j,\ jq,\ jq^2,\ jq^3,\ldots\ \bmod m\}\]
  Ein vollständiges Repräsentantensystem von Nebenklassen von $M_q(1\bmod m)$
  in $\Z_m$ sei mit $R_q(m)$ bezeichnet.
\end{definition}


\begin{beispiel}
  \label{beispiel:zerfall_x21_1_1}
  Wollen wir den Zerfall von $X^{21}-1$ über $\F_2$ untersuchen, so berechnen
  wir erst ein Vertretersystem von Restklassen $\bmod{21}$:
  \[\begin{array}[t]{r|l}
    l \in R_2(21) & M_2(l \bmod{21}) \\\hline
    0 & 0 \\
    1 & 1, 2, 4, 8, 11, 16 \\
    3 & 3, 6, 12 \\
    5 & 5, 10, 13, 17, 19, 20 \\
    7 & 7, 14 \\
    9 & 9, 15, 18
  \end{array}\]
  Nun wissen wir aus \thref{satz:zerfall_xn_1}, dass 
  \[ X^{21} -1 = \Phi_1(X) \cdot \Phi_3(X) \cdot \Phi_7(X) \cdot
  \Phi_{21}(X)\,.\]
  Die Nullstellen von $\Phi_{21}(X)$ partitionieren sich gerade in
  diejenigen $M_2(l\bmod{21})$ für die $l=1,5$.
  Also haben wir 
  \[\footnotesize\setlength{\arraycolsep}{2pt}\everymath{\displaystyle}
    \begin{array}{rcccc} 
      \Phi_{21}(X) &=& (X^{6} + X^{4} + X^{2} + X + 1) 
        &\cdot& (X^{6} + X^{5} + X^{4} + X^{2} + 1) \\
      &=& (X-\zeta)(X-\zeta^2)(X-\zeta^4)(X-\zeta^8)(X-\zeta^{11})
        (X-\zeta^{16}) &\cdot&
        (X-\zeta^5)(X-\zeta^{10})(X-\zeta^{13})(X-\zeta^{17})
        (X-\zeta^{19})(X-\zeta^{20})\,
    \end{array}\]
  falls wir $\zeta \in C\kl{21}$ als Nullstelle von
  $X^6+X^4+X^2+X+1$ setzen.
  Analog erhalten wir den Zerfall von $\Phi_7(X)$ durch Betrachtung der
  $M_2(l\bmod{21})$ für $l=3,9$. 
  \[\footnotesize\setlength{\arraycolsep}{2pt}\everymath{\displaystyle}
    \begin{array}{rcccc} 
      \Phi_7(X)&=& (X^{3} + X + 1) &\cdot& (X^{3} + X^{2} + 1) \\
      &=& (X-\zeta^3)(X-\zeta^{3\cdot 2})(X-\zeta^{3\cdot 4}) &\cdot&
        (X-\zeta^{3\cdot 3})(X-\zeta^{3\cdot 5})(X-\zeta^{3\cdot 6}) 
    \end{array}\]
  Sammeln wir den Rest auf, erhalten wir die Partitionierung für
  $\Phi_3(X)$ und den trivialen Fall $\Phi_1(X)$.
  \[ \begin{array}{rcc} 
    \Phi_3(X) &=& X^2 + X + 1\\
              &=& (X-\zeta^7)(X-\zeta^{14})\,,\\[10pt]
    \Phi_1(X) &=& X-1\\
             &=& X-\zeta^0\,.
    \end{array}\]
\end{beispiel}


Nun können wir uns überlegen, ob und wie unterschiedliche Kreisteilungspolynome
zusammenhängen und kommen dabei auf die bekannten Resultate, die z.B. in 
\autocite[Proposition 10.6, 10.7]{hachenberger1997finite} zu finden sind. Um
diese anzugeben, benötigen wir jedoch noch eine Definitionen und zitieren
einige Eigenschaften.

\begin{definition}
  Seien $r,n\in \N$, so definiere
  \[ \pi_r(n) \speq{:=} \max\{k \in \N^\ast:\ k \mid n,\ \nu(k) \mid \nu(r)
  \}\,.\]
\end{definition}


\begin{lemma}
  \label{lemma:cl_1}
  Seien $q>1$ eine ganze Zahl, $n\in \N^\ast$ und $r$ ein Primteiler von $q-1$.
  Dann gilt:
  \begin{enumerate}
    \item Ist $r\neq 2$ oder $q\equiv 1 \bmod 4$, so gilt
      \[ \pi_r(q^{r^n}-1) \speq= r^n\, \pi_r(q-1) \,.\]
    \item Ist $q \equiv 3 \bmod 4$, so gilt
      \[ \pi_2(q^{2^n}-1) \speq= 2^{n-1}\, \pi_2(q^2-1)\,.\]
  \end{enumerate}
\end{lemma}
\begin{proof}
  \autocite[Lemma 19.4]{hachenberger1997finite}.
\end{proof}


\begin{lemma}
  \label{lemma:cl_2}
  Seien $q,m,k > 1$ ganze Zahlen mit $\nu(k) \mid \nu(m)\mid q-1$. Dann gilt
  \begin{enumerate}
    \item Ist $m$ ungerade oder $q \equiv 1 \bmod 4$ oder $k$ ungerade, so
      gilt
      \[ \pi_m(q^k-1) \speq= k\,\pi_m(q-1)\,. \]
    \item Ist $m$ gerade, $q \equiv 3 \bmod 4$ und $k$ gerade, so ist
      \[ \pi_m(q^k-1) \speq= \tfrac k 2\, \pi_m(q^2-1)\,.\]
  \end{enumerate}
\end{lemma}
\begin{proof}
  \autocite[Lemma 19.5]{hachenberger1997finite}.
\end{proof}

\begin{satz}
  \label{satz:zusammenhang_unterschiedlicher_kreisteilungspolys}
  Seien $t,k \in \N^\ast$.
  \begin{enumerate}
    \item Ist $\nu(t) \mid k$, so gilt
      \[ \Phi_k(X^t) \speq= \Phi_{kt}(X) \ \in K[X]\,.\]
    \item Sind $t$ und $k$ teilerfremd, so gilt
      \[ \Phi_k(X^t) \speq= \prod_{d\mid t} \Phi_{kd}(X)\ \in K[X]\,.\]
    \item Insbesondere gilt: Ist $q = p^r$ eine Primzahlpotenz,
      $t,k\in\N^\ast$ mit $p\nmid t,k$ und $\pi$ eine Potenz von $p$. Sei ferner
      $t = \pi_k(t)\cdot \bar t$, so gilt
      \[ \Phi_k(X^{t\pi}) \speq= 
        \left(\prod_{d\mid \bar t} \Phi_{k\,d\,\pi_k(t)} (X)\right)^\pi
        \ \in \F_q[X]\,. \]
  \end{enumerate}
\end{satz}
\begin{proof}
  Dass sich Potenzen von $p$ aus dem Argument herausziehen lassen, ist klar,
  da $\id_P = (.)^\pi: P \to P$ für den Primkörper $P \subset K$ nach 
  \thref{satz:frob_auto} eine lineare Abbildung ist. 
  Ferner haben nach \thref{satz:zerfall_xn_1} die
  Kreisteilungspolynome nur Koeffizienten in $P$.

  Der Kern des Beweises des Rests liegt in der Betrachtung des 
  Gruppenhomomorphismus
  \[ \psi_n:\ \bar K^\ast \to \bar K^\ast,\ x \mapsto x^n\]
  für $p\nmid n$. Denn nun ist offensichtlich, dass die Nullstellen von 
  $\Phi_k(X^t)$ gerade alle Elemente in $\bar K^\ast$, deren $t$-te Potenz
  eine primitive $k$-te Einheitswurzel ist, sind, also $\psi_t\inv(C\kl k)$.
  Ergo formulieren sich die Aussagen wie folgt um:
  \begin{enumerate}[label=(\arabic*')]
    \item Ist $\nu(t) \mid k$, so gilt
      $\psi_t\inv(C\kl k) \speq= C\kl{kt}$.
    \item Ist $\ggT(t,k)=1$, so gilt
      $\psi_t\inv(C\kl k) \speq= \bigcupdot_{d\mid t} C\kl{kd}$.
    \item Ist $k,t\in \N^\ast$ mit $p\nmid t,k$ und $k = \pi_k(t) \bar t$,
      so gilt
      \[ \psi_t\inv(C\kl k) \speq= 
        \bigcupdot_{d\mid \bar t} C\kl{k\,d\,\pi_k(t)}\]
  \end{enumerate}
  Nun ist offensichtlich, dass es reicht (3') zu zeigen. 
  Dazu notiere $t_0 := \pi_k(t)$ und seien $d\mid \bar t$ 
  und $\zeta \in C\kl{kd t_0}$ beliebig. Dann ist 
  \[ \ord(\zeta^t) = \ord((\zeta^{t_0d})^{\frac{\bar t}{d}})
    = k\,,\]
  da per definitionem von $\pi_k(t)$ gerade $\ggT(\bar t, kt_0) = 1$.
  Also gilt $\psi_t(C\kl{kdt_0}) \subseteq C\kl k$
  und damit
  \[ \bigcupdot_{d\mid \bar t} C\kl{kdt_0} \speq\subseteq 
    \psi_t\inv \psi_t ( \cupdot_{d\mid \bar t} C\kl{kdt_0}) 
    \speq\subseteq \psi_t\inv(C\kl k)\]
  Die Gleichheit folgt mit einem Zählargument:
  Auf der einen Seite ist
  \[ \big|\bigcupdot_{d\mid \bar t} C\kl{kdt_0}\big| = 
    \sum_{d\mid \bar t} \varphi(kdt_0) = 
    \varphi(kt_0)\sum_{d\mid \bar t}\varphi(d) = \varphi(kt_0) \cdot \bar t
     = \varphi(k)t\,,\]
  wobei an \thref{lemma:rechenregeln_phifunktion} erinnert sei.
  Auf der anderen Seite haben wir
  \[ \big| \psi_t\inv(C \kl k)\big| \speq= t |C\kl k| \speq= t \varphi(k)\,,\]
  was den Beweis abschließt.
\end{proof}

Bevor wir den Zerfall der Kreisteilungspolynome noch genauer untersuchen,
kann man als einfache Folgerung angeben, wann genau ein
Binom $x^n-\beta \in \F_q[x]$ irreduzibel ist.

\begin{satz}
  \label{satz:binom_irreduzibel}
  Seien $\beta \in F_q^\ast$ und $n\in \N$. Es gilt:
  $x^n -\beta \in \F_q[x]$ ist genau
  dann irreduzibel, wenn 
  \begin{enumerate}
    \item $p := \charak \F_q \nmid n$,
    \item $\nu(n) \mid e := \ord(\beta)$ und
    \item $\ord_{ne}(q) = n$.
  \end{enumerate}
\end{satz}
\begin{proof}
  Zunächst ist klar, dass $p \nmid n$ erfüllt sein muss, da ansonsten 
  $\beta' \in \F_q^\ast$ existiert mit $\beta'^p = \beta$ ($(\cdot)^p$ ist ein
  Automorphismus auf $\F_q$ nach \thref{satz:frob_auto}). Damit wäre
  $x^n-\beta = (x^\frac n p - \beta')^p$ eine Faktorisierung.
  Nun sei $u$ eine Nullstelle von $x^n-\beta$, so lässt sich beobachten, dass
  (in Notation des Beweises von
  \thref{satz:zusammenhang_unterschiedlicher_kreisteilungspolys})
  $u \in \psi_n\inv(C\kl{e})$. Damit gilt nach
  \thref{satz:zusammenhang_unterschiedlicher_kreisteilungspolys} (3)
  \[ x^n-\beta \speq= \prod_{d\mid \bar n} \ggT(x^n-\beta, \Phi_{en_0d}(x))\]
  für $n = \pi_{e}(n) \bar n$ und diese Zerlegung ist, wie man sich analog zum
  Beweis von \thref{satz:zerfall_kreisteilungspolys} überlegen kann, 
  nicht trivial (vgl. \autocite[Theorem \ldots]{hachenberger2015}). 
  Damit ist (2) der Behauptung
  klar, so dass $x^n-\beta \mid \Phi_{ne}(x)$. Ferner zerfällt nach
  \thref{satz:zerfall_kreisteilungspolys} $\Phi_{ne}(x)$ in 
  $\frac{\varphi(ne)}{\ord_{ne}(q)}$ irreduzible Faktoren von jeweils Grad
  $\ord_{ne}(q)$. Damit wird auch (3) der Behauptung augenblicklich klar.
\end{proof}

Für die letzte Bedingung in obigem Satz existieren noch verschiedene weitere
äquivalente Charakterisierungen, die nachstehend zu finden sind.

\begin{satz}
  \label{satz:binom_irreduzibel_aquiv}
  Seien $p$ eine Primzahl, $q$ eine Potenz von $p$ und $n,e\in \N^\ast$ 
  mit $p\nmid n$ und $\nu(n)\mid e\mid q-1$. Dann sind äquivalent:
  \begin{enumerate}
    \item $\ord_{ne}(q)  = n$,
    \item $\ggT(\tfrac{q-1}{e},n)=1$ und $q\equiv 1 \bmod 4$, falls $4\mid n$,
      und
    \item $\pi_n(q-1) \mid e$ und $q\equiv 1 \bmod 4$, falls $4\mid n$.
  \end{enumerate}
\end{satz}
\begin{proof}
  \autocite[Theorem \ldots]{hachenberger2015}.
\end{proof}


Wir haben nun erkannt, wann genau Binome über einem endlichen Körper
irreduzibel sind. Doch wenn man dem Titel dieses Kapitels Glauben schenken mag,
interessieren wir uns hier vorangig für den Zerfall der Kreisteilungspolynome
über endlichen Körpern. Diese sind im Allgemeinen keine Binome, 
aber genau das Wissen über die Irreduzibiltät von
Binomen lässt uns Bedingungen formulieren, die dazu führen, dass ein
Kreisteilungspolynom über einem gegebenen endlichen Körper in irreduzible
Binome zerfällt
Später (\autoref{sec:stark_regulare_erweiterungen}) werden wir diese
Bedingungen \emph{stark regulär} (\thref{def:stark_regular}) nennen und
einsehen, dass sie eine wesentliche Rolle bei der expliziten Konstruktion 
von Normalbasen spielen.

\begin{satz}
  \label{satz:kreisteilungspolynome_binome}
  Seien $\F_q$ ein endlicher Körper von Charakteristik $p$ und $m \in \N^\ast$.
  Es gelte $p\nmid m$, $\nu(m)\mid q-1$ und $4\mid q-1$, falls $2\mid m$.
  Setze $l := \pi_m(q-1)$, $a := \ggT(l,m)$ und $I_a := \{ j\in \N^\ast:\ 
  j\geq a,\ \ggT(j,a)=1\}$. Ist $\zeta \in \F_q^\ast$ eine primitive 
  $a$-te Einheitswurzel, so ist
  \[ \Phi_m(x) \speq= \prod_{j\in I_a} \big( x^\frac m a - \zeta^j\big) \]
  die vollständige Faktorisierung des $m$-ten Kreisteilungspolynoms über
  $\F_q$.
\end{satz}
\begin{proof}
  Wir stellen fest, dass $\F_q$ in der Tat $a$-te Einheitswurzeln enthält, da 
  $\ord_a(q) = 1$. Dies ist klar, da $l$ per definitionem $q-1$ teilt und 
  $a = \ggT(l,m)$.
  Nun wollen wir uns klar werden, dass beide Seiten obiger Gleichung auch
  identisch sind: Für $j \in I_a$ durchläuft $\zeta^j$ alle primitiven $a$-ten
  Einheitswurzeln und damit sind die Nullstellen der rechten Seite der
  Gleichung gerade alle primitiven $m$-ten Einheitswurzeln.
  Bleibt die Irreduzbilität von $x^\frac m a - \zeta^j$ zu zeigen, wobei wir
  ohne Einschränkung $j=1$ wählen können: Klar ist, dass $p\nmid \frac m a$, da
  $p \nmid m$ nach Voraussetzungen. Ferner ist $\nu(l) = \nu(m)$, da wegen
  $\nu(m) \mid q-1$ gilt:
  \[ \pi_m(q-1) = \max\{k\in \N^\ast:\ k\mid q-1,\ \nu(k)= \nu(m) \}\,. \]
  Also ist auch $\nu(a) = \nu(\ggT(l,m)) = \nu(m)$ und damit folgt
  $\nu(\tfrac m a) \mid \nu(m) = \nu(a) \mid a$, womit auch (2) in 
  \thref{satz:binom_irreduzibel} erfüllt wäre.
  Da $\nu(m)\mid q-1$ ist $\pi_{\frac m a}(q-1) \mid m$, also auch
  $\pi_{\frac m a}(q-1)\mid a$. Damit wäre durch die Bedingung $q\equiv 1 \bmod
  4$, falls $2\mid m$, auch (3) in \thref{satz:binom_irreduzibel_aquiv}
  erfüllt.
\end{proof}

\begin{bemerkung}
  Man hätte obigen Beweis auch ohne das Wissen über irreduzible Binome führen
  können, in dem man sich \thref{satz:zerfall_kreisteilungspolys} bedient. So
  findet man dies auch in \autocite[Lemma 22.2]{hachenberger1997finite}.
\end{bemerkung}



Erinnert man sich nun erneut an \thref{satz:zerfall_kreisteilungspolys}, so
kann man sich die Frage stellen, ob man den
Zusammenhang unterschiedlicher Kreisteilungspolynome aus 
\thref{satz:zusammenhang_unterschiedlicher_kreisteilungspolys}
in dem Sinne verfeinern kann, dass man sich nicht für das gesamte 
Kreisteilungspolynom interessiert, sondern lediglich für einen irreduziblen
Teiler. Diese Frage beantwortet nachstehender Satz.


\begin{satz}
  \label{satz:zerfall_f_x_s}
  Seien $q=p^r$ eine Primzahlpotenz und
  $m,t\in \N$ mit $p\nmid m$, $p\nmid t$ und $\ggT(m,t) = 1$.
  Definieren wir ferner
  \[ \Delta_q(m,d) \speq{:=} \frac{\varphi(d) \ord_m(q)}{\ord_{md}(q)}\,,\]
  so gilt:
  \begin{enumerate}
    \item Ist $f(x) \mid \Phi_m(x)$ ein über $\F_q$ irreduzibler monischer Teiler des
      $m$-ten Kreisteilungspolynoms, so gilt
      \[ f(x^t) \speq= \prod_{d\mid t} 
        \prod_{i=1}^{\Delta_q(m,d)} f_{d,i}(x)\,,\]
      wobei für alle $i=1,\ldots,\Delta_q(m,d)$ 
      \[ f_{d,i}\in \F_q[x] 
        \text{ monisch, irreduzibel und } f_{d,i}(x) \mid \Phi_{md}(x)\,.\]
      Ferner sind alle $f_{d,i}(x)$ paarweise teilerfremd.
    \item Sind $f(x) \mid \Phi_m(x)$ und $g(x) \mid \Phi_m(x)$ zwei
      teilerfremde, monische, über $\F_q$ irreduzible Teiler des $m$-ten
      Kreisteilungspolynoms, so sind auch $f(x^t)$ und $g(x^t)$ teilerfremd.
  \end{enumerate}
\end{satz}
\begin{proof}
  Wie schon im Beweis von
  \thref{satz:zusammenhang_unterschiedlicher_kreisteilungspolys} betrachten wir
  den Gruppenhomomorphismus $\psi_t$, diesmal eingeschränkt auf $E\kl{mt}$:
  \[ \psi_t:\ \funcdef{E\kl{mt} & \to& E\kl m\,,\\
    x &\mapsto& x^t\,,}\]
  was offenbar ein wohldefinierter Gruppenhomomorphismus bleibt.
  Offensichtlich ist $\ker\psi_t = E\kl t$. Da $\ggT(m,t) = 1$, also 
  $E\kl{mt} = E\kl m \odot E\kl t$ als leichte Folgerung aus
  \thref{satz:zykl_gruppen}, ist $\psi$ auch surjektiv.

  Ist nun $\alpha \in C\kl m$ eine Nullstelle von $f(x)$, 
  so existiert -- wiederum
  weil $m$ und $t$ teilerfremd sind -- genau ein $\beta \in C\kl m$ mit
  $\beta^t = \alpha$. Damit ist also
  \[ \psi_t\inv(\{\alpha\}) \speq= \beta\, E\kl t \speq=
    \bigcupdot_{d\mid t}\ \beta\, C\kl d \,.\]
  Notiert wieder $\sigma$ der Frobenius von
  $\F_q$, so sind nach \thref{satz:nst_irred_polys}
  $\sigma^j(\alpha)$, $j=0,\ldots,\delta-1$ für 
  $\delta=\ord_q(m)$ die Nullstellen von $f(x)$. Da $p\nmid t$ bleibt
  die Menge der $t$-en Einheitswurzeln invariant unter $\sigma $ und 
  damit ist die Menge der Nullstellen von $f(x^t)$ gerade
  \begin{equation}
    \bigcupdot_{j=0}^{\delta-1} \sigma^j(\beta)\, E\kl t \speq=
    \bigcupdot_{j=0}^{\delta-1}\ \bigcupdot_{d\mid t}\ \beta^{q^j}\, C\kl d
    \speq=
    \bigcupdot_{d\mid t}\ \bigcupdot_{j=0}^{\delta-1}\ \beta^{q^j}\, C\kl d
    \speq{=:} \bigcupdot_{d\mid t} N_d\,. 
  \end{equation}
  %Wir bemerken kurz, dass $\alpha$ nicht in obiger Nullstellenmenge liegt, da
  %$q^i \neq t$ für alle $i=1,\ldots,\delta-1$ und damit 
  %$\beta^{q^i} \neq \alpha$ für alle $i=1,\ldots,\delta-1$, was
  %$f(X)\nmid f(X^t)$ zeigt.
  Wollen wir nun einsehen, wie $f(x^t)$ über $\F_q$ zerfällt, so müssen wir
  überlegen, wie obige Nullstellenmenge in $\sigma$-invariante Teilmengen
  zerfällt. Für jedes $d\mid t$ und jedes $j\in\{0,\ldots,\delta -1\}$ ist
  $\zeta \in \beta^{q^j}\,C\kl d$ ein Element mit $\ord(\zeta) = md$, also
  Nullstelle von $\Phi_{md}(x)$. Ferner gilt
  offenbar $\forall d\mid t:\ |N_d| = \delta\varphi(d)$
  und wir können folgern, dass $N_d$ in genau
  \[ \frac{\delta\varphi(d)}{\ord_{md}(q)} \speq= \frac{\ord_m(q)\,
    \varphi(d)}{\ord_{md}(q)} \speq=
    \Delta_q(m,d)\]
  $\sigma$-invariante Teilmengen zerfällt. $\Delta_q(m,d)$ ist in der Tat eine
  natürliche Zahl größer 0, da nach \thref{lemma:rechenregeln_ordn} (2)
  \[ \frac{\ord_m(q)\,\varphi(d)}{\ord_{md}(q)} \speq=
    \frac{\varphi(d)\, \ggT(\ord_m(q),\ord_d(q))}{\ord_d(q)}\]
  und $\ord_d(q) \mid \varphi(d)$ nach \thref{lemma:rechenregeln_ordn} (1).
  Damit ist alles in (1) gezeigt.
  Der Zusatz (2) folgt sofort, denn ist $\alpha_f \in C\kl m$ 
  bzw. $\alpha_g \in C\kl m$ Nullstelle von $f$ bzw. $g$, so gehören diese zu 
  verschiedenen $\sigma$-invarianten Teilmengen von $C\kl m$ (vgl. 
  auch \thref{beispiel:zerfall_x21_1_1}) und folglich gehören auch
  $\beta_f\in C\kl m$ bzw. $\beta_g\in C\kl m$ mit $\beta_f^t = \alpha_f$ bzw.
  $\beta_g^t = \alpha_g$ zu verschiedenen und damit disjunkten
  $\sigma$-invarianten Teilmengen von $\C\kl m$.
\end{proof}

%\begin{satz}
  %\label{satz:f_x_s_ist_teiler_von_phimd}
  %Seien $q=p^r$ eine Primzahlpotenz, $F = \F_q$ ein endlicher Körper,
  %$m$ eine natürliche Zahl und $s = \ord_{\nu(m)}(q)$ mit 
  %$\ggT(s,m) = 1$. Weiter sei $s = \bar s p^\beta$ mit 
  %$\ggT(\bar s, p) = 1$. Ist ferner $f(x) \mid \Phi_m(x)$ ein über $\F_q$
  %irreduzibler monischer Teiler des $m$-ten Kreisteilungspolynoms, so gilt:
  %\[ f(x^s) \speq= \prod_{i=1}^? f_{i}(x)^{p^\beta}\,,\]
  %so dass $f_i(x)$ ein irreduzibler monischer Teiler von $\Phi_{md(i)}(x)$ ist
  %für $d:\ \{1,\ldots,?\} \to \{\text{Teiler von } \bar s\},\ 
  %i\mapsto d(i)$ surjektiv und monoton. Ferner gilt sogar 
  %$d(1) < d(2)$.
%\end{satz}
%\marginpar{Leider habe ich keine Ahnung, wie $i\mapsto d(i)$ aussieht und
  %was $?$ ist. Insbesondere gelingt es mir nicht, die Surjektivität zu
  %beweisen.}
%\begin{proof}
  %Zunächst ist klar, dass wir \obda annehmen können, dass $p \nmid s$. Ist
  %nämlich $s = \bar s p^\beta$ wie oben, so gilt
  %\[ f(x^s) = f(x^{\bar s})^{p^\beta}\,,\]
  %da $\F_q \to \F_q, x \mapsto x^p$ bekanntlich ein Ringhomomorphismus ist.
  %Ist dann $\zeta \in E$ eine primitive $(ms)$-te Einheitswurzel für
  %$E = F_e$, $e = \ord_{ms}(q)$, so ist $\ord(\zeta^s) = m$ und es gilt nach
  %\cref{} \marginpar{Reference!}
  %\[ \Phi_m(x) \speq= \prod_{j \in R_q(m)\atop \ggT(j,m) = 1}
    %\prod_{i\in M_q(j\bmod m)} \ (x - \zeta^{si} ) \]
  %Also haben wir
  %\[ f(x) \speq= \prod_{i\in M_q(j_0 \bmod m)} \ (x - \zeta^{si})\]
  %für ein $j_0\in R_q(m)$.
  %Betrachten wir nun einen Linearfaktor von $f(x^s)$, so wollen wir 
  %zeigen, dass
  %\[g(x) := (x^s - \zeta^s) \speq= \prod_{i=0}^{s-1} ( x - \zeta^{im +1} )\,.\]
  %Dies ist aber nicht schwer zu sehen, da für $i \in \{0,\ldots,s-1\}$
  %\[ g(\zeta^{in+1}) \speq= \zeta^{ism + s} - \zeta^s \speq= 0 \,.\]
  %Da $\zeta$ primitive $(ms)$-te Einheitswurzel ist, haben wir $s$ paarweise
  %verschiedene Nullstellen von $g$ gefunden und obige Behauptung gezeigt.
  %Nun betrachten wir die Aufteilung der Kreisteilungspolynome:
  %Nach \cref{} haben wir
  %\[ \Phi_m(x^s) \speq= \prod_{d \mid s} \Phi_{md}(x)\,. \]
  %Nun wollen wir einsehen, dass jedes $\zeta^{in+1}$ ein 
  %$\Phi_{md}(x)$ für $d\mid s$ \glqq trifft\grqq: \TODO
  %Sei nun $I \subseteq \{0,\ldots,s-1\}$, so dass $\forall d\mid s$ genau ein
  %$i \in I$ existiert mit $(x-\zeta^{im+1}) \mid \Phi_{md}(x)$. Dann sind wir
  %jedoch fertig, da
  %\[ f(x^s) \speq= \prod_{i \in I} \prod_{j\in M_q(i \bmod md_i)}
    %\ (x - \zeta^{(im+1)j}) \, \]
  %wobei $d_i$ gerade der zu $i \in I$ korrespondierende Teiler $d_i\mid s$ sei,
  %das kleinste Produkt ist, das $f(x^s)$ zu einem Polynom über $F$ macht.
%\end{proof}


\begin{beispiel}
  Greifen wir noch einmal \thref{beispiel:zerfall_x21_1_1} auf und betrachten
  einen irreduziblen Teiler $f(x)$ von $\Phi_7(x)$ über $\F_2$, sagen wir
  \[ f(x) \speq{:=} x^3+ x +1\,.\]
  Sei $t := 3$.
  Nun wissen wir nach \thref{satz:zerfall_f_x_s}, dass $f(x^3)$ wie folgt über
  $\F_2$ zerfällt:
  \[ f(x^3) \speq= 
    \tikz[baseline]{\node[anchor=base,rounded corners,fill=gray!5]
      (n)
      {$\displaystyle\prod_{i=1}^{\Delta_2(7,1)} f_{1,i}(x)$};
      \node[above=0pt of n, font=\scriptsize, text=gray]{$d=1 \mid 3$};}
    \cdot
    \tikz[baseline]{\node[anchor=base,rounded corners,fill=gray!5]
      (n)
      {$\displaystyle\prod_{i=1}^{\Delta_2(7,3)} f_{3,i}(x)$};
      \node[above=0pt of n, font=\scriptsize, text=gray]{$d=3 \mid 3$};}
    \speq=
    \tikz[baseline]{\node[anchor=base,rounded corners,fill=gray!5]
      (n)
      {$\displaystyle f_{1,1}(x)$};}
    \cdot
    \tikz[baseline]{\node[anchor=base,rounded corners,fill=gray!5]
      (n)
      {$\displaystyle f_{3,1}(x)$};}
    \]
    da 
    \begin{alignat*}{6}
      \Delta_2(7,1) &\speq=& \frac{\varphi(1)\ord_7(2)}{\ord_{7}(2)} &\speq=&
        \frac{1\cdot 3}{3} &\speq=& 1\,, \\
      \Delta_2(7,3) &\speq=& \frac{\varphi(3)\ord_7(2)}{\ord_{21}(2)} &\speq=&
        \frac{2\cdot 3}{6} &\speq=& 1\,. 
    \end{alignat*}
    Wir wollen nun herausfinden, welche Teiler $f_{1,1}(x)$ und $f_{3,1}(x)$ 
    von $\varphi_7(x)$ und $\varphi_{21}(x)$ sind. Wir übernehmen den Zerfall
    der Kreisteilungspolynome aus \thref{beispiel:zerfall_x21_1_1}
    und können einsehen, dass
    \[ f_{1,1}(x) \speq= x^3+x^2+1\,,\qquad
      f_{3,1}(x) \speq= x^6+x^5+x^4+x^2+1\,.\]
\end{beispiel}

%\begin{beispiel}
  %Sei $q = p = 2$ und $m = 7$, so haben wir über $\F_2$
  %\begin{align*}
    %\Phi_m(x) = \Phi_7(x) &= 
      %x^{6} + x^{5} + x^{4} + x^{3} + x^{2} + x + 1 \\
    %&= (x^{3} + x + 1) \cdot (x^{3} + x^{2} + 1)
  %\end{align*}
  %Hieraus wählen wir einen irreduziblen Teiler, sagen wir
  %\[ f(x) = x^3 + x +1\,.\]
  %Nun wollen wir den Beweis von \cref{satz:f_x_s_ist_teiler_von_phimd}
  %nachvollziehen. Dazu berechnen wir erst $\ord_7(2) = 3  =: s$. 
  %Sei dann $\zeta$ eine 
  %$(ms) = 21$-te Einheitswurzel. Um $\Phi_7$ in Termen von $\zeta$ darstellen
  %zu können, brauchen wir ein Vertretersystem von Restklassen ${}\bmod 21$
  %\[\begin{array}[t]{r|l}
    %l \in R_2(21) & M_2(l \bmod 21) \\\hline
    %0 & 0 \\
    %1 & 1, 2, 4, 8, 11, 16 \\
    %3 & 3, 6, 12 \\
    %5 & 5, 10, 13, 17, 19, 20 \\
    %7 & 7, 14 \\
    %9 & 9, 15, 18
  %\end{array}\]
  %Damit haben wir 
  %\[ \setlength{\arraycolsep}{2pt}\begin{array}{rcccc}
    %\Phi_7(x) &=& (x^{3} + x + 1) &\cdot& (x^{3} + x^{2} + 1) \\
      %&=& (x-\zeta^3)(x-\zeta^{3\cdot 2})(x-\zeta^{3\cdot 4}) &\cdot&
        %(x-\zeta^{3\cdot 3})(x-\zeta^{3\cdot 5})(x-\zeta^{3\cdot 6})
  %\end{array}\]
  %Nun folgt 
  %\[\Phi_7(x^3) 
        %= \prod_{d \mid \bar s} \Phi_{md}(x) = \Phi_7(x) \cdot \Phi_{21}(x)\]
  %für 
  %\[\footnotesize\setlength{\arraycolsep}{2pt}\everymath{\displaystyle}
    %\begin{array}{rcccc} 
      %\Phi_7(x)&=& (x^{3} + x + 1) &\cdot& (x^{3} + x^{2} + 1) \\
      %&=& (x-\zeta^3)(x-\zeta^{3\cdot 2})(x-\zeta^{3\cdot 4}) &\cdot&
        %(x-\zeta^{3\cdot 3})(x-\zeta^{3\cdot 5})(x-\zeta^{3\cdot 6}) \\[10pt]
      %\Phi_{21}(x) &=& (x^{6} + x^{4} + x^{2} + x + 1) 
        %&\cdot& (x^{6} + x^{5} + x^{4} + x^{2} + 1) \\
      %&=& (x-\zeta)(x-\zeta^2)(x-\zeta^4)(x-\zeta^8)(x-\zeta^{11})
        %(x-\zeta^{16}) &\cdot&
        %(x-\zeta^5)(x-\zeta^{10})(x-\zeta^{13})(x-\zeta^{17})
        %(x-\zeta^{19})(x-\zeta^{20}) 
    %\end{array}\]
  %Ferner ist 
  %\begin{align*}
    %f(x^s) = f(x^3) &= x^{9} + x^{3} + 1\\
    %&= (x^{3} + x^{2} + 1) \cdot (x^{6} + x^{5} + x^{4} + x^{2} + 1)
  %\end{align*}
  %und wir erkennen, dass bereits alles durch
  %\[ (x-\zeta^3)(x^3) \speq= (x-\zeta) (x-\zeta^8) (x-\zeta^{15})\]
  %festgelegt ist.
  %Hier wäre also $f(x^s) = f_1(x)f_2(x)$ mit 
  %\[ d:\ \{1,2\} \to \{1,3\},\ 1\mapsto 1,\ 2\mapsto 3\,,\]
  %was in diesem Fall sogar bijektiv ist.
%\end{beispiel}

\begin{beispiel}
  Als zweites Beispiel wollen wir uns einen Fall betrachten, 
  in dem $\Delta_q(d,m)$ nicht immer $1$ ist.
  Sei $p=q=3$, $m=5$ und $t:= 4$.
  Also müssen wir ein Vertretersystem von Restklassen ${}\bmod 20$ betrachten:
  \[\begin{array}[t]{r|l}
    l \in R_3(20) & M_2(l \bmod 20) \\\hline
    0 & 0 \\
    1 & 1, 3, 7, 9 \\
    2 & 2, 6, 14, 18 \\
    4 & 4, 8, 12, 16 \\
    5 & 5, 15 \\
    10& 10 \\
    11& 11, 13, 17, 19 
    \end{array}\]
  Wir sehen, dass $\Phi_{20}(x)$ für $l=1,11$ in 
  2 Polynome von jeweils Grad $4$ zerfällt:
  \[\setlength{\arraycolsep}{2pt}\everymath{\displaystyle}
    \begin{array}{rcccc} 
      \Phi_{20}(x) &=& 
        (x^{4} + x^{3} + 2 x + 1) &\cdot& (x^{4} + 2 x^{3} + x + 1) \\
      &=& (x-\zeta^{11})(x-\zeta^{13})(x-\zeta^{17})(x-\zeta^{19}) &\cdot&
        (x-\zeta)(x-\zeta^{3})(x-\zeta^{7})(x-\zeta^{9})\,,
    \end{array}\]
  wobei wir $\zeta \in C\kl{20}$ mit Minimalpolynom $x^4+2x^3+x+1$ gewählt
  haben.
  Nun können wir den Zerfall von $\Phi_5(x)$ und $\Phi_{10}(x)$ in Termen von
  $\zeta$ anhand der Restklassen ${}\bmod{20}$ beschreiben:
  \[\setlength{\arraycolsep}{2pt}\everymath{\displaystyle}
    \begin{array}{rcccc} 
      \Phi_5(x)&=& x^4 + x^3 + x^2 + x + 1\\
      &=& (x-\zeta^4)(x-\zeta^{8})(x-\zeta^{12})(x-\zeta^{16})\,, \\[10pt]
      \Phi_{10}(x) &=& x^{4} + 2 x^{3} + x^{2} + 2 x + 1 \\
      &=& (x-\zeta^2)(x-\zeta^6)(x-\zeta^{14})(x-\zeta^{16})\,.
    \end{array}\]
  Die Restklassen für $l=0,5,10$ gehören zu den Kreisteilungspolynomen 
  $\Phi_1(x), \Phi_4(x)$ und $\Phi_2(x)$, die wir für ein Beispiel zu
  \thref{satz:zerfall_f_x_s} nicht benötigen.
  Nun brauchen wir wieder einen irreduziblen monischen Teiler von
  $\Phi_m(x)$ und setzen daher $f(x) = \Phi_m(x)$.
  Wir berechnen wie oben
  \begin{alignat*}{6}
    \Delta_3(5,1) &\speq=& \frac{\varphi(1)\ord_5(3)}{\ord_{5}(3)} &\speq=&
      \frac{1\cdot 4}{4} &\speq=& 1\,, \\
    \Delta_3(5,2) &\speq=& \frac{\varphi(1)\ord_5(3)}{\ord_{10}(3)} &\speq=&
      \frac{1\cdot 4}{4} &\speq=& 1\,, \\
    \Delta_3(5,4) &\speq=& \frac{\varphi(4)\ord_5(3)}{\ord_{20}(3)} &\speq=&
      \frac{2\cdot 4}{4} &\speq=& 2\,. 
  \end{alignat*}
  Nun ist klar, wie $f(x^t)$ über $\F_3$ zerfällt:
  \[\small f(x^3) \speq= \big(
    \tikz[baseline]{\node[anchor=base,rounded corners,fill=gray!5]
      (n)
      {$\displaystyle x^4+x^3+x^2+x+1$};
      \node[above=0pt of n, font=\scriptsize, text=gray]{$d=1 \mid 4$};}
    \big)\cdot\big(
    \tikz[baseline]{\node[anchor=base,rounded corners,fill=gray!5]
      (n)
      {$\displaystyle x^4+2x^3+x^2+2x+1$};
      \node[above=0pt of n, font=\scriptsize, text=gray]{$d=2 \mid 4$};}
    \big)\cdot\big(
    \tikz[baseline]{\node[anchor=base,rounded corners,fill=gray!5]
      (n)
      {$\displaystyle (x^4+x^3+2x+1)(x^4+2x^3+x+1)$};
      \node[above=0pt of n, font=\scriptsize, text=gray]{$d=4 \mid 4$};}
    \big) \]
\end{beispiel}

%\begin{beispiel}
  %Sei $q = p = 3$ und $m = 22$, so haben wir über $\F_q$
  %\begin{align*} 
    %\Phi_m(x) = \Phi_{22}(x) \speq{&=} 
      %x^{10} + 2 x^{9} + x^{8} + 2 x^{7} + x^{6} + 2 x^{5} + x^{4} + 
      %2 x^{3} + x^{2} + 2 x + 1 \\
    %\speq{&=}
      %(x^{5} + 2 x^{3} + 2 x^{2} + 2 x + 1) \cdot 
      %(x^{5} + 2 x^{4} + 2 x^{3} + 2 x^{2} + 1) \,.
  %\end{align*}
  %Es ist ferner $s = \ord_{\nu(m)}(q) = \ord_{22}(3) = 5 = \bar s$. Sei
  %\[ f(x) \speq= x^{5} + 2 x^{3} + 2 x^{2} + 2 x + 1 \quad\in\F_q[x]\]
  %ein irreduzibler Teiler von $\Phi_m(x)$ in $\F_q[x]$.
  
  %Wählen wir nun wie im Beweis von \cref{satz:f_x_s_ist_teiler_von_phimd} eine
  %$(ms) = 100$-te Einheitswurzel $\zeta$, so müssen wir ein Vertretersystem von
  %Restklassen ${}\bmod m$ berechnen, um die passenden Zerlegungen in
  %Linearfaktoren angeben zu können:
  %\begin{center}
    %\[\begin{array}{r|l}
      %l \in R_q & M_q(l \bmod m) \\\hline
      %0 & 0 \\
      %1 & 1, 3, 5, 9, 15    \\
      %2 & 2, 6, 8, 10, 18   \\
      %4 & 4, 12, 14, 16, 20 \\
      %7 & 7, 13, 17, 19, 21 \\
      %11 & 11
    %\end{array}\]
  %\end{center}
  %Da nur $1$ und $7$ teilerfremd zu $22$ sind, ist also
  %\[\setlength{\arraycolsep}{3pt}\begin{array}{rcccc}
    %\Phi_{22}(x) &=& (x^{5} + 2 x^{3} + 2 x^{2} + 2 x + 1) &\cdot& 
      %(x^{5} + 2 x^{4} + 2 x^{3} + 2 x^{2} + 1) \\
    %&=& (x - \zeta)(x-\zeta^3)(x-\zeta^5)(x-\zeta^9)(x-\zeta^{15}) &\cdot&
      %(x-\zeta^7)(x-\zeta^{13})(x-\zeta^{17})(x-\zeta^{19})(x-\zeta^{21})
  %\end{array}\]
  %Gehen wir nun über $f(x^s)$ zu betrachten, so erhalten wir über $\F_q$
  %\begin{align*}
    %f(x^s) &= x^{25} + 2 x^{15} + 2 x^{10} + 2 x^{5} + 1\\
    %&= (x^{5} + 2 x^{3} + 2 x^{2} + 2 x + 1) \\
    %&\ \  \cdot (x^{20} + x^{18} + x^{17} + 2 x^{16} + x^{15} + 
      %x^{14} + 2 x^{10} + 2 x^{9} + 2 x^{8} + x^{7} + 2 x^{5} + x^{4} + x^{3} + 
      %2 x^{2} + x + 1)\,.
  %\end{align*}

  %Dies teilt $\Phi_{m}(x^s) = \Phi_{22}(x^5)$, was über $\F_q$ wie folgt
  %zerfällt:
  %\[\begin{array}{rcc}
    %\Phi_{22}(x^5) &=& \prod_{d\mid \bar s} \Phi_{md}(x)
      %=  \Phi_{22}(x) \cdot \Phi_{110}(x) \\
    %&=&
  %\end{array}\]
  %Wie $\Phi_{22}$ in Termen von $\zeta$ zerfällt haben wir oben bereits
  %gesehen,

%\end{beispiel}
