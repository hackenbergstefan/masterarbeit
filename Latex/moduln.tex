\chapter{Moduln}

Nähern wir uns der Situation von Normalbasen in möglichst allgemeiner Form, so
beginnt die Reise bei der Betrachtung folgender Situation:
\begin{definition}[$(V,\tau)$]
  Sei $\K$ ein Körper und $V$ ein $\K$-Vektorraum und 
  $\tau \in \End_\K(V)$, so können wir $V$ als $\K[x]$-Modul auffassen:
  \[ f(x) \cdot v \speq{:=} f(\tau)(v)\]
  für alle $f(x) \in \K[x]$ und $v\in V$.
  Nenne das Paar $(V,\tau)$ \emph{$\K[x]$-Modul bzgl. $\tau$}.
\end{definition}

\begin{notation}
  Sei $\tau\in \End_\K(V)$.
  \begin{itemize}
  \item Es bezeichne $\mu_\tau$ das Minimalpolynom von 
    $\tau$, also das normierte Polynom kleinsten Grades $f(x)\in \K[x]$ mit 
    $f(\tau) = 0$.
  \item Ferner schreibe $\chi_\tau$ für das charakteristische Polynom von 
    $\tau$, also $\chi_\tau(x) := \det(x \id_V - \tau) \in \K[x]$.
  \end{itemize}
\end{notation}


\begin{bemerkung}
  Ist $\K  = F :=\F_q$ ein endlicher Körper, 
  $V = E := \F_{q^n}$ eine Körpererweiterung
  von Grad $n$ und 
  \[\tau = \sigma: \funcdef{E & \to & E\\
    v &\mapsto & v^q}\]
  der Frobenius von $E$, so ist
  \[ \mu_\tau(x) \ =\ \chi_\tau(x) \ =\ x^n - 1\,,\]
  denn: Es ist klar, dass $n = \deg \chi_\tau$ und da nach dem Satz von
  Cayley-Hamilton ist $\sigma$ Nullstelle von $\chi_\tau$. Daher teilt
  $\mu_\tau$ das charakteristische Polynom. Jedoch kennen wir das
  Minimalpolynom von $\tau$: Nach Dedekinds-Unabhängigkeitslemma ist 
  $\id_E,\sigma,\ldots,\sigma^{n-1}$ linear unabhänig über $E$, also insbesondere
  über $F$, und $\sigma^n = \id_E$.\marginpar{References!}
\end{bemerkung}


\begin{definition}[$\tau$-Ordnung, Teilmodul]
  Sei $(V,\tau)$ ein $\K[x]$-Modul. Zu jedem $v \in V$ betrachte den
  $\K[x]$-Modulhomomorphismus
  \[ \psi_v: \funcdef{\K[x] & \to & V \\
    f(x) & \mapsto & f(x)\cdot v }  \]
  Sei ferner $\dim V < \infty$.
  \begin{enumerate}
    \item Ist $\ker\psi_v = (g(x))$ für $g(x)\in \K[x]$ normiert, so heißt
      $g(x)$ \emph{$\tau$-Ordnung von $v$}\@. Ferner ist $g(x)$ eindeutig.
      Schreibe $\Ord_\tau(v) := g(x)$.
    \item $\K[\tau]\cdot v := \im{\psi_v}$ heißt der von \emph{$v$ erzeugte
      $\K[x]$-Teilmodul von $V$}.
  \end{enumerate}
\end{definition}
\marginpar{Eindeutigkeit!}


\begin{notation}
  Für $\K = \F_q$ einen endlichen Körper, $V = E \mid \F_q$ eine 
  Körpererweiterung und $\tau = \sigma$ den Frobenius-Endomorphismus schreibe
  \[ \Ord_q := \Ord_\tau \]
  und bezeichne $\Ord_q$ mit \emph{$q$-Ordnung}.
\end{notation}

\begin{lemma}
  \label{lemma:eigenschaften_tau_ordnung}
  Sei $(V,\tau)$ ein $\K[x]$-Modul. Ferner seien
  $u,v\in V$ mit $g(x) := \Ord_\tau(u)$, $h(x) := \Ord_\tau(v)$ und 
  $f(x) \in \K[x]$. Dann gilt
  \begin{enumerate}
    \item $\Ord_\tau(f(x)\cdot u) = \frac{g(x)}{\ggT(f(x),g(x))}$.
    \item $\Ord_\tau(u+v) = g(x)h(x)$, falls $\ggT(g,h) = 1$.
  \end{enumerate}
\end{lemma}
\begin{proof}
  \begin{enumerate}
    \item \TODO
    \item \TODO
  \end{enumerate}
\end{proof}

\begin{lemma}
  Sei $(V,\tau)$ ein $\K[x]$-Modul. Sei $v\in V$. Dann gilt:
  \[ \dim_\K( \K[x]\cdot v ) \speq= \deg( \Ord_\tau(v) )\,.\]
\end{lemma}
\begin{proof}
  Nach dem Homomorphiesatz gilt: \marginpar{References!}
  $ \im\psi_v \cong \K[x] \big/ \ker \psi_v$.
\end{proof}


\begin{definition}[zyklischer Modul]
  $(V,\tau)$ heißt \emph{zyklischer $\K[x]$-Modul bzgl. $w$}, falls es ein 
  $w\in \K$ gibt, sodass $K[\tau]\cdot w = V$.
\end{definition}


\begin{satz}
  Es gilt:
  \[ (V,\tau) \text{ ist ein zyklischer Modul} \quad\Leftrightarrow\quad
    \mu_\tau = \chi_\tau\]
\end{satz}
\begin{proof}
  Fassen wir zunächst ein paar einfache Tatsachen zusammen:
  Ist $u \in V$, so haben wir 
  \[ \dim(\K[x]\cdot v) = \deg( \Ord_\tau(v) ) \speq\leq 
    \deg\mu_\tau \speq\leq \deg \chi_\tau \]
  und 
  \[ \Ord_\tau (v) \speq\mid \mu_\tau \speq\mid \chi_\tau \,,\]
  wobei die erste Teilbarkeitsrelation per definitionem erfüllt ist und die
  zweite gerade der Satz von Cayley-Hamilton ist.
  Damit kommen wir zum direkten Beweis:
  \begin{description}
    \item["`$\Rightarrow$"'] Sei $V$ also zyklisch bzgl. $w$, so ist dies nach
      obigem äquivalent zu $\deg(\Ord_\tau(w)) = n$. Daraus folgt aber sofort
      $\mu_t = \chi_\tau$, da beide normiert sind.
    \item["`$\Leftarrow$"'] Zunächst sei behauptet, dass es stets ein 
      $w \in V$ gibt mit $\Ord_\tau(w) = \mu_\tau$. Sei dazu 
      $\mu_\tau(x) = \prod_{i=1}^r p_i(x)^{a_i}$ die Zerlegung in irreduzible
      Faktoren über $\K[x]$, so existieren $w_i \in V$ mit
      $\Ord_\tau(w_i) = p_i^{a_i}$. Andernfalls hätten wir einen Widerspruch 
      zum Minimalpolynom von $\tau$!
      Nach \autoref{lemma:eigenschaften_tau_ordnung} ist dann aber 
      $w := \sum_{i=1}^r w_i$ ein Element in $V$ mit $\tau$-Ordnung $\mu_\tau$.

      Ist dann also $\mu_\tau = \chi_\tau$, so hat obiges $w$ genau
      $\tau$-Ordnung $\chi_\tau$; erzeugt also $V$ als $\K[x]$-Modul.
  \end{description}
\end{proof}

Nun wollen wir spezielle Untermoduln von $V$ betrachten, welche uns guten
Aufschluss über die Struktur von $V$ geben können:

\begin{notation}
  Seien $(V,\tau)$ ein $\K[x]$-Modul und $g(x) \in \K[x]$.
  Definiere
  \[ V_g \speq{:=} \{ v \in V \mid g(x)\cdot v = 0 \}\,.\]
\end{notation}

Zunächst ist klar, dass $V_g \neq 0$ nur für $g$ Teiler von $\mu_\tau$ gelten
kann. Damit können wir folgende "`Rechenregeln"' formulieren:

\begin{lemma}
  \label{lemma:moduln:ueber_schnitt_und_vereinigung_von_v_g}
  Seien $g(x), h(x) \in \K[x]$ mit $g,h \mid \mu_\tau$. Dann gilt:
  \begin{enumerate}
    \item $V_g \cap V_h \speq= V_{\ggT(g,h)}$
    \item $V_g + V_h \speq= V_{\kgV(g,h)}$
  \end{enumerate}
\end{lemma}
\begin{proof}
  Per definitionem ist klar, dass für $v \in V_g$ gerade
  $\Ord_\tau(v) \mid g$. Also können wir $V_g$ auch wie folgt auffassen:
  \[ V_g \speq= \{ v\in V:\ \Ord_\tau(v) \mid g\} \,,\]
  Damit sind die Behautungen nach \cref{lemma:eigenschaften_tau_ordnung} klar,
  denn für $v \in V$ gilt:
  \[ v\in V_g \cap V_h \speq\Leftrightarrow 
    \Ord_\tau(v) \mid g \land \Ord_\tau(v) \mid h \speq\Leftrightarrow
    \Ord_\tau(v) \mid \ggT(g,h) \speq\Leftrightarrow v \in V_{\ggT(g,h)}\]
  und ebenso
  \[ v \in V_g + V_h \speq\Leftrightarrow 
    \Ord_\tau(v) \mid g \lor \Ord_\tau(v) \mid h \speq\Leftrightarrow
    \Ord_\tau(v) \mid \kgV(g,h) \speq\Leftrightarrow v \in V_{\kgV(g,h)}\,.\]
\end{proof}

Da wir letztlich zyklische Moduln und darin Untermoduln untersuchen wollen, ist
brauchen wir noch ein Lemma, das garantiert, dass auch die Untermoduln von
zyklischen Moduln wieder zyklisch sind.

\begin{lemma}
  \label{lemma:untermoduln_bleiben_zyklisch}
  Sei $R$ ein Hauptidealbereich. Ferner seien $V$ ein zyklischer $R$-Modul und
  $W \subseteq V$ ein $R$-Untermodul. Dann ist auch $W$ zyklisch.
\end{lemma}
\begin{proof}
\TODO
\marginpar{References!}
\end{proof}


\begin{satz}
  \label{satz:moduln_ueber_v_g}
  Sei $(V,\tau)$ ein zyklischer Modul mit $\dim(V) = n$. Sei ferner 
  $g(x)\in \K[x]$ normiert mit $g\mid \mu_\tau$. Dann gilt:
  \begin{enumerate}
    \item $V_g$ ist ein $\K[x]$-Teilmodul von $V$.
    \item Alle $\K[x]$-Teilmoduln von $V$ sind von dieser Form.
    \item $V_g$ ist zyklisch bzgl. $\tau$ mit Minimalpolynom $g(x)$.
      Ferner ist $\dim(V_g) = \deg(g)$.
    \item Die Erzeuger von $V_g$ sind genau die Elemente $v\in V$ mit 
      $\Ord_\tau(v) = g$, d.h. für diese gilt
      $\K[x]\cdot v = V_g$.
  \end{enumerate}
\end{satz}
\begin{proof}
  \begin{enumerate}
    \item Klar: $0\in V_g$. Weiter seien $f(x) \in \K[x]$ und 
      $v \in V_g$ mit $h(x) := \Ord_\tau(v)$, 
      so ist nach \cref{lemma:eigenschaften_tau_ordnung}
      $\Ord_\tau(f(x)\cdot v) = \frac{h(x)}{\ggT(f,h)} \mid g(x)$.
      Damit liegt auch $f(x)\cdot v$ in $V_g$.
    \item Da $V$ ein zyklischer Modul ist, existiert ein $v\in V$ mit 
      $V = \K[x]\cdot v$. Sei nun $W\subseteq V$ ein Untermodul,
      so ist dieser nach \cref{lemma:untermoduln_bleiben_zyklisch} ebenfalls
      zyklisch; sagen wir zu $w \in W$. 

    \item
    \item
  \end{enumerate}
\end{proof}

\begin{kor}
  \label{kor:moduln_ueber_v_g}
  Sei $v \in V$ mit $\Ord_\tau(v) = g(x)$.
  Für $w \in V$ gilt dann:
  \[ w \in V_g \quad\Leftrightarrow\quad 
    w = f(x)\cdot v \text{ für ein } f(x) \in \K[x]_{< \deg g},\ \ggT(f,g)=1\]
\end{kor}
\begin{proof}
  Nach \thref{satz:moduln_ueber_v_g} ist $v$ Erzeuger von $V_g$, d.h.
  $V_g = \im \psi_v \cong \K[x]\big/(g(x))$, wobei letztere Isomorphie nach dem
  Homomorphiesatz gilt. Dies zeigt die Behauptung.
\end{proof}
