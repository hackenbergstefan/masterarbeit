\chapter{Moduln}
\label{chap:moduln}

\section{Über Moduln über Hauptidealbereichen}
\label{sec:Moduln}

Nähern wir uns der Situation von Normalbasen in möglichst allgemeiner Form, so
beginnt die Reise bei der Betrachtung von Moduln über Hauptidealbereichen. Dazu
wiederholen wir die wichtigsten Definitionen und Aussagen. Für eine intensivere
Betrachtung sei auf Standardwerke der Algebra, z.B. 
\autocite{lang2002algebra} oder \autocite{hartley1974rings}, verwiesen.
Die Referenzen als Beweise seien ohne explizite Erwähnung immer 
als beispielhafte Angabe zu verstehen und es
sei bemerkt, dass jene grundlegenden Resultate auch in anderen Werken zu 
finden sind.

\begin{definition}[Integritätsbereich]
  Sei $R$ ein kommutativer Ring, so heißt $R$ \emph{Integritätsbereich},
  falls $R$ nullteilerfrei ist und $1\neq 0$ in $R$.
\end{definition}

\begin{lemma}
  Sei $R$ ein Integritätsbereich. Dann ist $R[x]$, also der 
  univariate Polynomring über $R$, ein Integritätsbereich und
  $R[x]^\times = R^\times$.
\end{lemma}
\begin{proof}
  \autocite[Lemma 13.4]{karpfinger2010algebra}.
\end{proof}

\begin{definition}[assoziierte Elemente]
  Sei $R$ ein Integritätsbereich. Zwei Elemente $r,s\in R$ heißen 
  \emph{assoziiert}, falls sie sich nur um eine Einheit unterscheiden, d.h. ein
  $u \in R^\times$ existiert mit $r = us$.
\end{definition}

\begin{bemerkung}
  Man sieht leicht ein, dass Assoziiertheit eine Äquivalenzrelation definiert.
\end{bemerkung}

\begin{definition}[Hauptidealbereich]
  Sei $R$ ein Integritätsbereich, so heißt $R$ \emph{Hauptidealbereich}
  oder \emph{Hauptidealring}, falls jedes Ideal $I\subseteq R$ ein
  Hauptideal ist, d.h. ein $r \in R$ existiert mit
  $I = (r)$. Dabei notiere wie üblich $(r)$ das von $r$ erzeugte Ideal.
\end{definition}


\begin{lemma}
  Sei $K$ ein Körper, so sind $K$ und $K[x]$ Hauptidealbereiche.
\end{lemma}
\begin{proof}
  \autocite[Satz 17.6]{karpfinger2010algebra}.
\end{proof}

\begin{definition}[$\ggT$ und $\kgV$]
  Sei $R$ ein Hauptidealbereich und $a,b \in R$. Dann heißt $t\in R$ mit 
  \[ (a)+(b) \speq= (a,b) \speq= (t)\]
  \emph{größter gemeinsamer Teiler von $a$ und $b$}; 
  geschrieben $t=\ggT(a,b)$.\\
  Ferner heißt $T\in R$ mit 
  \[ (a)\cap (b) \speq= (T)\]
  \emph{kleinstes gemeinsames Vielfaches von $a$ und $b$};
  geschrieben $T = \kgV(a,b)$.
\end{definition}


\begin{bemerkung}
  $\ggT$ und $\kgV$ sind jeweils nur bis auf Assoziiertheit 
  eindeutig, wie man sich leicht überlegen kann.
\end{bemerkung}

\begin{definition}[irreduzibles Element]
  Sei $R$ ein Integritätsbereich, so heißt $p\in R$ 
  \emph{irreduzibel}, falls $p\neq 0$, $p\notin R^\times$ und
  gilt:
  \[ \forall a,b \in R:\ p \mid ab \ \Rightarrow\ 
    p\mid a \lor p\mid b\,.\]
\end{definition}

\begin{definition}[faktorieller Ring]
  Ein Integritätsbereich $R$ heißt \emph{faktorieller Ring}, falls jedes
  Element $0\neq r \in R$ eine Zerlegung in irreduzible Faktoren besitzt, d.h.
  \[ r \speq= e a_1 \ldots a_n\]
  mit $e \in R^\times$, $n\geq 0$ und $a_i\in R$ irreduzibel und diese
  Zerlegung eindeutig ist, d.h.  sind 
  \[ e a_1 \ldots a_n \speq= f b_1 \ldots b_m \]
  zwei Zerlegungen mit $e, f\in R^\times$, $a_i,b_j$ irreduzibel, so
  folgt $n = m$ und $a_i = u_i b_{\pi(i)}$ für eine Permutation
  $\pi$ von $\{1,\ldots,n\}$ und Einheiten $u_i$ für alle $i=1,\ldots,n$.
\end{definition}

\begin{satz}
  \label{satz:hid_sind_faktoriell}
  Hauptidealbereiche sind faktorielle Ringe.
\end{satz}
\begin{proof}
  \autocite[Theorem II.5.2]{lang2002algebra}.
\end{proof}

\begin{definition}[Modul]
  Sei $R$ ein kommutativer Ring, so ist ein \emph{$R$-Modul}
  eine abelsche Gruppe $(M,+,0)$ zusammen mit einer 
  Abbildung 
  \[ \cdot:\ R \times M \to M,\ (r,m) \mapsto r\cdot m\,,\]
  sodass für alle $r, r' \in R$, $m,m' \im M$ gilt
  \begin{enumerate}
    \item $r\cdot (r'\cdot m) = (rr')\cdot m$,
    \item $(r+r')\cdot m = r\cdot m + r'\cdot m$ und 
    \item $r\cdot(m+m') = r\cdot m + r\cdot m'$.
  \end{enumerate}
\end{definition}


\begin{definition}[zyklischer Modul]
  Ein $R$-Modul $M$ heißt \emph{zyklisch}, falls
  ein $x\in M$ existiert, so dass
  \[ M = Rx\,.\]
\end{definition}

\begin{bemerkung}
  In mancher Literatur wird auch die Schreibweise $M = (x)$ für einen
  zyklischen Modul benutzt, jedoch suggeriert $(x)$ ein Ideal in $R$ zu
  bezeichnen. Da dies im Allgemeinen nicht der Fall ist, verzichten wir auf
  diese Schreibweise.
\end{bemerkung}


\begin{satz}
  \label{satz:untermoduln_bleiben_zyklisch}
  Untermoduln zyklischer Moduln sind wieder zyklisch.
\end{satz}
\begin{proof}
  Die Aussage ist nicht anderes, als ein Spezialfall der Tatsache, dass
  Untermoduln freier Moduln wieder frei sind und die Dimension des Untermoduls
  nie größer ist als die des Moduls 
  (\autocite[Theorem 7.1]{lang2002algebra}).
\end{proof}

\begin{definition}[Ordnung eines Moduls]
  Seien $R$ ein Hauptidealbereich und $M$ ein $R$-Modul. 
  Falls $r \in R\setminus\{0\}$ existiert mit $rM = 0$, so heißt
  das bezüglich der Zerlegung in irreduzible Faktoren kleinste 
  $r \in R$, das weder $0$ noch eine Einheit ist, mit
  \[ rM \speq= 0\]
  \emph{Ordnung von $M$}.
\end{definition}

Nun können wir ein zentrales Resultat über Moduln mit Ordnung beweisen, das
sich so auch beispielsweise in \autocite[Lemma 8.10]{hartley1974rings}
wiederfindet.

\begin{satz}[Zerlegungssatz für Moduln mit Ordnung]
  \label{satz:zerlegungssatz_moduln}
  Seien $R$ ein Hauptidealbereich und 
  $M$ ein nicht-trivialer $R$-Modul mit Ordnung $r$, also $rM = 0$.
  Sei $r = e p_1^{\alpha_1}\ldots p_k^{\alpha_k}$ eine Zerlegung in irreduzible
  Faktoren, sodass die $p_i$ paarweise nicht assoziiert sind, 
  so existieren eindeutig bestimmte $R$-Moduln 
  $M_1,\ldots,M_k$ mit $p_i^{\alpha_i}M_i = 0$ für alle $i=1,\ldots,k$, sodass
  \[ M \speq= \bigoplus_{i=1}^k M_i\,.\]
\end{satz}
\begin{proof}
  Zu Beginn stellen wir fest, dass die geforderte Zerlegung für $r$ existiert
  und eindeutig ist, in dem wir ($R$ ist faktoriell nach 
  \thref{satz:hid_sind_faktoriell}) $r$ in irreduzible
  Elemente zerlegen und dann diejenigen, die zueinander assoziiert sind,
  zusammenfassen. Ferner notieren wir $d_i = \frac{r}{p_i^{\alpha_i}}$.
  Kümmern wir uns nun um die Eindeutigkeit. Sei also 
  \[ M = M_1 \oplus \ldots \oplus M_k \quad\text{mit } p_i^{\alpha_i} M_i = 0\]
  gegeben, so wollen wir zeigen, dass $M_i = d_i M$ und dadurch die Komponenten
  $M_i$ eindeutig festgelegt sind. Es gilt offenbar für alle $i=1,\ldots,k$
  \[ d_i M \speq\subseteq d_i M_1 + \ldots + d_i M_k \speq\speq= d_i M_i 
    \speq\subseteq M_i\,,\]
  denn $d_i M_j = 0$ für $i\neq j$ nach Voraussetzung. 
  Wählen wir ein $i = 1,\ldots,k$ beliebig, so ist 
  $(d_i, p_i^{\alpha_i}) = (1)$, d.h. es existieren $s,t \in R$ mit
  $sd_i + t p_i^{\alpha_i} = 1$. Für $m\in M_i$ folgt dann
  \[ m \speq= 1m \speq= (sd_i + tp_i^{\alpha_i})m \speq= d_i (sm)\ 
    \in d_i M\,. \]
  Zusammen haben wir 
  \[ M_i \speq\subseteq d_iM \subseteq d_iM_i \subseteq M_i\]
  und damit Gleichheit.

  Um die Existenz zu zeigen, definieren wir einmal $M_i := d_iM$ und müssen nun
  die geforderten Eigenschaften nachprüfen. Zunächst ist klar, dass
  $p_i^{\alpha_i}M_i = 0$. Da $(d_1,\ldots,d_k) = (1)$ existeren 
  $t_1,\ldots,t_k$ mit $\sum_{i=1}^kt_id_i$ und für alle $x \in M$ folgt
  \[ x \speq= 1x \speq= \sum_{i=1}^k d_i(t_ix) 
    \speq\in \sum_{i=1}^k d_iM \speq= \sum_{i=1}^k M_i\,.\]
  Es fehlt nur noch zu zeigen, dass diese Summe auch direkt ist. 
  Dazu sei wieder $i\in\{1,\ldots,k\}$ beliebig und 
  $y \in \sum_{i\neq j} M_j$. Also ist $d_iy = 0$. Ist ferner zusätzlich
  $y \in M_i$, so ist $p_i^{\alpha_i}y = 0$. Wie oben existieren $s,t \in R$
  mit $sd_i + tp_i^{\alpha_i} = 1$. Also
  \[ y \speq= 1y \speq= (s_di + tp_i^{\alpha_i})y \speq= 0\,,\]
  was den Beweis abschließt.
\end{proof}



\begin{definition}
  \label{def:V_r}
  Seien $M$ ein $R$-Modul und $r \in R$, so definiere
  \[ V_r \speq{:=} \{ a\in M:\ ra = 0\}\,.\]
\end{definition}


\begin{bemerkung}
  Es ist klar, dass $V_r$ wieder zu einem $R$-Modul wird, da sich $V_r$ auch
  lesen lässt als der Kern des Modulhomomorphismus
  \[ M\to M,\ a \mapsto ra\,.\]
\end{bemerkung}


\begin{definition}[Annihilator]
  Sei $M$ ein $R$-Modul. Für $S\subset M$ heißt
  \[ \Ann_R(S) := \{ r \in R:\ sr = 0\ \forall s\in S\}\]
  der \emph{Annihilator von $S$ in $R$}. 
  Für $S= \{x\}$ schreibe $\Ann_R(x) := \Ann_R(\{x\})$.
\end{definition}

\begin{bemerkung}
  Man spricht in obiger Definition auch vom \emph{Annihilator-Ideal}, da in der
  Tat $\Ann_R(S)$ ein Ideal in $R$ ist. Insbesondere, falls 
  $M = xR$ ein zyklischer $R$-Modul über einem Hauptidealbereich $R$ ist, so
  ist 
  \[ \Ann_R(x) \speq= (r)\]
  für ein $r\in R$.
\end{bemerkung}

\begin{satz}
  \label{satz:schnitt_plus_vs}
  Seien $M$ ein Modul über einem Hauptidealbereich $R$ und $r,s,t,T\in R$ mit 
  $t = \ggT(r,s)$ und $T = \kgV(r,s)$. Dann gilt
  \begin{enumerate}
    \item $V_r \cap V_s \speq= V_t$.
    \item $V_r + V_s \speq= V_T$.
  \end{enumerate}
\end{satz}
\begin{proof}
  Zunächst ist klar, dass $V_r+V_s$ und $V_r\cap V_s$ wiederum $R$-Moduln sind.
  \begin{enumerate}
    \item Sei $x\in V_t$, so ist $t \in \Ann_R(x)$ nach Definition des
      Annihilators. Dieser ist ein Ideal, also sind auch $s,t\in \Ann_R(x)$.
      Damit folgt sofort $x \in V_r\cap V_s$.
      Sei umgekehrt $x \in V_r \cap V_s$, also $rx = 0$ und $sx = 0$. 
      Nach Definition des $\ggT$ existieren $r',s'\in R$ mit 
      $t = r'r + s's$, also 
      \[ tx = r'rx + s'sx = 0\,.\]
    \item Da $r$ und $s$ Teiler von $T$ sind, ist klar, dass
      $V_r+V_s \subseteq V_T$. Sei umgekehrt $z\in V_T$.
      Schreibe nun $r = r't$, $s = s't$ und setze
      $x := s'z$, $y:=r'z$. Dann ist 
      ohne Einschränkung $rs'=T=sr'$ und 
      \[ rx \speq= rs'z \speq= Tz \speq= r'sz \speq= sy\,. \]
      Wegen $Tz = 0$ folgt $x\in V_r$ und $y \in V_s$. Da nach Wahl nun
      $(r')+(s') = (1)$, existieren $\alpha,\beta\in R$ mit 
      $\alpha r' + \beta s' = 1$ und wir folgern
      \[ z \speq= \alpha r'z + \beta s'z \speq= \alpha y + \beta x \,.\]
  \end{enumerate}
\end{proof}


\begin{lemma}
  \label{lemma:zyklischer_modul_iso}
  Sei $M = Rx$ ein zyklischer $R$-Modul. Dann gilt
  \[M \speq\cong R\big/\Ann_R(x)\]
  als $R$-Moduln und dieser Isomorphismus ist kanonisch.
\end{lemma}
\begin{proof}
  $\phi:R \to M,\ r \mapsto rx$ liefert einen surjektiven Homomorphismus von
  $R$-Moduln, dessen Kern gerade $\Ann_R(x)$ ist. Damit folgt die Behauptung
  sofort aus dem Homomorphiesatz für Moduln.
\end{proof}

\begin{lemma}
  \label{lemma:teilmoduln_prim_moduln}
  Sei $Z = Rz$ ein zyklischer $R$-Modul von Ordnung $p^\alpha$ für ein
  Primelement $p\in R$. Dann sind die einzigen Teilmoduln von $Z$
  \[ 0 = Z_\alpha \speq\subseteq Z_{\alpha-1} \speq\subseteq \ldots
    \speq\subseteq Z_0 = Z \,,\]
  wobei $Z_\beta = p^\beta Z$.
\end{lemma}
\begin{proof}
  Nach \thref{lemma:zyklischer_modul_iso} ist $Z \cong R\big/(p^\alpha)$. Damit
  stehen die Teilmoduln von $Z$ in Bijektion zu den Teilmoduln von
  $R$ (gelesen als $R$-Modul), die $(p^\alpha)$ enthalten. Solch ein
  Teilmodul ist also gerade ein Ideal $(r)$ mit 
  $(p^\alpha)\subseteq (r)$. Da $p$ prim ist, folgt $r = up^\beta$ für
  $0\leq\beta \leq \alpha$ und $u\in R^\times$, wobei $u=1$ \obda angenommen
  werden kann. Damit sind die $Z_\beta$ die einzigen Teilmoduln von $Z$.
\end{proof}



\begin{lemma}
  \label{lemma:annihilator_ggt}
  Sei $x \in M$ mit $\Ann_R(x) = (\lambda)$. Für $r \in R$ gilt
  \[ \Ann_R(rx) \speq= (\delta) \]
  mit $\lambda = \delta t$ für $t = \ggT(r,\lambda)$.
\end{lemma}
\begin{proof}
  Schreibe $r = r't$, so ist $\delta rx = \delta r'tx = r'\lambda x = 0$. Also
  $\delta \in \Ann_R(rx)$. Für die andere Inklusion sei $s \in \Ann_R(rx)$,
  also $srx = 0$. Damit ist aber $sr \in \Ann_R(x)$ und
  $\lambda \mid sr$. Mit $\lambda = \delta t$ und $sr = sr't$ folgt
  $\delta \mid sr'$.
  Nach Definition des $\ggT$ sind $r'$ und $\delta$ teilerfremd, wodurch
  $\delta \mid s$. Also $s \in (\delta)$.
\end{proof}


\begin{lemma}
  \label{lemma:annihilator_teilerfremd}
  Seien $x,y \in M$ mit $\Ann_R(x) = (a)$ und $\Ann_R(y) = (b)$. Sind 
  $a$ und $b$ teilerfremd, so gilt $\Ann_R(x+y) = (ab)$.
\end{lemma}
\begin{proof}
  Zunächst ist klar, dass $ab(x+y) = abx + aby = 0$, also 
  $(ab) \subseteq \Ann_R(x+y)$. Ist nun $t(x+y) = 0$ für ein $t \in R$, so
  ist $z:= tx = -ty \in (x)\cap(y)$. Ferner ist 
  $(x)\cap(y) \subseteq V_a \cap V_b$ nach Voraussetzung. 
  Nach \thref{satz:schnitt_plus_vs} (1) ist aber 
  $V_a \cap V_b = V_1 =  (0)$,
  also auch $z=0$. Damit ist 
  \[ t \in \Ann_R(x)\cap \Ann_R(y) \speq= (a)\cap(b) \speq= 
    \speq= (\kgV(a,b)) \speq= (ab)\,,\]
  da $a$ und $b$ nach Voraussetzung teilerfremd sind.
  Also ist $\Ann_R(x+y) \subseteq (ab)$.
\end{proof}


\section{Vektorräume als Moduln}

\begin{definition}[$(V,\tau)$]
  \label{def:V_tau}
  Sei $\K$ ein Körper, $V$ ein $\K$-Vektorraum und 
  $\tau \in \End_\K(V)$, so können wir $V$ als $\K[x]$-Modul auffassen:
  \[ f(x) \cdot v \speq{:=} f(\tau)(v)\]
  für alle $f(x) \in \K[x]$ und $v\in V$.
  Nenne das Paar $(V,\tau)$ ein \emph{$\K[x]$-Modul bzgl. $\tau$}.
\end{definition}

\begin{notation}
  Sei $\tau\in \End_\K(V)$.
  \begin{itemize}
  \item Es bezeichne $\mu_\tau$ das Minimalpolynom von 
    $\tau$, also das normierte Polynom kleinsten Grades $f(x)\in \K[x]$ mit 
    $f(\tau) = 0$.
  \item Ferner schreibe $\chi_\tau$ für das charakteristische Polynom von 
    $\tau$, also $\chi_\tau(x) := \det(x \id_V - \tau) \in \K[x]$.
  \end{itemize}
\end{notation}


\begin{bemerkung}
  \label{bem:mipo_frob}
  Ist $\K  = F :=\F_q$ ein endlicher Körper, 
  $V = E := \F_{q^n}$ eine Körpererweiterung
  von Grad $n$ und 
  \[\tau = \sigma: \funcdef{E & \to & E\\
    v &\mapsto & v^q}\]
  der Frobenius von $F$, so ist
  \[ \mu_\sigma(x) \ =\ \chi_\sigma(x) \ =\ x^n - 1\,,\]
  denn: Es ist klar, dass $n = \deg \chi_\sigma$ und nach dem Satz von
  Cayley-Hamilton ist $\sigma$ Nullstelle von $\chi_\sigma$. Daher teilt
  $\mu_\sigma$ das charakteristische Polynom. Jedoch kennen wir das
  Minimalpolynom von $\sigma$: Nach dem Dedekindschen Lemma
  (\thref{satz:dedekindsches_lemma}) ist 
  $\id_E,\sigma,\ldots,\sigma^{n-1}$ linear unabhängig über $E$, 
  also insbesondere über $F$, und $\sigma^n = \id_E$.
\end{bemerkung}


\begin{definition}[$\tau$-Ordnung, Teilmodul]
  Sei $(V,\tau)$ ein $\K[x]$-Modul. Zu jedem $v \in V$ betrachte den
  $\K[x]$-Modulhomomorphismus
  \[ \psi_v: \funcdef{\K[x] & \to & V\,, \\
    f(x) & \mapsto & f(x)\cdot v }  \]
  Sei ferner $\dim V < \infty$.
  \begin{enumerate}
    \item Ist $\ker\psi_v = (g(x))$ für $g(x)\in \K[x]$ normiert, so heißt
      $g(x)$ \emph{$\tau$-Ordnung von $v$}\@. Ferner ist $g(x)$ eindeutig.
      Schreibe $\Ord_\tau(v) := g(x)$.
    \item $\K[\tau]\cdot v := \im{\psi_v}$ heißt der von \emph{$v$ erzeugte
      $\K[x]$-Teilmodul von $V$}.
  \end{enumerate}
\end{definition}

\begin{bemerkung}
  Die Eindeutigkeit der $\tau$-Ordnung wird sofort klar, wenn man sich
  überlegt, dass für einen Hauptidealbereich $R$ mit $(a) = (b) \subseteq R$ 
  gilt: Schreibe $a = b b'$ und $b = a a'$, so gilt $a = a a'b'$, also 
  $a'b' = 1$. Damit unterscheiden sich die Erzeuger zweier gleicher Hauptideale
  lediglich um eine Einheit. Wir haben jedoch $\K[x]^\times = \K^\times$. Die
  Forderung nach Normiertheit klärt damit abschließend die Eindeutigkeit.
\end{bemerkung}


\begin{bemerkung}
  In Notation von \autoref{sec:Moduln} gilt offensichtlich
  \[ \Ann_{\K[x]}(v) \speq= \big(\Ord_\tau(v)\big) \,.\]
\end{bemerkung}

\begin{notation}
  Für $\K = \F_q$ einen endlichen Körper, $V = E \mid \F_q$ eine 
  Körpererweiterung und $\tau = \sigma$ den Frobenius-Endomorphismus schreibe
  \[ \Ord_q := \Ord_\tau \]
  und bezeichne $\Ord_q$ mit \emph{$q$-Ordnung}.
\end{notation}

\begin{lemma}
  \label{lemma:eigenschaften_tau_ordnung}
  Sei $(V,\tau)$ ein $\K[x]$-Modul. Ferner seien
  $u,v\in V$ mit $g(x) := \Ord_\tau(u)$, $h(x) := \Ord_\tau(v)$ und 
  $f(x) \in \K[x]$. Dann gilt
  \begin{enumerate}
    \item $\Ord_\tau(f(x)\cdot u) = \frac{g(x)}{\ggT(f(x),g(x))}$.
    \item $\Ord_\tau(u+v) = g(x)h(x)$, falls $\ggT(g,h) = 1$.
  \end{enumerate}
\end{lemma}
\begin{proof}
  \begin{enumerate}
    \item Dies ist lediglich eine Umformulierung von
      \thref{lemma:annihilator_ggt}.
    \item In \thref{lemma:annihilator_teilerfremd} haben wir gesehen, dass
      in diesem Fall $\Ann_{\K[x]}(u+v) = (g(x)h(x))$ gilt.
  \end{enumerate}
\end{proof}

\begin{lemma}
  \label{lemma:modul_dim_deg}
  Sei $(V,\tau)$ ein $\K[x]$-Modul. Sei $v\in V$. Dann gilt:
  \[ \dim_\K( \K[x]\cdot v ) \speq= \deg( \Ord_\tau(v) )\,.\]
\end{lemma}
\begin{proof}
  Nach dem Homomorphiesatz gilt: 
  $ \im\psi_v \cong \K[x] \big/ \ker \psi_v$ als $\K[x]$-Moduln.
\end{proof}


\begin{definition}[zyklischer {$\K\lbr x \rbr$}-Modul]
  $(V,\tau)$ heißt \emph{zyklischer $\K[x]$-Modul bzgl. $w$}, falls es ein 
  $w\in \K$ gibt, sodass $\K[\tau]\cdot w = V$.
\end{definition}


\begin{satz}
  Es gilt:
  \[ (V,\tau) \text{ ist ein zyklischer Modul} \quad\Leftrightarrow\quad
    \mu_\tau = \chi_\tau\]
\end{satz}
\begin{proof}
  Fassen wir zunächst ein paar einfache Tatsachen zusammen:
  Ist $v \in V$, so haben wir 
  \[ \dim(\K[x]\cdot v) = \deg( \Ord_\tau(v) ) \speq\leq 
    \deg\mu_\tau \speq\leq \deg \chi_\tau \]
  und 
  \[ \Ord_\tau (v) \speq\mid \mu_\tau \speq\mid \chi_\tau \,,\]
  wobei die erste Teilbarkeitsrelation per Definition erfüllt ist und die
  zweite gerade der Satz von Cayley-Hamilton ist.
  Damit kommen wir zum eigentlichen Beweis:
  \begin{description}
    \item["`$\Rightarrow$"'] Sei $V$ also zyklisch bzgl. $w$, so ist dies nach
      Obigem äquivalent zu $\deg(\Ord_\tau(w)) = n$. Daraus folgt aber sofort
      $\mu_t = \chi_\tau$, da beide normiert sind.
    \item["`$\Leftarrow$"'] Zunächst sei behauptet, dass es stets ein 
      $w \in V$ gibt mit $\Ord_\tau(w) = \mu_\tau$. Sei dazu 
      $\mu_\tau(x) = \prod_{i=1}^r p_i(x)^{a_i}$ die Zerlegung in irreduzible
      Faktoren über $\K[x]$, so existieren $w_i \in V$ mit
      $\Ord_\tau(w_i) = p_i^{a_i}$. Andernfalls hätten wir einen Widerspruch 
      zum Minimalpolynom von $\tau$!
      Nach \autoref{lemma:eigenschaften_tau_ordnung} ist dann aber 
      $w := \sum_{i=1}^r w_i$ ein Element in $V$ mit $\tau$-Ordnung $\mu_\tau$.

      Ist dann also $\mu_\tau = \chi_\tau$, so hat obiges $w$ genau
      $\tau$-Ordnung $\chi_\tau$; erzeugt also $V$ als $\K[x]$-Modul.
  \end{description}
\end{proof}


\begin{satz}
  \label{satz:moduln_ueber_v_g}
  Sei $(V,\tau)$ ein zyklischer Modul über einem endlich dimensionalem
  Vektorraum. Sei ferner $g(x)\in \K[x]$ normiert mit $g\mid \mu_\tau$. 
  Dann gilt:
  \begin{enumerate}
    \item $V_g$ (siehe \thref{def:V_r}) ist ein $\K[x]$-Teilmodul von $V$.
    \item Alle $\K[x]$-Teilmoduln von $V$ sind von dieser Form.
    \item Die Erzeuger von $V_g$ sind genau die Elemente $v\in V$ mit 
      $\Ord_\tau(v) = g$, d.h. für diese gilt
      $\K[x]\cdot v = V_g$.
      Insbesondere sind die Erzeuger von $V$ gerade die Elemente $u\in V$ mit
      $\Ord_\tau(u) = \mu_\tau$.
    \item $V_g$ ist zyklisch bzgl. $\tau$ mit Minimalpolynom $g(x)$.
      Ferner ist $\dim(V_g) = \deg(g)$.
  \end{enumerate}
\end{satz}
\begin{proof}
  \begin{enumerate}
    \item Klar: $0\in V_g$. Weiter seien $f(x) \in \K[x]$ und 
      $v \in V_g$ mit $h(x) := \Ord_\tau(v)$, 
      so ist nach \thref{lemma:eigenschaften_tau_ordnung}
      $\Ord_\tau(f(x)\cdot v) = \frac{h(x)}{\ggT(f,h)} \mid g(x)$.
      Damit liegt auch $f(x)\cdot v$ in $V_g$.
    \item Dies ist lediglich eine Umformulierung von
      \thref{satz:zerlegungssatz_moduln}
      und \thref{lemma:teilmoduln_prim_moduln}.
    \item Sei $v \in V$ ein Erzeuger von $V_g$, so ist 
      $g(x)\cdot v = 0$. Also $\Ord_\tau(v) \mid g(x)$. Schreibe
      $\Ord_\tau(v) =: h(x)$, so folgt
      \[ h(x)\cdot V_g \speq= \K[x]\cdot (h(x)\cdot v) \speq=0\,.\]
      Also $g(x) = h(x)$.
      Ist andererseits $w \in V$ mit $\Ord_\tau(w) = g(x)$, 
      so können wir zunächst festhalten, dass $\K[x] \cdot w \subseteq V_g$.
      Sei ferne $x \in V$ ein Erzeuger von $V_g$. Dann ist $w = f(x)\cdot v$ für
      ein $f(x)\in \K[x]$ mit $\ggT(f(x),g(x)) = 1$ nach 
      \thref{lemma:eigenschaften_tau_ordnung}. Also existieren
      $f'(x),g'(x)\in \K[x]$ mit $f'f + g'g = 1$. Wir folgern:
      \[ v \speq= (f'(x)f(x) + g'(x)g(x)) \cdot v \speq=
        f'(x)\cdot w\,,\]
      d.h. $v \in \K[x]\cdot w$; mithin $V_g \subseteq \K[x]\cdot w$.
    \item Klar nach (3) und \thref{lemma:modul_dim_deg}.
  \end{enumerate}
\end{proof}


\begin{bemerkung}
  Die Punkte (1) und (2) hätten bereits in \autoref{sec:Moduln} aufgeführt
  werden können, da die Zyklizität des Moduls hier ausreichend war.
  In \autocite[Theorem 7.10]{hachenberger1997finite} ist dies auch so
  vorzufinden.
\end{bemerkung}

\begin{kor}
  \label{kor:moduln_ueber_v_g}
  Sei $v \in V$ mit $\Ord_\tau(v) = g(x)$.
  Für $w \in V$ gilt dann:
  \[ w \in V_g \quad\Leftrightarrow\quad 
    w = f(x)\cdot v \text{ für ein } f(x) \in \K[x]_{< \deg g}\]
\end{kor}
\begin{proof}
  Nach \thref{satz:moduln_ueber_v_g} ist $v$ Erzeuger von $V_g$, d.h.
  $V_g = \im \psi_v \cong \K[x]\big/(g(x))$, wobei letztere Isomorphie nach dem
  Homomorphiesatz gilt. Dies zeigt die Behauptung.
\end{proof}


Wir schließen dieses Kapitel mit einer wesentlichen Beobachtung, über den
Zusammenhang von Zerlegungen im Ring $\K[x]$ und im Vektorraum $V$. 
Für den Spezialfall von zyklischen Galoiserweiterung findet man den Satz auch
in \autocite[Theorem 8.6]{hachenberger1997finite} wieder.

\begin{definition}[Zerlegung]
  Sei $f(x)\in \K[x]$ normiert mit $\deg f \geq 1$, so heißt 
  $\Delta \subseteq \K[x]$ \emph{Zerlegung von $f(x)$}, falls gilt: Alle
  $\delta \in \Delta$ sind normiert, vom Grad größer gleich 1, paarweise
  teilerfremd und es gilt: $f(x) = \prod_{\delta\in \Delta} \delta(x)$.
\end{definition}


\begin{satz}
  \label{satz:zerlegungssatz_zykl_vektorraume}
  Sei $(V,\tau)$ ein zyklischer Modul von endlicher Dimension. Seien ferner 
  $g(x) \in \K[x]$ normiert mit $g\mid \mu_\tau$ und $\Delta$ eine 
  Zerlegung von $g$. Dann gilt:
  \begin{enumerate}
    \item $V_g = \oplus_{\delta\in\Delta} V_\delta$ ist eine direkte Summe 
      von zyklischen Moduln bzgl. $\tau$.
    \item Jedes $w \in V_g$ lässt sich eindeutig schreiben als 
      $w = \sum_{\delta\in\Delta} w_\delta$ mit $w_\delta \in V_\delta$. Ferner
      gilt 
      \[ \Ord_\tau(w) \speq= \prod_{\delta\in \Delta} \Ord_\tau(w_\delta)\]
      und $\Ord_\tau(w)$ ist ein normierter Teiler von $g(x)$.
    \item $w$ ist ein Erzeuger von $V_g$ genau dann, wenn für alle 
      $\delta\in\Delta$ auch $w_\delta$ Erzeuger von $V_\delta$ sind.
    \item Ist $V_g = \oplus_{i \in I} V_i$ eine Zerlegung in Teilmoduln,
      so existieren eine Zerlegung $\Delta$ von $g$ und 
      eine Bijektion $\pi:I\to\Delta$, so dass $V_i = V_{\pi(i)}$.
  \end{enumerate}
\end{satz}
\begin{proof}
  \begin{enumerate}
    \item Betrachten wir $V_g$ selbst als $\K[x]$-Modul, so ist diese Aussage
      lediglich eine Anwendung von \thref{satz:zerlegungssatz_moduln}.
    \item Dass die geforderte Zerlegung in $w_\delta$ existiert und eindeutig
      ist, ist eine direkte Konsequenz aus (1). Da $w$ von $g(x)$ annihiliert
      wird, ist klar, dass $\Ord_\tau(w) \mid g$. Analog gilt auch
      $\Ord_\tau(w_\delta)\mid \delta$ für alle $\delta \in \Delta$. Diese sind
      jedoch teilerfremd und wir erhalten die postulierte Gleichung aus
      \thref{lemma:eigenschaften_tau_ordnung} (2).
    \item Dies ist mit (2) und \thref{satz:moduln_ueber_v_g} (3) sofort klar.
    \item Nach \thref{satz:moduln_ueber_v_g} (2) sind alle $V_i = V_{\pi(i)}$
      für einen normierterten Teiler $\pi(i) \mid g$. Da die Summe direkt ist,
      folgt die Behauptung mit \thref{satz:schnitt_plus_vs}.
  \end{enumerate}
\end{proof}


\begin{kor}
  \label{kor:anzahl_erzeuger}
  Seien $(V,\tau)$ und $g(x)\in \K[x]$ wie oben. Sei ferner
  $g(x) = \prod_{i=1}^k g_i(x)^{\nu_i}$ die vollständige Faktorisierung von 
  $g(x)$ über $\K[x]$. Dann ist die Menge aller Elemente 
  in $V$ mit $\tau$-Ordnung $g(x)$ gleich
  \[ \bigoplus_{i=1}^k \big( V_{g_i^{\nu_i}} \setminus V_{g_i^{\nu_i-1}} \big)\]
\end{kor}
\begin{proof}
  Da obige Faktorisierung vollständig ist, sind die $g_i(x)$ irreduzibel über
  $\K[x]$ für alle $i=1,\ldots,k$. Also sind alle Elemente 
  in $V_{g_i^{\nu_i}}$ die nicht von $g_i^{\nu_i-1}$ anihiliert werden bereits
  Erzeuger von $V_{g_i^{\nu_i}}$; haben mithin $\tau$-Ordnung $g_i^{\nu_i}$.
  Der Rest ist klar nach obigem Satz.
\end{proof}
