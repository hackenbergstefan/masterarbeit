\chapter{Explizite Konstruktion von Normalbasen}

Wir haben nun kennengelernt, wie man einen 
Vektorraum $V$ über $\K$ zusammen mit einem Endomorphismus $\tau$ als
$\K[x]$-Modul lesen kann. Analog dazu wollen wir nun eine Erweiterung
endlicher Körper $E$ über $F$ in diesem Kontext verstehen: $E$ ist ein
Vektorraum über $F$ und wird mit Hilfe des Frobenius 
$\sigma: \bar F \to \bar F$ zu einem $F[x]$-Modul im Sinne von 
\thref{def:V_tau}. Dies wird uns helfen, normale (und später vollständig
normale) Elemente zu konstruieren.

\section{Grundlegende Ideen}

Seien im Folgenden stets $F := \F_q$ ein endlicher Körper von 
Charakteristik $p$ und $E := \F_{q^n} \mid F$ eine Körpererweiterung.
Wir wiederholen kurz die Definition einer \emph{Normalbasis}.

\begin{definition}[normales Element, normales Polynom, Normalbasis]
  Sei $F$ ein Körper und $E \mid F$ eine endliche Galoiserweiterung von Grad
  $n$. Sei ferner $w\in E$ mit $F(w) = E$. $w$ heißt \emph{normal über $F$},
  falls
  \[ \{ \gamma(w) \mid \gamma \in \Gal(E\mid F) \}\]
  eine $F$-Basis von $E$ ist. 
  $\{ \gamma(w) \mid \gamma \in \Gal(E\mid F)\}$ heißt entsprechend 
  \emph{Normalbasis} und $g(x) \in F[x]$ mit 
  \[ g(x) = \prod_{\gamma \in \Gal(E\mid F)}(x - \gamma(w))\]
  heißt \emph{normales Polynom}.
\end{definition}

Wir wollen nun die Begriffe der normalen Elemente und der Erzeuger von
zyklischen Moduln aus vorhergehendem Kapitel in Zusammenhang bringen. Dazu
müssen wir uns Gedanken machen, wie sich Normalität im Kontext der
Modulstruktur einer Körpererweiterung lesen lässt.

\begin{satz}
  Ein Element $u\in E$ ist genau dann normal über $F$, wenn
  \[ \Ord_q(u) \speq= x^n-1 \quad\in F[x]\,.\]
\end{satz}
\begin{proof}
  Nach \thref{bem:mipo_frob} ist das Minimalpolynom des Frobenius von $F$
  gerade $x^n-1$. Ferner wissen wir nach \thref{satz:frob_sind_alle_autos},
  dass $\Gal(E\mid F) = \langle\sigma\rangle$. 
  Letztlich liefert \thref{satz:moduln_ueber_v_g} (3), dass die Erzeuger von 
  $E$ als $F[x]$-Modul, also gerade die normalen Elemente, jene mit $q$-Ordnung
  $x^n-1$ sind.
\end{proof}

\begin{kor}
  Sei $x^n-1 = \prod_{i=1}^s r_i(x)$ eine Zerlegung in paarweise teilerfremde
  Polynome. Seien ferner $u_i \in V_{r_i}$ Elemente mit 
  $\Ord_q(u_i) = r_i(x)$ $\forall i=1,\ldots,s$. Dann ist
  \[ u \speq= u_1 + u_2 + \ldots + u_s\]
  normal in $E \mid F$.
\end{kor}
\begin{proof}
  %Da obige Zerlegung von $x^n-1$ paarweise teilerfremd ist, folgt nach
  %\thref{lemma:eigenschaften_tau_ordnung}
  %$\Ord_q(u) = \prod_{i=1}^s \Ord_q(u_i) = \prod_{i=1}^s r_i(x) = x^n -1$.
  \thref{satz:zerlegungssatz_zykl_vektorraume}.
\end{proof}

\section{Stark reguläre Erweiterungen}
\label{sec:stark_regulare_erweiterungen}

\begin{beispiel}
  \label{bsp:stark_regular_1}
  Wählen wir einmal $q = 7$ und $n=9$. Also $F = \F_7$.
  Ferner wissen wir aus \autoref{chap:kreisteilungspolynome}, dass
  \[ x^n-1 \speq= x^9-1 \speq= 
    \Phi_1(x)\Phi_3(x)\Phi_9(x) \speq=
    \big(x-1\big)\big( (x+3)(x+5)\big)
    \big( (x^3+3)(x^3+5) \big) \ \in \F_7[x]\]
  die vollständige Faktorisierung von $x^n-1$ über $\F_7$ ist.
  Da es sich hier bei den Faktoren lediglich um Binome handelt, können wir
  relativ einfach Elemente passender $q$-Ordnungen angeben: Beginnen wir 
  mit $\Phi_9 = (x^3+3)(x^3+5)$. Gesucht ist nun ein Element $u$ in einer 
  passenden Erweiterung von $F$ mit $\Ord_q(u)= \Phi_9$.
  Sei dazu $u \in \bar F$ eine primitive $27$-te Einheitswurzel. Dann
  gilt:
  \[ \ord(u^{342}) \speq= \ord(u^{18}) \speq= \frac{27}{\ggT(27,18)} \speq= 3\]
  Wenn man sich die Frage stellt, warum an diesem Punkt gerade $342$ eine
  interessante Zahl ist, so wird man diese sofort wiederfinden, wenn man
  versucht ein Element $w\in \bar F^\ast$ mit $q$-Ordnung $x^3+3$ (bzw.
  $x^3+5$) wiederzufinden:
  \[ (x^3+3)\cdot w \speq= w^{q^3} + 3 w \speq= 
    w(w^{7^3-1} +3) \speq= w(w^{342}+3) \speq{\overset{!}{=}} 0.\]
  Da $w \neq 0$ und per definitionem der Kreisteilungspolynome 
  $(-3),(-5)\in \F_7$ primitive $3$-te Einheitswurzeln sind, 
  brauchen wir ein $w$ mit $\ord(w^{342}) = 3$; 
  was obiges $u$ gerade erfüllt! Damit ist also $u^{18} = u^{342}$ 
  eine der beiden dritten Einheitswurzeln $(-3)$ oder $(-5) \in \F_7$ und wir
  können mit \thref{lemma:eigenschaften_tau_ordnung} folgern:
  \[ \Ord_q(u + u^2) \speq= \Phi_9\]

  Auch die Suche nach einem Element mit $q$-Ordnung $\Phi_3$ ist damit
  erledigt: Mit analoger Argumentation wie oben erhalten wir, dass 
  \[ \Ord_q(u^{3} + u^{6}) \speq= \Phi_3.\]
  Zusammengefasst ist also $u$ ein normales Element von $\F_{7^9} \mid \F_7$.
\end{beispiel}

Ein entscheidender Vorteil in der Konstruktion eines normalen Elements in
obigem Beispiel war der Zerfall der Kreisteilungspolynome in Binome. Aus 
\thref{satz:kreisteilungspolynome_binome}
wissen wir bereits, wann die Kreisteilungspolynome in Binome zerfallen.
Damit können wir diese sehr einfache Möglichkeit Normalbasen explizit anzugeben,
als Reihe von Aussagen formulieren:

\begin{definition}[stark regulär]
  \label{def:stark_regular}
  Das Paar $(q,n) \in \N^2$ heißt \emph{stark regulär}, falls
  \begin{itemize}
    \item $q = p^r$ mit $p$ einer Primzahl und $r>0$,
    \item $p \nmid n$,
    \item $\nu(n) \mid q-1$,
    \item $4 \mid q-1$, falls $n$ gerade.
  \end{itemize}
  Schreibe $n \in \S_q$, falls $(q,n)$ stark regulär.
  Eine Körpererweiterung $\F_{q^n} \mid \F_q$ heißt \emph{stark regulär}, falls
  $n \in \S_q$.
\end{definition}


\begin{satz}
  \label{satz:konstruktion_q_ordnung}
  Sei $F = \F_q$ ein endlicher Körper und $m \in \S_q$. Seien
  $l := \pi_m(q-1)$, $a := \ggT(l,m)$ und $u \in \bar\F_q$ eine primitive
  $(ml)$-te Einheitswurzel. Dann gilt:
  \begin{enumerate}
    \item $\F_q(u) = \F_{q^m} =: E$.
    \item $\Ord_q(u)$ ist ein irreduziblerTeiler von $\Phi_m$ in $\F_q[x]$.
    \item $v := \sum_{i \in I_a} u^i$ hat $q$-Ordnung $\Phi_m$.
  \end{enumerate}
\end{satz}
\begin{proof}
  \begin{enumerate}
  \item Aus \thref{lemma:cl_2} haben wir 
   $\pi_m(q^m-1) = m\pi_m(q-1) = ml$ und
    damit $[F(u):F] = \ord_{ml}(q) = m$. 
  \item Sei nun $\zeta \in F^\ast$ eine primitive $a$-te Einheitswurzel. Dann
    ist nach \thref{satz:kreisteilungspolynome_binome}
    \[ \Phi_m(x) \speq= \prod_{i\in I_a} (x^{\frac m a} - \zeta^i).\]
    Wie in \thref{bsp:stark_regular_1} betrachten wir für $i \in I_a$:
    \begin{align*}
      (x^{\frac m a} - \zeta^i)\cdot u 
      \quad&=\quad \big( \sigma^{\frac m a} - \zeta^i \id_E \big)\\
        &=\quad u^{q^{\frac m a}} - \zeta^i u \\
        &=\quad u\big( u^{q^{\frac m a}-1} - \zeta^i \big)\,, \tag{$\ast$}
    \end{align*}
    wobei wie üblich $\sigma: E \to E, x \mapsto x^q$ den Frobenius von
    $E\mid F$ meint. Wir wollen nun zeigen, dass $\Ord_q(u) = (x^{\frac m a} -
    \zeta^i)$ für ein $i\in I_a$, explizit also, dass $(\ast) = 0$ für ein
    $i\in I_a$ gilt. Dazu müssen wir $q^{\frac m a} -1$ genauer untersuchen:
    Wiederum nach \thref{lemma:cl_2} haben wir:
    \[ \pi_m(q^{\frac m a} -1 ) \speq= \frac m a \pi_m(q-1) \speq=
      \frac{ml}{a} \speq= \kgV(m,l) \speq{=:} c\,.\]
    Nun ist $q^{\frac m a}-1 = ck$ für 
    ein $k\in\N$ mit $\ggT(k,ml) = 1$, wie man sich anhand der 
    Primfaktorzerlegungen leicht klar machen kann. Erinnern wir uns kurz, 
    dass $u$ eine primitive $(ml)$-te Einheitswurzel war, so folgern wir:
    \[ \ord(u^{ck}) \speq= \ord(u^c) \speq= \frac{ml}{\ggT(ml,c)}
      \speq= \frac{ml}{c} \speq= a\,.\]
    Ergo gibt es $j\in I_a$ mit $u^{ck} = \zeta^j$ und für dieses $j$ ist
    $(\ast)$ gerade $0$, was zu zeigen war.
  \item Seien $j\in I_a$ mit $\Ord_q(u) = x^\frac m a - \zeta^j$ und 
    $i\in I_a$ beliebig, so gilt
    \begin{align*}
      \big( x^\frac m a - \zeta^{ji} \big) u^i \speq{&=}
      (u^{q^\frac m a})^i - (\zeta^ju)^i \\
      \speq{&=} (u^{q^\frac m a} - \zeta^ju) 
        \sum_{k=0}^{i-1} u^{k\,q^\frac m a} \zeta^{(i-1-k)\,j}u^{i-1-k} \\
        \speq{&=} 0\,.
    \end{align*}
    Da $i$ teilerfremd zu $ml$ ist, haben wir ferner $\ord(u) = \ord(u^i)$ und
    können folgern, dass $\Ord_q(u^i) = x^\frac m a - \zeta^{ji}$.
    Packen wir nun alles zusammen und bedenken, dass
    $I_a \to I_a, i\mapsto ij$ eine Bijektion ist, können wir den letzten
    Schritt im Beweis führen:
    \[ \Ord_q\big( \sum_{i\in I_a} u^i \big)
      \speq= \prod_{i\in I_a} \Ord_q(u^i) \speq=
      \prod_{i \in I_a} \big( x^\frac m a - \zeta^{ji} \big)
      \speq= \prod_{k\in K_a} \big(x^\frac m a - \zeta^k \big)
      \speq= \Phi_m(x)\,.\]
  \end{enumerate}
\end{proof}

Da wir nun die einzelnen Bausteine kennen, können wir auf die Faktorisierung
und damit auf eine explizite Angabe eines normalen Elements zu schließen:

\begin{kor}
  \label{kor:konstruktion-q-ordnung}
  Für $F = \F_q$, $n\in \S_q$ sei $\lambda \in F^\ast$ eine primitive 
  $\pi_n(q-1)$-te Einheitswurzel. Zu jedem $m \mid n$ seien
  \begin{itemize}
    \item $a(q,m) := \ggT(m,\pi_m(q-1))$ und
    \item $I_a := \{ i\leq a:\ \ggT(i,a) = 1\}$.
  \end{itemize}
  Dann ist
  \[ x^n-1 \speq= \prod_{m\mid n}\ \prod_{i\in I_{a(q,m)}} 
    \left( x^\frac{m}{a(q,m)} - \lambda^{\frac{\pi_n(q-1)}{a(q,m)} i}\right)\]
  die vollständige Faktorisierung von $x^n-1$ über $\F_q$.
\end{kor}
\begin{proof}
  Da  $\ord\big(\lambda^{\frac{\pi_n(q-1)}{a(q,m}}\big) = a(q,m)$
  und für $m\mid n$ offensichtlich auch $m\in \S_q$, ist obige Aussage lediglich
  eine Anwendung von \thref{satz:konstruktion_q_ordnung}.
\end{proof}

\begin{satz}
  Seien $F = \F_q$ ein endlicher Körper, $n\in \S_q$ und $L:=\pi_n(q-1)$.
  Ferner sei $u \in \bar F$ eine primitive $(nL)$-te Einheitswurzel. Dann ist
  mit Notation aus \thref{kor:konstruktion-q-ordnung}
  \[ w \speq{:=} \sum_{m\mid n} \ \sum_{i\in I_a}
    u^{\frac{n\, L}{m\, \pi_m(q-1)}i}\]
  normal in $E := \F_{q^n}$ über $F$.
\end{satz}
\begin{proof}
  Im Grunde haben wir bereits alles gezeigt. Daher reicht ein kurzer Kommentar,
  warum wir \thref{satz:konstruktion_q_ordnung} anwenden können aus. Es ist
  trivialerweise 
  \[ \ord\big( u^{\frac{nL}{m\pi_m(q-1)}} \big) \speq=
    m\, \pi_m(q-1)\]
  und damit sind alle Voraussetzungen erfüllt.
\end{proof}

\begin{bemerkung}
  Wie wir in \thref{bsp:stark_regular_1} und \thref{satz:konstruktion_q_ordnung}
  gesehen haben, sind die $(ml)$-ten Einheitswurzeln die Elemente von kleinster
  multiplikativer Ordnung, deren $q$-Ordnung ein irreduzibler Teiler von
  $\Phi_m(x)$ wird. Natürlich können wir dieses Konzept auch erweitern und uns
  überlegen, welche primitiven Einheitswurzeln die selbe Eigenschaft erfüllen.
  Darüber hinaus können wir die Elemente deren $q$-Ordnung ein irreduzibler
  Teiler von $\Phi_m(x)$ über $F[x]$ ist, auch durch die Modulstruktur selbst
  beschreiben.
  Diese beiden Überlegungen wollen wir in den nächsten beiden Lemmas beweisen.
\end{bemerkung}

\begin{lemma}
  \label{lemma:hoehere_wurzeln_auch_erzeuger}
  Seien die Voraussetzungen wie in \thref{satz:konstruktion_q_ordnung}, also
  $F = \F_q$ ein endlicher Körper und $m\in \S_q$. Setze ferner 
  $l := \pi_m(q-1)$, $a := \ggT(l,m)$. 
  Ist nun $\theta$ eine primitive $(nf)$-te Einheitswurzel für 
  $l \mid f \mid q-1$, so ist $\Ord_q(\theta)$ ein irreduzibler Teiler
  von $\Phi_m(x)$ über $F[x]$.
\end{lemma}
\begin{proof}
  Sei $f = le$ mit $\ggT(e,l) = 1$. Diese Zerlegung ist möglich, da
  $l$ per definitionem der größte Teiler von $q-1$ ist, dessen Primfaktoren
  allesamt in $m$ vorkommen, d.h. jeder Primfaktor von $q-1$ der $l$ teilt,
  kommt dort bereits in maximaler Potenz vor.
  Damit reicht es -- wenn wir wir den Beweis von
  \thref{satz:konstruktion_q_ordnung} noch einmal nachvollziehen -- zu zeigen,
  dass $\ggT(mle, ck) = ce$ für $c := \kgV(m,l)$ und 
  $q^{\frac m a} -1 = ck$ für ein $k$ mit $\ggT(k,ml) = 1$. Da $\ggT(e,l) = 1$,
  ist per definitionem von $l$ auch $\ggT(e,m) = 1$ und damit auch
  $\ggT(e,c) = 1$. Da $e \mid q-1\mid q^{\frac m a -1}$ zerfällt $k$ 
  in $k = \bar k k_e$ mit $e \mid k_e$ und $\ggT(\bar k,e) = 1$. Damit
  folgt $\ggT(mle, ck) = ce$ und wir haben
  $\ord(\theta^{ck}) = \frac{mle}{\ggT(mle,ck)} = \frac{mle}{ce} = a$.
\end{proof}

Als Korollar dieses Lemmas können wir einen Satz von \citeauthor{semaev:1989}
\citeyear{semaev:1989} in 
\autocite{semaev:1989} beweisen, welcher erneut von \citeauthor{gao:1997}
\citeyear{gao:1997} in \autocite{gao:1997} bewiesen wurde:

\begin{satz}[{\autocite[Theorem 2.7]{gao:1997}, \autocite{semaev:1989}}]
  \label{satz:gao1}
  Sei $q = p^r$ eine Primzahlpotenz. Sei $n \in \N$ mit $\ggT(n,q)=1$.
  Ist $x^n - a \in \F_q[x]$ irreduzibel mit Wurzel $\theta\in \F_{q^n}$, so
  gilt 
  \[ \F_{q^n} \speq= \bigoplus_{l\in R_q(n)} \langle \theta^l\rangle_q\,,\]
  wobei $\langle \theta^l\rangle_q := 
  \spann_{\F_q}\{ \theta^i \mid i\in M_q(l\bmod n)\}$.
\end{satz}
\begin{proof}
  Aus \thref{satz:binom_irreduzibel} und \thref{satz:binom_irreduzibel_aquiv}
  wissen wir, dass für $a \in \F_q^\ast$ das Polynom
  $x^n-a \in \F_q[x]$ genau dann irreduzibel ist, wenn
  $p\nmid n$, $\nu(n) \mid f := \ord(a)$, $l := \pi_n(q-1)\mid f$ und
  $q \equiv 1 \bmod 4$, falls $4\mid n$. 
  Ist bereits $q\equiv 1 \bmod 4$, falls $n$ gerade, so sind wir genau in der 
  Situation, dass $n$ stark regulär ist.
  Damit ist $\theta$ eine primitive $(nf)$-te Einheitswurzel und wir wissen
  nach \thref{lemma:hoehere_wurzeln_auch_erzeuger}, dass
  $\Ord_q(\theta)$ ein irreduzibler Teiler von $\Phi_n(x)$ über $\F_q[x]$ ist,
  womit alles gezeigt wäre.

  Es bleibt ein Wort zur Situation $q\equiv 3 \bmod 4$, falls $n$ gerade, zu
  verlieren. Dieser Fall wird von den bisherigen Resultaten nicht erfasst.
  Mit etwas mehr Aufwand ist es jedoch möglich, die explizite Konstruktion 
  von Normalbasen mit primitiven Einheitswurzeln zu erweitern, wie Hachenberger
  in \autocite[Section 22]{hachenberger1997finite} zeigt.
\end{proof}

Bevor wir die Ideen der stark regulären Körpererweiterungen verallgemeinern
wollen, betrachten wir noch ein Lemma, das die irreduziblen Teilmodule genauer
beschreibt und in ganz ähnlicher Form in \autocite[Theorem
22.5]{hachenberger1997finite} wiederzufinden ist.

\begin{lemma}
  \label{lemma:erzeuger_von_irred_teilmoduln}
  Seien die Voraussetzungen wie in \thref{satz:konstruktion_q_ordnung}, also
  $F = \F_q$ ein endlicher Körper und $m\in \S_q$. Setze ferner 
  $l := \pi_m(q-1)$, $a := \ggT(l,m)$. Ist dann $u$ eine primitive 
  $(ml)$-te Einheitswurzel mit $\Ord_q(u) = f(x)$ für $f(x)$ einen irreduziblen
  monischen Teiler von $\Phi_m(x)$, so gilt für $v \in E:= \F_{q^m}$:
  \[ \Ord_q(v) = f(x) \quad\Leftrightarrow\quad
    v = g(x) \cdot u\quad\text{ für ein } 0\neq g\in F[x]_{<\frac m a}\]
\end{lemma}
\begin{proof}
  In \thref{kor:moduln_ueber_v_g} haben wir gezeigt, dass für $v \in E$ gilt
  \[ v \in E_f \quad\Leftrightarrow\quad v = g(x)\cdot u 
    \text{ für ein } g(x) \in F[x]_{<\deg f},\ \ggT(f,g) = 1\]
  Da $f$ irreduzibel von Grad $\frac m a$ ist, sind alle $v \in
  E_f\setminus\{0\}$ von $q$-Ordnung $f$ und es folgt die Behauptung.
\end{proof}

\begin{bemerkung}
  Obiges Lemma gilt nicht nur für primitive $(ml)$-te Einheitswurzeln, sondern
  -- wie man offensichtlich erkennen kann -- für jedes Element mit gleicher
  $q$-Ordnung!
\end{bemerkung}




\section{Reguläre Erweiterungen}

Die Erkenntnisse über stark reguläre Erweiterungen wollen wir nutzen, um
Normalbasen auch in einem allgemeineren Kontext angeben zu können. Ist nämlich
$m\in\N$ ungerade mit $p\nmid m$ und setzen wir $s = \ord_{\nu(m)}(q)$, 
so erkennen wir, dass $\nu(m) \mid q^s-1$ per definitionem von
$\ord_{\nu(m)}(q)$. Mit anderen Worten: $m \in \S_{q^s}$! 

\begin{definition}[regulär]
  Sei $q = p^r$ eine Primzahlpotenz und $n\in \N_{>0}$. Setze 
  $n = n' p^c$ mit $p\nmid n'$. Das Paar $(q,n)$ heißt \emph{regulär}, falls
  $\ggT(n, \ord_{\nu(n')}(q)) = 1$. 
  Ist $F = \F_q$ und $E = \F_{q^n}$, so nenne die Erweiterung $E\mid F$ 
  \emph{regulär}.
\end{definition}

\begin{satz}
  \label{satz:konstruktion_q_ordnung_reg}
  Seien $F := \F_q$ ein endlicher Körper für eine Primzahlpotenz $q = p^r$ 
  und $m\in\N$ ungerade mit $p\nmid m$.
  Setze $s := \ord_{\nu(m)}(q)$, 
  $l := \pi_m(q^s-1)$, $b:= b(q,m) = \ggT(l,m)$, $E := \F_{q^m}$ und
  $E' = \F_{q^{sm}}$. Ist dann
  $u$ eine primitive $(ml)$-te Einheitswurzel, so gilt
  \begin{enumerate}
    \item $F(u) = E'$,
    \item $\Ord_q(u) = f(x^s)$ für einen monischen irreduziblen Teiler 
      $f(x) \in F[x]$ von $\Phi_m(x)$ und 
    \item $v := \sum_{j \in R_q(m)\atop \ggT(j,m)=1} 
      u^j$ hat $q$-Ordnung $\Phi_m(x^s)$.
  \end{enumerate}
\end{satz}
\begin{proof}
  \TODO
\end{proof}

Nun haben wir also Elemente mit $q$-Ordnung $\Phi_m(x^s)$ gefunden. Es stellt
sich jedoch die Frage, wir wir daraus Elemente mit $q$-Ordnung $\Phi_m(x)$ 
\glqq basteln\grqq können. Dies zeigt uns der folgende Satz:

\begin{satz}
  \label{satz:q_ordnung_von_trace}
  Seien alle Voraussetzungen und Notationen wie in 
  \thref{satz:konstruktion_q_ordnung_reg}.
  Sei jedoch zusätzlich $\ggT(m,s) = 1$.
  Bezeichne ferner 
  $\sigma_E: E'\to E',\ x \mapsto x^{q^m}$ den Frobenius von $E'$ auf $E$.
  Ist dann $u$ eine primitive $(ml)$-te Einheitswurzel
  und $v = \sum_{j\in R_q(m)} u^j$, so gilt:
  \begin{enumerate}
    \item $H(\sigma_E)(v)$ hat $q$-Ordnung $\Phi_m(x)$ für 
      $H(x) := \frac{\Phi_m(x^s)}{\Phi_m(x)}$.
    \item $\Tr_{E'\mid E}(v)$ hat $q$-Ordnung $\Phi_m(x)$.
    \item $\Ord_q\big(\Tr_{E'\mid E}(u)\big)$ ist ein irreduzibler 
      monischer Teiler von $\Phi_m(x)$.
  \end{enumerate}
\end{satz}
\begin{proof}
  Zerlegen wir $s = \bar s p^\beta$ mit $p\nmid s'$, so ist 
  nach Voraussetzungen $\ggT(\bar s, m) = 1$ und wir sind in der Situation von
  \thref{satz:zusammenhang_unterschiedlicher_kreisteilungspolys}. Damit gilt
  \[ \Phi_m(x^s) \speq= \Phi_m(x^{\bar s})^{p^\beta}
    \speq= \prod_{d \mid \bar s} \Phi_{md}(x)^{p^\beta}\,.\]
  Mit konsequenter Anwendung von \thref{lemma:eigenschaften_tau_ordnung}
  folgern wir die Behauptungen:
  \[ \Ord_q(H(\sigma_E)v) = \frac{\Phi_m(x^s)}{\ggT(\Phi_m(x^s), H(x))}
    = \Phi_m(x)\,.\]
  Für (2) und (3) überlegen wir uns, dass für $a \in E'$
  \[ \Tr_{E'\mid E}(a) = \sum_{i=0}^{s-1} a^{q^{im}} = 
    \left[\frac{x^{sm}-1}{x^m-1}\right](\sigma_E)(a)\]
  und 
  \[ \frac{x^{sm} -1}{x^m-1} \speq= 
    \frac{ \prod_{d\mid \bar sm} \Phi_d(x)^{p^\beta}}{
      \prod_{d\mid m} \Phi_d(x)} \speq=
    \Phi_m(x)^{p^\beta-1}\ 
    \prod_{l\mid m\atop l\neq m} \Phi_l(x)^{p^\beta-1}\ 
    \prod_{d\mid \bar s\atop d\neq 1} \Phi_{md}(x)^{p^\beta} \,.\]
  Ergo ist
  \[ \Ord_q(\Tr_{E'\mid E}(v)) \speq= 
    \frac{\Phi_m(x^s)}{\ggT(\Phi_m(x^s), \frac{x^{sm}-1}{x^m-1})}
    \speq= \Phi_m(x)\]
  und
  \[ \Ord_q(\Tr_{E'\mid E}(u)) \speq= 
    \frac{f(x^s)}{\ggT(f(x^s), \frac{x^{sm}-1}{x^m-1})} \speq= f_1(x) \,,\]
  wobei wir uns hierfür noch einmal an \thref{satz:f_x_s_ist_teiler_von_phimd} 
  erinnern müssen, wo wir gezeigt haben, dass in genau dieser Situation 
  $f(x^s) = \prod_{i=1}^? f_i(x) \in F[x]$ mit $f_i$ irreduzibler monischer 
  Teiler von $\Phi_{md}(x)$ für $d\mid \bar s$, wobei wir \obda 
  $f_1 \mid \Phi_m(x)$ für den einzigen Teiler von $\Phi_m(x)$ wählen.
\end{proof}


Wir können sogar noch in gewisser Weise eine Verschärfung dieses Satzes
angeben, die genauer charakterisiert, warum die Spurfunktion in diesem Fall
gute Dienste leistet.

\begin{satz}
  \label{satz:q_ordnung_von_zusammensetzen}
  Seien die Voraussetzungen wie in \thref{satz:q_ordnung_von_trace}.
  Ist dann wiederum $u$ eine primitive $(nl)$-te Einheitswurzel und
  $k(x) = \sum_{i=0}^j a_i x^i \in F[x]$, $j\in\N$, so dass
  $\Ord_q(u) = \Ord_q(u^i) = f(x^s)$ für einen irreduziblen monischen Teiler 
  $f(x)$ von $\Phi_m(x)$ über $F$ für alle $i=0,\ldots,j$. Gilt letztlich
  $[F(k(u)):F]=n$, so gilt:
  $\Ord_q(k(u))$ ist ein irreduzibler monischen Teiler 
  von $\Phi_m(x)$ über $F$.
\end{satz}
\begin{proof}
  Sammeln wir einmal den Wissensstand für gegebene Situation zusammen:
  \begin{itemize}
    \item $\Ord_q(u) = \Ord_q(u^i) = f(x^s)$ teilt $\Phi_m(x^s)$.
    \item Für $s = \bar s p^\beta$ mit $p\nmid \bar s$ gilt nach
      \thref{satz:zusammenhang_unterschiedlicher_kreisteilungspolys}
      \[\Phi_m(x^s) = \prod_{d\mid \bar s} \Phi_{md}(x)^{p^\beta}\]
    \item Als leichte Konsequenz aus 
      \thref{lemma:} haben wir 
      \[\Ord_q(k(u)) \speq\mid \kgV\{\Ord_q(a_iu^i): i=1,\ldots,j\} = f(x^s) \]
    \item Da $k(u)\in E$ und die $q$-Ordnung immer das Minimalpolynom 
      des zugehörigen Frobenius teilt, folgt
      \[\Ord_q(k(u)) \speq\mid x^m-1 = \prod_{d\mid m} \Phi_d(x)\]
  \end{itemize}
  Zusammengefasst teilt also $\Ord_q(k(u))$ nur den Anteil von $f(x^s)$,
  der auch in $\Phi_m(x)$ liegt. Damit folgt nach
  \thref{satz:f_x_s_ist_teiler_von_phimd} die Behauptung.
\end{proof}

\begin{bemerkung}
  Auf diese Weise kann man auch den Beweis von \thref{satz:q_ordnung_von_trace}
  führen, wenn man sich überlegt, dass für $\Ord_q(u) = f(x^s)$ auch
  $\Ord_q(u^{q^m}) = f(x^s)$ ($m$ und $s$ sind teilerfremd!) 
  gilt und gerade $\Tr_{E'\mid E}(u) \in E$ gilt.
\end{bemerkung}


Auch für reguläre Erweiterungen wollen wir ein Analogon von 
\thref{lemma:hoehere_wurzeln_auch_erzeuger} beweisen:

\begin{lemma}
  \label{lemma:hoehere_wurzeln_auch_erzeuger_reg}
  Seien die Voraussetzungen wie in \thref{satz:konstruktion_q_ordnung_reg}, also
  also $F := \F_q$ ein endlicher Körper für eine Primzahlpotenz $q = p^r$ 
  und $m\in\N$ ungerade mit $p\nmid m$.
  Setze $s := \ord_{\nu(m)}(q)$, 
  $l := \pi_m(q^s-1)$, $b:= b(q,m) := \ggT(l,m)$, $E := \F_{q^m}$ und
  $E' = \F_{q^{sm}}$. 
  Ist nun $\theta$ eine primitive $(nf)$-te Einheitswurzel für 
  $l \mid f \mid q^s-1$, so ist $\Ord_q(\theta) = f(x^s)$ für 
  $f$ einen irreduzibler Teiler von $\Phi_m(x)$ über $F[x]$.
\end{lemma}
\begin{proof}
  \thref{lemma:hoehere_wurzeln_auch_erzeuger} mit dem Beweis von
  \thref{satz:konstruktion_q_ordnung_reg}.
\end{proof}


\section{Normalbasen mit Dickson-Polynomen}
\citeauthor{scheerhorn:1996} zeigt in
\autocites{scheerhorn:1996}{scheerhorn:1997}, dass sich Dickson-Polynome
eignen, um Normalbasen zu beschreiben. Insbesondere zeigt er die beiden
folgenden Resultate.

\begin{satz}[{\autocite[Theorem 2]{scheerhorn:1997}}]
  \label{satz:scheerhorn1}
  Seien $n\geq 3$ ein Produkt ungerader Primzahlen von $(q+1)$ und 
  $a,b\in \F_q =: F$, so dass $D_n(x,a)-b \in F[x]$ irreduzibel ist.
  Sei $\gamma \in E := \F_{q^n}$ eine Wurzel von $D_n(x,a)$. Dann ist
  \[ E \speq= \langle 1\rangle_q \oplus 
    \bigoplus_{l\in R_q(n)\setminus \{0\}} \langle D_l(\gamma,a)\rangle_q\,,\]
  eine Zerlegung von $E$ in irreduzible $F[x]$-Teilmoduln, wobei
  \[ \langle D_l(\gamma,a)\rangle_q \speq{:=}
    \spann_F\{ D_i(\gamma,a) \mid i\in M_q(l\bmod n)\} \,,\]
  so dass $\{ D_i(\gamma,a) \mid i\in M_q(l\bmod n)\}$ eine $F$-Basis von
  $\langle D_l(\gamma,a)\rangle_q$ ist.
\end{satz}

\begin{satz}[{\autocite[Theorem 3]{scheerhorn:1997}}] 
  \label{satz:scheerhorn2}
  Seien $n\geq 3$ ein Produkt ungerader Primzahlen von $(q-1)$ und
  $a,b \in \F_q =: F$, so dass $D_n(x,a)-b\in F[x]$ irreduzibel ist.
  Seien ferner $x^2+bx +a^n = (x-\beta)(x-a^n\beta\inv) \in \F_q$ und
  $\theta \in \F_{q^n}$ eine Wurzel von $x^n-\beta$. Ist schießlich
  $\gamma\in \F_{q^n}$ eine Wurzel von $D_n(x,a) -b$, so gilt für
  $l\in R_q(n) \setminus\{0\}$
  \[ \langle \theta^l\rangle_q \oplus \langle \theta^{n-l}\rangle_q
  \speq= \langle D_l(\gamma,a)\rangle_q\]
\end{satz}

Wir wollen nun im Folgenden zeigen, dass dies relativ einfach aus der
allgemeinen Theorie aus dem vorherigen Abschnitt folgt.
Zunächst brauchen wir jedoch eine wesentliche Eigenschaft der Dickson-Polynome.

\begin{lemma}[{??}]
  \label{lemma:dickson:dickson_polynome_potenz}
  Es gilt $D_n(x+a x\inv,a) = x^n + a^n x^{-n}$.
\end{lemma}

Nun können wir obige Sätze beweisen.

\begin{proof}[von \thref{satz:scheerhorn1}]
  Betrachten wir die vorliegende Situation, so sehen wir, dass
  $s := \ord_{\nu(n)}(q) = 2$ und damit $n \in \S_{q^2}$.
  Nach \thref{} ist $x^2 +bx+a^n \in F[x]$ irreduzibel. Sei 
  $x^2+bx+a^n = (x -\beta)(x -a^n\beta\inv)$ über $K := \F_{q^2}$.
  Dann ist $x^n-\beta \in \F_{q^2}[x]$ irreduzibel.
  Ist dann $\theta \in \F_{q^{2n}} =: E'$ eine Wurzel, so wissen wir nach
  \thref{satz:gao1} und \thref{lemma:hoehere_wurzeln_auch_erzeuger_reg}, dass
  $\Ord_q(\theta) = f(x^s)$ für einen irreduziblen Teiler $f(x)$ von
  $\Phi_n(x)$ über $F$. Nach \thref{satz:q_ordnung_von_trace}
  ist
  \[ f_1(x) \speq{:=} \Ord_q( \Tr_{E'\mid E}(\theta) ) \speq=
    \Ord_q(\theta + \theta^{q^n})\]
  ein irreduzibler Teiler von $\Phi_n(x)$ in $F[x]$.
  
  Es ist nun $\theta^{q^n} = \theta\inv$: 
  Da $\theta \in E'$, gilt $\theta^{q^{2n}} = \theta$, also
  \[ \theta^{q^{2n}-1} \speq= 1 \speq= \theta^{(q^n+1)(q^n-1)}\,. \]
  wäre $\theta^{q^n-1} = 1$, so läge $\theta$ bereits in $\F_{q^n}$, im Widerspruch
  zur Definition von $\theta$.
  Also haben wir nach \thref{satz:q_ordnung_von_zusammensetzen} 
  gerade 
  \[ \Ord_q(\theta + a\theta\inv) = f_1(x)\,,\]
  da mit \thref{lemma:dickson:dickson_polynome_potenz}
  \[ D_n(\theta + a\theta\inv,a) -b \speq= \theta^n + a^n\theta^{-n} -b
    \speq= \beta +a^n\beta\inv -b \speq= b - b\speq= 0\,,\]
  also $\theta + a\theta\inv$ Grad $n$ über $F$ hat.
  Da wir mit $\theta^l$ für $l \in R_q(n)$ alle irreduziblen Teilmoduln
  %
  \footnote{hier vielleicht noch ein Satz/Lemma/Bemerkung im vorherigen 
  Abschnitt über die Erzeugung \emph{aller} irreduziblen Teilmodule}
  %
  von $x^n-1$ erzeugen können, also $\Ord_q(\theta^i) = g(x^s)$ für einen
  irreduziblen monischen Teiler $g(x)$ von $x^n-1$ über $F$, ist alles gezeigt,
  da analog zu oben
  \[ \Ord_q(\theta^l + a^l\theta^{-l}) \speq\mid x^n-1 \]
  für $l\in R_q(n)$ ein irreduzibler Teiler von $x^n-1$ ist
  (für $l=0$ hat $\theta^l+a^l\theta^{-l}$ natürlich Grad $0$ über $F$).
  $\gamma := \theta + a\theta\inv$ und $D_l(\gamma,a) =
  \theta^l+a^l\theta^{-l}$ schließt den Beweis ab.
\end{proof}
