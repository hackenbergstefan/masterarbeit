\chapter{Explizite Konstruktion von Normalbasen}

Wir haben nun kennengelernt, wie man einen 
Vektorraum $V$ über $\K$ zusammen mit einem Endomorphismus $\tau$ als
$\K[x]$-Modul lesen kann. Analog dazu wollen wir nun eine Erweiterung
endlicher Körper $E$ über $F$ in diesem Kontext verstehen: $E$ ist ein
Vektorraum über $F$ und wird mit Hilfe des Frobenius 
$\sigma: \bar F \to \bar F$ zu einem $F[x]$-Modul im Sinne von 
\thref{def:V_tau}. Dies wird uns helfen, normale (und später vollständig
normale) Elemente zu konstruieren.

\section{Grundlegende Ideen}

Seien im Folgenden stets $F := \F_q$ ein endlicher Körper von 
Charakteristik $p$ und $E := \F_{q^n}$ über $F$ eine Körpererweiterung.
Wir wiederholen kurz die Definition einer \emph{Normalbasis}.

\begin{definition}[normales Element, normales Polynom, Normalbasis]
  \label{def:normal}
  Sei $F$ ein Körper und $E \mid F$ eine endliche Galoiserweiterung von Grad
  $n$. Sei ferner $w\in E$ mit $F(w) = E$. $w$ heißt \emph{normal über $F$},
  falls
  \[ \{ \gamma(w):\ \gamma \in \Gal(E\mid F) \}\]
  eine $F$-Basis von $E$ ist. 
  $\{ \gamma(w):\ \gamma \in \Gal(E\mid F)\}$ heißt entsprechend 
  \emph{Normalbasis} und $g(x) \in F[x]$ mit 
  \[ g(x) = \prod_{\gamma \in \Gal(E\mid F)}(x - \gamma(w))\]
  heißt \emph{normales Polynom}.
\end{definition}

Wir wollen nun die Begriffe der normalen Elemente und der Erzeuger von
zyklischen Moduln aus dem vorhergehenden Kapitel in Zusammenhang bringen. Dazu
müssen wir uns Gedanken machen, wie sich Normalität im Kontext der
Modulstruktur einer Körpererweiterung lesen lässt.

\begin{satz}
  Ein Element $u\in E$ ist genau dann normal über $F$, wenn
  \[ \Ord_q(u) \speq= x^n-1 \quad\in F[x]\,.\]
\end{satz}
\begin{proof}
  Nach \thref{bem:mipo_frob} ist das Minimalpolynom des Frobenius von $F$
  gerade $x^n-1$. Ferner wissen wir nach \thref{satz:frob_sind_alle_autos},
  dass $\Gal(E\mid F) = \langle\sigma\rangle$. 
  Letztlich liefert \thref{satz:moduln_ueber_v_g} (3), dass die Erzeuger von 
  $E$ als $F[x]$-Modul, also gerade die normalen Elemente, jene mit $q$-Ordnung
  $x^n-1$ sind.
\end{proof}

\begin{kor}
  \label{kor:summe_erzeuger_normal}
  Sei $x^n-1 = \prod_{i=1}^s r_i(x)$ eine Zerlegung in paarweise teilerfremde
  Polynome. Seien ferner $u_i \in V_{r_i}$ Elemente mit 
  $\Ord_q(u_i) = r_i(x)$ $\forall i=1,\ldots,s$. Dann ist
  \[ u \speq= u_1 + u_2 + \ldots + u_s\]
  normal in $E \mid F$.
\end{kor}
\begin{proof}
  %Da obige Zerlegung von $x^n-1$ paarweise teilerfremd ist, folgt nach
  %\thref{lemma:eigenschaften_tau_ordnung}
  %$\Ord_q(u) = \prod_{i=1}^s \Ord_q(u_i) = \prod_{i=1}^s r_i(x) = x^n -1$.
  \thref{satz:zerlegungssatz_zykl_vektorraume}.
\end{proof}

Darüber hinaus können wir  eine
Formel präsentieren, die die Anzahl aller Elemente mit gegebener $q$-Ordnung
liefert. Insbesondere gewinnen wir so eine Formel für die Anzahl aller
normalen Elemente.

\begin{definition}
  \label{def:polynom_phi}
  Sei $f(x) \in \F_q[x]$ ein Polynom über einem endlichen Körper. Definiere
  \[ \phi_q(f) \speq{:=} |\{ g(x) \in \F_q[x]:\ 
    \deg g < \deg f,\ \ggT(f,g) = 1\}|\,.\]
\end{definition}

\begin{satz}
  \label{satz:anzahl_normal}
  Seien $\F_{q^n}$ über $F:=\F_q$ eine Erweiterung endlicher
  Körper und $g(x) \mid x^n-1$ monisch mit $g(x) = \prod_{i=1}^k
  g_i(x)^{\nu_i}$ seiner vollständigen Faktorisierung über $F$. Dann 
  existieren genau
  \[\phi_q(g) \speq= \prod_{i=1}^k \big( q^{\nu_i\deg g_i} - q^{(\nu_i-1)\deg
    g_i}\big)\]
  Elemente in $E$ mit $q$-Ordnung $g(x)$.
  Insbesondere existieren
  \[ \phi_q(x^n-1) \speq= q^{n'(\pi-1)}\,\prod_{d\mid n}
    \left( q^{\ord_d(q)} -1 \right)^{\frac{\varphi(d)}{\ord_d(q)}}\]
  Elemente, die normal über $\F_q$ sind, wobei $n = n'\pi$ mit $\ggT(n',q) = 1$.
\end{satz}
\begin{proof}
  Mit \thref{kor:anzahl_erzeuger} müssen wir lediglich argumentieren, warum
  es $q^{\nu_i\deg g_i}$ viele Elemente von $q$-Ordnung $g_i^{\nu_i}$ und
  $q^{(\nu_i-1)\deg g_i}$ viele von $q$-Ordnung $g_i^{\nu_i-1}$ gibt. Dies ist
  jedoch klar mit \thref{lemma:modul_dim_deg}: Ist $v\in E$ mit 
  $\Ord_q(v) = g_i^{\nu_i}$, so gilt
  \[ \dim_F(F[x]\cdot v) \speq= \deg(\Ord_q(v))\speq= \deg(g_i^{\nu_i})
    \speq= \nu_i \deg(g_i)\,.\]
  Damit ist auch die Gleichheit mit $\phi_q(g)$ klar.
  Die Anzahl der normalen Elemente erhalten wir aus dem bekannten
  Zerfall von $x^n-1$ über $F$ nach 
  \thref{satz:zerfall_xn_1} und \thref{satz:zerfall_kreisteilungspolys}.
\end{proof}


\section{Stark reguläre Erweiterungen}
\label{sec:stark_regulare_erweiterungen}

\begin{beispiel}
  \label{bsp:stark_regular_1}
  Wählen wir einmal $q = 7$ und $n=9$. Also $F = \F_7$.
  Ferner wissen wir aus \autoref{chap:kreisteilungspolynome}, dass
  \[ x^n-1 \speq= x^9-1 \speq= 
    \Phi_1(x)\Phi_3(x)\Phi_9(x) \speq=
    \big(x-1\big)\big( (x+3)(x+5)\big)
    \big( (x^3+3)(x^3+5) \big) \ \in \F_7[x]\]
  die vollständige Faktorisierung von $x^n-1$ über $\F_7$ ist.
  Da es sich hier bei den Faktoren lediglich um Binome handelt, können wir
  relativ einfach Elemente passender $q$-Ordnungen angeben: Beginnen wir 
  mit $\Phi_9 = (x^3+3)(x^3+5)$. Gesucht ist nun ein Element $u$ in einer 
  passenden Erweiterung von $F$ mit $\Ord_q(u)= \Phi_9$.
  Sei dazu $u \in \bar F$ eine primitive $27$-te Einheitswurzel. Dann
  gilt:
  \[ \ord(u^{342}) \speq= \ord(u^{18}) \speq= \frac{27}{\ggT(27,18)} 
    \speq= 3\,.\]
  Wenn man sich die Frage stellt, warum an diesem Punkt gerade $342$ eine
  interessante Zahl ist, so wird man diese sofort wiederfinden, wenn man
  versucht, ein Element $w\in \bar F^\ast$ mit $q$-Ordnung $x^3+3$ (bzw.
  $x^3+5$) wiederzufinden:
  \[ (x^3+3)\cdot w \speq= w^{q^3} + 3 w \speq= 
    w(w^{7^3-1} +3) \speq= w(w^{342}+3) \speq{\overset{!}{=}} 0.\]
  Da $w \neq 0$ und per Definition der Kreisteilungspolynome 
  $(-3),(-5)\in \F_7$ primitive $3$-te Einheitswurzeln sind, 
  brauchen wir ein $w$ mit $\ord(w^{342}) = 3$; 
  was obiges $u$ gerade erfüllt! Damit ist also $u^{18} = u^{342}$ 
  eine der beiden dritten Einheitswurzeln $(-3)$ oder $(-5) \in \F_7$ und wir
  können mit \thref{lemma:eigenschaften_tau_ordnung} folgern:
  \[ \Ord_q(u + u^2) \speq= \Phi_9\]

  Auch die Suche nach einem Element mit $q$-Ordnung $\Phi_3$ ist damit
  erledigt: Mit analoger Argumentation wie oben erhalten wir, dass 
  \[ \Ord_q(u^{3} + u^{6}) \speq= \Phi_3.\]
  Zusammengefasst ist also $u$ ein normales Element von $\F_{7^9} \mid \F_7$.
\end{beispiel}

Ein entscheidender Vorteil in der Konstruktion eines normalen Elements in
obigem Beispiel war der Zerfall der Kreisteilungspolynome in Binome. Aus 
\thref{satz:kreisteilungspolynome_binome}
wissen wir bereits, wann die Kreisteilungspolynome in Binome zerfallen.
Damit haben wir eine sehr einfache Möglichkeit, Normalbasen explizit anzugeben.

\begin{definition}[stark regulär]
  \label{def:stark_regular}
  Das Paar $(q,n) \in \N^2$ heißt \emph{stark regulär}, falls
  \begin{itemize}
    \item $q = p^r$ mit $p$ einer Primzahl und $r>0$,
    \item $p \nmid n$,
    \item $\nu(n) \mid q-1$,
    \item $4 \mid q-1$, falls $n$ gerade.
  \end{itemize}
  Schreibe $n \in \S_q$, falls $(q,n)$ stark regulär.
  Eine Körpererweiterung $\F_{q^n} \mid \F_q$ heißt \emph{stark regulär}, falls
  $n \in \S_q$.
\end{definition}

\begin{lemma}
  \label{lemma:ordn_1}
  Seien $q,m \in \N^\ast$ mit $q,m > 1$ und $\nu(m) \mid \nu(q-1)$. Dann gilt:
  \begin{enumerate}
    \item Ist $m$ ungerade oder $q\equiv 1 \bmod 4$, so gilt
      \[ \ord_m(q) \speq= \frac{m}{\ggT(\cl_m(q-1),m)}\,.\]
    \item Ist $m\equiv 0\bmod 4$ und $q\equiv 2 \bmod 4$, so gilt
      \[ \ord_m(q) \speq= 2\,\frac{m}{\ggT(\cl_m(q^2-1),m)}\,.\]
    \item Ist $m\equiv 2\bmod 4$ und $q\equiv 2 \bmod 4$, so gilt
      \[ \ord_m(q) \speq= \ord_{\frac m 2}(q) \speq=
        \frac{m}{\ggT(\cl_m(q-1),m)}\,.\]
  \end{enumerate}
\end{lemma}
\begin{proof}
  \autocite[Lemma 19.6]{hachenberger1997finite}.
\end{proof}

\begin{lemma}
  \label{lemma:ordn_2}
  Seien $q,m>1$ zwei teilerfremde ganze Zahlen. Setze 
  $s :=\ord_{\nu(m)}(q)$, so gilt
  \[ \ord_m(q) \speq= s\,\ord_m(q^s)\,.\]
\end{lemma}
\begin{proof}
  \autocite[Lemma 19.7]{hachenberger1997finite}.
\end{proof}

\begin{satz}[{\autocite[Satz 8.12]{hachenberger2013vorlesung}}]
  \label{satz:konstruktion_q_ordnung}
  Sei $F := \F_q$ ein endlicher Körper und $m \in \S_q$. Seien
  $l := \cl_m(q-1)$, $a := \ggT(l,m)$ und $u \in \bar\F_q$ eine primitive
  $(ml)$-te Einheitswurzel. Dann gilt:
  \begin{enumerate}
    \item $\F_q(u) = \F_{q^m} =: E$.
    \item $\Ord_q(u)$ ist ein irreduzibler Teiler von $\Phi_m$ in $\F_q[x]$.
    \item $v := \sum_{i \in I_a} u^i$ hat $q$-Ordnung $\Phi_m$.
  \end{enumerate}
\end{satz}
\begin{proof}
  \begin{enumerate}
  \item Aus \thref{lemma:cl_2} haben wir 
   $\cl_m(q^m-1) = m\cl_m(q-1) = ml$ und
    damit nach \thref{lemma:ordn_1} 
    \[ [F(u):F] \speq= \ord_{ml}(q) \speq= 
      \frac{ml}{\ggT(\cl_{ml}(q-1), ml} \speq=
      \frac{ml}{l} \speq= m\,,\]
    da $\nu(ml) = \nu(m)$, wegen $l = \cl_m(q-1)$.
  \item Sei nun $\zeta \in F^\ast$ eine primitive $a$-te Einheitswurzel. Dann
    ist nach \thref{satz:kreisteilungspolynome_binome}
    \[ \Phi_m(x) \speq= \prod_{i\in I_a} (x^{\frac m a} - \zeta^i).\]
    Wie in \thref{bsp:stark_regular_1} betrachten wir für $i \in I_a$:
    \begin{align*}
      (x^{\frac m a} - \zeta^i)\cdot u 
      \quad&=\quad \big( \sigma^{\frac m a} - \zeta^i \id_E \big)(u)\\
        &=\quad u^{q^{\frac m a}} - \zeta^i u \\
        &=\quad u\big( u^{q^{\frac m a}-1} - \zeta^i \big)\,, \tag{$\ast$}
    \end{align*}
    wobei wie üblich $\sigma: E \to E, x \mapsto x^q$ den Frobenius von
    $F$ meint. Wir wollen nun zeigen, dass $\Ord_q(u) = (x^{\frac m a} -
    \zeta^i)$ für ein $i\in I_a$, explizit also, dass $(\ast) = 0$ für ein
    $i\in I_a$ gilt. Dazu müssen wir $q^{\frac m a} -1$ genauer untersuchen:
    Wiederum nach \thref{lemma:cl_2} haben wir:
    \[ \cl_m(q^{\frac m a} -1 ) \speq= \frac m a \cl_m(q-1) \speq=
      \frac{ml}{a} \speq= \kgV(m,l) \speq{=:} c\,.\]
    Nun ist $q^{\frac m a}-1 = ck$ für 
    ein $k\in\N$ mit $\ggT(k,ml) = 1$, wie man sich anhand der 
    Primfaktorzerlegungen leicht klar machen kann. Erinnern wir uns kurz, 
    dass $u$ eine primitive $(ml)$-te Einheitswurzel war, so folgern wir:
    \[ \ord(u^{ck}) \speq= \ord(u^c) \speq= \frac{ml}{\ggT(ml,c)}
      \speq= \frac{ml}{c} \speq= a\,.\]
    Ergo gibt es $j\in I_a$ mit $u^{ck} = \zeta^j$ und für dieses $j$ ist
    $(\ast)$ gerade $0$, was zu zeigen war.
  \item Seien $j\in I_a$ mit $\Ord_q(u) = x^\frac m a - \zeta^j$ und 
    $i\in I_a$ beliebig, so gilt
    \begin{align*}
      \big( x^\frac m a - \zeta^{ji} \big) u^i \speq{&=}
      (u^{q^\frac m a})^i - (\zeta^ju)^i \\
      \speq{&=} (u^{q^\frac m a} - \zeta^ju) 
        \sum_{k=0}^{i-1} u^{k\,q^\frac m a} \zeta^{(i-1-k)\,j}u^{i-1-k} \\
        \speq{&=} 0\,.
    \end{align*}
    Da $i$ teilerfremd zu $ml$ ist, haben wir ferner $\ord(u) = \ord(u^i)$ und
    können folgern, dass $\Ord_q(u^i) = x^\frac m a - \zeta^{ji}$.
    Packen wir nun alles zusammen und bedenken, dass
    $I_a \to I_a, i\mapsto ij$ eine Bijektion ist, können wir den letzten
    Schritt im Beweis führen:
    \[ \Ord_q\big( \sum_{i\in I_a} u^i \big)
      \speq= \prod_{i\in I_a} \Ord_q(u^i) \speq=
      \prod_{i \in I_a} \big( x^\frac m a - \zeta^{ji} \big)
      \speq= \prod_{k\in K_a} \big(x^\frac m a - \zeta^k \big)
      \speq= \Phi_m(x)\,.\]
  \end{enumerate}
\end{proof}

Da wir nun die einzelnen Bausteine kennen, können wir auf die Faktorisierung
und damit auf eine explizite Angabe eines normalen Elements schließen:

\begin{kor}[{{\autocite[Korollar 8.11]{hachenberger2013vorlesung}}}]
  \label{kor:konstruktion-q-ordnung}
  Für $F := \F_q$, $n\in \S_q$ sei $\lambda \in F^\ast$ eine primitive 
  $\cl_n(q-1)$-te Einheitswurzel. Zu jedem $m \mid n$ seien
  \begin{itemize}
    \item $a(q,m) := \ggT(m,\cl_m(q-1))$ und
    \item $I_a := \{ i\leq a:\ \ggT(i,a) = 1\}$.
  \end{itemize}
  Dann ist
  \[ x^n-1 \speq= \prod_{m\mid n}\ \prod_{i\in I_{a(q,m)}} 
    \left( x^\frac{m}{a(q,m)} - \lambda^{\frac{\cl_n(q-1)}{a(q,m)} i}\right)\]
  die vollständige Faktorisierung von $x^n-1$ über $\F_q$.
\end{kor}
\begin{proof}
  Da  $\ord\big(\lambda^{\frac{\cl_n(q-1)}{a(q,m}}\big) = a(q,m)$
  und für $m\mid n$ offensichtlich auch $m\in \S_q$, ist obige Aussage lediglich
  eine Anwendung von \thref{satz:konstruktion_q_ordnung}.
\end{proof}

\begin{satz}[{{\autocite[Korollar 8.13]{hachenberger2013vorlesung}}}]
  Seien $F := \F_q$ ein endlicher Körper, $n\in \S_q$ und $L:=\cl_n(q-1)$.
  Ferner sei $u \in \bar F$ eine primitive $(nL)$-te Einheitswurzel. Dann ist
  mit Notation aus \thref{kor:konstruktion-q-ordnung}
  \[ w \speq{:=} \sum_{m\mid n} \ \sum_{i\in I_a}
    u^{\frac{n\, L}{m\, \cl_m(q-1)}i}\]
  normal in $E := \F_{q^n}$ über $F$.
\end{satz}
\begin{proof}
  Im Grunde haben wir bereits alles gezeigt. Daher reicht ein kurzer Kommentar,
  warum wir \thref{satz:konstruktion_q_ordnung} anwenden können, aus. Es ist
  trivialerweise 
  \[ \ord\big( u^{\frac{nL}{m\cl_m(q-1)}} \big) \speq=
    m\, \cl_m(q-1)\]
  und damit sind alle Voraussetzungen erfüllt.
\end{proof}

\begin{bemerkung}
  Wie wir in \thref{bsp:stark_regular_1} und \thref{satz:konstruktion_q_ordnung}
  gesehen haben, sind die $(ml)$-ten Einheitswurzeln die Elemente von kleinster
  multiplikativer Ordnung, deren $q$-Ordnung ein irreduzibler Teiler von
  $\Phi_m(x)$ wird. Natürlich können wir dieses Konzept auch erweitern und uns
  überlegen, welche primitiven Einheitswurzeln die selbe Eigenschaft erfüllen.
  Darüber hinaus können wir die Elemente, deren $q$-Ordnung ein irreduzibler
  Teiler von $\Phi_m(x)$ über $F[x]$ ist, auch durch die Modulstruktur selbst
  beschreiben.
  Diese beiden Überlegungen wollen wir in den nächsten beiden Aussagen beweisen.
\end{bemerkung}

\begin{lemma}
  \label{lemma:hoehere_wurzeln_auch_erzeuger}
  Seien die Voraussetzungen wie in \thref{satz:konstruktion_q_ordnung}, also
  $F = \F_q$ ein endlicher Körper und $m\in \S_q$. Setze ferner 
  $l := \cl_m(q-1)$, $a := \ggT(l,m)$. 
  Ist nun $\theta$ eine primitive $(nf)$-te Einheitswurzel für 
  $l \mid f \mid q-1$, so ist $\Ord_q(\theta)$ ein irreduzibler Teiler
  von $\Phi_m(x)$ über $F[x]$.
\end{lemma}
\begin{proof}
  Sei $f = le$ mit $\ggT(e,l) = 1$. Diese Zerlegung ist möglich, da
  $l$ per Definition der größte Teiler von $q-1$ ist, dessen Primfaktoren
  allesamt in $m$ vorkommen, d.h. jeder Primfaktor von $q-1$, der $l$ teilt,
  kommt dort bereits in maximaler Potenz vor.
  Damit reicht es -- wenn wir wir den Beweis von
  \thref{satz:konstruktion_q_ordnung} noch einmal nachvollziehen -- zu zeigen,
  dass $\ggT(mle, ck) = ce$ für $c := \kgV(m,l)$ und 
  $q^{\frac m a} -1 = ck$ für ein $k$ mit $\ggT(k,ml) = 1$. Da $\ggT(e,l) = 1$,
  ist per Definition von $l$ auch $\ggT(e,m) = 1$ und damit auch
  $\ggT(e,c) = 1$. Da $e \mid q-1\mid q^{\frac m a -1}$ zerfällt $k$ 
  in $k = \bar k k_e$ mit $e \mid k_e$ und $\ggT(\bar k,e) = 1$. Damit
  folgt $\ggT(mle, ck) = ce$ und wir haben
  $\ord(\theta^{ck}) = \frac{mle}{\ggT(mle,ck)} = \frac{mle}{ce} = a$.
\end{proof}

Als Korollar dieses Lemmas können wir einen Satz von \citeauthor{semaev:1989}
\citeyear{semaev:1989} in 
\autocite{semaev:1989} beweisen, welcher erneut von \citeauthor{gao:1997}
\citeyear{gao:1997} in \autocite{gao:1997} bewiesen wurde:

\begin{satz}[{\autocite[Theorem 2.7]{gao:1997}, \autocite{semaev:1989}}]
  \label{satz:gao1}
  Sei $q = p^r$ eine Primzahlpotenz. Sei $n \in \N$ mit $\ggT(n,q)=1$.
  Ist $x^n - a \in \F_q[x]$ irreduzibel mit Wurzel $\theta\in \F_{q^n}$, so
  gilt 
  \[ \F_{q^n} \speq= \bigoplus_{l\in R_q(n)} \langle \theta^l\rangle_q\,,\]
  wobei $\langle \theta^l\rangle_q := 
  \spann_{\F_q}\{ \theta^i:\ i\in M_q(l\bmod n)\}$.
\end{satz}
\begin{proof}
  Aus \thref{satz:binom_irreduzibel} und \thref{satz:binom_irreduzibel_aquiv}
  wissen wir, dass für $a \in \F_q^\ast$ das Polynom
  $x^n-a \in \F_q[x]$ genau dann irreduzibel ist, wenn
  $p\nmid n$, $\nu(n) \mid f := \ord(a)$, $l := \cl_n(q-1)\mid f$ und
  $q \equiv 1 \bmod 4$, falls $4\mid n$. 
  Ist bereits $q\equiv 1 \bmod 4$, falls $n$ gerade, so sind wir genau in der 
  Situation, dass $n$ stark regulär ist.
  Damit ist $\theta$ eine primitive $(nf)$-te Einheitswurzel und wir wissen
  nach \thref{lemma:hoehere_wurzeln_auch_erzeuger}, dass
  $\Ord_q(\theta)$ ein irreduzibler Teiler von $\Phi_n(x)$ über $\F_q[x]$ ist,
  womit alles gezeigt wäre.

  Es bleibt ein Wort zur Situation $q\equiv 3 \bmod 4$, falls $n$ gerade, zu
  verlieren. Dieser Fall wird von den bisherigen Resultaten nicht erfasst.
  Mit etwas mehr Aufwand ist es jedoch möglich, die explizite Konstruktion 
  von Normalbasen mit primitiven Einheitswurzeln zu erweitern, wie Hachenberger
  in \autocite[Section 22]{hachenberger1997finite} zeigt.
\end{proof}

Bevor wir die Ideen der stark regulären Körpererweiterungen verallgemeinern
wollen, betrachten wir noch ein Lemma, das die irreduziblen Teilmodule genauer
beschreibt und in ganz ähnlicher Form in \autocite[Theorem
22.5]{hachenberger1997finite} wiederzufinden ist.

\begin{lemma}
  \label{lemma:erzeuger_von_irred_teilmoduln}
  Seien die Voraussetzungen wie in \thref{satz:konstruktion_q_ordnung}, also
  $F = \F_q$ ein endlicher Körper und $m\in \S_q$. Setze ferner 
  $l := \cl_m(q-1)$, $a := \ggT(l,m)$. Ist dann $u$ eine primitive 
  $(ml)$-te Einheitswurzel mit $\Ord_q(u) = f(x)$ für $f(x)$ einen irreduziblen
  monischen Teiler von $\Phi_m(x)$, so gilt für $v \in E:= \F_{q^m}$:
  \[ \Ord_q(v) = f(x) \quad\Leftrightarrow\quad
    v = g(x) \cdot u\quad\text{ für ein } 0\neq g\in F[x]_{<\frac m a}\,.\]
\end{lemma}
\begin{proof}
  In \thref{kor:moduln_ueber_v_g} haben wir gezeigt, dass für $v \in E$ gilt
  \[ v \in V_f \quad\Leftrightarrow\quad v = g(x)\cdot u 
    \text{ für ein } g(x) \in F[x]_{<\deg f}\,.\]
  Da $f$ irreduzibel von Grad $\frac m a$ ist, sind alle $v \in
  V_f\setminus\{0\}$ von $q$-Ordnung $f$ und es folgt die Behauptung.
\end{proof}

\begin{bemerkung}
  Obiges Lemma gilt nicht nur für primitive $(ml)$-te Einheitswurzeln, sondern
  -- wie man offensichtlich erkennen kann -- für jedes Element mit gleicher
  $q$-Ordnung!
\end{bemerkung}




\section{Reguläre Erweiterungen}

Die Erkenntnisse über stark reguläre Erweiterungen wollen wir nutzen, um
Normalbasen auch in einem allgemeineren Kontext angeben zu können. Ist nämlich
$m\in\N$ ungerade mit $p\nmid m$ und setzen wir $s = \ord_{\nu(m)}(q)$, 
so erkennen wir, dass $\nu(m) \mid q^s-1$ per Definition von
$\ord_{\nu(m)}(q)$. Mit anderen Worten: $m \in \S_{q^s}$! 

\begin{definition}[regulär]
  Sei $q = p^r$ eine Primzahlpotenz und $n\in \N^\ast$. Setze 
  $n = n' p^c$ mit $p\nmid n'$. Das Paar $(q,n)$ heißt \emph{regulär}, falls
  $\ggT(n, \ord_{\nu(n')}(q)) = 1$. 
  Ist $F = \F_q$ und $E = \F_{q^n}$, so nenne die Erweiterung $E\mid F$ 
  \emph{regulär}.
\end{definition}

\begin{satz}[{{\autocite[Satz 9.3]{hachenberger2013vorlesung}}}]
  \label{satz:konstruktion_q_ordnung_reg}
  Seien $F := \F_q$ ein endlicher Körper für eine Primzahlpotenz $q = p^r$ 
  und $m\in\N$ ungerade mit $p\nmid m$.
  Setze $s := \ord_{\nu(m)}(q)$, 
  $l := \cl_m(q^s-1)$, $b:= b(q,m) = \ggT(l,m)$, $E := \F_{q^m}$ und
  $E' := \F_{q^{sm}}$. Ist dann
  $u$ eine primitive $(ml)$-te Einheitswurzel, so gilt
  \begin{enumerate}
    \item $F(u) = E'$,
    \item $\Ord_q(u) = f(x^s)$ für einen monischen irreduziblen Teiler 
      $f(x) \in F[x]$ von $\Phi_m(x)$ und 
    \item $v := \sum_{j \in R_q(m)\atop \ggT(j,m)=1} 
      u^j$ hat $q$-Ordnung $\Phi_m(x^s)$.
  \end{enumerate}
\end{satz}
\begin{proof}
  Wie zu Beginn dieses Abschnitts erwähnt, sieht man leicht, dass
  $m \in \S_{q^s}$.
  \begin{enumerate}
    \item Analog zum Beweis von \thref{satz:konstruktion_q_ordnung} 
      folgern wir $\ord_{ml}(q) = sm$.
    \item Nach \thref{satz:konstruktion_q_ordnung} ist nun
      $g(x) := \Ord_{q^s}(u)$ ein über $K := \F_{q^s}$ irreduzibler Teiler von
      $\Phi_m(x)$. Sei nun $\zeta\in K^\ast$ eine primitive $b$-te Einheitswurzel,
      so gilt ohne Einschränkung
      \[ g(x) \speq= x^{\frac m b} -\zeta\,.\]
      Definiere nun $f(x) := \prod_{i=0}^{s-1} (x^{\frac m b} - \zeta^{q^i})$, 
      so ist $f(x)$ ein irreduzibler monischer Teiler von $\Phi_m(x)$ über $F$,
      wie man sich leicht überlegt, betrachtet man den Zerfall von
      $\Phi_m(x)$ über $F$. Per Definition von $g(x)$ ist auch
      $f(\sigma^s)(u) = 0$. Also $\Ord_q(u) \mid f(x^s)$.
      Es bleibt nun zu zeigen, dass $\Ord_q(u) = f(x^s)$:
      Schreibe dazu $h(x) := \Ord_q(u)$. 
      Betrachten wir nun 
      \[ \Omega_F:\funcdef{ F[x] &\to& E'\,, \\
        p(x) &\mapsto& p(\sigma^s)(u)\,,}\]
      so stellen wir fest, dass $\ker\Omega_F = (f(x))$, da $f(x)$ irreduzibel
      über $F$ ist.
      Bezeichne nun $M := \im\Omega_F$. Dann ist 
      \[ \dim_F(M) \speq= \deg f \speq= \ord_m(q) \speq= s\ \frac m b\,,\]
      wobei letzte Gleichheit aus \thref{lemma:ordn_1} und \thref{lemma:ordn_2}
      folgt. Analog sehen wir ein, dass für
      $\Omega_K: K[x] \to E',\ p(x) \mapsto p(\sigma^s)(u)$ wegen der
      Irreduzibilität von $g(x)$ über $K$
      \[ \ker\Omega_K = (g(x)) \quad\text{und}\quad
        \dim_F(N)= [K:F]\dim_K(N) \speq= s\ \frac m b\,,\]
      für $N := \im\Omega_K$. Ferner ist klar, dass $M\subseteq N$, da
      $\Omega_F = \Omega_K\mid_{F[x]}$, womit wir durch Gleichheit der
      $F$-Dimensionen $M=N$ schließen können.
      Schließlich sei $\Gamma_u: F[x]\to E',\ p(x) \mapsto p(\sigma)(u)$, so
      ist $\ker\Gamma_u = (h(x))$ per Definition von $h(x)= \Ord_q(u)$. 
      Nach \thref{satz:moduln_ueber_v_g} (3) folgt sofort
      $\im\Gamma_u = V_h$ mit $V_h = \{ w\in E':\ h(\sigma)(w) = 0 \}$
      und wegen $\Omega_F = \Gamma_u \circ (p(x)\mapsto p(x^s))$ ist
      $N = M = \im\Omega_F \subseteq \im \Gamma_h$. Nach 
      \thref{satz:moduln_ueber_v_g} (4) ist $V_h$ zyklisch, also invariant
      unter $\sigma$ und wir können folgern, dass
      $N_i := \sigma^i(N) \subseteq V_h$ für alle $i\in \N$ gilt, wobei klar
      ist, dass $N_i$ von $x^{\frac m b}-\zeta^{q^i} \in K[x]$ annihiliert wird.
      Für $i=0,\ldots,s-1$ sind diese jedoch über $K$ paarweise teilerfremd
      und wir folgern
      \[ \sum_{i=0}^{s-1} N_i \speq= \bigoplus_{i=0}^{s-1} N_i 
        \speq= \{ w \in E':\ f(\sigma^s)(w) = 0 \} = V_{f(x^s)} 
        \speq\subseteq V_h\,.\]
      Ergo $h(x) \mid f(x^s)$, mithin $h(x) = f(x^s)$.
    \item Nach Definition von $R_q(m)$ (vgl. auch 
      \thref{beispiel:zerfall_x21_1_1}) bestimmt $R_q(m)$ den Zerfall
      von $x^m-1$ über $\F_q$. Für $j\in R_q(m)$ mit $\ggT(j,m) = 1$ 
      ist damit nach (2) $\Ord_q(u^j) = f_j(x^s)$ für einen irreduziblen Teiler
      $f_j(x)$ von $\Phi_m(x)$ über $F$. Für $i,j\in R_q(m)$ mit 
      $\ggT(i,m) = \ggT(j,m) = 1$ und $i\neq j$ sind jene Teiler auch
      verschieden und damit teilerfremd. Mithin gilt die Behauptung
      nach \thref{lemma:eigenschaften_tau_ordnung} (2).
  \end{enumerate}
\end{proof}

Nun haben wir also Elemente mit $q$-Ordnung $\Phi_m(x^s)$ gefunden. Es stellt
sich jedoch die Frage, wir daraus Elemente mit $q$-Ordnung $\Phi_m(x)$ 
\glqq basteln\grqq können. Dies zeigt uns der folgende Satz, wobei 
die Punkte (1) und (2) aus 
\autocite[Satz 9.4]{hachenberger2013vorlesung} stammen:

\begin{satz}
  \label{satz:q_ordnung_von_trace}
  Seien alle Voraussetzungen und Notationen wie in 
  \thref{satz:konstruktion_q_ordnung_reg}.
  Sei jedoch zusätzlich $\ggT(m,s) = 1$.
  Bezeichne ferner 
  $\sigma_E: E'\to E',\ x \mapsto x^{q^m}$ den Frobenius von $E'$ auf $E$.
  Ist dann $u$ eine primitive $(ml)$-te Einheitswurzel
  und $v := \sum_{j\in R_q(m)\atop \ggT(j,m)=1} u^j$, so gilt:
  \begin{enumerate}
    \item $H(\sigma_E)(v)$ hat $q$-Ordnung $\Phi_m(x)$ für 
      $H(x) := \frac{\Phi_m(x^s)}{\Phi_m(x)}$.
    \item $\Tr_{E'\mid E}(v)$ hat $q$-Ordnung $\Phi_m(x)$.
    \item $\Ord_q\big(\Tr_{E'\mid E}(u)\big)$ ist ein irreduzibler 
      monischer Teiler von $\Phi_m(x)$.
  \end{enumerate}
\end{satz}
\begin{proof}
  Zerlegen wir $s = \bar s p^\beta$ mit $p\nmid \bar s$, so ist 
  nach Voraussetzung $\ggT(\bar s, m) = 1$ und wir sind in der Situation von
  \thref{satz:zusammenhang_unterschiedlicher_kreisteilungspolys}. Damit gilt
  \[ \Phi_m(x^s) \speq= \Phi_m(x^{\bar s})^{p^\beta}
    \speq= \prod_{d \mid \bar s} \Phi_{md}(x)^{p^\beta}\,.\]
  Mit konsequenter Anwendung von \thref{lemma:eigenschaften_tau_ordnung}
  folgern wir die Behauptungen:
  \[ \Ord_q(H(\sigma_E)v) = \frac{\Phi_m(x^s)}{\ggT(\Phi_m(x^s), H(x))}
    = \Phi_m(x)\,.\]
  Für (2) und (3) überlegen wir uns, dass für $a \in E'$
  \[ \Tr_{E'\mid E}(a) = \sum_{i=0}^{s-1} a^{q^{im}} = 
    \left[\frac{x^{sm}-1}{x^m-1}\right](\sigma_E)(a)\]
  und 
  \[ \frac{x^{sm} -1}{x^m-1} \speq= 
    \frac{ \prod_{d\mid \bar sm} \Phi_d(x)^{p^\beta}}{
      \prod_{d\mid m} \Phi_d(x)} \speq=
    \Phi_m(x)^{p^\beta-1}\ 
    \prod_{l\mid m\atop l\neq m} \Phi_l(x)^{p^\beta-1}\ 
    \prod_{d\mid \bar s\atop d\neq 1} \Phi_{md}(x)^{p^\beta} \,.\]
  Ergo ist
  \[ \Ord_q(\Tr_{E'\mid E}(v)) \speq= 
    \frac{\Phi_m(x^s)}{\ggT(\Phi_m(x^s), \frac{x^{sm}-1}{x^m-1})}
    \speq= \Phi_m(x)\]
  und
  \[ \Ord_q(\Tr_{E'\mid E}(u)) \speq= 
    \frac{f(x^s)}{\ggT(f(x^s), \frac{x^{sm}-1}{x^m-1})} \speq= f_{1,1}(x) \,,\]
  wobei wir uns hierfür noch einmal an \thref{satz:zerfall_f_x_s}
  erinnern müssen, wo wir gezeigt haben, dass in genau dieser Situation 
  \[ f(x^{\bar s}) = \prod_{d\mid m} \prod_{i=1}^{\Delta_q(m,d)} f_{d,i}(x) \]
  für monische, irreduzible Teiler $f_{d,i}$ von $\Phi_{md}(x)$ für alle 
  $i=1,\ldots,\Delta_q(m,d)$ und wir damit 
  \[ \ggT\big(f(x^{\bar s})^{p^\beta}, \tfrac{x^{sm}-1}{x^m-1}\big) \speq=
    \prod_{d\mid m\atop d \neq 1} \prod_{i=1}^{\Delta_q(m,d)} f_{d,i}(x) \]
  folgern können. Abschließend sehen wir, dass
  \[ \Delta_q(m,1) \speq= \frac{\varphi(1) \ord_m(q)}{\ord_{1\cdot m}(q)}
    \speq= 1\,.\]
\end{proof}


%Wir können sogar noch in gewisser Weise eine Verschärfung dieses Satzes
%angeben, die genauer charakterisiert, warum die Spurfunktion in diesem Fall
%gute Dienste leistet.

%\begin{satz}
  %\label{satz:q_ordnung_von_zusammensetzen}
  %Seien die Voraussetzungen wie in \thref{satz:q_ordnung_von_trace}.
  %Ist dann wiederum $u$ eine primitive $(nl)$-te Einheitswurzel und
  %$k(x) = \sum_{i=0}^j a_i x^i \in F[x]$, $j\in\N$, so dass
  %$\Ord_q(u) = \Ord_q(u^i) = f(x^s)$ für einen irreduziblen monischen Teiler 
  %$f(x)$ von $\Phi_m(x)$ über $F$ für alle $i=0,\ldots,j$. Gilt letztlich
  %$[F(k(u)):F]=n$, so gilt:
  %$\Ord_q(k(u))$ ist ein irreduzibler monischen Teiler 
  %von $\Phi_m(x)$ über $F$.
%\end{satz}
%\begin{proof}
  %Sammeln wir einmal den Wissensstand für gegebene Situation zusammen:
  %\begin{itemize}
    %\item $\Ord_q(u) = \Ord_q(u^i) = f(x^s)$ teilt $\Phi_m(x^s)$.
    %\item Für $s = \bar s p^\beta$ mit $p\nmid \bar s$ gilt nach
      %\thref{satz:zusammenhang_unterschiedlicher_kreisteilungspolys}
      %\[\Phi_m(x^s) = \prod_{d\mid \bar s} \Phi_{md}(x)^{p^\beta}\]
    %\item Als leichte Konsequenz aus 
      %\thref{lemma:} haben wir 
      %\[\Ord_q(k(u)) \speq\mid \kgV\{\Ord_q(a_iu^i): i=1,\ldots,j\} = f(x^s) \]
    %\item Da $k(u)\in E$ und die $q$-Ordnung immer das Minimalpolynom 
      %des zugehörigen Frobenius teilt, folgt
      %\[\Ord_q(k(u)) \speq\mid x^m-1 = \prod_{d\mid m} \Phi_d(x)\]
  %\end{itemize}
  %Zusammengefasst teilt also $\Ord_q(k(u))$ nur den Anteil von $f(x^s)$,
  %der auch in $\Phi_m(x)$ liegt. Damit folgt nach
  %\thref{satz:f_x_s_ist_teiler_von_phimd} die Behauptung.
%\end{proof}

%\begin{bemerkung}
  %Auf diese Weise kann man auch den Beweis von \thref{satz:q_ordnung_von_trace}
  %führen, wenn man sich überlegt, dass für $\Ord_q(u) = f(x^s)$ auch
  %$\Ord_q(u^{q^m}) = f(x^s)$ ($m$ und $s$ sind teilerfremd!) 
  %gilt und gerade $\Tr_{E'\mid E}(u) \in E$ gilt.
%\end{bemerkung}


Auch für reguläre Erweiterungen wollen wir ein Analogon von 
\thref{lemma:hoehere_wurzeln_auch_erzeuger} beweisen:

\begin{lemma}
  \label{lemma:hoehere_wurzeln_auch_erzeuger_reg}
  Seien die Voraussetzungen wie in \thref{satz:konstruktion_q_ordnung_reg}, also
  $F := \F_q$ ein endlicher Körper für eine Primzahlpotenz $q = p^r$ 
  und $m\in\N$ ungerade mit $p\nmid m$.
  Setze $s := \ord_{\nu(m)}(q)$, 
  $l := \cl_m(q^s-1)$, $b:= b(q,m) := \ggT(l,m)$, $E := \F_{q^m}$ und
  $E' = \F_{q^{sm}}$. 
  Ist nun $\theta$ eine primitive $(mf)$-te Einheitswurzel für 
  $l \mid f \mid q^s-1$, so ist $\Ord_q(\theta) = f(x^s)$ für 
  $f(x)$ einen irreduzibler Teiler von $\Phi_m(x)$ über $F[x]$.
\end{lemma}
\begin{proof}
  \thref{lemma:hoehere_wurzeln_auch_erzeuger} mit dem Beweis von
  \thref{satz:konstruktion_q_ordnung_reg}.
\end{proof}


Wir können in gewisser Weise sogar eine Erweiterung von 
\thref{satz:q_ordnung_von_trace} angeben und
gleichzeitig die Verwendung der Spurfunktion rechtfertigen. 
Wir müssen lediglich dafür
sorgen, dass wir ein Element desjenigen Körpers erhalten, 
für den wir Erzeuger (und letztendlich normale Elemente) konstruieren wollen.

\begin{satz}
  \label{satz:q_ordnung_von_zusammensetzen}
  Seien die Voraussetzungen wie in \thref{satz:q_ordnung_von_trace} und 
  sei wiederum $u$ eine primitive $(mf)$-te Einheitswurzel für
  $l\mid f \mid q^s-1$.
  Für $g(x) \in F[x]$ gilt:
  %Ist $[F(g(x)\cdot u) : F] = m$, so gilt:
  Ist $g(x)\cdot u \in \F_{q^m}^\ast$ so gilt:
  \[ \Ord_q(g(x)\cdot u) \text{ ist ein monischer irreduzibler Teiler von }
    \Phi_m(x)\,. \]
\end{satz}
\begin{proof}
  Auf der einen Seite haben wir 
  \[ \Ord_q(g(x)\cdot u) \speq\mid x^m-1 \speq= \prod_{d\mid m} \Phi_d(x)\,, \]
  da $g(x)\cdot u$ nach Voraussetzung in $\F_{q^m}$ liegt.
  %da $g(x) \cdot u$ nach Voraussetzung Grad $m$ über $F$ hat und auf der
  Andererseits gilt nach \thref{lemma:eigenschaften_tau_ordnung}
  \[ \Ord_q(g(x)\cdot u) \speq= \frac{f(x^s)}{\ggT(f(x^s), g(x))} 
    \speq\mid f(x^s) \,.\]
  Wir wissen jedoch aus \thref{satz:zerfall_f_x_s} (vgl. auch den Beweis von
  \thref{satz:q_ordnung_von_trace}), dass 
  \[ f(x^s) \speq= \prod_{d\mid \bar s} 
    \prod_{i=1}^{\Delta_q(m,d)} f_{d,i}(x)^{p^\beta}\]
  für $s = \bar sp^\beta$ mit $p\nmid \bar s$ und
  $f_{d,i}(x)$ monische, irreduzible Teiler von $\Phi_{md}(x)$. Da 
  $p\nmid m$ nach Voraussetzung, kommt $\Phi_m(x)$ in $x^m-1$ lediglich in
  einfacher Vielfachheit vor und damit folgt
  \[ \Ord_q(g(x)\cdot u) \speq= f_{1,1}(x)\]
  für den einzigen monischen irreduziblen Teiler in $f(x^s)$ von $\Phi_m(x)$.
\end{proof}

\begin{bemerkung}
  Wir hätten den Beweis von \thref{satz:q_ordnung_von_trace} auch mit obigem
  Satz führen können, da 
  \[ \Tr_{E'\mid E}(u) \speq= (1+x^m+\ldots + x^{m(s-1)})\cdot u \]
  und nach Definition der Spurfunktion von 
  $E'$ auf $E$ diese ein Element in $E$ liefert. 
\end{bemerkung}



\section{Normalbasen mit Dickson-Polynomen}

Im vorangegangenen Abschnitt haben wir uns primitiver Einheitswurzeln,
also Nullstellen der Kreisteilungspolynome, bedient,
um normale Elemente in (stark) regulären Erweiterungen zu konstruieren. Im
Folgenden werden wir erkennen, dass sich auch Nullstellen gewisser anderer
Polynome, namentlich Dickson-Polynome, nutzen lassen, um in speziellen
regulären Erweiterungen normale Elemente zu konstruieren.

\begin{definition}[Dickson-Polynom]
  \label{def:dickson}
  Sei $n \in \N^\ast$. Für $a \in \F_q$ definieren wir das
  \emph{$n$-te Dickson-Polynom erster Art über $\F_q$} 
  (hier auch nur \emph{$n$-tes Dickson-Polynom}) als
  \[ D_n(x,a) \speq{:=} \sum_{i=0}^{\lfloor\frac n 2\rfloor} \frac{n}{n-i}
    \binom{n-i}{i} (-a)^i x^{n-2i}\,.\]
  Das \emph{$n$-te Dickson-Polynom zweiter Art über $\F_q$} ist gegeben durch
  \[ E_n(x,a) \speq{:=} \sum_{i=0}^{\lfloor\frac n 2\rfloor} 
    \binom{n-i}{i} (-a)^i x^{n-2i}\,.\]
\end{definition}

\begin{bemerkung}
  Über den komplexen Zahlen sind die Dickson-Polynome 
  nahe verwandt mit den
  Chebyshev-Polynomen, wie man z.B. in \autocite[Absatz nach Corollary
  7.15]{lidl1997finite} nachlesen kann.
  Über endlichen Körpern liefern sie eine spezielle Klasse von
  Permutations-Polynomen (vgl. \autocite[Theorem 7.16]{lidl1997finite}, 
  \autocite[Section 9.6]{mullen2013handbook}).
\end{bemerkung}

Die Dickson-Polynome besitzen einige interessante Eigenschaften, 
welche wir im Folgenden zitieren wollen.

\begin{proposition}
  \label{satz:dickson_1}
   Seien $n\in \N^\ast$ und $a,y\in \F_q$ beliebig, so gilt
  \[ D_n(y+ay\inv, a) \speq= y^n +a^ny^{-n}\,.\]
\end{proposition}
\begin{proof}
  \autocite[Gleichung (7.8)]{lidl1997finite}.
\end{proof}

\begin{proposition}
  \label{satz:dickson_2}
  Sei $a \in \F_q$. Die Dickson-Polynome erster und zweiter Art 
  erfüllen für alle $n\geq 2$ folgende Rekursionsgleichungen:
  \begin{align*}
    D_n(x,a) &\speq= xD_{n-1}(x,a) - a D_{n-2}(x,a) \,, \\
    E_n(x,a) &\speq= xE_{n-1}(x,a) - a E_{n-2}(x,a)
  \end{align*}
  mit den Startwerten $D_0(x,a) = 2$ und $D_1(x,a)=x$ bzw.
  $E_0(x,a) = 1$ und $E_1(x,a) = x$. Ferner gilt
  \[ D_{mn}(x,a) \speq= D_m(D_n(x,a),a^n)\,.\]
\end{proposition}
\begin{proof}
  \autocites[Lemma 2.3, Lemma 2.6 (i)]{lidl1993dickson}.
\end{proof}

\begin{proposition}
  \label{prop:dickson_linear_unabhangig}
  Sei $a \in \F_q$. Für alle $n\geq 0$ sind die Polynome
  $D_0(x,a), \ldots, D_n(x,a)$ linear unabhängig über $\F_q$.
\end{proposition}
\begin{proof}
  Dies lässt sich per Induktion und obiger Rekursionsgleichung
  sehr leicht einsehen: Der Induktionsanfang ist
  mit $n = 1$ durch $D_0(x,a) = 2$ und $D_1(x,a) = x$ erledigt.
  Sei also $n > 1$, so ist $D_n(x,a) = xD_{n-1}(x,a)-aD_{n-2}(x,a)$.
  Die Polynome $D_1(x,a),\ldots,D_{n-1}(x,a)$ sind nach Induktionsvoraussetzung
  jedoch linear unabhängig.
\end{proof}

Ferner können wir eine  relativ einfache Charakterisierung angeben, wann ein
Polynom der Form $D_n(x,a)-b$ irreduzibel ist.

\begin{proposition}
  \label{satz:dickson_irred}
  Seien $n\geq 2$ eine ganze Zahl und $a,b \in \F_q$ mit $a \neq 0$.
  Seien
  \[ x^2 + bx +a^n \speq= (x-\beta_1)(x-\beta_2), \qquad 
    \beta_1,\beta_2 \in \F_{q^2}\]
  und $e_i = \ord(\beta_i)$ für $i=1,2$. Es ist
  \[ D_n(x,a) + b \]
  irreduzibel genau dann, wenn sowohl (1), als auch (2) für $i=1,2$ erfüllt
  sind:
  \begin{enumerate}
    \item Jeder ungerade Primteiler von $n$ teilt $e_i$, jedoch nicht
      $\tfrac{q^2-1}{e_i}$.
    \item Ist $n$ gerade, so ist $\charak \F_q$ ungerade und eine der beiden
      folgenden Aussagen gilt:
      \begin{enumerate}[label=(\arabic*')]
        \item $b^2-4a^n \neq 0$ ist ein quadratischer Rest in $\F_q$, $2\mid e_i$
          und falls $4\mid n$, so $4 \mid (q-1)$.
        \item $b^2-4a^n$ ist ein quadratischer Nichtrest in $\F_q$,
          $-b-2a^\frac n 2$ ist ein quadratischer Rest in $\F_q$, 
          $2 \mid \tfrac{q^2-1}{e_i}$ und falls $4\mid n$, so $2\mid e_i$, 
          aber nicht $\tfrac{q^2-1}{2e_i}$.
      \end{enumerate}
  \end{enumerate}
\end{proposition}
\begin{proof}
  \autocite[Theorem 4]{gao1994}
  %\autocite[Theorem 7.39]{lidl1993dickson}.
\end{proof}

Da die weiteren Aussagen lediglich für ungerade Erweiterungen bewiesen werden,
wollen wir obige Proposition in einer handlicheren Form für unsere Zwecke 
noch einmal verfassen.

\begin{kor}
  \label{kor:dickson_irred}
  Seien $n\geq 3$ eine ungerade ganze Zahl und $a,b\in \F_q^\ast$. Seien
  \[ x^2 - bx +a^n \speq= (x-\beta_1)(x-\beta_2), \qquad 
    \beta_1,\beta_2 \in \F_{q^2}\]
  und $e_i = \ord(\beta_i)$ für $i=1,2$. Es ist
  \[ D_n(x,a) - b \]
  irreduzibel genau dann, wenn $\nu(n)\mid e_i$, aber 
  $\nu(n)\nmid \tfrac{q^2-1}{e_i}$ für $i=1,2$.
\end{kor}

Ferner lässt sich spezifizieren, in welchem Körper die Nullstellen $\beta_1$ 
und $\beta_2$ liegen.

\begin{kor}
  \label{kor:dickson_quadr}
  Seien $n\geq 3$ eine ungerade Zahl und $a,b \in \F_q^\ast$. Ist $D_n(x,a)-b$
  irreduzibel über $\F_q$, so gilt:
  \begin{enumerate}
    \item Ist $\nu(n) \mid (q-1)$, so zerfällt
      $x^2-bx+a^n = (x-\beta_1)(x-\beta_2)$ über $\F_q$.
    \item Ist $\nu(n) \mid (q^2-1)$, aber $\nu(n)\nmid (q-1)$, so ist
      $x^2-bx+a^n$ über $\F_q$ irreduzibel.
  \end{enumerate}
\end{kor}
\begin{proof}
  Seien $\beta_1$ und $\beta_2$ die Nullstellen von $x^2-bx+a^n$ mit 
  $\ord(\beta_i) = e_i$ für $i=1,2$, so ist $\nu(n) \mid e_i$ nach 
  \thref{kor:dickson_irred}, aber $\nu(n)\nmid \frac{q^2-1}{e_i}$ für $i=1,2$.
  \begin{enumerate}
    \item Wäre $x^2-bx+a^n$ über $\F_q$ irreduzibel und
      $\beta \in \F_{q^2}\setminus\F_q$ eine Nullstelle, so
      ist $\ord(\beta) \mid q^2-1$ aber $\ord(\beta) \nmid q-1$,
      im Widerspruch zu $\nu(n)\mid q-1$.
    \item Lägen $\beta_1,\beta_2$ in $\F_q$, so folgte $e_i \mid q-1$ im
      Widerspruch zu $\nu(n)\mid e_i$ und $\nu(n)\nmid q-1$.
  \end{enumerate}
\end{proof}


\citeauthor{scheerhorn:1996} zeigt in
\autocites{scheerhorn:1996}{scheerhorn:1997}, dass sich Dickson-Polynome
eignen, um Normalbasen zu beschreiben. Im weiteren Verlauf werden wir sogar
sehen, dass sich daraus vollständig normale Polynome, also Polynome, deren
Nullstellen vollständig normale Elemente (\thref{def:vollst_normal}) sind,
konstruieren lassen. Alle Folgerungen über (vollständig) normale Polynome
basieren jedoch auf zwei zentralen Resultaten über die Modulstruktur der
betrachteten Erweiterungskörper, wie die beiden nachstehenden Sätze angeben.

\begin{satz}[{\autocite[Theorem 2]{scheerhorn:1997}}]
  \label{satz:scheerhorn1}
  Seien $n\geq 3$ ein Produkt ungerader Primzahlen von $(q+1)$ und 
  $a,b\in \F_q =: F$, sodass $D_n(x,a)-b \in F[x]$ irreduzibel ist.
  Sei $\gamma \in E := \F_{q^n}$ eine Wurzel von $D_n(x,a)$. Dann ist
  \[ E \speq= \langle 1\rangle_q \oplus 
    \bigoplus_{l\in R_q(n)\setminus \{0\}} \langle D_l(\gamma,a)\rangle_q\]
  eine Zerlegung von $E$ in irreduzible $F[x]$-Teilmoduln, wobei
  \[ \langle D_l(\gamma,a)\rangle_q \speq{:=}
    \spann_F\{ D_i(\gamma,a):\ i\in M_q(l\bmod n)\} \,,\]
  sodass $\{ D_i(\gamma,a):\ i\in M_q(l\bmod n)\}$ eine $F$-Basis von
  $\langle D_l(\gamma,a)\rangle_q$ ist.
\end{satz}

\begin{satz}[{\autocite[Theorem 3]{scheerhorn:1997}}] 
  %\marginpar{Genauere Betrachtung, warum $x^2+bx+a^n$ hier nicht irreduzibel
  %ist!?}
  \label{satz:scheerhorn2}
  Seien $n\geq 3$ ein Produkt ungerader Primzahlen von $(q-1)$ und
  $a,b \in \F_q =: F$, sodass $D_n(x,a)-b\in F[x]$ irreduzibel ist.
  Seien ferner $x^2+bx +a^n = (x-\beta)(x-a^n\beta\inv) \in \F_q$ und
  $\theta \in \F_{q^n}$ eine Wurzel von $x^n-\beta$. Setze 
  $\gamma := \theta + a\theta\inv$, so gilt für 
  $l\in R_q(n) \setminus\{0\}$
  \[ \langle \theta^l\rangle_q \oplus \langle \theta^{n-l}\rangle_q
  \speq= \langle D_l(\gamma,a)\rangle_q\]
\end{satz}

In \autocites{scheerhorn:1996}{scheerhorn:1997} werden die Beweise dieser
Resultate ohne Zuhilfenahme der Theorie über (stark) reguläre
Erweiterungen geführt, wie wir sie im letzten Abschnitt kennengelernt haben.
Wir wollen uns nun im Folgenden überlegen, dass die Theorie der vorherigen
Abschnitte an diesem Punkt ein besseres Verständnis der Struktur der Zerlegung 
des Erweiterungskörpers in Teilmoduln liefert. Ferner sind wir in der Lage,
das vielleicht überraschende Auftauchen von Dickson-Polynomen im Kontext
(vollständig) normaler Elemente zu rechtfertigen und die benötigten 
Eigenschaften herzuleiten, welche an diesem Punkt eine zentrale Rolle spielen
und die Verwendung der Dickson-Polynome motivieren.

\begin{proof}[von \thref{satz:scheerhorn1}]
  Betrachten wir die vorliegende Situation, so sehen wir, dass
  $s := \ord_{\nu(n)}(q) = 2$ und damit $n \in \S_{q^2}$.
  Sei nun 
  $x^2-bx+a^n = (x -\beta)(x -a^n\beta\inv)$ über $K := \F_{q^2}$
  (vgl. \thref{kor:dickson_quadr}).
  Dann ist 
  nach \thref{satz:binom_irreduzibel} und \thref{satz:binom_irreduzibel_aquiv} 
  $x^n-\beta \in \F_{q^2}[x]$ irreduzibel.
  Ist dann $\theta \in \F_{q^{2n}} =: E'$ eine Wurzel, so wissen wir nach
  \thref{satz:gao1} und \thref{lemma:hoehere_wurzeln_auch_erzeuger_reg}
  ($\theta$ ist eine $(n\ord(\beta))$-te primitive Einheitswurzel), dass
  $\Ord_q(\theta) = f(x^s)$ für einen irreduziblen Teiler $f(x)$ von
  $\Phi_n(x)$ über $F$. 
  %Nach \thref{satz:q_ordnung_von_trace}
  %ist
  %\[ f_1(x) \speq{:=} \Ord_q( \Tr_{E'\mid E}(\theta) ) \speq=
    %\Ord_q(\theta + \theta^{q^n})\]
  %ein irreduzibler Teiler von $\Phi_n(x)$ in $F[x]$.
  
  Es ist nun $\theta^{q^n} = \theta\inv$: 
  Da $\theta \in E'$, gilt $\theta^{q^{2n}} = \theta$, also
  \[ \theta^{q^{2n}-1} \speq= 1 \speq= \theta^{(q^n+1)(q^n-1)}\,. \]
  Wäre $\theta^{q^n-1} = 1$, so läge $\theta$ bereits in $\F_{q^n}$, im Widerspruch
  zur Definition von $\theta$.
  Also haben wir nach \thref{satz:q_ordnung_von_zusammensetzen} 
  gerade 
  \[ \Ord_q(\theta + a\theta\inv) \speq=
    \Ord_q((1+ax^n)\cdot \theta) \speq= f_{1,1}(x)\]
  für einen monischen, irreduziblen Teiler $f_{1,1}(x)$ von $\Phi_n(x)$
  (wieder in Notation von \thref{satz:zerfall_f_x_s}), denn
  mit \thref{satz:dickson_1} sehen wir, dass
  \[ D_n(\theta + a\theta\inv,a) -b \speq= \theta^n + a^n\theta^{-n} -b
    \speq= \beta +a^n\beta\inv -b \speq= b - b\speq= 0\,.\]
  %und da $D_n(x,a)-b \in F[x]$ irreduzibel nach Voraussetzung, hat
  %$\theta + a\theta\inv$ Grad $n$ über $F$.
  %Da mit $\theta^l$ für $l \in R_q(n)$ alle irreduziblen Teilmoduln
  %%
  %\footnote{hier vielleicht noch ein Satz/Lemma/Bemerkung im vorherigen 
  %Abschnitt über die Erzeugung \emph{aller} irreduziblen Teilmodule}
  %%
  %von $x^n-1$ erzeugbar sind, also $\Ord_q(\theta^i) = g(x^s)$ für einen
  %irreduziblen monischen Teiler $g(x)$ von $x^n-1$ über $F$, ist alles gezeigt,
  %da analog zu oben
  %\[ \Ord_q(\theta^l + a^l\theta^{-l}) \speq\mid x^n-1 \]
  %für $l\in R_q(n)$ ein irreduzibler Teiler von $x^n-1$ ist
  %(für $l=0$ hat $\theta^l+a^l\theta^{-l}$ natürlich Grad $0$ über $F$).
  %$\gamma := \theta + a\theta\inv$ und $D_l(\gamma,a) =
  %\theta^l+a^l\theta^{-l}$ schließt den Beweis ab.
  %Wählen wir $1 \neq l \in R_q(n)$, so entspricht 
  %$\Ord_q(\theta^l) = g(x^s)$ für einen monischen irreduziblen Teiler 
  %$g(x) \mid \Phi_m(x)$ mit $m\mid n$ und $\ggT(f,g) = 1$. 
  %Wegen $\theta^l+a^l\theta^{-l} = D_l(\theta+a\theta\inv,a)$ hat 
  %$\theta^l+a^l\theta^{-l}$ Grad kleiner oder gleich $n$ über $F$ 
  %(vgl. \thref{bem:grad_phi_l})
  %und damit sind auch
  %$\Ord_q(\theta+a\theta\inv)$ und $\Ord_q(\theta^l+a^l\theta^{-l})$ 
  %nach \thref{satz:zerfall_f_x_s} (2) teilerfremd.
  %Wegen $\theta^l+a^l\theta^{-l} = D_l(\theta+a\theta\inv,a)$ liegt 
  %$\theta^l+a^l\theta^{-l}$ immernoch in $\F_{q^n}$ und damit sind auch
  %$\Ord_q(\theta+a\theta\inv)$ und $\Ord_q(\theta^l+a^l\theta^{-l})$ 
  %teilerfremd (für $m \neq n$ klar und für $m=n$ wegen 
  %\thref{satz:zerfall_f_x_s} (2)).
  Also liegt $\theta +a\theta\inv$ in $E$, sein Minimalpolynom über
  $F$ ist $D_n(x,a)-b$ und wir setzen ohne Einschränkung 
  $\gamma:= \theta+a\theta\inv$.
  Die weiteren irreduziblen Teiler von $\Phi_n(x)$
  erhalten wir analog zu \thref{satz:konstruktion_q_ordnung_reg}: Für 
  $l \in R_q(n)\setminus \{ 0\}$ mit $\ggT(l,n) = 1$ ist
  $\Ord_q(\theta^l) = g(x^s)$ für einen irreduziblen monischen Teiler 
  $g(x)$ von $\Phi_n(x)$ mit $f(x) \neq g(x)$. 
  Nun vollführen wir erneut obigen Kniff und erhalten 
  \[ \Ord_q(D_l(\theta+a\theta\inv,a)) \speq=
    \Ord_q(\theta^l+a^l\theta^{-l}) \speq= 
    \Ord_q( (1+a^lx^n)\cdot \theta^l ) \speq= g_{1,1}(x)\]
  mit $\ggT(f_{1,1}(x), g_{1,1}(x)) = 1$ nach \thref{satz:zerfall_f_x_s} (2).
  Auf diese Weise lassen sich für jeden irreduziblen Teilmodul von 
  $V_{\Phi_n}$ Erzeuger angeben:
  \[ V_{\Phi_n} \speq= \bigoplus_{l\in R_q(n)\atop \ggT(l,n)=1} 
    \big( F[x] \cdot D_l(\theta+a\theta\inv,a)\big) \speq=
    \bigoplus_{l\in R_q(n)\atop \ggT(l,n)=1} 
    \big( F[x]\cdot (\theta^l+a^l\theta^{-l}) \big)\,. \]
  Für alle $i \in M_q(l\mod n)$ erhalten wir 
  selbstverständlich immer Erzeuger des
  gleichen irreduziblen Teilmoduls: Für alle 
  $i\in M_q(l\mod n)$ für ein festes $l\in R_q(n)$ mit 
  $\ggT(l,n)=1$ gilt: $\Ord_q(\theta^i) = h(x^s)$ für einen
  irreduziblen Teiler $h(x)$ von $\Phi_n(x)$. (Genauer gesagt haben wir 
  im Beweis von \thref{satz:konstruktion_q_ordnung_reg} gesehen,
  dass sich die $q^s$-Ordnungen von $\theta^i$ und $\theta^j$ für 
  $i,j \in M_q(l\mod n)$ für $i\neq j$ unterscheiden; jedoch nicht ihre
  $q$-Ordnungen.) 

  Alle weiteren Teiler $\Phi_m(x)$ von $x^n-1$ für $m\mid n$ decken wir mit
  obiger Argumentation analog ab: Für $l \in R_q(n)$ mit 
  $r_q(l\mod n) = \ord_m(q)$ ist 
  $\Ord_q(\theta^l+a^l\theta^{-l})$ ein irreduzibler monischer Teiler
  von $\Phi_m(x)$.

  Letztlich haben wir bisher nur Erzeuger der irreduziblen Teilmoduln 
  im Sinne der Modulstruktur angegeben, jedoch keine $F$-Basis wie gefordert.
  Dazu müssen wir uns nur noch kurz überlegen, dass
  \[\{1\}\cup \{ D_i(\theta+a\theta\inv,a):\ i=1,\ldots,n-1\} \]
  eine $F$-Basis von $\F_{q^n}$ ist, da alle $D_i(x,a)$ für
  $i\geq 1$ linear unabhängig über $F$ nach 
  \thref{prop:dickson_linear_unabhangig} sind. Diese Basis lässt sich nach 
  Erzeugern wie oben gezeigt partitionieren.
\end{proof}

%\begin{bemerkung}
  %\label{bem:grad_phi_l}
  %In der Tag können wir sogar genau angeben, welchen Grad 
  %$\theta^l+a^l\theta^{-l}$ über $F$ hat: Ist $l\mid n$, 
  %so hat $\theta^l+a^l\theta^{-l}$ Grad $\tfrac n l$ über $F$, denn nach
  %\thref{satz:dickson_2} ist
  %\[ \F_q\ \ni\ D_n(\theta + a \theta\inv, a) \speq= 
    %D_{\frac n l}\big( D_l(\theta+a\theta\inv,a),\, a^l\big) \speq=
    %D_{\frac n l}\big(\theta^l + a^l \theta^{-l}, a^l\big)\,.\]
  %Für $\ggT(n,l) = 1$ hat $\theta^l + a^l\theta^{-l}$ weiterhin Grad $n$ über 
  %$F$, da $\theta^l + a^l\theta^{-l} = D_l(\theta+a\theta\inv,a)$ 
  %\TODO
%\end{bemerkung}


\begin{proof}[von \thref{satz:scheerhorn2}]
  Im Grunde brauchen wir nichts zu zeigen, wenn wir uns überlegen, dass
  hier $n$ stark regulär ist! Nach \thref{lemma:hoehere_wurzeln_auch_erzeuger}
  ist dann $\Ord_q(\theta)$ ein irreduzibler monischer Teiler von $x^n-1$. Also
  werden alle irreduziblen Teilmoduln von $\F_{q^n}$ von $\theta^l$ mit 
  $l\in R_q(n)$ erzeugt.  Wie im Beweis von \thref{satz:scheerhorn1} ist
  natürlich 
  \[ D_n(\theta+a\theta\inv,a)-b \speq= 0 \quad\text{und}\quad
    D_l(\theta+a\theta\inv,a) \speq= \theta^l+a^l\theta^{-l},\ l=1,\ldots,n\,. \]
  Letztlich überlege man sich, dass $\Ord_q(a^l\theta^{-l}) =
  \Ord_q(\theta^{n-l})$, da $a \in F$. 
\end{proof}


Man bemerke, dass im Beweis von \thref{satz:scheerhorn1}
$\theta + a\theta\inv$ lediglich eine
Abwandlung von $\theta + \theta^{q^n} = \Tr_{E'\mid E}(\theta)$ ist!
Daher ist das Resultat vielleicht auch nicht besonders überraschend,
da wir in \thref{satz:q_ordnung_von_trace} erkannt haben, dass die Spurfunktion
zur Konstruktion von normalen Elementen in regulären Erweiterungen zentral ist.
%In Notation des vorherigen Abschnittes würde man die beiden Sätze von
%Scheerhorn vielleicht auf die folgende Art und Weise aufschreiben. 
%Insbesondere erlaubt es die Theorie aus den vorherigen Abschnitten, den Fall
%$\nu(n)\mid (q-1)$ zu erweitern, um auch dort (unter leichten
%Zusatzvoraussetzungen) ein normales Element zu konstruieren.
Insbesondere erlaubt es die Theorie des vorherigen Abschnitts, die beiden Sätze 
auf den Fall $\nu(n)\mid q^2-1$ zu erweitern.

\begin{satz}
  \label{satz:scheerhorn_erweitert}
  Seien $F := \F_q$ ein endlicher Körper von Charakteristik $p$ und $m\in
  \N^\ast$ ungerade mit $p\nmid n$ und $\nu(n)\mid q^2-1$. Sind $a,b\in F$, 
  sodass $D_n(x,a)-b \in F[x]$ irreduzibel ist mit Nullstelle $\gamma \in E:=
  \F_{q^n}$, so gilt:
  \begin{enumerate}
    \item $E = F(\gamma)$,
    \item Ist $\nu(n)\nmid q-1$, so gilt:
      Für alle $l\in R_q(n)$ mit $r_q(l\bmod n) = \ord_q(m)$ 
      für einen Teiler $m$ von $n$ ist 
      $\Ord_q(D_l(\gamma,a))$ ein irreduzibler monischer Teiler von
      $\Phi_m(x)$.
    \item Setze 
      \[ v := \sum_{l\in R_q(n)} D_l(\gamma,a)\,. \]
      Dann ist $v \in E$ normal über $F$, falls
      $\nu(n)\nmid q-1$ oder $a^l\neq -1$ für alle $l \in R_q(n)$.
  \end{enumerate}
\end{satz}
\begin{proof}
  Für $\nu(n)\nmid q-1$ 
  sehen wir, dass weiterhin $n\in \S_{q^2}$ und damit der Beweis 
  von \thref{satz:scheerhorn1} auch hier gültig ist.
  Für $\nu(n)\mid q-1$ bleibt anzugeben, warum $v$ normal über $F$ ist, wenn
  $a^l\neq -1$ für alle $l\in R_q(n)$:
  Da in $v$ gerade je zwei Erzeuger pro irreduziblem Teilmodul von $E$
  vorkommen, ist sicherzustellen, dass diese sich nicht gegenseitig aufheben:
  Existierte $a^l = -1$ für ein $l\in R_q(n)$, so wäre
  \[ \theta^{n-l} + a^l\theta^{n-l} \speq= 0\,.\]
\end{proof}


\section{Normale und vollständig normale Polynome mit
  Dickson-Poly\-nomen}

Mit Hilfe dieser Resultate lassen sich nun relativ einfach vollständig normale
Polynome angeben, wie sie in \autocite[Section 3]{scheerhorn:1997} und 
\autocite[Section 4]{scheerhorn:1996} zu finden sind. Ihre Beweise beinhalten
keine besondere Beachtung der Modulstrukturen und sollen daher hier nur
nachvollzogen werden. Wie im vorhergehenden Abschnitt können wir den Fall
$\nu(n)\mid q+1$ wieder erweitern.
Zunächst brauchen wir jedoch eine Transformation eines Polynoms, welches gerade
die Nullstellen desselbigen invertiert.

\begin{definition}[reziprokes Polynom]
  \label{def:reziprokes_polynom}
  Sei $f(x) \in \K[x]$ ein Polynom über einem Körper $\K$ mit 
  $f(0)\neq 0$. Dann heißt
  \[ f^\ast(x) \speq{:=} \tfrac{1}{f(0)}\, x^{\deg f}\, f(\tfrac 1 x)
    \quad \in \K[x] \]
  \emph{reziprokes Polynom von $f(x)$}.
\end{definition}

\begin{bemerkung}
  Man bemerke, dass das reziproke Polynom stets normiert ist.
\end{bemerkung}

\begin{bemerkung}
  Ist $u$ eine Nullstelle von $f(x)$, so ist $u\inv$ eine Nullstelle von 
  $f^\ast(x)$, wie man sofort erkennen kann.
\end{bemerkung}

\begin{satz}%[{\autocite[Theorem 4]{scheerhorn:1997}}]
  \label{satz:scheerhorn3}
  Seien $n\geq 3$ ein Produkt aus ungeraden Primteilern von 
  $q^2-1$ mit $\nu(n) \nmid q-1$ und
  $D_n(x,a)-b \in \F_q[x]$ irreduzibel. Seien ferner $s,t\in \F_q$ mit
  $s\neq 0$. Dann ist
  \[ \big(D_n(sx+t,a)-b\big)^\ast\]
  normal über $\F_q$ genau dann, wenn für jedes $l\in R_q(n)$ ein
  $i\in M_q(l\bmod n)$ existiert, sodass $E_{n-1-i}(t,a)\neq 0$.
\end{satz}
\begin{proof}
  Setzen wir wieder $x^2-bx+a^n = (x-\beta)(x-a^n\beta\inv)$ über $\F_{q^2}$
  wie im Beweis von \thref{satz:scheerhorn1} mit $\theta \in \F_{q^{2n}}$ einer
  Nullstelle von $x^n-\beta$, so erkennen wir, dass
  $s(\theta-t+a\theta\inv)\inv$ eine Nullstelle von 
  $\big(D_n(sx+t,a)-b\big)^\ast$ ist.
  Dort haben wir ebenfalls gesehen, dass
  $\{ 1\} \cup \{\theta^l+a^l\theta^{-l}:\ l=1,\ldots,n-1\}$ eine 
  $\F_q$-Basis von $F_{q^n}$ bildet. Offensichtlich ist
  $s(\theta-t+a\theta\inv)\inv$ genau dann normal über $\F_q$, wenn
  $\theta+t+a\theta\inv$ normal über $\F_q$ ist. Also überlegen wir uns,
  für welche $\alpha = a_0 + \sum_{i=1}^{n-1} a_i(\theta^i+a^i\theta^{-i})$
  das Produkt $\alpha(\theta-t+a\theta\inv)$ in $\F_q^\ast$ liegt.
  Durch Ausmultiplizieren erkennt man, dass die Koeffizienten $a_i$ gerade 
  die Rekursionsgleichung der Dickson-Polynome zweiter Art erfüllen müssen 
  (siehe \autocite[Beweis zu Theorem 4]{scheerhorn:1997} für die konkrete
  Rechnung). Damit ist $\alpha(\theta-t+a\theta\inv) \in \F_q^\ast$, falls
  \[ \alpha \speq= E_{n-1}(t,a) + \sum_{i=1}^{n-1} E_{n-1-i}(t,a)
    (\theta^ia^i+\theta^{-i})\,.\]
  Die Behauptung des Satzes folgt dann sofort mit \thref{satz:scheerhorn1} 
  (oder \thref{satz:scheerhorn_erweitert}).
\end{proof}


Ferner können wir genau diese Folgerung benutzen, um vollständig normale
Polynome, also solche, die über jedem Zwischenkörper normal sind, anzugeben.

\begin{satz}%[{\autocite[Theorem 5]{scheerhorn:1997}}]
  \label{satz:scheerhorn4}
  Seien $n\geq 3$ ein Produkt aus ungeraden Primteilern von 
  $q^2-1$ mit $\nu(n)\nmid q-1$ und
  $D_n(x,a)-b \in \F_q[x]$ irreduzibel. Seien ferner $s,t\in \F_q$ mit
  $s\neq 0$. Ist $d$ ein Teiler von $n$, so ist
  \[ \big(D_n(sx+t,a)-b\big)^\ast\]
  normal über $\F_{q^d}$ genau dann, wenn für jedes 
  $l\in R_{q^d}(\tfrac n d)$ ein
  $i\in M_{q^d}(l\bmod \tfrac n d)$ existiert, sodass 
  $E_{\frac n d -1-i}(t,a)\neq 0$.
\end{satz}
\begin{proof}
  Wie man sich leicht überlegt, ist 
  $\theta^{\frac n d}+a^{\frac n d} \theta^{-\frac n d} \in \F_{q^d}$ 
  und das Minimalpolynom von 
  $(\theta -t+a\theta\inv)$ über $F_{q^d}$ ist 
  $D_{\frac n d}(x+t,a) - (\theta^{\frac n d}+a^{\frac n d}\theta^{-\frac n
  d})$. Dann folgt die Aussage analog zu vorherigem Satz.
\end{proof}


Nun lassen sich einige Korollare ziehen, in denen spezielle Werte für $a,b,s,t$
dazu führen, dass die geforderten Eigenschaften aus 
\thref{satz:scheerhorn3} bzw. \thref{satz:scheerhorn4} erfüllt sind.
Diese werden wir hier nicht beweisen und der Leser sei auf
\autocite{scheerhorn:1997} verwiesen. Erneut sind wir in der Lage, die
Korollare für den erweiterten Fall $\nu(n)\mid q^2-1$ mit $\nu(n)\nmid q-1$ zu
formulieren.


\begin{kor}
  Sei $n\geq 3$ ein Produkt ungerader Primfaktoren von $q^2-1$ mit
  $\nu(n)\nmid q-1$ und sei
  $D_n(x,a) -b$ irreduzibel über $\F_q$. Dann ist
  \[ \big(D_n(x,1) -b\big)^\ast\]
  ein vollständig normales Polynom über $\F_q$.
\end{kor}
\begin{proof}
  \autocite[Corollary 1]{scheerhorn:1997} mit
  \thref{satz:scheerhorn4}.
\end{proof}


\begin{kor}
  Sei $n\geq 3$ ein Produkt ungerader Primfaktoren von $q^2-1$
  mit $\nu(n)\nmid q-1$ und sei
  $D_n(x,a) -b$ irreduzibel über $\F_q$. Ist ferner $n\nequiv 0 \bmod 3$, so
  ist
  \[ \big(D_n(x\pm 1,1) -b\big)^\ast\]
  ein vollständig normales Polynom über $\F_q$.
\end{kor}
\begin{proof}
  \autocite[Corollary 2]{scheerhorn:1997} mit 
  \thref{satz:scheerhorn4}.
\end{proof}

\begin{kor}
  Sei $n\geq 3$ ein Produkt ungerader Primfaktoren von $q^2-1$ mit
  $\nu(n)\nmid q-1$ und sei
  $D_n(x,a) -b$ irreduzibel über $\F_q$. Dann ist
  \[ \big(D_n(x\pm 2,1) -b\big)^\ast\]
  ein vollständig normales Polynom über $\F_q$.
\end{kor}
\begin{proof}
  \autocite[Corollary 3]{scheerhorn:1997} mit 
  \thref{satz:scheerhorn4}.
\end{proof}



