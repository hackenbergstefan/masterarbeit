\addchap*{Liste verwendeter Symbole}

\renewcommand{\arraystretch}{1.2}

\subsubsection{Allgemeines}
\begin{longtable}[h]{>{\raggedright}p{4cm}@{\qquad}p{10cm}}
$\N$ & Menge der natürlichen Zahlen mit $0$\\
$\Z$ & Menge der ganzen Zahlen\\
$\Z_m$ & ganze Zahlen modulo $m$\\
$F[x]$ & univariater Polynomring über $F$\\
$F[x]_{<k}$ & Menge der univariaten Polynome über $F$ vom Grad kleiner $k$\\
$\varphi$ & Eulersche Phifunktion (\thref{def:euler_phi})\\
$M^\ast$ & Menge $M$ ohne das Element $0$\\
\end{longtable}

\subsubsection{Gruppentheoretisches}
\begin{longtable}[h]{>{\raggedright}p{4cm}@{\qquad}p{10cm}}
$\langle a \rangle$ & von $a$ erzeugte Gruppe\\
$\ord(u)$ & gruppentheoretische Ordnung des Elements $u$\\
\end{longtable}

\subsubsection{Ringe und Moduln}
\begin{longtable}[h]{>{\raggedright}p{4cm}@{\qquad}p{10cm}}
$(r)$ & von $r$ erzeugtes Ideal\\
$M^\times$ & Einheiten von $M$\\
$V_r$ & die von $r$ annihilierten Elemente
  (\thref{def:V_r})\\
$\Ann_R(S)$ & Annihilator von $S$ in $R$
  (\thref{def:annihilator})\\
$f(x)\cdot v$ & Multiplikation im $K[x]$-Modul bzgl. $\tau$
  (\thref{def:V_tau})\\
$\Ord_\tau(v)$ & $\tau$-Ordnung von $v$ 
  (\thref{def:tau_ordnung})\\
\end{longtable}

\subsubsection{Zahlentheoretisches}
\begin{longtable}[h]{>{\raggedright}p{4cm}@{\qquad}p{10cm}}
$\nu(n)$ & quadratfreier Teil von $n$ (\thref{def:quadratfreier_teil})\\
$\ord_n(q)$ & multiplikative Ordnung von $q$ modulo $n$
  (\thref{def:multiplikative_ordnung_mod})\\
$M_q(l\bmod m)$ & $:=\{l\,q^i\bmod m:\ i\in \N\}$ 
  (\thref{def:nebenklassen_mod_m})\\
$R_q(m)$ & vollständiges Repräsentantensystem von $M_q(1\bmod n)$
  (\thref{def:nebenklassen_mod_m})\\
$r_q(l\bmod m)$ & Länge der Bahn von $M_q(l\bmod m)$
  (\thref{def:nebenklassen_mod_m})\\
$\cl_r(n)$ & Abschluss von $r$ in $n$
  (\thref{def:closure})\\
$\Delta_q(m,d)$ & $:= \tfrac{\varphi(d) \ord_m(q)}{\ord_{md}(q)}$
  (\thref{satz:zerfall_f_x_s})\\
\end{longtable}

\subsubsection{Endliche Körper}
\begin{longtable}[h]{>{\raggedright}p{4cm}@{\qquad}p{10cm}}
$\F_q$ & Endlicher Körper mit $q$ Elementen\\
$\bar F$ & algebraischer Abschluss eines Körpers $F$\\
$\Tr_{E\mid F}$ & Spurfunktion von $E$ nach $F$\\
$\Nm_{E\mid F}$ & Normfunktion von $E$ nach $F$\\
$\charak(\F_q)$ & Charakteristik von $\F_q$\\
$\sigma$ & Frobenius-Endomorphismus
  (\thref{satz:frob_fix})\\
$\Gal(E\mid F)$ & Galoisgruppe einer Körpererweiterung $E$ über $F$\\
$K\kl n$ & $n$-ter Kreisteilungskörper
  (\thref{def:kreisteilungskorper})\\
$U\kl n$ & Menge der $n$-ten Einheitswurzeln
  (\thref{def:kreisteilungskorper})\\
$C\kl n$ & Menge der primitiven $n$-ten Einheitswurzeln
  (\thref{def:primitive_einheitswurzeln})\\
$\phi_q(f)$ & Polynomversion der Eulerschen Phifunktion
  (\thref{def:polynom_phi})\\
$\C_{k,t}$ & verallgemeinerter Kreisteilungsmodul
  (\thref{def:verallgemeinerter_kreisteilungsmodul})\\
$\tau(q,k)$ & $\tau$-Teiler
  (\thref{def:tau})\\
\end{longtable}


\subsubsection{Spezielle Polynome}
\begin{longtable}[h]{>{\raggedright}p{4cm}@{\qquad}p{10cm}}
$\Phi_n(x)$ & $n$-tes Kreisteilungspolynom 
  (\thref{def:kreisteilungspolynom})\\
$\Phi_{k,t}(x)$ & verallgemeinertes Kreisteilungspolynom
  (\thref{def:verallgemeinertes_kreisteilungspolynom})\\
$D_n(x,a)$ & Dickson-Polynom erster Art
  (\thref{def:dickson})\\
$E_n(x,a)$ & Dickson-Polynom zweiter Art
  (\thref{def:dickson})\\
$f^\ast(x)$ & reziprokes Polynom von $f(x)$
  (\thref{def:reziprokes_polynom})\\
\end{longtable}

\subsubsection{Spezielle Mengen}
\begin{longtable}[h]{>{\raggedright}p{4cm}@{\qquad}p{10cm}}
$\S_q$ & Menge der Grade stark regulärer Erweiterungen über $\F_q$
  (\thref{def:stark_regular})\\
$\cal N(q,n)$, $\CN(q,n)$, $\PN(q,n)$, $\PCN(q,n)$ &
  Anzahl normaler, vollständig normaler, primitiv normaler, 
  primitiv vollständig normaler Elemente von $\F_{q^n}$ über $\F_q$
  (\thref{def:anzahlen})\\
$\G$ & Menge der $n$, für die für alle Primzahlpotenzen $q$
  ein primitiv vollständig normales Element in $\F_{q^n}$ über $\F_q$ existiert\\
\end{longtable}

\renewcommand{\arraystretch}{1}
