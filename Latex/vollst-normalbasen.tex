\chapter{Vollständige Normalbasen}

In den vorherigen Kapiteln haben wir einige Resultate zu Normalbasen kennen
gelernt. Es stellt sich jedoch ganz natürlich die Frage, ob dieser Begriff
nicht erweitert werden kann: Für eine Körpererweiterung $E$ über $F$ existieren
im Allgemeinen Zwischenkörper $F \mid K \mid E$, so wollen wir untersuchen,
ob ein Element $w \in E$, welches normal über $F$ ist, auch normal über allen
Zwischenkörpern bleibt. Diese Eigenschaft verdient einen eigenen Namen:

\begin{definition}[vollständig normal]
  Sei $E \mid F$ eine Körpererweiterung endlicher Körper $E$ über $F$.
  $w\in E$ heißt \emph{vollständig normal}, falls $w$ normal über jedem
  Zwischenkörper $E \mid K\mid F$ ist.
\end{definition}

Ungeklärt ist in diesem Moment, ob überhaupt vollständig normale Elemente
existieren. Greifen wir auf das zurück, was wir über normale Elemente wissen,
so treffen wir folgende Definition und erkennen 

\begin{definition}[einfach]
  \label{def:einfach}
  Eine Körpererweiterung $E \mid F$ endlicher Körper $F$ und $E$ heißt 
  \emph{einfach}, falls jedes normale Element von $E$ über $F$ bereits
  vollständig normal ist.
\end{definition}


\begin{satz}
  \label{satz:einfache_erweiterungen}
  Sei $\F_{q^m} \mid \F_q$ eine Erweiterung endlicher Körper. Dann ist
  äquivalent:
  \begin{enumerate}
    \item $\F_{q^m} \mid \F_q$ ist einfach.
    \item Für jeden Primteiler $r \mid m$ ist jedes normale Element in 
      $\F_{q^m}$ über $\F_q$ auch normal in $\F_{q^m}$ über $\F_{q^r}$.
    \item Für jeden Primteiler $r \mid m$ teilt $r$ nicht
      $\ord_{(\frac m r)'}(q)$.
  \end{enumerate}
\end{satz}

\begin{definition}[verallgemeinertes Kreisteilungspolynom]
  Sei $F$ ein endlicher Körper. Seien $k,t\geq 1$ natürliche Zahlen und 
  $k$ teilerfremd zu $\charak F$, so heißt
  \[ \Phi_{k,t}(x) \speq{:=} \Phi_k(x^t) \ \in F[x]\]
  \emph{verallgemeinertes Kreisteilungspolynom}.
\end{definition}


\begin{definition}[verallgemeinerter Kreisteilungsmodul, Modulcharakter]
  \label{def:verallgemeinerter_kreisteilungsmodul}
  Sei $\Phi_{k,t}$ ein verallgemeinertes Kreisteilungspolynom über einem
  endlichen Körper $F$. Notiere ferner $\sigma: \bar F\to \bar F$ 
  den Frobenius von $F$,so heißt
  \[ \C_{k,t} \speq{:=} \{ w \in \bar F:\ \Phi_{k,t}(\sigma)(w) = 0 \}\]
  \emph{verallgemeinerter Kreisteilungsmodul}.

  Der \emph{Modulcharakter} von $\C_{k,t}$ ist $\frac{k\,t}{\nu(k)}$.
\end{definition}

\begin{definition}[vollständiger Erzeuger]
  \label{def:vollst_erzeuger}
  Sei $\C_{k,t}$ ein verallgemeinerter Kreisteilungsmodul über $\F_q$.
  $w \in \bar F$ heißt 
  \emph{vollständiger Erzeuger von $\C_{k,t}$}, falls
  $w$ ein Erzeuger von $\C_{k,t}$ als $\F_{q^d}[x]$-Modul 
  für alle Teiler $d$ des Modulcharakters $\frac{kt}{\nu(k)}$ ist.
\end{definition}

\begin{definition}[Zerlegung in verallgemeinerte Kreisteilungsmoduln]
  Sei $\Phi_{k,t}$ ein verallgemeinertes Kreisteilungspolynom über $F$.
  $\Delta \subseteq F[x]$ heißt eine 
  \emph{Zerlegung von $\Phi_{k,t}$ in verallgemeinerte Kreisteilungspolynome}, 
  falls $\Delta$ nur verallgemeinerte Kreisteilungspolynome enthält, diese
  paarweise teilerfremd sind und
  \[ \Phi_{k,t}(x) \speq= \prod_{\Psi \in \Delta} \Psi(x)\,. \]
  Definiere ferner
  \[ i(\Delta) \speq{:=} \{ (l,s) \in \N^2:\ 
    \Phi_{l,s} \in \Delta\}\,.\]
\end{definition}


\begin{definition}[verträgliche Zerlegung]
  \label{def:vertraeglich}
  Sei $\Delta$ eine Zerlegung von $\Phi_{k,t}$ in verallgemeinerte
  Kreisteilungspolynome über $F$.
  Dann heißt $\Delta$ 
  \emph{verträgliche Zerlegung} falls gilt: Sei für jedes 
  $(l,s) \in i(\Delta)$
  $w_{l,s} \in \bar F$ ein vollständiger Erzeuger von 
  $\C_{l,s}$ über $F$,
  so ist 
  \[ w = \sum_{(l,s) \in i(\Delta)} w_{l,s} \]
  ein vollständiger Erzeuger von $\C_{k,t}$ über $F$.
\end{definition}

\begin{satz}[Zerlegungssatz für verallgemeinerte Kreisteilungsmoduln]
  \label{satz:zerlegungssatz}
  Sei $\Phi_{k,t}$ ein verallgemeinertes Kreisteilungspolynom über einem
  endlichen Körper $\F_q$ mit Charakteristik $p$. Sei $r$ eine Primzahl
  mit

  \begin{itemize*}[itemjoin={\qquad}]
    \item $r \mid t$,
    \item $r \neq p$,
    \item $r \nmid k$.
  \end{itemize*}

  Dann ist 
  \[ \Delta_r \speq{:=} \{ \Phi_{k,\frac{t}{r}},\ \Phi_{kr, \frac{t}{r}}\}\]
  eine Zerlegung von $\Phi_{k,t}$ in verallgemeinerte Kreisteilungspolynome und
  diese ist verträglich genau dann, wenn
  \[ r^a \nmid \ord_{\nu(kt')}(q) \]
  mit $a = \max\{ b\in \N: r^b \mid t\}$.
\end{satz}



\begin{definition}[regulär]
  \label{def:regulaer}
  Ein verallgemeinerter Kreisteilungsmodul $\C_{k,t}$ 
  mit $\ggT(k,t)=1$ heißt \emph{regulär},
  falls $\ord_{\nu(k\,t')}(q)$ und $k\,t$ teilerfremd sind.
\end{definition}


\begin{definition}[ausfallend]
  \label{def:ausfallend}
  Sei $\C_{k, p^b}$ ein regulärer verallgemeinerter Kreisteilungsmodul 
  über $\F_q$ mit $\charak \F_q = p$. Schreibe $k = 2^c \cdot \bar k$ mit $\bar
  k$ ungerade. Dann heißt $\C_{k,p^b}$ \emph{ausfallend}, falls gilt:
  \begin{itemize}
    \item $q \equiv 3 \bmod 4$,
    \item $c \geq 3$ und 
    \item $\ord_{2^c}(q) = 2$.
  \end{itemize}
\end{definition}

\begin{definition}[$\tau$-Teiler]
  Sei $\C_{k,p^b}$ ein regulärer verallgemeinerter Kreisteilungsmodul über
  $\F_q$. Schreibe
  \[ \ord_k(q) \speq= \ord_{\nu(k)}(q) \ \prod_{r \in \pi(k)} r^{\alpha_r}\,,\]
  wobei $\pi(k)$ die Primteiler von $k$ bezeichnen.
  Dann heißt
  \[ \tau \speq{:=} \tau(q,k) \speq{:=} \prod_{r\in \pi(k)} 
    r^{\lfloor \frac{\alpha_r}{2}\rfloor}\]
  der \emph{$\tau$-Teiler von $\C_{k,p^b}$}.
\end{definition}

\begin{satz}[Über reguläre Erweiterungen]
  \label{satz:regulare_erweiterungen}
  Sei $\F_q$ ein endlicher Körper von Charakteristik $p$. 
  Seien $k$ eine positive ganze Zahl teilerfremd zu $q$ und 
  $\C_{k,p^b}$ ein regulärer verallgemeinerter Kreisteilungsmodul. Dann gilt:
  \begin{enumerate}
    \item Ist $\C_{k,p^b}$ nicht ausfallend, so ist $u \in \bar F$ genau dann
      ein vollständiger Erzeuger von $\C_{k,p^b}$, falls
      \[ \Ord_{q^\tau}(u) \speq= \Phi_{\frac k \tau,\, p^b} \,.\]
    \item Ist $\C_{k,p^b}$ ausfallend, so ist $u\in \bar F$ genau dann
      ein vollständiger Erzeuger von $\C_{k,p^b}$, falls
      \[ \Ord_{q^\tau}(u) \speq= \Phi_{\frac k \tau,\, p^b} \quad
        \text{und}\quad 
        \Ord_{q^{2\tau}}(u) \speq= \Phi_{\frac{k}{2\tau},\, p^b}\,.\]
  \end{enumerate}
\end{satz}
