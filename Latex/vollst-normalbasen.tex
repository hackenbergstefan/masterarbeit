\chapter{Vollständige Normalbasen}
\label{chap:vollst_normalbasen}

In den vorherigen Kapiteln haben wir einige Resultate zu Normalbasen 
kennengelernt. Es stellt sich jedoch ganz natürlich die Frage, ob dieser Begriff
nicht erweitert werden kann: Für eine Körpererweiterung $E$ über $F$ existieren
im Allgemeinen Zwischenkörper $E \mid K \mid F$ und wir wollen untersuchen,
ob ein Element $w \in E$, welches normal über $F$ ist, auch normal über allen
Zwischenkörpern bleibt. Solche Elemente wollen wir \emph{vollständig normal}
nennen. Ähnlich zu normalen Elementen kann man mit Hilfe der Modulstrukturen
von Körpererweiterungen eine Theorie aufbauen, die es erlaubt, vollständig
normale Elemente in \emph{vollständige Erzeuger} (\thref{def:vollst_erzeuger}) zu
zerlegen, wie man es für normale Elemente aus \thref{kor:summe_erzeuger_normal}
kennt. Hachenberger konnte in \autocite{hachenberger1997finite} ausarbeiten, wie
die simultan auftretenden Modulstrukturen zu behandeln sind. Wir wollen hier
die zentralen Resultate lediglich ohne Beweise zitieren. Eine ebenfalls gute
Übersicht dazu findet man in \autocite{hachenberger2013handbook}.
Wir beginnen bei der grundlegendsten Definition, die sich ja bereits in der
Kapitelüberschrift wiederfinden lässt.

\begin{definition}[vollständig normal]
  \label{def:vollst_normal}
  Sei $E \mid F$ eine Körpererweiterung endlicher Körper $E$ über $F$.
  $w\in E$ heißt \emph{vollständig normal}, falls $w$ normal über jedem
  Zwischenkörper $E \mid K\mid F$ ist.

  Die Begriffe \emph{vollständige Normalbasis}, \emph{vollständig normales
  Polynom} sind analog zu \thref{def:normal} zu setzen.
\end{definition}

Ein besonderer Fall würde natürlich auftreten, wenn bereits alle normalen
Elemente einer Körpererweiterung auch vollständig normal wären. Man kann zeigen,
dass dies in der Tat unter gewissen Bedingungen auftreten kann und verleiht
dieser Konstellation den Namen \emph{einfach}%
\footnote{Im englischsprachigen Raum wird
dieser Begriff \emph{completely basic} genannt.}.

\begin{definition}[einfach]
  \label{def:einfach}
  Eine Körpererweiterung $E \mid F$ endlicher Körper $F$ und $E$ heißt 
  \emph{einfach}, falls jedes normale Element von $E$ über $F$ bereits
  vollständig normal ist.
\end{definition}


\begin{satz}
  \label{satz:einfache_erweiterungen}
  Sei $\F_{q^m} \mid \F_q$ eine Erweiterung endlicher Körper
  von Charakteristik $p$. Dann sind äquivalent:
  \begin{enumerate}
    \item $\F_{q^m} \mid \F_q$ ist einfach.
    \item Für jeden Primteiler $r \mid m$ ist jedes normale Element in 
      $\F_{q^m}$ über $\F_q$ auch normal in $\F_{q^m}$ über $\F_{q^r}$.
    \item Für jeden Primteiler $r \mid m$ teilt $r$ nicht
      $\ord_{r'}(q)$.
  \end{enumerate}
  Dabei ist $\tfrac m r = r'\,p^b$ mit $\ggT(r',p) = 1$.
\end{satz}
\begin{proof}
  \autocite[Corollary 15.8]{hachenberger1997finite}.
\end{proof}

\begin{kor}
  \label{kor:einfache_erweiterungen}
  Insbesondere ist $\F_{q^m}$ über $\F_q$ einfach, falls
  \begin{enumerate}
    \item $m = r$ oder $m=r^2$ für eine Primzahl $r$.
    \item $m' \mid (q-1)$, wobei $m=m'p^b$ mit $\ggT(m',p) = 1$.
    \item $m = p^b$ für $b\geq 0$.
  \end{enumerate}
\end{kor}
\begin{proof}
  (1) ist klar. Für (2) sei auf \autocite[Theorem 15.9]{hachenberger1997finite}
  verwiesen und (3) ist eine Folgerung aus (2).
\end{proof}

Im Abschnitt über normale Elemente konnten wir herausarbeiten, dass die
Zerlegung von $x^n-1$ in Kreisteilungspolynome ein guter Startpunkt ist, um
normale Elemente zu konstruieren und die Modulstrukturen zu beschreiben. Jedoch
zeigt es sich, dass im Allgemeinen ein Element, dessen $q$-Ordnung einem
Kreisteilungspolynom entspricht, eine $q^d$-Ordnung für Teiler $d$ von
$n$ besitzt, die kein reines Kreisteilungspolynom mehr ist. 
Also muss eine passende Klasse von Polynomen gefunden werden, um die
verschiedenen simultan auftauchenden $q^d$-Ordnungen zu erfassen: 
\emph{verallgemeinerte Kreisteilungspolynome}.

\begin{definition}[verallgemeinertes Kreisteilungspolynom]
  \label{def:verallgemeinertes_kreisteilungspolynom}
  Sei $F$ ein endlicher Körper. Seien $k,t\geq 1$ natürliche Zahlen und 
  $k$ teilerfremd zur Charakteristik von $F$, so heißt
  \[ \Phi_{k,t}(x) \speq{:=} \Phi_k(x^t) \ \in F[x]\]
  \emph{verallgemeinertes Kreisteilungspolynom}.
\end{definition}


\begin{definition}[verallgemeinerter Kreisteilungsmodul, Modulcharakter]
  \label{def:verallgemeinerter_kreisteilungsmodul}
  Sei $\Phi_{k,t}$ ein verallgemeinertes Kreisteilungspolynom über einem
  endlichen Körper $F$. Notiere ferner $\sigma: \bar F\to \bar F$ 
  den Frobenius von $F$, so heißt
  \[ \C_{k,t} \speq{:=} \{ w \in \bar F:\ \Phi_{k,t}(\sigma)(w) = 0 \}\]
  \emph{verallgemeinerter Kreisteilungsmodul}.

  Der \emph{Modulcharakter} von $\C_{k,t}$ ist $\frac{k\,t}{\nu(k)}$.
\end{definition}

\begin{definition}[vollständiger Erzeuger]
  \label{def:vollst_erzeuger}
  Sei $\C_{k,t}$ ein verallgemeinerter Kreisteilungsmodul über $\F_q$.
  $w \in \bar \F_q$ heißt 
  \emph{vollständiger Erzeuger von $\C_{k,t}$}, falls
  $w$ ein Erzeuger von $\C_{k,t}$ als $\F_{q^d}[x]$-Modul 
  für alle Teiler $d$ des Modulcharakters $\frac{kt}{\nu(k)}$ ist.
\end{definition}

\begin{definition}[Zerlegung in verallgemeinerte Kreisteilungsmoduln]
  Sei $\Phi_{k,t}$ ein verallgemeinertes Kreisteilungspolynom über $\F_q$.
  $\Delta \subseteq \F_q[x]$ heißt eine 
  \emph{Zerlegung von $\Phi_{k,t}$ in verallgemeinerte Kreisteilungspolynome
  über $\F_q$}, 
  falls $\Delta$ nur verallgemeinerte Kreisteilungspolynome enthält, diese
  paarweise teilerfremd sind und
  \[ \Phi_{k,t}(x) \speq= \prod_{\Psi \in \Delta} \Psi(x)\,. \]
  Definiere ferner
  \[ i(\Delta) \speq{:=} \{ (l,s) \in \N^2:\ 
    \Phi_{l,s} \in \Delta\}\,.\]
\end{definition}


\begin{definition}[verträgliche Zerlegung]
  \label{def:vertraeglich}
  Sei $\Delta$ eine Zerlegung von $\Phi_{k,t}$ in verallgemeinerte
  Kreisteilungspolynome über $\F_q$.
  Dann heißt $\Delta$ 
  \emph{verträgliche Zerlegung}, falls gilt: Für jedes 
  $(l,s) \in i(\Delta)$ sei
  $w_{l,s} \in \bar \F_q$ ein vollständiger Erzeuger von 
  $\C_{l,s}$ über $\F_q$,
  so ist 
  \[ w = \sum_{(l,s) \in i(\Delta)} w_{l,s} \]
  ein vollständiger Erzeuger von $\C_{k,t}$ über $\F_q$.
\end{definition}


Nun können wir einen zentralen Satz formulieren, der eine passende
Zerlegung eines erweiterten Kreisteilungspolynoms herstellt, sodass sich ein
vollständiger Erzeuger als Summe von vollständigen Erzeugern der entsprechenden
Teilmoduln zusammensetzen lässt. Man bemerke an dieser Stelle, dass das Problem
der vollständigen Erzeuger (und damit der vollständigen Normalbasen) ungleich
schwerer ist, als das der normalen Elemente, da sich dort Elemente 
mit teilerfremden $q$-Ordnungen \emph{immer} zu einem Element summieren, dessen
$q$-Ordnung gerade das Produkt der $q$-Ordnungen ist 
(vgl. \thref{satz:zerlegungssatz_zykl_vektorraume}); mit anderen Worten 
also die Summe von Erzeugern bis auf das Nullelement disjunkter 
Teilmoduln stets wieder einen Erzeuger
liefert. Dies ist bei vollständigen Erzeugern nur bedingt gegeben, wie
nachstehender Zerlegungssatz beschreibt.

\begin{satz}[Zerlegungssatz für verallgemeinerte Kreisteilungsmoduln]
  \label{satz:zerlegungssatz}
  Sei $\Phi_{k,t}$ ein verallgemeinertes Kreisteilungspolynom über einem
  endlichen Körper $\F_q$ mit Charakteristik $p$. Sei $r$ eine Primzahl
  mit

  \begin{itemize*}[itemjoin={\qquad}]
    \item $r \mid t$,
    \item $r \neq p$,
    \item $r \nmid k$.
  \end{itemize*}

  Dann ist 
  \[ \Delta_r \speq{:=} \{ \Phi_{k,\frac{t}{r}},\ \Phi_{kr, \frac{t}{r}}\}\]
  eine Zerlegung von $\Phi_{k,t}$ in verallgemeinerte Kreisteilungspolynome und
  diese ist verträglich genau dann, wenn
  \[ r^a \nmid \ord_{\nu(kt')}(q) \]
  mit $a = \max\{ b\in \N: r^b \mid t\}$ und 
  $t=t'p^b$ für $\ggT(t',p)=1$.
\end{satz}
\begin{proof}
  \autocite[Decomposition Theorem, Section 19]{hachenberger1997finite}.
\end{proof}


Sicherlich kann man sich nun fragen, in welchen Fällen die kanonische Zerlegung
eines erweiterten Kreisteilungspolynoms in Kreisteilungspolynome noch
verträglich ist. Nach \autocite[Theorem 19.10]{hachenberger1997finite} 
ist die kanonische Zerlegung von $\Phi_{k,t}(x)^\pi$ für $\pi$ eine Potenz der
Charakteristik verträglich über 
$\F_q$, falls $\ord_{\nu(kt')}(q)$ und $t'$ teilerfremd sind,
wobei wieder $t = t'p^b$ mit $\ggT(t',p)=1$ für $p=\charak\F_q$. Dies motiviert
dazu, dieser Klasse von Kreisteilungsmoduln einen eigenen Namen zu geben:

\begin{definition}[regulär]
  \label{def:regulaer}
  Ein verallgemeinerter Kreisteilungsmodul $\C_{k,t}$ 
  mit $\ggT(k,t)=1$ heißt \emph{regulär} über 
  einem endlichen Körper $\F_q$ der Charakteristik $p$,
  falls $\ord_{\nu(k\,t')}(q)$ und $k\,t$ teilerfremd sind für
  $t=t'p^b$ mit $\ggT(t',p)=1$.

  Eine Körpererweiterung $\F_{q^m} \mid \F_q$ heißt \emph{regulär}, falls
  $\C_{1,m}$ regulär ist.
\end{definition}


\begin{definition}[ausfallend]
  \label{def:ausfallend}
  Sei $\C_{k, p^b}$ ein regulärer verallgemeinerter Kreisteilungsmodul 
  über $\F_q$ mit $\charak \F_q = p$. Schreibe $k = 2^c \cdot \bar k$ mit $\bar
  k$ ungerade. Dann heißt $\C_{k,p^b}$ \emph{ausfallend}, falls gilt:
  \begin{itemize}
    \item $q \equiv 3 \bmod 4$,
    \item $c \geq 3$ und 
    \item $\ord_{2^c}(q) = 2$.
  \end{itemize}
\end{definition}


Hachenberger war es nun möglich, für reguläre Kreisteilungsmoduln zu beweisen,
dass alle auftretenden Zwischenkörper, deren Betrachtung bei der Suche nach
vollständigen Erzeugern notwendig ist, von einem einzigen 
Zwischenkörper (oder zwei Zwischenkörpern) dominiert werden. Das bedeutet, dass
ein Element eines regulären Kreisteilungsmoduls maximal zwei bestimmte
$q^\bullet$-Ordnungen besitzen muss, um bereits den Kreisteilungsmodul
vollständig zu erzeugen. Die
geforderten $q^\bullet$-Ordnungen werden durch nachstehende Definition gegeben
und wir schließen dieses Kapitel mit der Angabe dieses wahrlich beachtlichen
Resultats.

\begin{definition}[$\tau$-Teiler]
  \label{def:tau}
  Sei $\C_{k,p^b}$ ein regulärer verallgemeinerter Kreisteilungsmodul über
  einem endlichen Körper $\F_q$ von Charakteristik $p$. Schreibe
  \[ \ord_k(q) \speq= \ord_{\nu(k)}(q) \ \prod_{r \in \pi(k)} r^{\alpha_r}\,,\]
  wobei $\pi(k)$ die Menge der Primteiler von $k$ bezeichne.
  Dann heißt
  \[ \tau \speq{:=} \tau(q,k) \speq{:=} \prod_{r\in \pi(k)} 
    r^{\lfloor \frac{\alpha_r}{2}\rfloor}\]
  der \emph{$\tau$-Teiler von $\C_{k,p^b}$}.
\end{definition}

\begin{satz}[über reguläre Erweiterungen]
  \label{satz:regulare_erweiterungen}
  Sei $\F_q$ ein endlicher Körper von Charakteristik $p$. 
  Seien $k$ eine positive ganze Zahl teilerfremd zu $q$ und 
  $\C_{k,p^b}$ ein regulärer verallgemeinerter Kreisteilungsmodul. Dann gilt:
  \begin{enumerate}
    \item Ist $\C_{k,p^b}$ nicht ausfallend, so ist $u \in \bar \F_q$ genau dann
      ein vollständiger Erzeuger von $\C_{k,p^b}$, falls
      \[ \Ord_{q^\tau}(u) \speq= \Phi_{\frac k \tau,\, p^b} \,.\]
    \item Ist $\C_{k,p^b}$ ausfallend, so ist $u\in \bar \F_q$ genau dann
      ein vollständiger Erzeuger von $\C_{k,p^b}$, falls
      \[ \Ord_{q^\tau}(u) \speq= \Phi_{\frac k \tau,\, p^b} \quad
        \text{und}\quad 
        \Ord_{q^{2\tau}}(u) \speq= \Phi_{\frac{k}{2\tau},\, p^b}\,.\]
  \end{enumerate}
\end{satz}
\begin{proof}
  \autocite[Theorem 20.3]{hachenberger1997finite}.
\end{proof}
