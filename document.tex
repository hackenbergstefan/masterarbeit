%% Basierend auf einer TeXnicCenter-Vorlage von Mark Müller
%%%%%%%%%%%%%%%%%%%%%%%%%%%%%%%%%%%%%%%%%%%%%%%%%%%%%%%%%%%%%%%%%%%%%%%

% Wählen Sie die Optionen aus, indem Sie % vor der Option entfernen  
% Dokumentation des KOMA-Script-Packets: scrguide

%%%%%%%%%%%%%%%%%%%%%%%%%%%%%%%%%%%%%%%%%%%%%%%%%%%%%%%%%%%%%%%%%%%%%%%
%% Optionen zum Layout des Artikels                                  %%
%%%%%%%%%%%%%%%%%%%%%%%%%%%%%%%%%%%%%%%%%%%%%%%%%%%%%%%%%%%%%%%%%%%%%%%
\documentclass[%
%a5paper,             % alle weiteren Papierformat einstellbar
%landscape,           % Querformat
%10pt,                % Schriftgröße (12pt, 11pt (Standard))
%BCOR1cm,             % Bindekorrektur, bspw. 1 cm
%DIVcalc,             % führt die Satzspiegelberechnung neu aus
%                       s. scrguide 2.4
%twoside,             % Doppelseiten
%twocolumn,           % zweispaltiger Satz
halfparskip*,       % Absatzformatierung s. scrguide 3.1
%headsepline,         % Trennline zum Seitenkopf  
%footsepline,         % Trennline zum Seitenfuß
titlepage,            % Titelei auf eigener Seite
%normalheadings,      % Überschriften etwas kleiner (smallheadings)
%idxtotoc,            % Index im Inhaltsverzeichnis
%liststotoc,          % Abb.- und Tab.verzeichnis im Inhalt
bibtotoc,           % Literaturverzeichnis im Inhalt
%abstracton,          % Überschrift über der Zusammenfassung an 
%leqno,               % Nummerierung von Gleichungen links
%fleqn,               % Ausgabe von Gleichungen linksbündig
%draft                % überlangen Zeilen in Ausgabe gekennzeichnet
DIV = 15,
headsepline,
openany,
BCOR=0.8cm,
pointlessnumbers,        %keine Punkte nach Überschriften
chapterprefix=true
]
{scrbook}



%% Deutsche Anpassungen %%%%%%%%%%%%%%%%%%%%%%%%%%%%%%%%%%%%%

\usepackage[ngerman]{babel}
\usepackage[T1]{fontenc}
\usepackage[utf8]{inputenc}


\usepackage{lmodern} %Type1-Schriftart für nicht-englische Texte

\usepackage{amsmath,amssymb,MnSymbol}


\usepackage{enumitem}
\setlist[enumerate]{label=(\arabic*)}


%% Packages für Grafiken & Abbildungen %%%%%%%%%%%%%%%%%%%%%%
\usepackage{graphicx} %%Zum Laden von Grafiken
%\usepackage{subfig} %%Teilabbildungen in einer Abbildung
\usepackage{calc}

\usepackage{tikz}
\usetikzlibrary{calc}
\usepackage{tikzpagenodes}
\usepackage{tikz-cd} %%PSTricks - nicht verwendbar mit pdfLaTeX
\usepackage[colorlinks=false, pdfborder={0 0 0}]{hyperref}
\usepackage[nameinlink,german]{cleveref}

\usepackage{listings}
\usepackage[automark]{scrpage2} % Headline styles 
\usepackage[square,numbers]{natbib}


%% Listings setup %%%%%%%%%%%%%%%%
\lstset{
  mathescape = true,
  basicstyle = \small\normalfont\sffamily,
  frame = tb,
  framexleftmargin = 15pt,
% numbers = left,
  numberstyle = \tiny,
% numbersep = 5pt,
  breaklines = true,
  xleftmargin = 0.1\linewidth,
  xrightmargin = 0.1\linewidth,
  escapeinside = {(*}{*)},
  tabsize=3,
  morekeywords={if, and, or, is, then, else, endif, while, endwhile, for, from,
  to, do, endfor, Input, Output, Algorithmus, return},
  morecomment=[l]{//},
  columns=flexible
}

%%%%%%%%%%%%%%%%%%%%%%%%%%%%%%%%%%%%%%%%%%%%%%%%%%%%%%%%%%%%%%%%%%%%%%%%%%%%%%%
%% Style Anpassungen
%%%%%%%%%%%%%%%%%%%%%%%%%%%%%%%%%%%%%%%%%%%%%%%%%%%%%%%%%%%%%%%%%%%%%%%%%%%%%%%
%\usepackage[sc]{mathpazo}
%\renewcommand{\sfdefault}{fav}
%\setkomafont{disposition}{\sffamily}
%%%%%%%%%%%%%%%%%%%%%%%%%%%%%%%%%%%%%%%%%%%%%%%%%%%%%%%%%%%%%%%%%%%%%%%%%%%%%%%

\usepackage{lipsum}


\usepackage[framemethod=tikz]{mdframed}

%% Theorems %%%%%%%%%%%%%%%%%%%%%%
\usepackage{framed}
\usepackage[thref, hyperref, thmmarks, amsmath, framed]{ntheorem}

%\theoremseparator{.}
%\theoremheaderfont{\normalfont\sffamily\bfseries}
%\newtheorem*{beh}{Behauptung}
%\theorembodyfont{\normalfont}

\theoremseparator{.}

\theoremstyle{plain}
\theoremheaderfont{\normalfont\sffamily\bfseries}
\theorembodyfont{\normalfont\itshape}      
%\newcommand{\thmbox}[1]{
%  \tikzstyle{thmnode}=[inner sep=0pt, outer sep=0pt]
%  \begin{tikzpicture}
%    \node[thmnode] (thm) {#1}; 
%    \path[draw] (thm.north west) -- (thm.south west);
%  \end{tikzpicture}}
%\newcommand{\thmbox}[1]{\def\FrameCommand{\vrule width 1pt \hspace{10pt}}
%  \MakeFramed{\advance\hsize-\width \FrameRestore} #1\endMakeFramed}
\renewcommand*\FrameCommand{{%
  \hspace*{-10pt} \vrule width 0.5pt\hspace{1pt}\vrule width 0.5pt 
  \hspace{3pt}\FrameRestore}}
\newframedtheorem{satz}{Satz}[chapter]
\newtheorem{lemma}[satz]{Lemma}
\newtheorem{kor}[satz]{Korollar}
\newtheorem{prop}[satz]{Proposition}
\newtheorem{algorithmus}[satz]{Algorithmus}

%Custom theorems
\makeatletter
\newenvironment{plainthm}[1]{\let\plthm\@undefined
\newtheorem{plthm}[satz]{#1} \begin{plthm}}{\end{plthm}}
\makeatother

\theoremstyle{plain}
\theoremheaderfont{\normalfont\sffamily\itshape}
\theorembodyfont{\normalfont}
\newtheorem{bemerkung}[satz]{Bemerkung}
\newtheorem{beispiel}[satz]{Beispiel}
                                     
\theoremstyle{plain}
\theoremheaderfont{\normalfont\sffamily\bfseries}
\theorembodyfont{\normalfont}
\newframedtheorem{definition}[satz]{Definition}
\newtheorem{deflemma}[satz]{Definition/Lemma}

\newtheorem{notation}[satz]{Notation}

\theoremstyle{nonumberplain}
\theoremindent0pt
\theoremheaderfont{\sffamily\itshape}
\theorembodyfont{\normalfont}
\theoremsymbol{\ensuremath{_\square}}
\newtheorem{proof}{Beweis}
\qedsymbol{\ensuremath{_\square}}


%% Autoref Names %%%%%%%%%%%%%%%%%
\crefname{lemma}{Lemma}{Lemmas}
\crefname{equation}{Gleichung}{Gleichungen}
\crefname{definition}{Definition}{Definitionen}
\crefname{algorithmus}{Algorithmus}{Algorithmen}
\crefname{kor}{Korollar}{Korollare}
\crefname{satz}{Satz}{Sätze}

%% Amsmath options %%%%%%%%%%%%%%%%%
\numberwithin{equation}{chapter}
\allowdisplaybreaks

%% Pagestyle %%%%%%%%%%%%%%%%%%%%%%%%%%
\addtokomafont{pagenumber}{\sffamily}
\addtokomafont{pagehead}{\sffamily\upshape}
\usepackage{scrpage2}
\pagestyle{scrheadings}
\setheadsepline{0pt}
\chead{\headmark}
\ohead{\tikz[remember picture]{\node[outer sep=5pt,inner sep=0pt]
  (a) {\sffamily\pagemark};}%
  \tikz[remember picture, overlay]{
    \ifthenelse{\isodd{\thepage}}{
      \draw[line width=1pt]
        ($(a.north west)+(-5pt,0)$) |- ($(a.south east)+(0,-5pt)$);
      \draw ($(a.south west)+(-5pt,0)$)  -- (a.south west -| current page text area.north
      west);
    }{
      \draw[line width=1pt]
        ($(a.north east)+(5pt,0)$) |- ($(a.south west)+(3pt,-5pt)$);
      \draw ($(a.south east)+(5pt,0)$)-- (a.south east -| current page text area.north east);
    }
  }
}
\ofoot{}


%% Makros %%%%%%%%%%%%%%%%%%%%%%
\newcommand{\A}{\ensuremath \mathbb{A}}
\newcommand{\R}{\ensuremath \mathbb{R}}
\newcommand{\N}{\ensuremath \mathbb{N}}
\newcommand{\Q}{\ensuremath \mathbb{Q}}
\newcommand{\Z}{\ensuremath \mathbb{Z}}
\newcommand{\C}{\ensuremath \mathbb{C}}
\newcommand{\F}{\ensuremath \mathbb{F}}
\newcommand{\K}{\ensuremath \mathbb{K}}
\renewcommand{\P}{\ensuremath \mathbb{P}}
\newcommand{\Kb}{\ensuremath \overline K}
\renewcommand{\O}{\ensuremath \mathcal{O}}
\renewcommand{\L}{\ensuremath \mathcal{L}}
\renewcommand{\l}{\ensuremath \ell}
\newcommand{\m}{\ensuremath \mathfrak{m}}
\newcommand{\speq}[1]{\ #1\ }
\newcommand{\const}{\ensuremath \mathrm{const}}
\newcommand{\divp}[1]{\ensuremath [#1]} %Divisorpunkt
\newcommand{\probn}[1]{{\sffamily #1}} %Problem Name

\let\div\undefined
\DeclareMathOperator{\charak}{char}
\DeclareMathOperator{\div}{div}
\DeclareMathOperator{\ord}{ord}
\DeclareMathOperator{\summ}{sum}
\DeclareMathOperator{\comp}{\circ}
\DeclareMathOperator{\ggT}{ggT}
\DeclareMathOperator{\supp}{supp}
\DeclareMathOperator{\Div0}{Div^0}
\DeclareMathOperator{\Divv}{Div}
\DeclareMathOperator{\Pic0}{Pic^0}
\DeclareMathOperator{\Gal}{Gal}
\DeclareMathOperator{\End}{End}
\DeclareMathOperator{\Ord}{Ord}
\DeclareMathOperator{\im}{im}
\DeclareMathOperator{\id}{id}


\newcommand{\funcdef}[1]{%
  \begin{array}[t]{>{\displaystyle}r>{\displaystyle}c>{\displaystyle}l}%
  #1\end{array}}



%% Others  %%%%%%%%%%%%%%%%%%%%%%%
\newcommand{\?}{{\huge \color{red} ?}}
\newcommand{\TODO}{{\sffamily\bfseries\large \color{red} TODO}}

\newcommand{\overbox}[2]{\ensuremath\begin{array}[b]{c}%
\makebox[0pt]{\fbox{\scriptsize#2}}\\[-2pt]\text{\small$\downarrow$}\\[-3pt]%
{\displaystyle#1}\end{array}}%

\let\marginparold\marginpar
\renewcommand{\marginpar}[1]{\marginparold{\scriptsize\sffamily #1}}

%% Bibliographiestil %%%%%%%%%%%%%%%%%%%%%%%%%%%%%%%%%%%%%%%%%%%%%%%%%%




\begin{document}

%% Trennungen %%%%%%%%%%%%%%%%%%%%%%%%%%%%%%%%%%%%%%%%%%%%%%%%%%%%%%%%%
%%%%%%%%%%%%%%%%%%%%%%%%%%%%%%%%%%%%%%%%%%%%%%%%%%%%%%%%%%%%%%%%%%%%%%%

\frontmatter


%%%%%%%%%%%%%%%%%%%%%%%%%%%%%%%%%%%%%%%%%%%%%%%%%%%%%%%%%%%%%%%%%%%%%%%
%% Ihr Artikel                                                       %%
%%%%%%%%%%%%%%%%%%%%%%%%%%%%%%%%%%%%%%%%%%%%%%%%%%%%%%%%%%%%%%%%%%%%%%%

%% eigene Titelseitengestaltung %%%%%%%%%%%%%%%%%%%%%%%%%%%%%%%%%%%%%%%    
\begin{titlepage}
\thispagestyle{empty}
\newcommand{\Rule}{\rule{\textwidth}{1mm}}
\begin{center}\sffamily\bfseries
\LARGE Masterarbeit
\vfill
\Rule\vspace{5mm}
\Huge
\vspace{1mm}\Rule
\vfill
\normalfont\sffamily\large vorgelegt von\par
\bfseries\LARGE Stefan Hackenberg
\vfill
\normalfont\sffamily\large am\\
\bfseries\Large Institut für Mathematik\\
\normalfont\sffamily\large der\\
\bfseries\Large Universität Augsburg
\vfill
\normalfont\sffamily\large betreut durch \\
\bfseries\Large Prof. Dr. Dirk Hachenberger\par
\vfill
\normalfont\sffamily\large abgegeben am \\
\bfseries\Large 
\end{center}
\end{titlepage}


%% Angaben zur Standardformatierung des Titels %%%%%%%%%%%%%%%%%%%%%%%%
%\titlehead{Titelkopf }
\subject{\large Masterarbeit}
\title{\Huge }
%\subtitle{Grundlegende Resultate zu elliptischen Kurven, Konstruktionen und
%Eigenschaften der Weil-Paarung, ein
%algorithmischer Überblick zu elliptischen Kurven in der Kryptographie und die
%kryptographische Anwendung der Weil-Paarung mit Hilfe des MOV-Algorithmus}
\author{\vspace*{2cm}\\\normalsize von\\\Large Stefan Hackenberg}
%\and{Der Name des Co-Autoren}
%\thanks{Fußnote}     % entspr. \footnote im Fließtext
%\date{}              % falls anderes, als das aktuelle gewünscht
\publishers{{\small geschrieben an der} \\ Universität Augsburg}

%% Widmungsseite %%%%%%%%%%%%%%%%%%%%%%%%%%%%%%%%%%%%%%%%%%%%%%%%%%%%%%
%\dedication{Für Sandra}

%\maketitle             % Titelei wird erzeugt

%% Zusammenfassung nach Titel, vor Inhaltsverzeichnis %%%%%%%%%%%%%%%%%
%\begin{abstract}
% Für eine kurze Zusammenfassung des folgenden Artikels.
% Für die Überschrift s. \documentclass[abstracton].
%\end{abstract}


\cleardoubleemptypage


%% Der Text %%%%%%%%%%%%%%%%%%%%%%%%%%%%%%%%%%%%%%%%%%%%%%%%%%%%%%%%%%%

%\include{intro.tex}
\mainmatter
\chapter{Moduln}

Nähern wir uns der Situation von Normalbasen in möglichst allgemeiner Form, so
beginnt die Reise bei der Betrachtung folgender Situation:
\begin{definition}[$(V,\tau)$]
  Sei $\K$ ein Körper und $V$ ein $\K$-Vektorraum und 
  $\tau \in \End_\K(V)$, so können wir $V$ als $\K[x]$-Modul auffassen:
  \[ f(x) \cdot v \speq{:=} f(\tau)(v)\]
  für alle $f(x) \in \K[x]$ und $v\in V$.
  Nenne das Paar $(V,\tau)$ \emph{$\K[x]$-Modul bzgl. $\tau$}.
\end{definition}

\begin{notation}
  Sei $\tau\in \End_\K(V)$.
  \begin{itemize}
  \item Es bezeichne $\mu_\tau$ das Minimalpolynom von 
    $\tau$, also das normierte Polynom kleinsten Grades $f(x)\in \K[x]$ mit 
    $f(\tau) = 0$.
  \item Ferner schreibe $\chi_\tau$ für das charakteristische Polynom von 
    $\tau$, also $\chi_\tau(x) := \det(x \id_V - \tau) \in \K[x]$.
  \end{itemize}
\end{notation}


\begin{bemerkung}
  Ist $\K  = F :=\F_q$ ein endlicher Körper, 
  $V = E := \F_{q^n}$ eine Körpererweiterung
  von Grad $n$ und 
  \[\tau = \sigma: \funcdef{E & \to & E\\
    v &\mapsto & v^q}\]
  der Frobenius von $E$, so ist
  \[ \mu_\tau(x) \ =\ \chi_\tau(x) \ =\ x^n - 1\,,\]
  denn: Es ist klar, dass $n = \deg \chi_\tau$ und da nach dem Satz von
  Cayley-Hamilton ist $\sigma$ Nullstelle von $\chi_\tau$. Daher teilt
  $\mu_\tau$ das charakteristische Polynom. Jedoch kennen wir das
  Minimalpolynom von $\tau$: Nach Dedekinds-Unabhängigkeitslemma ist 
  $\id_E,\sigma,\ldots,\sigma^{n-1}$ linear unabhänig über $E$, also insbesondere
  über $F$, und $\sigma^n = \id_E$.\marginpar{References!}
\end{bemerkung}


\begin{definition}[$\tau$-Ordnung, Teilmodul]
  Sei $(V,\tau)$ ein $\K[x]$-Modul. Zu jedem $v \in V$ betrachte den
  $\K[x]$-Modulhomomorphismus
  \[ \psi_w: \funcdef{\K[x] & \to & V \\
    f(x) & \mapsto & f(x)\cdot v }  \]
  Sei ferner $\dim V < \infty$.
  \begin{enumerate}
    \item Ist $\ker\psi_v = (g(x))$ für $g(x)\in \K[x]$ normiert, so heißt
      $g(x)$ \emph{$\tau$-Ordnung von $v$}\@. Ferner ist $g(x)$ eindeutig.
      Schreibe $\Ord_\tau(v) := g(x)$.
    \item $\K[\tau]\cdot v := \im{\psi_v}$ heißt der von \emph{$v$ erzeugte
      $\K[x]$-Teilmodul von $V$}.
  \end{enumerate}
\end{definition}
\marginpar{Eindeutigkeit!}


\begin{notation}
  Für $\K = \F_q$ einen endlichen Körper, $V = E \mid \F_q$ eine 
  Körpererweiterung und $\tau = \sigma$ den Frobenius-Endomorphismus schreibe
  \[ \Ord_q := \Ord_\tau \]
  und bezeichne $\Ord_q$ mit \emph{$q$-Ordnung}.
\end{notation}

\begin{lemma}
  \label{lemma:eigenschaften-tau-ordnung}
  Sei $(V,\tau)$ ein $\K[x]$-Modul. Ferner seien
  $u,v\in V$ mit $g(x) := \Ord_\tau(u)$, $h(x) := \Ord_\tau(v)$ und 
  $f(x) \in \K[x]$. Dann gilt
  \begin{enumerate}
    \item $\Ord_\tau(f(x)\cdot u) = \frac{g(x)}{\ggT(f(x),g(x))}$.
    \item $\Ord_\tau(u+v) = g(x)h(x)$, falls $\ggT(g,h) = 1$.
  \end{enumerate}
\end{lemma}
\begin{proof}
  \begin{enumerate}
    \item \TODO
    \item \TODO
  \end{enumerate}
\end{proof}

\begin{lemma}
  Sei $(V,\tau)$ ein $\K[x]$-Modul. Sei $v\in V$. Dann gilt:
  \[ \dim_\K( \K[x]\cdot v ) \speq= \deg( \Ord_\tau(v) )\,.\]
\end{lemma}
\begin{proof}
  Nach dem Homomorphiesatz gilt: \marginpar{References!}
  $ \im\psi_v \cong \K[x] \big/ \ker \psi_v$.
\end{proof}


\begin{definition}[zyklischer Modul]
  $(V,\tau)$ heißt \emph{zyklischer $\K[x]$-Modul bzgl. $w$}, falls es ein 
  $w\in \K$ gibt, sodass $K[\tau]\cdot w = V$.
\end{definition}


\begin{satz}
  Es gilt:
  \[ (V,\tau) \text{ ist ein zyklischer Modul} \quad\Leftrightarrow\quad
    \mu_\tau = \chi_\tau\]
\end{satz}
\begin{proof}
  Fassen wir zunächst ein paar einfache Tatsachen zusammen:
  Ist $u \in V$, so haben wir 
  \[ \dim(\K[x]\cdot v) = \deg( \Ord_\tau(v) ) \speq\leq 
    \deg\mu_\tau \speq\leq \deg \chi_\tau \]
  und 
  \[ \Ord_\tau (v) \speq\mid \mu_\tau \speq\mid \chi_\tau \,,\]
  wobei die erste Teilbarkeitsrelation per definitionem erfüllt ist und die
  zweite gerade der Satz von Cayley-Hamilton ist.
  Damit kommen wir zum direkten Beweis:
  \begin{description}
    \item["`$\Rightarrow$"'] Sei $V$ also zyklisch bzgl. $w$, so ist dies nach
      obigem äquivalent zu $\deg(\Ord_\tau(w)) = n$. Daraus folgt aber sofort
      $\mu_t = \chi_\tau$, da beide normiert sind.
    \item["`$\Leftarrow$"'] Zunächst sei behauptet, dass es stets ein 
      $w \in V$ gibt mit $\Ord_\tau(w) = \mu_\tau$. Sei dazu 
      $\mu_\tau(x) = \prod_{i=1}^r p_i(x)^{a_i}$ die Zerlegung in irreduzible
      Faktoren über $\K[x]$, so existieren $w_i \in V$ mit
      $\Ord_\tau(w_i) = p_i^{a_i}$. Andernfalls hätten wir einen Widerspruch 
      zum Minimalpolynom von $\tau$!
      Nach \autoref{lemma:eigenschaften-tau-ordnung} ist dann aber 
      $w := \sum_{i=1}^r w_i$ ein Element in $V$ mit $\tau$-Ordnung $\mu_\tau$.

      Ist dann also $\mu_\tau = \chi_\tau$, so hat obiges $w$ genau
      $\tau$-Ordnung $\chi_\tau$; erzeugt also $V$ als $\K[x]$-Modul.
  \end{description}
\end{proof}

Nun wollen wir spezielle Untermoduln von $V$ betrachten, welche uns guten
Aufschluss über die Struktur von $V$ geben können:

\begin{notation}
  Seien $(V,\tau)$ ein $\K[x]$-Modul und $g(x) \in \K[x]$.
  Definiere
  \[ V_g \speq{:=} \{ v \in V \mid g(x)\cdot v = 0 \}\,.\]
\end{notation}

Zunächst ist klar, dass $V_g \neq 0$ nur für $g$ Teiler von $\mu_\tau$ gelten
kann. Damit können wir folgende "`Rechenregeln"' formulieren:

\begin{lemma}
  Seien $g(x), h(x) \in \K[x]$ mit $g,h \mid \mu_\tau$. Dann gilt:
  \begin{enumerate}
    \item $V_g \cap V_h \speq= V_{\ggT(g,h)}$
    \item $V_g + V_h \speq= V_{\kgV(g,h)}$
  \end{enumerate}
\end{lemma}
\begin{proof}
  Per definitionem ist klar, dass für $v \in V_g$ gerade
  $\Ord_\tau(v) \mid g$. Also können wir $V_g$ auch wie folgt auffassen:
  \[ V_g \speq= \{ v\in V:\ \Ord_\tau(v) \mid g\} \,,\]
  Damit sind die Behautungen nach \cref{lemma:eigenschaften-tau-ordnung} klar,
  denn für $v \in V$ gilt:
  \[ v\in V_g \cap V_h \speq\Leftrightarrow 
    \Ord_\tau(v) \mid g \land \Ord_\tau(v) \mid h \speq\Leftrightarrow
    \Ord_\tau(v) \mid \ggT(g,h) \speq\Leftrightarrow v \in V_{\ggT(g,h)}\]
  und ebenso
  \[ v \in V_g + V_h \speq\Leftrightarrow 
    \Ord_\tau(v) \mid g \lor \Ord_\tau(v) \mid h \speq\Leftrightarrow
    \Ord_\tau(v) \mid \kgV(g,h) \speq\Leftrightarrow v \in V_{\kgV(g,h)}\,.\]
\end{proof}


\begin{satz}
  Sei $(V,\tau)$ ein zyklischer Modul mit $\dim(V) = n$. Sei ferner 
  $g(x)\in \K[x]$ normiert mit $g\mid \mu_\tau$. Dann gilt:
  \begin{enumerate}
    \item $V_g$ ist ein $\K[x]$-Teilmodul von $V$.
    \item Alle $\K[x]$-Teilmoduln von $V$ sind von dieser Form.
    \item $V_g$ ist zyklisch bzgl. $\tau$ mit Minimalpolynom $g(x)$.
      Ferner ist $\dim(V_g) = \deg(g)$.
    \item Die Erzeuger von $V_g$ sind genau die Elemente $v\in V$ mit 
      $\Ord_\tau(v) = g$.
  \end{enumerate}
\end{satz}
\begin{proof}
  \begin{enumerate}
    \item 
    \item
    \item
    \item
  \end{enumerate}
\end{proof}

\chapter{Normalbasen -- Ein Überblick}

Seien wieder $F := \F_q$ ein endlicher Körper von Charakteristik $p$ und 
$E := \F_{q^n} \mid F$ eine Körpererweiterung.
Wir wiederholen kurz die Definition einer \emph{Normalbasis}

\begin{definition}[normales Element, normales Polynom, Normalbasis]
  Sei $F$ ein Körper und $E \mid F$ eine endliche Galoiserweiterung von Grad
  $n$. Sei ferner $w\in E$ mit $F(w) = E$. $w$ heißt \emph{normal über $F$},
  falls
  \[ \{ \gamma(w) \mid \gamma \in G\}\]
  eine $F$-Basis von $E$ ist. 
  $\{ \gamma(w) \mid \gamma \in G\}$ heißt entsprechend \emph{Normalbasis} und
  $g(x) \in F[x]$ mit 
  \[ g(x) = \prod_{\gamma \in G}(x - \gamma(w))\]
  heißt \emph{normales Polynom}.
\end{definition}

Um effizient normale Elemente in $E\mid F$ zu finden, betrachten wir 
$(E,\sigma)$ als $F[x]$-Modul und nutzen die Aussagen aus
\autoref{chap:moduln}.

\begin{satz}
  \begin{enumerate}
    \item Die Erzeuger von $(E,\sigma)$ als $F[x]$-Modul sind genau die 
      normalen Elemente in $E\mid F$.
    \item Man hat eine Bijektion von Mengen
      \[ \{V_g :\ g(x) \in F[x] \text{ monisch mit } g(x) \mid x^n-1\}
        \overset{1-1}{\longleftrightarrow}
        \{F[x] \text{-Teilmoduln von }E\}\]
        wobei $V_g := \{v \in E : g(x)\cdot v = 0\} = \ker(g(\sigma))$.
    \item Jedes $V_g$ ist ein zyklischer Modul und es gilt
      \[u \text{ erzeugt } V_g \quad\Leftrightarrow \quad
        \Ord_q(u) = g(x).\]
        Insbesondere sind die Erzeuger von $V$ genau die Elemente $v \in V$ mit 
        $\Ord_\tau(v) = x^n-1$.
  \end{enumerate}
\end{satz}
\begin{proof}
  Alles schon in \TODO~gezeigt.
\end{proof}

Dies liefert uns die grundlegende Idee für das Auffinden von normalen
Elementen:
\begin{lemma}
  Sei $x^n-1 = \prod_{i=1}^s r_i(x)$ eine Zerlegung in paarweise teilerfremde
  Polynome, so gilt:
  \[ E \speq= \bigoplus_{i=1}^s V_{r_i} \,.\]
\end{lemma}
\begin{proof}
  \TODO.
\end{proof}

\begin{kor}
  Sei $x^n-1 = \prod_{i=1}^s r_i(x)$ eine Zerlegung in paarweise teilerfremde
  Polynome. Seien ferner $u_i \in V_{r_i}$ Elemente mit 
  $\Ord_q(u_i) = r_i(x)$ $\forall i=1,\ldots,s$. Dann ist
  \[ u \speq= u_1 + u_2 + \ldots + u_s\]
  normal in $E \mid F$.
\end{kor}
\begin{proof}
  Da obige Zerlegung von $x^n-1$ paarweise teilerfremd ist, folgt nach
  \cref{lemma:eigenschaften-tau-ordnung}
  $\Ord_q(u) = \prod_{i=1}^s \Ord_q(u_i) = \prod_{i=1}^s r_i(x) = x^n -1$.
\end{proof}

\begin{beispiel}
  
\end{beispiel}


An diesem Punkt stellt sich natürlich die Frage, wie wir dies nutzbar machen
können. Ist nämlich $p\nmid n$, so kennen wir eine Faktorisierung von $x^n-1$:
\[ x^n-1 \speq= \prod_{d\mid n} \Phi_d(x)\,.\]
Des Weiteren können wir eine Zerlegung von $\Phi_d(x)$ sogar genauer angeben:

%\chapter{Normalbasen mit Dickson Polynomen}

Sei hier stets $n$ eine zu $q$ teilerfremde natürliche Zahl.




\end{document}
