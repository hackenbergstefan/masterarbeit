\chapter{Normalbasen -- Ein Überblick}

Seien wieder $F := \F_q$ ein endlicher Körper von Charakteristik $p$ und 
$E := \F_{q^n} \mid F$ eine Körpererweiterung.
Wir wiederholen kurz die Definition einer \emph{Normalbasis}

\begin{definition}[normales Element, normales Polynom, Normalbasis]
  Sei $F$ ein Körper und $E \mid F$ eine endliche Galoiserweiterung von Grad
  $n$. Sei ferner $w\in E$ mit $F(w) = E$. $w$ heißt \emph{normal über $F$},
  falls
  \[ \{ \gamma(w) \mid \gamma \in G\}\]
  eine $F$-Basis von $E$ ist. 
  $\{ \gamma(w) \mid \gamma \in G\}$ heißt entsprechend \emph{Normalbasis} und
  $g(x) \in F[x]$ mit 
  \[ g(x) = \prod_{\gamma \in G}(x - \gamma(w))\]
  heißt \emph{normales Polynom}.
\end{definition}

Um effizient normale Elemente in $E\mid F$ zu finden, betrachten wir 
$(E,\sigma)$ als $F[x]$-Modul und nutzen die Aussagen aus
\autoref{chap:moduln}.

\begin{satz}
  \begin{enumerate}
    \item Die Erzeuger von $(E,\sigma)$ als $F[x]$-Modul sind genau die 
      normalen Elemente in $E\mid F$.
    \item Man hat eine Bijektion von Mengen
      \[ \{V_g :\ g(x) \in F[x] \text{ monisch mit } g(x) \mid x^n-1\}
        \overset{1-1}{\longleftrightarrow}
        \{F[x] \text{-Teilmoduln von }E\}\]
        wobei $V_g := \{v \in E : g(x)\cdot v = 0\} = \ker(g(\sigma))$.
    \item Jedes $V_g$ ist ein zyklischer Modul und es gilt
      \[u \text{ erzeugt } V_g \quad\Leftrightarrow \quad
        \Ord_q(u) = g(x).\]
        Insbesondere sind die Erzeuger von $V$ genau die Elemente $v \in V$ mit 
        $\Ord_\tau(v) = x^n-1$.
  \end{enumerate}
\end{satz}
\begin{proof}
  Alles schon in \TODO~gezeigt.
\end{proof}

Dies liefert uns die grundlegende Idee für das Auffinden von normalen
Elementen:
\begin{lemma}
  Sei $x^n-1 = \prod_{i=1}^s r_i(x)$ eine Zerlegung in paarweise teilerfremde
  Polynome, so gilt:
  \[ E \speq= \bigoplus_{i=1}^s V_{r_i} \,.\]
\end{lemma}
\begin{proof}
  \TODO.
\end{proof}

\begin{kor}
  Sei $x^n-1 = \prod_{i=1}^s r_i(x)$ eine Zerlegung in paarweise teilerfremde
  Polynome. Seien ferner $u_i \in V_{r_i}$ Elemente mit 
  $\Ord_q(u_i) = r_i(x)$ $\forall i=1,\ldots,s$. Dann ist
  \[ u \speq= u_1 + u_2 + \ldots + u_s\]
  normal in $E \mid F$.
\end{kor}
\begin{proof}
  Da obige Zerlegung von $x^n-1$ paarweise teilerfremd ist, folgt nach
  \cref{lemma:eigenschaften-tau-ordnung}
  $\Ord_q(u) = \prod_{i=1}^s \Ord_q(u_i) = \prod_{i=1}^s r_i(x) = x^n -1$.
\end{proof}

\begin{beispiel}
  
\end{beispiel}


An diesem Punkt stellt sich natürlich die Frage, wie wir dies nutzbar machen
können. Ist nämlich $p\nmid n$, so kennen wir eine Faktorisierung von $x^n-1$:
\[ x^n-1 \speq= \prod_{d\mid n} \Phi_d(x)\,.\]
Des Weiteren können wir eine Zerlegung von $\Phi_d(x)$ sogar genauer angeben:
