\chapter{Normalbasen mit Dickson Polynomen}

\chapter{Vollständige Normalbasen mit Dickson Polynomen}

\section{Hier eine Section}

Sei hier stets $n$ eine zu $q$ teilerfremde natürliche Zahl.

\begin{lemma}
  Es gilt:
  \begin{enumerate}
    \item $| M_q(l\bmod n)| = \ord_{\tilde n}(q)$, 
      wobei $\tilde n(q) = \tfrac{n}{\ggT(n,l)}$.
    \item Es gibt ein vollständiges Repräsentantensystem $R_q \subset \Z_n$, so
      dass 
      \[ \{0,1,...,n-1\} = \bigcupdot_{l\in R_q} M_q(l \bmod n) \,.\]
  \end{enumerate}
\end{lemma}
\begin{proof}
  \begin{enumerate}
    \item Per definitionem ist die $\tilde n$ Ordnung von $q$ gerade die
      kleinste natürliche Zahl $k$, so dass $q^k \equiv 1 \pmod{\tilde n}$ gilt.
      Damit folgt die Behauptung in offensichtlicher Weise.
    \item 
  \end{enumerate}
\end{proof}

